% Degrees
%\newcommand{\degree}{\ensuremath{^\circ}}
\DeclareMathOperator*{\argmax}{arg\,max}


% References
\newcommand{\secref}[1]{afsnit \ref{#1}}
\newcommand{\chapref}[1]{kapitel \ref{#1}}
\newcommand{\figref}[1]{figur \ref{#1}}
\newcommand{\lstref}[1]{eksempel \ref{#1}}
\newcommand{\apref}[1]{bilag \ref{#1}}
\newcommand{\tableref}[1]{tabel \ref{#1}}
\newcommand{\itemref}[1]{element \ref{#1}}
\newcommand{\coderef}[1]{eksempel \ref{#1}} % deprecated, use \lstref

\newcommand{\fileref}[1]{\texttt{#1}}
\newcommand{\classref}[1]{\textcolor{purple}{\texttt{#1}}}
\newcommand{\methodref}[1]{\textcolor{blue}{\texttt{#1}}}
\newcommand{\dbtableref}[1]{\texttt{#1}}
\newcommand{\classmethodref}[2]{\classref{#1}\texttt{.}\methodref{#2}}

\newcommand{\tablespacing}{\newline \qquad \newline}

\floatname{algorithm}{Algoritme}
\renewcommand\lstlistingname{Eksempel}
\renewcommand\lstlistlistingname{Eksempler}

% Other
\newcommand{\capt}[1]{\caption{\emph{#1}}} % Emphesize caption
\newcommand{\arrows}[2]{#1$\Leftrightarrow$#2} % Pil mellem to inputs
\newcommand{\op}[1]{$\Uparrow$ #1} % Pil op
\newcommand{\ned}[1]{$\Downarrow$ #1} % Pil ned

%\titleformat{\subparagraph}
%    {\normalfont\normalsize\bfseries}{\thesubparagraph}{1em}{}
%\titlespacing*{\subparagraph}{\parindent}{3.25ex plus 1ex minus .2ex}{.75ex plus .1ex}

\newcommand{\skippar}{}

\newcommand{\fx}{f.eks. }
\newcommand{\Fx}{F.eks. }

\newcommand{\todo}[1]{\colorbox{yellow}{\color{red}\textbf{TODO} \hspace{1ex} #1}}
\newcommand{\tjek}[1]{TJEK MIG: \guillemotright #1 \guillemotleft}


\newcommand{\tabelmedoverskrift}[2]{\begin{tabular}{p{\textwidth}}
    \hline
     \parbox[t][0.4cm][t]{\textwidth}{\begin{vplace}[-0.25]\centering\textbf{\textit{#1}}\end{vplace}}\\ \hline
     \vspace{-2mm}#2 \\ \hline
  \end{tabular}}
\newcommand{\aktortabel}[3]{\tabelmedoverskrift{#1}{
\textbf{Formål:} #2 \\ 
\textbf{Karakteristik:} #3
}}

\newcommand{\aktortabelEx}[4]{\brugtabel{#1}{#2}{#3 \\ 
    \textbf{Eksempler:} #4}}

\newcommand{\brugtabel}[4]{\tabelmedoverskrift{#1}{
\textbf{Brugsmønster:} #2 \\ 
\textbf{Objekter:} #3 \\ 
\textbf{Funktioner:} #4
}}