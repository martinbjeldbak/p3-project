\chapter{Samarbejde med informanter}
\label{akademiskinformanter}

Vi har haft et tæt samarbejde med to informanter fra Aalborg. De har hjulpet os med at fortolke problemet ved at deltage i interviews og diskussioner vedr. problemet. Dette har gjort det muligt for os at skabe en brugbar analyse til design af vores system. Informanter bliver beskrevet her:

\begin{description}
\item[Merete] er en familiemor fra Gistrup, der bor med sin mand. De har tre børn, som alle er flyttet hjemmefra. Merete står for madlavningen i husstanden. Både Merete og hendes mand har et arbejde, der skal passes, derfor føler de, at det ofte er svært at planlægge madlavningen i forvejen.

\item[Keld] er en familiefar fra Aalborg, der bor med sin kone og deres to små døtre. Keld står for både madlavningen og indkøb af varer til familien. Familien føler ikke, at de har meget fritid i hverdagen, derfor vælger Keld at lave store portioner aftensmad, så han ikke behøver at lave mad hver aften. Både Keld og hans kone har et arbejde, der skal passes, og når de kommer hjem fra arbejde, så ønsker de at bruge så meget tid som muligt på at være sammen med deres døtre.
\end{description}

Vi valgte at arbejde med netop disse informanter, fordi de var potentielle brugere af det system, som vi ønskede at udvikle. Begge informanter har været yderst interesserede i at hjælpe os igennem projektet og de har været gode til at udtrykke deres idéer, når vi havde diskussioner.

To familier kan ikke repræsentere en befolknings madvaner eller madlavningspolitik. Vi er helt klare over, at vores to informanters madvaner ikke kan generaliseres til hele Danmark. Vi vurderer dog, at samtalerne med informanterne giver et godt billede af, hvordan stituationen, mht. madlavningen og madvaner, ser ud i nogle danske husstande. 