\chapter{Samarbejde med informanter}
\label{akademiskinformanter}

Vi har haft et tæt samarbejde med to informanter fra Aalborg. De har hjulpet os med at fortolke problemet ved at deltage i interviews og diskussioner vedr. problemet. Dette har gjort det muligt for os at skabe en brugbar analyse til design af vores system. Informanter har været inddraget i forbindelse med 2 interviews hver, afprøvning af 3 prototyper og endelig en færdig test af programmet. Informanterne er kort beskrevet her, for en længere beskrivelse, se \secref{sec:samarbejde}.

\begin{description}
\item[Merete] er en familiemor, der bor med sin mand. De har tre børn, som alle er flyttet hjemmefra. Merete står for madlavningen i husstanden. Både Merete og hendes mand har et arbejde, der skal passes- Derfor føler de, at det ofte er svært at planlægge madlavningen på en måde så deres madrester bliver brugt.

\item[Keld] er en familiefar, der bor med sin kone og deres to små døtre. Keld står for både madlavningen og indkøb af varer til familien. Familien føler ikke, at de har meget fritid i hverdagen, derfor vælger Keld at lave store portioner aftensmad, så han ikke behøver at lave mad hver aften, men tilgengæld tit har rester til overs. Både Keld og hans kone har et arbejde, der skal passes, og når de kommer hjem fra arbejde, så ønsker de at bruge så meget tid som muligt på at være sammen med deres døtre.
\end{description}

Vi valgte at arbejde med netop disse informanter, fordi de var potentielle brugere af det system, som vi ønskede at udvikle. Begge informanter har været yderst interesserede i at hjælpe os igennem projektet, og de har været gode til at udtrykke deres idéer, når vi havde diskussioner.

Vi er klare over, at vores to informanters madvaner ikke kan generaliseres til hele Danmark. Vi vurderer dog, at samtalerne med informanterne giver et godt billede af, hvordan situationen, mht. madlavningen og madvaner, ser ud i nogle danske husstande. 

De to informanter har ca. samme vilkår mht. familien, job og IT-kompetencer, hvilket var meget tydeligt, da vi ofte fik meget lignende respons fra begge informanter. Hvis vi havde ressourcer til at arbejde sammen med flere informanter, der havde flere forskelle, der kunne påvirke hvordan de bruger systemet, ville vi helt sikkert gøre det. Det kan være at folk, der er eneboende sammenlignet med familieboende oplever problemet med madspild og brugen af madrester på en anden måde. Det kunne også være der var en forskel på rige kontra fattige. Vi mener, at dette ville være en fordel, og hvis der er mulighed for at have flere informanter i et fremtidigt projekt, så vil vi helt klart forsøge at få informanter med forskellige baggrunde med i projektet.

Vi synes informanternes inddragelse var helt tilpas. Vi havde hele tiden viden nok at arbejde med til at kunne videreudvikle systemet. Med færre afholdte møder ville vi nok have været i tvivl om hvordan dele af systemet skulle se ud med risiko for at skulle lave det om. Med flere afholdte møder ville vi trække for meget på vores egen men også informanternes tid, selvom det måske kunne have afklaret nogle mindre ting vi var i tvivl om. Vi valgte i stedet at ringe vores informanter op når vi havde nogle små tvivlsspørgsmål. Det skete \fx da vi var i tvivl om hvorvidt informanterne ønskede mængdeangivelser på deres indkøbsliste.

\section{Møder med informanter}
Projektets problemstilling var meget bred, og vi ønskede at skabe os et lidt bedre overblik og få mulighed for at fortolke problemet, som vi ønskede at arbejde med. De indledende møder med informanterne var derfor fokuseret på selve problemet, og hvordan de oplevede madspildet i deres respektive husstande. Hver gang vi skulle holde et møde eller præsentere noget for informanter, så tog vi hjem til dem og holdte møderne. Dette valgte vi at gøre, fordi vi vurderede, at informanterne måske ville føle sig mindre pressede af os og mere trygge, hvis de var i deres naturlige omgivelser, end hvis de skulle med os op til Universitet hver gang. Vi mente, at hvis informanterne var trygge, så ville de også være mere villige til at diskutere og tale med os.

På baggrund af de indledende møder havde vi viden nok til at diskutere og fortolket problemet, og deraf formulere to systemdefinitioner, som vi kunne præsentere for informanterne. Systemdefinitioner er en del af den objektorienterede tilgang til et projekt, hvilket vi mener fungerede rigtig godt, fordi systemdefinitionerne gav os mulighed for at præsentere vores idéer for informanterne, og informanterne havde mulighed for at give os feedback på, hvad de kunne se som en fornuftig løsning på problemet. Hvis det ikke var for systemdefinitionerne, så havde vi højst sandsynligt endt ud med et produkt, som indeholdt rigtig mange funktioner, som brugerne ikke ønskede at tage i brug. Det feedback, vi har fået af informanterne bl.a. vedr. systemdefinitionerne, har været rigtig godt, og det har ledt os i den rigtig retning. 

Vi ønskede ikke at have truffet nogle faste beslutninger omkring systemet, inden vi havde afholdt møder med informanterne. Derfor arbejdede vi med at udvikle semi-strukturerede interviews. Dette betyder, at vi havde en struktur og nogle spørgsmål, som vi ønskede at stille informanterne, men hvis de introducerede et emne, som vi ikke havde taget højde for, så ville det ikke ødelægge interviewets flow. Der var altså luft til eventuelle spørgsmål og nye emner midt i interviewet.

\section{Prototyper og usabilitytest}
Ud over interviews og diverse diskussioner med informanterne, præsenterede vi vores idéer for dem ved at udvikle prototyper. Vi benyttede os af papirsprototyper til de initierende afprøvninger. Denne form for prototype er ikke tidskrævende og er en billig form for præsentation. Da det var initierende afprøvninger, var det godt, ikke at bruge for mange kræfter på at udvikle prototyperne, da vi det sted i processen havde stor usikkerhed omkring hvordan systemet skulle fungere. Samtidig kunne det også have en ulempe at bruge meget tid på prototyperne, nemlig at vi ville have svært ved at give slip på idéerne, hvis informanterne ønskede noget helt andet end det prototyperne illustrerede.

I de senere forløb præsenterede vi informanterne for en hifi-prototype, der var brugt længere tid på at lave, og som til forveksling kunne ligne et rigtigt system. Vi lavede en diasshow-prototype, der havde til formål at undersøge hvilke funktioner systemet skulle bestå af. Prototypen var dynamiske, så man kunne klikke på knapper og navigere rundt i diasshowet. Systemets brugbarhed var en vigtig faktor for os, og vi ønskede at gøre det lettere for den madansvarlige i husstanden at bruge sine madrester i madlavningen. Derfor skulle systemet være intuitivt og nemt at gå til. Med diasshow-prototypen fik vi mulighed for at sikre os, at vores designidéer blev forstået af informanterne. Hvis informanterne \fx havde svært ved at finde nogle funktioner, så kunne det være, at de skulle gøres mere synlige. Lignende spørgsmål blev besvaret relativt tidligt i systemets udviklingsfase, hvilket var en fornuftig ting. Det gav os mere tid til at rette fejl og komme på nye designidéer, hvilket blev nødvendigt.

Det fungerede rigtig godt med papirsprototyper til de indledende afprøvninger. Vi kunne overraskende hurtigt præsentere vores idéer for informanterne og afprøve om idéerne helt basalt var brugbare. Informanterne kunne nemt forestille sig prototyperne afspejle et rigtigt system og finde fordele og ulemper ved dem, selvom de blot var lavet af papir. Det var ikke nødvendigt for os at lave helt nye prototyper, fordi informanterne gav os rigtig god feedback, som vi arbejdede videre med, men hvis det skulle blive nødvendigt, så ville det have været nemt at udvikle nye prototyper i de indledende faser af projektet, fordi de var hurtige og nemme at lave. Et eksempel kunne være, hvis det viste sig, i vores to første prototyper, at informanterne ikke kunne finde ud af at tilføje råvaretyper at søge på. Diasshow-prototyperne blev også præsenteret på et fornuftigt tidspunkt. Informanterne fik en smagsprøve af vores idéer til hvordan systemet skulle se ud, og hvilke funktioner man kunne bruge. Det var rigtig godt for os som udviklere at få lov til at afprøve vores idéer for at se, hvordan de ville reagere på systemet. Vi fik rigtig meget ud af det, og vi reviderede systemet ud fra de test, vi fik lavet med informanterne.

Brugen af prototyperne var generelt en god idé, for det gjorde det meget nemmere for informanterne at følge med i vores tankegange, fordi vi visualiserede idéerne for dem. Det er helt sikkert noget, vi vil tage med os videre i fremtidige projekter.

Som en sidste afprøvning af systemet præsenterede vi det funktionsdygtige system for informanterne. 
%Har fjernet dette, synes det er overflødigt DSU: Ved at præsentere informanterne for det funktionsdygtige system, var vi i stand til at sikre os en vis kvalitet, inden vi afsluttede projektet og produktudviklingen.
Til de sidste afprøvninger var der mulighed for at opdage kritiske fejl, fordi systemet nu var færdigt. Med et færdigt system har brugeren større mulighed for at være kritisk. I en papirsprototype vil en bruger måske tænke, at en fejl blot er med vilje og vil blive rettet, for det er jo bare en skitse han har fået. En anden forskel er, at vi nu sætter krav til systemet. Et eksempel er vores papirsprototyper, hvor det var facilitatoren, der kom med forslag når en bruger indtastede starten af en råvaretype i søgefeltet. I det færdige system lå ansvaret på systemet, og systemet gør ikke brugeren opmærksom på forslagene til det han indtaster i søgefeltet på samme måde, som når facilitatoren ved papirsprototyperne siger ``Vent lidt, jeg skal lige skifte layout, du får lige denne liste sat ind under dit søgefelt''.

Til den afsluttende test af systemet afholdt vi en test-session, hvor begge informanter skulle udføre en case. Efter casen benyttede vi Instant Data Analysis\cite{ida} til at analysere hvad der skete under casen og finde flest mulige usability-problemer. Casen var lavet for at få informanterne omkring flest mulige dele af systemet. Et punkt i casen kunne \fx være ``Opret en bruger på Foodl''. Fra tidligere afprøvninger havde vi bemærket, at informanterne virkede en smule anspændte under afprøvninger, så i stedet for at hele gruppen var til stede under vores afprøvning, installerede vi TeamViewer på den bærbare computer, som informanterne udførte casen på. TeamViewer sørgede for at dele bærbarens webcam, mikrofon og en live stream af skærmbilledet med en anden computer. På den måde kunne vi nøjes med kun at have 2 personer til stede ved casens udførelse. En facilitator, der guider informanten igennem casen og en logfører, der noterer hvad der sker. Resten af gruppen, der ikke var til stede under afprøvningen, fulgte meget præcist med i casen fra grupperummet på Cassiopeia og førte også en log imens.

Da de to cases var blevet udført, forlod vi efterfølgende informanterne og begyndte at brainstorme hvilke usability-problemer de to cases havde afsløret. Vi brugte hukommelsen og logbøgerne, og da vi efterhånden følte os meget sikre på, at vi havde fået alle de væsentlige problemer med, men kunne samtidig mærke en vis usikkerhed på, hvornår vi skulle stoppe, for der kunne jo lige pludselig dukke et problem mere op, som logføreren ikke havde noteret og som lå gemt dybt i hukommelsen. På dette punkt ville en videodata-analyse have en fordel, da vi blot ville kunne analysere et lille stykke film af gangen og være sikker på at finde alle usability-problemer i hver stump film. På den måde ville der ikke være samme usikkerhed omkring, hvornår vi burde stoppe, og hvor mange fejl vi ikke fandt. Når alt film var analyseret ville vi være færdige. Vi mener dog, at vi identificerede langt de fleste problemer, og at eventuelle mangler primært ville være kosmetiske problemer. Hvilken analysemetode vi vil vælge for et fremtidigt projekt vil selvfølgelig afhænge af hvilket system, der testes. Til et projekt lignende dette, ville vi til enhver tid benytte Instant Data Analysis, hvis lave tidsforbrug var altafgørende.
Informanterne virkede trygge ved brugen af Teamviewer, og de gruppemedlemmer, der ikke var til stede, mener alle at have fået lige så stort udbytte, hvis ikke større, som hvis de havde været til stede under testen. Det var nemlig nemmere at se informantens handlinger og tilhørende ansigtsudtryk på samme tid igennem Teamviewer. På den måde kunne man lægge mærke til, når der skete noget uventet for informanten.

I flere tilfælde i dette projekt har vi følt at informanterne var håbløst uvidende om hvordan en meget simpel ting skulle gøres i \Foodl. Det kunne \fx være at gå ind på sin indkøbsliste. Årsagen kan være at vi, datalogistuderende, har stor erfaring med brug af systemer, hvilket betyder at vi har brugt almindelige måder at gøre ting på \Foodl{}, som informanterne ikke har kendt til. Et eksempel kan være at man ved at trykke på sidens logo kommer tilbage til forsiden.
Vi har erfaret, at ved at arbejde på et projekt i næsten et halvt år, så får vi et indgående kendskab til mange små dele af systemet. Det indgående kendskab får os til at føle at en avanceret funktion er meget simplere end den vil fremstå for en ny bruger af siden. Derfor har vi nogle steder ikke givet nødvendig vejledning til hvordan en funktion bruges, som \fx søgefeltet på forsiden af \Foodl. I det store hele har vi lært hvor vigtigt det er at inddrage brugere når man designer et system.


\subsection{Løsning af usability-problemer}
Der var ikke tid til at løse alle usability-problemerne og vi opsummerer her de ændringer vi ville have lavet hvis vi havde haft tid.

\subsubsection{Indtastning af råvaretyper}
Begge informanter havde problemer med at indtaste råvaretyper første gang de brugte siden.
Dette mener vi er fordi det ikke er tydeligt at man skal vælge en råvaretype på en liste, i stedet for at man kan skrive fritekst som i de fleste andre søgemaskiner.

Vi vil derfor prøve at give bedre tilbagemeldning så brugeren bedre kan forstå hvad man skal gøre. 
Hvis der skrives en råvaretype som ikke eksistere i systemet dukker der ikke nogen forslag op og det ser ud som om at alt er som det skal være.
Men da dette ikke er tilfældet vil vi vise en besked hvor forslagne normalt dukker op som gør brugeren opmærksom på problemet. Man kunne supplere dette med et billede på forsiden første gang \Foodl besøges, hvor søgefeltet vises med en halvt indtastet tekst og nogle forslag i drop-down boksen. 

En skrivefejl førte også til at systemet ikke kunne give nogle forslag, hvilket førte til samme problem som før. 
Men da dette let kan ske burde systemet se bort fra disse og stadig give relevante forslag.

Et andet problem var at man hvis man ikke havde tilføjet den sidste ingrediens, men bare havde skrevet den i feltet, så søgte den uden den sidste ingrediens. 
For at undgå at dette sker skal knappen kun virke hvis feltet er tomt, så brugeren kan se at man skal tage stilling til det.

\subsubsection{Navigation}
Indkøbslisten var svært at finde for informanterne fordi de ikke lagde mærke til toolbaren i toppen af skærmen. 
For at gøre brugeren opmærksom på indkøbslisten, skal den oplyses når man tilføjer en opskrift eller ingrediens til indkøbslisten.

Konventionen at sidens logo fører til forsiden var heller ikke tydelig, og for at forbedre dette vil vi tilføje et ikon af et hus ved siden af logoet.

At skifte sortering på resultatsiden var også svært at finde, for at gøre det lettere at forstå vil vi gøre teksten mere selvforklarende. Man kunne eventuelt tilføje teksten ``Sorter efter:'' i forbindelse med knapperne.