\chapter{Samarbejde med informanter}
\label{akademiskinformanter}

Vi har haft et tæt samarbejde med to informanter fra Aalborg. De har hjulpet os med at fortolke problemet ved at deltage i interviews og diskussioner vedr. problemet. Dette har gjort det muligt for os at skabe en brugbar analyse til design af vores system. Informanter bliver beskrevet her:

\begin{description}
\item[Merete] er en familiemor, der bor med sin mand. De har tre børn, som alle er flyttet hjemmefra. Merete står for madlavningen i husstanden. Både Merete og hendes mand har et arbejde, der skal passes, derfor føler de, at det ofte er svært at planlægge madlavningen i forvejen.

\item[Keld] er en familiefar, der bor med sin kone og deres to små døtre. Keld står for både madlavningen og indkøb af varer til familien. Familien føler ikke, at de har meget fritid i hverdagen, derfor vælger Keld at lave store portioner aftensmad, så han ikke behøver at lave mad hver aften. Både Keld og hans kone har et arbejde, der skal passes, og når de kommer hjem fra arbejde, så ønsker de at bruge så meget tid som muligt på at være sammen med deres døtre.
\end{description}

Vi valgte at arbejde med netop disse informanter, fordi de var potentielle brugere af det system, som vi ønskede at udvikle. Begge informanter har været yderst interesserede i at hjælpe os igennem projektet, og de har været gode til at udtrykke deres idéer, når vi havde diskussioner.

To familier kan ikke repræsentere en befolknings madvaner eller madlavningspolitik. Vi er helt klare over, at vores to informanters madvaner ikke kan generaliseres til hele Danmark. Vi vurderer dog, at samtalerne med informanterne giver et godt billede af, hvordan stituationen, mht. madlavningen og madvaner, ser ud i nogle danske husstande. 

De to informanter har ca. samme vilkår mht. familien, job og IT-kompetencer, hvilket var meget tydeligt, da vi ofte fik meget lignende respons fra begge informanter. Hvis vi havde ressourcer til at arbejde sammen med flere informanter, som \fx kom fra forskellige dele af landet og havde forskellige IT-kompentencer, så ville dette være optimalt for kvalitetssikringen af produktet, fordi vi på den måde ville dække et større område af de potentielle brugere. Vi mener, at dette ville være en fordel, og hvis der er mulighed for at have flere informanter i et fremtidigt projekt, så vil vi helt klart forsøge at få forskellige informanter med i projektet.

\section{Møder}
Projektets problemstilling var meget bred, og vi ønskede at skabe os et lidt bedre overblik og få mulighed for at fortolke problemet, som vi ønskede at arbejde med. De indledende møder med informanterne var derfor fokuseret på selve problemet, og hvordan de oplevede madspildet i deres respektive husstande. Hver gang vi skulle holde et møde eller præsentere noget for informanter, så tog vi hjem til dem og holdte møderne. Dette valgte vi at gøre, fordi vi vurderede, at informanterne måske ville føle sig mindre pressede af os og mere trygge, hvis de var i deres naturlige omgivelser, end hvis de skulle med os op til Universitet hver gang. Vi mener, at hvis informanterne er trygge, så er de også mere villige til at diskutere og tale med os.

Efter vi havde diskuteret og fortolket problemet, var vi nu i stand til at formulere to systemdefinitioner, som vi kunne præsentere for informanterne. Systemdefinitioner er en del af den objektorienterede tilgang til et projekt, hvilket vi mener fungerede rigtig godt, fordi systemdefinitionerne gav os mulighed for at præsentere vores idéer for informanterne, og informatnerne havde mulighed for at give os feedback på, hvad de kunne se som en fornuftig løsning på problemet. Hvis det ikke var for systemdefinitionerne, så havde vi højst sandsynligt endt ud med et produkt, som indeholdt rigtig mange funktioner, som brugerne ikke ønskede at tage i brug. Det feedback, vi har fået af informanterne bl.a. vedr. systemdefinitionerne, har været rigtig godt, og det har ledt os i den rigtig retning. 

Vi ønskede ikke at have truffet nogle faste beslutninger, inden vi havde afholdt møder med informanterne. Derfor arbejdede vi med at udvikle semi-strukturerede interviews. Dette betyder, at vi havde en struktur og nogle spørgsmål, som vi ønskede at stille informanterne, men hvis de introducerede et emne, som vi ikke havde taget højde for, så ville det ikke ødelægge interviewets flow. Der var altså luft til eventuelle spørgsmål og nye emner midt i interviewet.

\section{Prototyper og kvalitetssikring}
Ud over interviews og diverse diskussioner med informanterne, så præsenterede vi vores idéer for dem ved at udvikle prototyper. Vi benyttede os af papirsprototyper til de initierende afprøvninger. Denne form for prototype er ikke tidskrævende og er en billig form for præsentation. I og med at det var initierende afprøvninger, syntes vi det var en god idé, ikke at bruge for mange kræfter på at udvikle prototyperne, da vi det sted i processen havde stor usikkerhed omkring hvordan systemet skulle fungere. Samtidig kunne det også have en ulempe at bruge meget tid på prototyperne, nemlig at vi ville have svært ved at give slip på idéerne, hvis informanterne ønskede noget helt andet end prototyperne illustrerede. I de senere afprøvninger brugte vi mere tid på prototyperne, da der her var blevet sat nogle fastere rammer og vi var blevet mere bevidste om informanternes behov. Med denne viden kunne vi afgrænse systemets udviklingsproces, og vi undgik at ende med et system, som ikke lå inden for rammerne. Dermed var risikoen, for at det system vi endte op med, var helt anderledes i forhold til informanternes ønsker, meget lille.

I de senere forløb præsenterede vi informanterne for diasshow-prototyper, der havde til formål at undersøge hvilke funktioner systemet skulle bestå af. Med sådan en prototype blev informanterne præsenteret for en prototype, der var dynamisk. Man kunne klikke på knapper og navigere rundt i diasshowet. Systemets brugbarhed var en vigtig faktor for os, og vi ønskede at gøre det lettere for den madansvarlige i husstanden at bruge sine madrester i madlavningen. Derfor skulle systemet være intuitivt og nemt at gå til. Med diasshow-prototyper fik vi mulighed for at sikre os, at vores designidéer blev forstået af informanterne. Hvis informanterne \fx havde svært ved at finde nogle funktioner, så kunne det være, at de skulle gøres mere synlige. Lignende spørgsmål blev besvaret relativt tidligt i systemets udviklingsfase, hvilket var en fornuftig ting. Det gav os mere tid til at rette fejl og komme på nye designidéer, hvilket blev nødvendigt.

Efter vores vurdering af udbyttet af de forskellige prototyper, vi har benyttet i projektet, synes vi, at det fungerede rigtig godt med papirsprototyper til de indledende afprøvninger. Vi kunne overraskende hurtigt præsentere vores idéer for informanterne og afprøve om idéerne helt basalt var brugbare. Det var ikke nødvendigt for os at lave helt nye prototyper, fordi informanterne gav os rigtig god feedback, som vi arbejdede videre med, men hvis det skulle blive nødvendigt, så ville det have været nemt at udvikle nye prototyper i de indledende faser af projektet, fordi de var hurtige og nemme at lave. Diasshow-prototyperne blev også præsenteret på et fornuftigt tidspunkt. Informanterne fik en smagsprøve af vores idéer til hvordan systemet skulle se ud, og hvilke funktioner man kunne bruge. Det var rigtig godt for os som udviklere at få lov til at afprøve vores idéer for at se, hvordan de ville reagere på systemet. Vi fik rigtig meget ud af det, og vi reviderede systemet ud fra de test, vi fik lavet med informanterne.

Brugen af prototyperne var generelt en god idé, for det gjorde det meget nemmere for informanterne at følge med i vores tankegange, fordi vi visualiserede idéerne for dem. Det er helt sikkert noget, vi vil tage med os videre i fremtidige projekter.

Som en sidste afprøvning af systemet præsenterede vi det funktionsdygtige system for informanterne. Ved at præsentere informanterne for det funktionsdygtige system, var vi i stand til at sikre os en vis kvalitet, inden vi afsluttede projektet og produktudviklingen. Til de sidste afprøvninger var der mulighed for at opdage kritiske fejl, fordi vi lod en person, der ikke havde været med til at udvikle systemet, bruge og teste det. Her kunne brugeren foretage sig nogle valg i systemet, som vi ikke havde tænkt over. Disse uforudsete valg måtte ikke være skyld i, at systemet gik ned. 

Til denne afsluttende test af systemet benyttede vi os af Instant Data Analysis. Vi afholdt en test-session, hvor begge informanter skulle udføre en case for at komme omkring flest mulige dele af systemet. Fra tidligere afprøvninger havde vi bemærket, at informanterne virkede en smule anspændt under afprøvninger, så i stedet for at hele gruppen var til stede under vores Instant Data Analysis, installerede vi TeamViewer på den bærbare computer, som informanterne udførte casen på. TeamViewer sørgede for at dele bærbarens webcam, mikrofon og en live stream af skærmbilledet med en anden computer. På den måde kunne resten af gruppen, der ikke var til stede under Instant Data Analysis, følge meget præcist med i casen.

Da de to cases var blevet udført, forlod vi efterfølgende informanterne og begyndte at brainstorme, hvilke usability-problemer de to cases havde afsløret. Vi brugte hukommelsen fra 5 mand og noter fra 1 logfører, der under hver case havde noteret hver gang han identificerede et problem. Vi følte os efterhånden meget sikre på, at vi havde fået alle de væsentlige problemer med, men kunne samtidig mærke en vis usikkerhed på, hvornår vi skulle stoppe, for der kunne jo lige pludselig dukke et problem mere op, som logføreren ikke havde noteret og som lå gemt dybt i hukommelsen. På dette punkt ville en videodata-analyse have en fordel, da vi blot ville kunne analysere et lille stykke film af gangen og være sikker på at finde alle usabality-problemer i hver stump film, vi analyserer af gangen. På den måde ville der ikke være samme usikkerhed omkring, hvornår vi burde stoppe, og hvor mange fejl vi ikke fandt. Når alt film var analyseret ville vi være færdige. Vi mener dog, at vi identificerede langt de fleste problemer, og at eventuelle mangler primært ville være kosmetiske problemer. Hvilken analysemetode vi vil vælge for et fremtidigt projekt vil selvfølgelig afhænge af hvilket system, der testes. Til et projekt lignende dette, ville vi til hver en tid benytte Instant Data Analysis, hvis lave tidsforbrug var altafgørende.