\chapter{Ruby on Rails}
\label{akademiskror}

Vi valgte Ruby og frameworket Ruby on Rails som platform, efter flere overvejelser om hvad der var bedst at implementere vores system i og hvad vi ville lære mest af. Det var ikke kun vigtig at det var hurtigt at implementere et godt system, vi ville også udvide vores kompentanser ved at vælge en platform vi ikke var fuldt bekendt i.

Vi overvejede følgende programmeringssprog og frameworks:
\begin{itemize}
\item C\# med ASP.NET MVC
\item Java
\item PHP med CakePHP
\item Ruby med Ruby on Rails
\end{itemize}

Java blev overvejet, men blev fravalgt fordi vi ikke fandt et lovende framework i Java. PHP er anerkendt som et programmeringssprog for webudvikling, men blev også fravalgt da flere af os ikke kan lide sproget. I forige semester blev vi oplært og udarbejdede et system i C\#, og vi valgte derfor Ruby i stedet, for at blive bekendt i et nyt programmeringssprog.

Ruby on Rails benytter 'konvention over specifikation' og 'scaffolding' til let at oprette et fungerene system. Det fik Ruby on Rails til at virke ``magisk'' hvilket var set både positivt og negativt. På den ene side var det hurtigt at få et fungerende system, på den anden side var det til tider svært præcist at vide hvad der foregik.



