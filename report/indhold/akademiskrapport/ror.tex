\chapter{Ruby on Rails}
\label{akademiskror}

Ruby on Rails er et framework til at udvikle webapplikationer i programmeringssproget Ruby som vi har benyttet i dette projekt til at implementere {\Foodl}.

Vi valgte Ruby og frameworket Ruby on Rails som platform, efter flere overvejelser om hvad der var bedst at implementere vores system i og hvad vi ville lære mest af. Det var ikke kun vigtig at det var hurtigt at implementere et godt system, vi ville også udvide vores kompentanser ved at vælge en platform vi ikke var fuldt bekendt i.

Vi overvejede følgende programmeringssprog og frameworks:
\begin{itemize}
\item C\# med ASP.NET MVC
\item Java
\item PHP med CakePHP
\item Ruby med Ruby on Rails
\end{itemize}

Java blev overvejet, men blev fravalgt fordi vi ikke fandt et lovende framework i Java. PHP er anerkendt som et programmeringssprog for webudvikling, men blev også fravalgt da flere af os ikke kan lide sproget. I forige semester blev vi oplært og udarbejdede et system i C\#, og vi valgte derfor Ruby i stedet, for at blive bekendt i et nyt programmeringssprog.

At lære et nyt sprog tager selvfølgelig ekstra tid. Udover Ruby var der mange relaterede sprog (HTML, CSS, JavaScript og MySQL) som der var meget svingende erfaring i mellem gruppens medlemmer. For at mindske oplæringstiden og undgå begynderfejl benyttede vi parprogrammering i starten, hvor vi kombinerede folks kompetanser.

Ingen fra gruppen havde dog erfaring med Ruby on Rails og det var derfor sværere at komme i gang med. Ruby on Rails benytter 'konvention over specifikation' og 'scaffolding' til let at oprette et fungerene system. Det fik Ruby on Rails til at virke ``magisk'' hvilket var set både positivt og negativt. På den ene side var det hurtigt at få et fungerende system, på den anden side var det til tider svært præcist at vide hvad der foregik. 'Konventioner' minsker forståeligheden hvis man ikke kender dem, for eksempel er variabler hvis navne starter med et stort bogstav konstanter.

\todo{memory leaks, performance (compile), development vs production, japaner, ustabilt}

