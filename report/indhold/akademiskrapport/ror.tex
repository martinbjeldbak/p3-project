\chapter{Ruby on Rails}
\label{akademiskror}

Vi valgte Ruby og frameworket Ruby on Rails som platform, efter flere overvejelser om hvad der var bedst at implementere vores system i og hvad vi ville l�re mest af. Det var ikke kun vigtig at det var hurtigt at implementere et godt system, vi ville ogs� udvide vores kompentanser ved at v�lge en platform vi ikke var fuldt bekendt i.

Vi overvejede f�lgende programmeringssprog og frameworks:
\begin{itemize}
\item C\# med ASP.NET MVC
\item Java
\item PHP med CakePHP
\item Ruby med Ruby on Rails
\end{itemize}

Java blev overvejet, men blev fravalgt fordi vi ikke fandt et lovende framework i Java. PHP er anerkendt som et programmeringssprog for webudvikling, men blev ogs� fravalgt da flere af os ikke kan lide sprogets opbygning. I forige semester blev vi opl�rt og udarbejdede et system i C\#, og vi valgte derfor Ruby i stedet for at blive bekendt i et nyt programmeringssprog.

Ruby on Rails var sv�rt at komme i gang med for flere af gruppemedlemmerne. Kombinationen af 'konvention over specifikation' og scaffolding medf�rte at udover at et kompliseret mappehieraki med tilh�rerende filer blev oprettet, s� var opf�rsel ikke altid specifiseret i den autogenerede kode da det var konventioner. Det bet�d at det var sv�rt at f� et overblik over hvilke filer gjorde hvad, og i selve filerne, hvordan koden p�virkede resten af systemet.

Efter folk blev bekendt med frameworket minskede disse problemer betydeligt, men det betyder stadig at forst�eligheden af vores implementation er lav hvis man ikke har brugt Ruby on Rails f�r, selvom man er en erfaren Ruby programm�r.