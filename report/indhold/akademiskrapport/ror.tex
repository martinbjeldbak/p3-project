\chapter{Ruby on Rails}
\label{akademiskror}

Ruby on Rails er et framework til at udvikle webapplikationer i programmeringssproget Ruby som vi har benyttet i dette projekt til at implementere {\Foodl}.

Vi valgte Ruby og frameworket Ruby on Rails som platform, efter flere overvejelser om hvad der var bedst at implementere vores system i og hvad vi ville lære mest af. Det var ikke kun vigtig at det var hurtigt at implementere et godt system, vi ville også udvide vores kompentanser ved at vælge en platform vi ikke var fuldt bekendt i.

Vi overvejede følgende programmeringssprog og frameworks:
\begin{itemize}
\item C\# med ASP.NET MVC
\item Java
\item PHP med CakePHP
\item Ruby med Ruby on Rails
\end{itemize}

Java blev overvejet, men blev fravalgt fordi vi ikke fandt et lovende framework i Java. PHP er anerkendt som et programmeringssprog for webudvikling, men blev også fravalgt da flere af os ikke kan lide sproget. I forige semester blev vi oplært og udarbejdede et system i C\#, og vi valgte derfor Ruby i stedet, for at blive bekendt i et nyt programmeringssprog.

At lære et nyt sprog tager selvfølgelig ekstra tid. Udover Ruby var der mange relaterede sprog (HTML, CSS, JavaScript og MySQL) som der var meget svingende erfaring i mellem gruppens medlemmer. For at mindske oplæringstiden og undgå begynderfejl benyttede vi parprogrammering i starten, hvor vi kombinerede folks kompetanser.

Ingen fra gruppen havde dog erfaring med Ruby on Rails og vi kunne ikke bruge samme metode her. Ruby on Rails benytter 'konvention over specifikation' og 'scaffolding' til let at oprette et fungerene system. Det fik Ruby on Rails til at virke ``magisk'' hvilket var set både positivt og negativt. På den ene side var det hurtigt at få et fungerende system, på den anden side var det til tider svært præcist at vide hvad der foregik. 'Konventioner' minsker forståeligheden hvis man ikke kender dem, for eksempel er variabler hvis navne starter med et stort bogstav konstanter i Ruby. Det tog derfor længere tid at lære Rails at kende end forventet. 
Rails fokus på MVC passede til gengæld godt med vores design og efter vi havde lært det grundlæggende hjalp det til at gøre det lettere at implementere vores system.

\todo{ overskrift? }
Under udviklingen af systemet oplevede vi stabilitets- og ydelsesproblemer med Rails.

Vores server havde kun 1GB RAM og efter et par dage stoppede Rails med at fungere fordi der ikke var mere hukommelse tilbage. Vi blev nød til at periodisk genstarte Rails \todo{og MySQL? hvad præcist?} for at undgå at systemet gik ned mens vi ikke holdt øje med det.

Rails køres enten i 'developement mode' eller 'production mode', hvor den første tillader at man kan let ændre i kodefilerne og se ændringerne med det samme, hvor 'production mode' er beregnet til et færdigudviklet system hvor man ikke har brug for ofte at ændre kodefilerne. En af forskellende er at alle JavaScript- og CSS-filer bliver kombineret til en enkelt JavaScript-fil og en CSS-fil i 'production mode', hvilket ikke sker i 'development mode'. For at øge organisesringen af vores kode og for at minske konflikter i forbindelse med vores revisionskontrolsystem, var vores kode delt op i mange filer. Men da alle filer skal overføres individuelt hver gang man tilgår en side, blev systemet langsomt i 'development mode'.

Et andet ydelsesproblem var i forbindelse med databasen. Mens vi udviklede på systemet på vores egne computere, valgte vi at tilgå databasen på serveren i stedet for at bruge en lokal kopi på vores egne computere. Vi valgte at gøre det på denne måde for at være sikker på at alle arbejdede på de samme data. Men vi oplevede at systemet kørte meget langsomere på vores egne computere. Vi fandt årsagen til at være at nogle sider, specielt søgeresultatsiden, udførte mange SQL-forespørgsler. Fordi at hver forespørgsel skulle sendes over internettet, tog det cirka 150 millisekunder at udføre hver forespørgsel.

På søgeresultatsiden blev der udført så mange forespørgsler at det tog næsten 1,5 minut for siden at blive generet og yderligere et halvt minut for at indlæse alle JavaScript- og CSS-filerne. Dette betød at udviklingen af den side tog længere tid, da hver gang man havde lavet en lille ændring skulle man vente 2 minutter for at se effekten af den. Dette var specielt problematisk i forhold til det visuelle design, da man ofte justere størrelser, farver og lignende og skal evaulere forskellen. Selvom det ikke var nær det samme problem på serveren da den kunne tilgå databasen direkte, tydeliggjorde det at der var et problem på den måde vi tilgik data i databasen, som senere blev rettet.

I fremtiden vil vi have en lokal kopi af databasen på vores computere og tilgå denne i stedet. På trods af at vi fandt og rettede et ydelsesproblem på grund af det, påvirkede det vores produktivitet for meget. Yderligere giver det også den fordel at vi kan arbejde på systemet uden at være forbundet til internettet, så man for eksempel kan arbejde på systemet mens man sidder i toget.

Selvom situationen blev bedre efter at serveren blev sat til at køre i 'production mode', har det efterladt os kritisk over om Rails kan benyttes i et system for effektivitet er meget vigtigt.

Alt i alt overvejede produktiviteten foresaget af brugen af Rails dens problemer hvilket betyder at vi gerne vil benytte Rails i fremtiden for webudvikling. Men ønsket om at lære alternative systemer vejer tungt og kan resultere i valget af et andet system.

