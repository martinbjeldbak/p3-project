\chapter{Indledning}

Til et systems udviklingsproces kan der være anvendt en række metoder og rekvisitter. Med rekvisitter menes eksempelvis softwareværktøjer, informanter eller andre remedier. Metoderne og rekvisitterne anvendes for at opnå en effektiv og flydende udviklingsproces, og for at opnå et system af så høj kvalitet som mulig, set i forhold til de givne rammer. Nogle metoder og rekvisitter kan være mere anerkendte og gennemarbejdede end andre, men det betyder ikke nødvendigvis, at en metode eller en rekvisit ville have været bedre at anvende end en anden. Uanset graden af anerkendelse, så er det vigtigt at reflektere over, hvorvidt den givne metode eller rekvisit, der har været anvendt, har været optimal. Hvis det ikke er tilfældet, hvad er årsagen så til dette? Hvilke andre metoder eller rekvisitter eksisterer, som i så fald kunne være blevet anvendt i stedet?

I projektforløbet har vi været seks datalogistuderende, som har haft en tidsperiode på cirka fire måned, til at udvikle et system, som skulle løse et realistik og eksisterende problem. I vores tilfælde valgte vi at arbejde med problemet madspild i et forsøg på at gøre det nemmere for danskerne at mindske dette. I udviklingsprocessen anvendte vi en anerkendt arbejdsmetode, kaldet den evolutionære arbejdsmetode. Vi havde en objetorienteret tilgang til problemet, hvor vi fulgte en metode fra bogen Objekt Orienteret Analyse \& Design \cite{ooad}. Vi havde et tæt samarbjede med to informanter. Vi anvendte et framework kaldet Ruby on Rails. Derudover anvendte vi en række andre rekvisitter, som vi ikke har tænkt os at beskrive her, da vi ikke mener, at de har være af lige så stor betydning for udviklingsprocessen som de førnævnte.

I denne akademiske rapport har vi valgt at beskrive og reflektere over følgende tre punkter:

\begin{itemize}[noitemsep]
  \item Samarbejdet med informanter (se \chapref{akademiskinformanter})
  \item Deb evolutionære- kontra den konstruktive arbejdsmetode, samt den objektorienterede tilgang (se \chapref{arbejdsmetoden})
  \item Vores tekniske platform, som var Ruby on Rails (se \chapref{akademiskror})
\end{itemize}

Desuden vil der til sidste i rapporten forefindes en konklusion (se \chapref{akademiskkonklusion}) på den akademiske rapport, hvori de væsentligste punkter fra hver af de tre overnævnte metoder/rekvisitter vil blive opsummeret og konkluderet på.

%% Informanter
%% Beskrive udviklingsprocessen
%% Den evolutionære arbejdsmetode
%% Anvendelse af RoR kontra .Net, Java eller Php
