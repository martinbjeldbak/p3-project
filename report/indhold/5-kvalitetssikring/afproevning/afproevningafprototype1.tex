\section{Afprøvning af prototype 1}

\paragraph{Mål}
Blandt 2 prototyper (1A og 1B), at udvælge den bedste metode til at finde og tilføje ingredienser, slette ingredienser, samt søge efter opskrifter med de valgte ingredienser.
Prototyperne er radikalt forskellige på dette område. Med prototype 1A vælges en ingrediens kun med musen. Dette gøres ved først at vælge en kategori, der muligvis bliver til en underkategori op til flere gange, og til sidst fås en liste af ingredienser indenfor den valgte kategori / underkategori, hvorfra en enkelt ingrediens nu kan vælges. Med prototype 1B vælges en ingrediens ved at indtaste dele af- eller hele navnet på ingrediensen i et søgefelt. Under indtastningen af navnet, kommer der løbende forslag til ingredienser, der starter med den indtastede tekststreng.
\begin{table}[H]
   \centering
    \begin{tabular}{|l|l|l|l|}
        \hline
        ~                    & \textbf{Fokus} & \textbf{Afgræsninger}              & \textbf{Forudsætninger}            \\ \hline
        \textbf{Grænseflade} & Forsiden       & Der fokuseres på, hvordan          & Brugeren kender formålet med siden \\
        ~                    & ~              & ingredienser findes og tilføjes    &                                    \\ \hline 
        \textbf{Funktioner}  & ~              & ~                                  & ~                                  \\  \hline
       \textbf{Model}        & ~              & ~                                  & ~                                  \\
        \hline
    \end{tabular}
    \capt{Hvad vides inden afprøvningen?}
    \label{table:afproevning1}
\end{table}	


\textbf{Opgaver til informant}

\textbf{For hver prototype:}
\begin{enumerate}[noitemsep]
\item Du ønsker at finde en opskrift, der indeholder både kylling, broccoli og ananas. Hvad gør du?
\item Du kommer i tanke om, at din ananas er alt for gammel, og ønsker at fjerne denne fra listen. Hvad gør du?
\end{enumerate} 

\textbf{Evaluering:}
\begin{enumerate}[noitemsep]
\item Hvilken prototype foretrak du? 
\end{enumerate}

\textbf{Afprøvning foretaget af informant Merete Munthe, 24 september, Gistrup:}
\tjek{Afprøvning kan ses på linket: kommer snart}

\textbf{Prototype 1A:}
\begin{enumerate}[noitemsep]
\item Informanten valgte at trykke søg hver eneste gang hun havde valgt en ingrediens. Altså vælg kylling, tryk søg, vælg broccoli, tryk søg og vælg ananas, tryk søg. Efter test af prototypen blev der for en sikkerheds skyld spurgt hvordan hun havde forstået opgaven. Hun havde forstået opgaven korrekt, nemlig at hun skulle søge efter de opskrifter, der indeholdt alle 3 ingredienser, og ikke foretage 3 forskellige søgninger på opskrifter indeholdende hver af disse 3 ingredienser.
\item Informanten slettede korrekt ananas ved at trykke på krydset.
\end{enumerate}

\textbf{Prototype 1B:}
\begin{enumerate}[noitemsep]
\item Informanten forstod at vælge ingredienserne korrekt, og trykkede først søg efter alle 3 ingredienser var korrekt valgt.
\item Informanten slettede korrekt ananas ved at trykke på krydset.
\end{enumerate}

\textbf{Evaluering}
Informanten foretrak prototype 1B, hvor man kan søge på ingredienser ved brug af tastaturet. Dette mente hun ville være hurtigere for hende, da hun ikke skulle tænke over hvilken kategori en ting ligger under, og valget af en ingrediens synes hun gik hurtigere med prototype 1B.