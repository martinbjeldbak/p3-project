\chapter{Konklusion}
\label{chap:konklusion}
%Problemformulering
%Opfyldelse af kriterier
%Implementering
%Informanternes inddragelse og samarbejde

I vores projektforløb har vi arbejdet med at løse et problem, som i problemformuleringen blev beskrevet som følger:

``Hvordan kan man, vha. et system, øge brugen af madrester i madlavningen i danske husstande?''

Til at løse problemet, udviklede vi en web-applikation, kaldet \Foodl{} (\url{http://foodl.dk}), som ved hjælp af indtastning af nogle ingredienser og madvarer, kommer med forslag til relevante opskrifter. Med relevante, menes opskrifter, som indeholder så mange af de indtastede ingredienser og madvarer som muligt, så disse kan blive anvendt.

Hvordan \Foodl{} fungerer på det tekniske niveau, er beskrevet i kapitel 6, kaldet Implementering \ref{chap:implementering}. Til udvikling af systemet, anvendte vi en objektorienteret metode fra bogen ``Objekt Orienteret Analyse \& Design'' \cite{ooad}. Metoden er meget anvendelig i forhold til at analysere og designe objektorienterede systemer, som eksempelvis web-applikationer. 

Hvordan \Foodl{} fungerer på det tekniske niveau, er bekskrevet i \chapref{chap:implementering}, implementering. Til udvikling af systemet, anvendte vi en objektorienteret metode fra bogen ``Objekt Orienteret Analyse \& Design'' \cite{ooad}. Metoden er meget anvendelig i forhold til at analysere og designe objektorienterede systemer, som eksempelvis web-applikationer. 

I projektforløbet har vi haft et tæt samarbejde med to informanter, Merete og Keld. Sammen med informanterne, formulerede vi en systemdefinition, som kan ses i \chapref{chap:systemvalg}, \secref{subsec:alternativesystemdefinitioner}. Systemdefinitionen opfylder BATOFF-modellens punkter, som beskriver hvad systemet skal kunne, hvor det skal kunne tilgåes fra, under hvilke forhold systemet bliver lavet, mm. Vi konkludere, at \Foodl{} lever op til systemdefinitionen, og at det, ved hjælp af \Foodl{}, er blevet lettere for danskerne, at få inspiration til, hvordan de kan få anvendt deres madrester.

Samarbejdet med informanterne omfatter møder, som har hjulpet os med, at blive mere bevidste om, hvilke kriterier systemet skulle opfylde, for at blive så attraktiv som muligt. Samarbejdet har desuden omfattet test af systemet ved hjælp af prototyper og en usability test af det endelige system. Ved brug af protoyper, blev vi mere bevidste om, hvordan \Foodl{}'s brugergrænseflade skulle opstilles, for at være brugbar, samt helt konkret, hvilke funktioner systemet skulle indeholde.

Alt det analyserede er blevet designet. Af det designede, er der dog ting vi ikke har valgt at implementere af hensyn til deadline for projektet:
\begin{itemize}
\item Dataudtrækkeren gennemgår kun opskriftssiderne på Arla én gang. Der er ikke implementeret muligheden for at vende tilbage til disse sider og tjekke om der er sket ændringer, for i givet fald at opdatere modellen.
\item Fejlrapportering af opskrifter er kun delvist implementeret. Man kan rapportere fejl i en opskrift fra søgesiden ud fra den opskrift, der er fejl i. Den generelle fejlrapporteringsknap, der findes som en footer på alle sider har dog ingen funktion.
\end{itemize}


\section{Perspektivering}
\label{chap:perspektivering}


