\chapter{Konklusion}
\label{chap:konklusion}
%Problemformulering
%Opfyldelse af kriterier
%Implementering
%Informanternes inddragelse og samarbejde

I vores projektforløb har vi arbejdet med at løse et problem, som i problemformuleringen blev beskrevet som følger:

``Hvordan kan man, vha. et system, øge brugen af madrester i madlavningen i danske husstande?''

Til at løse problemet, udviklede vi en web-applikation, kaldet \Foodl{} (\url{http://foodl.dk}), som ved hjælp af indtastning af nogle ingredienser og madvarer, kommer med forslag til relevante opskrifter. Med relevante, menes opskrifter, som indeholder så mange af de indtastede ingredienser og madvarer som muligt, så disse kan blive anvendt.

Hvordan \Foodl{} fungerer på det tekniske niveau, er beskrevet i kapitel 6, kaldet Implementering \ref{chap:implementering}. Til udvikling af systemet, anvendte vi en objektorienteret metode fra bogen ``Objekt Orienteret Analyse \& Design'' \cite{ooad}. Metoden er meget anvendelig i forhold til at analysere og designe objektorienterede systemer, som eksempelvis web-applikationer. 

Hvordan \Foodl{} fungerer på det tekniske niveau, er bekskrevet i \chapref{chap:implementering}, implementering. Til udvikling af systemet, anvendte vi en objektorienteret metode fra bogen ``Objekt Orienteret Analyse \& Design'' \cite{ooad}. Metoden er meget anvendelig i forhold til at analysere og designe objektorienterede systemer, som eksempelvis web-applikationer. 

I projektforløbet har vi haft et tæt samarbejde med to informanter, Marete Munthe, fra Gistrup, og Keld Kjær, fra Aalborg. Sammen med Marete Munthe og Keld Kjær, formulerede vi en systemdefinition, som kan ses i \chapref{chap:systemvalg}, \secref{subsec:alternativesystemdefinitioner}. Systemdefinitionen opfylder BATOFF-modellens punkter, som beskriver hvad systemet skal kunne, hvor det skal kunne tilgåes fra, under hvilke forhold systemet bliver lavet, mm. Vi konkludere, at \Foodl{} lever op til systemdefinitionen, og at det, ved hjælp af \Foodl{}, er blevet lettere for danskerne, at få inspiration til, hvordan de kan få anvendt deres madrester.

Samarbejdet med informanterne omfatter møder, som har hjulpet os med, at blive mere bevidste om, hvilke kriterier systemet skulle opfylde, for at blive så attraktiv som muligt. Samarbejdet har desuden omfattet tests af informanter. Test ved hjælp af protoyper, samt usability tests. Ud fra tests med protoyper, blev vi mere bevidste om, hvordan \Foodl{}'s brugergrænseflade skulle opstilles, for at være brugbar. 

Alt det analyserede er blevet designet. Af det designede, er der dog ting vi ikke har valgt at implementere af hensyn til deadline for projektet:
\begin{itemize}
\item Dataudtrækkeren gennemgår kun opskriftssiderne på Arla én gang. Der er ikke implementeret muligheden for at vende tilbage til disse sider og tjekke om der er sket ændringer, for i givet fald at opdatere modellen.
\item Fejlrapportering af opskrifter er kun delvist implementeret. Man kan rapportere fejl i en opskrift fra søgesiden ud fra den opskrift, der er fejl i. Den generelle fejlrapporteringsknap, der findes som en footer på alle sider har dog ingen funktion.
\end{itemize}

Samarbejdet med informanterne omfatter flere møder, som har hjulpet os med at blive mere bevidste om, hvilke kriterier systemet skulle opfylde, for at blive så attraktiv som muligt og hvordan det færdige system fungere i drift. Samarbejdet har desuden omfattet usability tests med informanter. Test ved hjælp af protoyper og case. Ud fra tests med protoyper, blev vi bevidste om, hvordan \Foodl{}'s brugergrænseflade skulle opstilles, for at blive så brugbar som muligt. 


\section{Perspektivering}
\label{sec:perspektivering}
%problem: madspild - mål:mindske madspild
Formålet med \Foodl{}; at mindske madspild i danske husstande, er ikke et nyt koncept. Vi har undersøgt tre systemer, der bygger på samme idé. Disse har vi forklaret i \secref{sec:eksisterendesystemer}. Systemet er relativt nyt, så der vil højst sandsynligt være mange punkter, hvor der kan ske ændringer, som vil gøre systemet bedre for befolkningen. Her prøver vi at reflektere over, hvordan dette kunne ske.

%flere begrænsningsmuligheder - allergikere
Systemet \Foodl{} implementerer ikke nogle muligheder for \fx allergikere at begrænse søgeresultater til \fx at udelukke opskrifter med nødder eller lignende. Dette kunne være en fornuftig feature at have, hvis man nemt og hurtigt ville have et overblik over de opskrifter, man selv kan have glæde af at lave.

%flere opskrifter
%andre sprog
%involvere flere websider - mindske madspild
Til webapplikationen \Foodl{} har vi kun kontaktet Arla og fået lov til at benytte os af deres, omkring 1000, opskrifter. Ved kun at hente opskrifter fra denne ene side støder vi på nogle ulemper, på grund af de særtræk, der er ved Arlas opskrifter; der er ikke opskrifter nok til at alle råvarer henviser til en opskrift. Enkelte ingredienser er Arlas egne produkter, hvilket giver en anledning til at tro at en lignende ingrediens måske kunne være lige så god eller bedre. Alle opskrifter virker til at være ret krævende med hensyn til antallet af ingredienser der bruges, \fx Arlas opskrift på øllebrød, der udover det basale: øl, brød og sukker, også indeholder kanel, appelsinskal, appelsinsaft og flødeskum. Ved at indhente opskrifter fra flere forskellige opskriftshjemmesider, vil det blive muligt at give brugerne adgang til mange flere opskrifter. Det vil hjælpe på problemet med at ikke alle råvarer er forbundet med en opskrift, og det vil også afhjælpe ulemperne ved de særtræk Arlas opskrifter har. 

Ved kun at benytte Arla som kilde, vil vi få et problem hvis Arlas sider er nede i et par dage. Ved at benytte 100 forskellige kilder, ville vi blot kunne deaktivere søgningen på Arlas opskrifter mens deres side er nede, hvilket ville få minimal betydning blandt 99 andre kilder, sammenlignet med nu, hvor \Foodl{} er afhængig af tilgængeligheden af Arlas hjemmeside.

Madspild forekommer ikke kun i Danmark. Derfor kan man også udvikle systemet til at understøtte andre sprog, og muligvis arbejde sammen med opskriftshjemmesider fra forskellige lande for at kunne levere webapplikationens ydelse til andre lande, der også kunne have gavn af mindre madspild.

%mere intelligent søgningsmetode - flest ud af, hvor mange ingredienser - hvis opskriften har mange ingredienser i forhold til opskrifter med færre ingredienser!
Ser vi på optimering af algoritmer, så skal der være særlig fokus på, hvordan en søgning foretages. På nuværende tidspunkt, så vises søgeresultater, der indeholder ingredienser, der er indtastet af brugeren. Altså den opskrift, der indeholder de fleste matchende ingredienser bliver vist først. Man kunne \fx sortere ud fra hvor stor en procentdel af alle opskriftens ingredienser, der matcher. På denne måde mindsker vi det eventuelle indkøb, der skal til for at lave en opskrift. 

Et eksempel kunne være en søgning på råvarerne kylling og pastinak. Hvis 2 opskrifter fremkommer, den ene med 10 ingredienser og den anden med 25, så vil det, at lave en af opskrifterne, kræve at man skaffer yderligere 8 eller 23 nye råvarer (forudsat man ikke i forvejen har disse). Det ville være smartest at få de opskrifter præsenteret først, der kræver mindst muligt indkøb. Ved at udvikle en mere intelligent søgning, så kan vi også mindske det fremtidige madspild, fordi vi begrænser det eventuelle indkøb af manglende råvarer.

