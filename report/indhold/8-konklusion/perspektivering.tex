\section{Perspektivering}
\label{sec:perspektivering}
%problem: madspild - mål:mindske madspild
I dette afsnit vil vi reflektere og persepktivere over de valg, vi har taget mht. det endelige system (\Foodl{}).

Formålet med \Foodl{}; at mindske madspild i danske husstande, er ikke et nyt koncept. \Foodl{} er relativt nyt, så der vil højst sandsynligt være mange punkter, hvor der kan ske ændringer, som vil gøre systemet bedre for befolkningen. Her prøver vi at reflektere over, hvordan dette kunne ske.

%flere begrænsningsmuligheder - allergikere
\Foodl{} implementerer ikke nogle muligheder for \fx allergikere at begrænse søgeresultater til \fx at udelukke opskrifter med nødder eller lignende. Dette kunne være en fornuftig feature at have, hvis man nemt og hurtigt ville have et overblik over de opskrifter, man selv kan have glæde af at lave.

%flere opskrifter
%andre sprog
%involvere flere websider - mindske madspild
Til webapplikationen \Foodl{} har vi kun kontaktet Arla og fået lov til at benytte os af deres, omkring 1000, opskrifter. Ved kun at hente opskrifter fra denne ene side støder vi på nogle ulemper, på grund af de særtræk, der er ved Arlas opskrifter. Enkelte ingredienser er Arlas egne produkter, hvilket giver en anledning til at tro at en lignende ingrediens måske kunne være lige så god eller bedre. Alle opskrifter virker til at være ret krævende med hensyn til antallet af ingredienser der bruges, \fx Arlas opskrift på øllebrød, der udover det basale: øl, brød og sukker, også indeholder kanel, appelsinskal, appelsinsaft og flødeskum. Ved at indhente opskrifter fra flere forskellige opskriftshjemmesider, vil det blive muligt at give brugerne adgang til mange flere opskrifter. Det vil hjælpe på problemet med at ikke alle råvarer er forbundet med en opskrift, og det vil også afhjælpe ulemperne ved de særtræk Arlas opskrifter har. 

Ved kun at benytte Arla som kilde, vil vi få et problem hvis Arlas sider er nede i et par dage. Ved at benytte 100 forskellige kilder, ville vi blot kunne deaktivere søgningen på Arlas opskrifter mens deres side er nede, hvilket ville få minimal betydning blandt 99 andre kilder, sammenlignet med nu, hvor \Foodl{} er afhængig af tilgængeligheden af Arlas hjemmeside.

Madspild forekommer ikke kun i Danmark. Derfor kan man også udvikle systemet til at understøtte andre sprog, og muligvis arbejde sammen med opskriftshjemmesider fra forskellige lande for at kunne levere webapplikationens ydelse til andre lande, der også kunne have gavn af mindre madspild.

%mere intelligent søgningsmetode - flest ud af, hvor mange ingredienser - hvis opskriften har mange ingredienser i forhold til opskrifter med færre ingredienser!
Ser vi på optimering af algoritmer, så skal der være særlig fokus på, hvordan en søgning foretages. På nuværende tidspunkt, så vises søgeresultater, der indeholder ingredienser, der er indtastet af brugeren. Altså den opskrift, der indeholder de fleste matchende ingredienser bliver vist først. Man kunne \fx sortere ud fra hvor stor en procentdel af alle opskriftens ingredienser, der matcher. På denne måde mindsker vi det eventuelle indkøb, der skal til for at lave en opskrift. 
Et eksempel kunne være en søgning på råvarerne kylling og pastinak. Hvis 2 opskrifter fremkommer, den ene med 10 ingredienser og den anden med 25, så vil det, at lave en af opskrifterne, kræve at man skaffer yderligere 8 eller 23 nye råvarer (forudsat man ikke i forvejen har disse). Det ville være smartest at få de opskrifter præsenteret først, der kræver mindst muligt indkøb. Ved at udvikle en mere intelligent søgning, så kan vi også mindske det fremtidige madspild, fordi vi begrænser det eventuelle indkøb af manglende råvarer.
