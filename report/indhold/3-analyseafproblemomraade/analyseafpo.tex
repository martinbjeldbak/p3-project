\chapter{Analyse af problemområde}
\label{chap:analyseafpo}

I følgendende kapitel vil vi analysere problemområdet. Analysemetoden tager udgangspunkt i bogen \cite[s. ~43]{ooad}. Problemområdet er den del af omgivelserne, som skal administreres, styres og overvåges af vores system, Foodl. I vores tilfælde er problemområdet: madresterne og madlavningen i de danske hustande. Formålet med analysen, er at skabe et overblik over, hvilke klasser og hændelser Foodl skal have, for at kunne udføre de funktioner vi ønsker, og for at systemet bliver så brugbart som muligt for dets fremtidige brugere. Udover at identificere klasser og hændelser, vil vi i starten af kapitlet undersøge og analysere nogle allerede eksisterende systemer, som tilbyder lignende funktioner. Dette gør vi for at finde ud af hvilke kvaliteter og mangler disse har, så vi derved opnår en bedre forståelse for, hvad Foodl skal kunne tilbyde brugerne, for at blive så attraktiv som mulig.



% Problemområde
\section{Eksisterende systemer}
Der eksisterer allerede en lang række systemer på internettet, som tilbyder en service, der ligner den, som vi ønsker Foodl skal kunne tilbyde. Disse systemer kalder vi for forbilleder. Heriblandt er Forbrugerrådets For Resten, DK-kogebogen, Opskrifter.dk samt flere. Der eksisterer desuden også en række engelske hjemmesider, som ligeledes tilbyder lignende service, som derfor også kunne være interessante at undersøge nærmere. Vi har dog valgt at fokusere på de tre førnævnte systemer: 

\begin{enumerate}[noitemsep]
  \item \href{https://play.google.com/store/apps/details?id=com.nodes.forresten}{For Resten} \cite{forresten}
  \item \href{http://www.dk-kogebogen.dk/}{DK-kogebogen} \cite{dkkogebogen}
  \item \href{http://opskrifter.dk/}{Opskrifter.dk} \cite{opskrifterdk}
\end{enumerate}

Dette er grundet, at de er dansksprogede, og derfor bør anses som værende direkte konkurrenter til Foodl. I følgende afsnit vil de tre systemer blive undersøgt og analyseret, for at finde ud af hvilke kvaliteter og mangler, systemerne har. Derved vil vi opnå en bedre forståelse for, hvad Foodl skal kunne tilbyde brugerne, for at gøre systemet bedre end allerede eksisterende systemer. DK-kogenbogen og Opskrifter.dk er begge web-applikationer, som kan tilgåes fra enhver web-browser, mens For Resten er en mobil-applikation udviklet til iOS- og Android-smartphones. Nogle kriterier er mere relevante at undersøge end andre.Vi vil undersøge og analysere de følgende features:

\begin{itemize}[noitemsep]
  \item Antal opskrifter
  \item Kvalitet af opskrifter
  \item Fleksibilitet
  \item Opskriftssøgningsfunktion
\end{itemize}

Det er relevant at undersøge, hvor mange opskrifter de tre forbilleder har i deres databaser. Jo færre opskrifter de har, jo større er risikoen nemlig for, at man, som bruger, ikke får nogle resultater, når man søger efter opskrifter med en specifik ingrediens. Kvaliteten af opskrifterne er også vigtig. Er hjemmesiden \fx et system, der tillader alle og enhver at uploade deres opskrifter til hjemmesiden, så er der en risiko for at nogle opskrifter vil være dårlige, eller ligefrem ubrugelige. Oplever brugeren, gang på gang, at han/hun, under en søgning, får dårlige eller ubrugelige opskrifter som søgningsresultater, så vil sidens troværdighed mindskes. 
Hjemmesidens fleksibilitet vurderes ud fra om brugeren har mulighed for \fx at op-skalére eller ned-skalére opskrifter således, at de er tilpasset flere eller færre personer; om det er muligt at sortere efter tilberedningstid eller andre ting, og om brugeren har mulighed for at sætte begrænsninger op for, hvilke opskrifter han/hun ønsker skal vises (\fx kun opskrifter uden svinekød, laktose, nødder osv.). Den sidste egenskab, som gruppen ønsker at analysere og undersøge, er opskriftssøgningsfunktionen. Dette er hovedfunktionen for systemet. Her vil vi undersøge, hvordan de tre forbillederne har valgt at bygge deres tøm-køleskabs-funktion op. \Fx hvordan man søger på ingredienser, hvor let tilgængelig funktionen er mm.
 
\section{Klasser}
\label{sec:klasser}

En klasse er en beskrivelse af en samling af objekter med samme struktur, adfærdsmønster og attributter \cite[s. ~51]{ooad}. To klasser kan have en association mellem dem, men det er ikke altid helt klart, hvad denne association skal betyde. Vi har løst dette problem ved at navngive nogle af associationerne mellem klasser. Ser vi på \figref{fig:klassediagram} i \secref{sec:struktur}, så ser vi eksempler på navngivne associationer. Her er en ``Vare'' associeret med en ``Person'', og associationen er navngivet ``Indkøbsliste''. 

Overblikket over hvilke klasser systemet skal have, kan skabes ved først at finde så mange klasserkandidater som muligt, og dernæst at afgrænse disse, så kun de mest relevante klasser er tilbage. I vores tilfælde anvendte vi en kombination af rigebilleder, som kan ses i \figref{fig:rigbillede1} og \figref{fig:rigbillede2}, samt systemdefinitionen som hjælpemidler til at finde frem til de klasser, som vi vil modellere i problemområdet.

Grundet den evolutionære arbejdsmetode, som vi har arbejdet ud fra, er klasser undervejs i forløbet blevet tilføjet og fjernet. De klasser som er blevet fravalgt, kan ses i \apref{ap:fravalgteklasser}. Herunder ses de valgte klasser, samt beskrivelser og begrundelser for, hvorfor de er med i vores problemområde.

\begin{description}
\item[Person] \hfill \\
En person er en, der er ansvarlig eller hjælper til med madlavningen i husstanden. En person kan være i besiddelse af en indkøbsliste, der bruges til at håndtere fremtidige indkøb af varer. Derudover kan en person have bogmærket nogle opskrifter, som vedkommende synes godt om.

\item[Opskrift] \hfill \\
En opskrift er den centrale klasse i problemområdet. En opskrift kan findes i en opskriftsamling, i en kogebog eller på nettet. Opskrifter indeholder en ingrediensliste. En opskrift kan være ikke-bogmærket eller bogmærket, hvis en person ønsker, at opskriften skal være let tilgængelig eller ej.

\item[Råvaretype] \hfill \\
En råvaretype findes i køleskabene, på madhylderne i husstanden og i supermarkederne. Råvaretype adskiller sig fra klassen ingrediens, ved at en råvaretype ikke har nogen enhed eller mængde. Et eksempel på råvaretyper kunne være ``gulerødder'', ``mælk'', ``hakket oksekød'' osv. 

\item[Ingrediens] \hfill \\ 
En ingrediens tilhører en råvaretype, og består af en mængde og en enhed. En ingrediens kan befinde sig på en ingrediensliste på en opskrift. Klassen ingrediens skiller sig ud fra råvaretype, netop fordi den udover at være en råvaretype, også har en mængde og en enhed. Eksempler på ingredienser er: ``3 styk gulerødder'', ``4 liter mælk'', ``500 gram hakket oksekød'' osv.

\item[Vare] \hfill \\
En vare er en vare som man kender det fra supermarkeder. En vare, i modsætning til ingredienser, tilhører ikke en råvaretype. En vare kan \fx være toiletpapir, vaskepulver eller blyanter. En vare kan befinde sig på en persons indkøbsliste.

\item[Fejl] \hfill \\
Fejl opstår som \fx en mængdefejl i en opskrift eller en manglende instruktion i fremgangsmåden. Fejl er uventede, og det er svært at sige, hvor de opstår og i hvilket omfang.

\end{description}

Ud fra de valgte klasser, skal vi have formuleret nogle hændelser, som beskriver klassernes adfærd. For klassen ``opskrift'', kan en hændelse \fx være ``opskrift fundet''. En hændelse er en bestemt adfærd for en klasse, som beskrives med udsagnsord. En hændelse kan involvere en til flere klasser. Som i eksemplet før, så involverer hændelsen ``opskrift fundet'', både ``opskrift'' men også ``ingrediens'', da vi vurderer, at man skal finde en opskrift før man kan arbejde med ingredienser, fordi en opskrift består af ingredienser. Ligesom med klasser, så har vi gennem forløbet, pga. den evolutionære arbejdsmetode, tilføjet og fjernet hændelser, igen og igen. De hændelser, som er blevet fravalgt, kan ses i \apref{ap:fravalgteklasseroghaendelser}.

I \tableref{table:haendelsestabel} ses de valgte hændelser, og hvilke klasser, disse hændelser, har en indflydelse på. Hver hændelse forårsager et tilstandsskift, og disse tilstande kan ses i tilstandsdiagrammerne i \secref{sec:adfaerd}. Hændelsestabellen er resultatet af analysen af problemområdet, men vi præsenterer den allerede nu, da den giver et godt overblik over de hændelser og klasser, vi er endt op med. 

I hændelsestabellen benytter vi \iter-symbolet til at illustrere, at den tilhørende hændelse kan forekomme flere gange i den pågældende klasse. Det betyder, at denne hændelse kan ses som en iteration i tilstandsdiagrammerne i \secref{sec:adfaerd}. Derudover benytter vi \once-symbolet til at illustrere, at den tilhørende hændelse blot kan forekomme en gang i klassen. Det betyder, at hændelsen kan ses som en selektion eller en sekvens i tilstandsdiagrammerne.

\ourtable{haendelsestabel}{5}{Hændelsestabel for klasserne person, opskrift, ingrediens, råvaretype, vare og fejl}
                                                             {Klasser}
       {Hændelser             	}{ Opskrift & Person & Vare  & Ingrediens & Råvaretype & Fejl  }{
\ourrow{Opskrift smidt ud     	}{ \once    &        &       & \once      &            &       }
\ourrow{Opskrift fundet         }{ \once    &        &       & \once      &            &       }
\ourrow{Bogmærke sat ind      	}{ \iter    & \iter  &       &            &            &       }
\ourrow{Bogmærke fjernet      	}{ \iter    & \iter  &       &            &            &       }
\ourrow{Skrevet på indkøbsliste	}{ \iter    & \iter  & \once & \iter      &            &       }
\ourrow{Fjernet fra indkøbsliste}{          & \iter  & \once & \iter      &            &       }
\ourrow{Råvare opbrugt        	}{          &        &       &            & \iter      &       }
\ourrow{Råvare købt           	}{          &        &       &            & \iter      &       }
\ourrow{Fejl rapport\'{e}ret    }{ \iter    & \iter  &       &            &            & \once }
\ourrow{Tilbagemelding modtaget }{          & \once  &       &            &            & \once }
\ourrow{Fejl set bort fra       }{          &        &       &            &            & \once }
}


Problemområdet består af madrester og madlavning i husstanden, derfor er det vigtigt, at problemområdet bliver repræsenteret ordentligt, i form af klasser. Madlavning, skal ikke forståes, som at vi ønsker at overvåge, om der bliver lavet mad i husstanden, da det vi ønsker blot er at udvikle et system, der giver mulighed for at genbruge madrester, og give inspiration til madlavningen.
                %klasser + klassediagram
\section{Hændelser}
\label{sec:haendelser}
Ved hjælp af klassekandidaterne, er det nu muligt at finde hændelseskandidater. Hændelserne er kommet til verden ud fra diverse forskellige forløb, der kan påvirke klasser. For at opretholde en konsistens mellem hændelserne, er de formuleret i datid. Det er igen vigtigt at nævne, at følgende er \emph{kandidater} og kan ændres under den iterative arbejdsproces. 

\subsection{Valgte hændelser}
Følgende hændelser er blevet skabt ud fra de valgte klassekandidater, som gruppen kom frem til i \secref{sec:klasser}. Klassekandidaterne er karaktiseret ved hjælp af følgende hændelseskandidater: 

\begin{itemize} [noitemsep]
\item Råvare opbrugt
\item Råvare smidt ud
\item Råvare købt
\item Opskrift fundet
\item Opskrift valgt
\item Opskrift smidt ud
\item Bogmærke tilføjet
\item Bogmærke fjernet
\item Indkøbsliste oprettet
\item Indkøbsliste færdig
\item Indkøbsliste smidt ud
\item Ingrediens tilføjet
\item Ingrediens fjernet
\end{itemize}

\subsection{Hændelsestabel}
Når de valgte klasser og hændelser er kommet på plads, giver det mulighed at fremstille et hændelsestabel, der danner overblik over sammenhæng mellem klasser og fælles hændelser. Herunder ses hændelsestabellen:

\ourtable{haendelsestabel}{5}{Hændelsestabel for klasserne person, opskrift, ingrediens, råvaretype, vare og fejl}
                                                             {Klasser}
       {Hændelser             	}{ Opskrift & Person & Vare  & Ingrediens & Råvaretype & Fejl  }{
\ourrow{Opskrift smidt ud     	}{ \once    &        &       & \once      &            &       }
\ourrow{Opskrift fundet         }{ \once    &        &       & \once      &            &       }
\ourrow{Bogmærke sat ind      	}{ \iter    & \iter  &       &            &            &       }
\ourrow{Bogmærke fjernet      	}{ \iter    & \iter  &       &            &            &       }
\ourrow{Skrevet på indkøbsliste	}{ \iter    & \iter  & \once & \iter      &            &       }
\ourrow{Fjernet fra indkøbsliste}{          & \iter  & \once & \iter      &            &       }
\ourrow{Råvare opbrugt        	}{          &        &       &            & \iter      &       }
\ourrow{Råvare købt           	}{          &        &       &            & \iter      &       }
\ourrow{Fejl rapport\'{e}ret    }{ \iter    & \iter  &       &            &            & \once }
\ourrow{Tilbagemelding modtaget }{          & \once  &       &            &            & \once }
\ourrow{Fejl set bort fra       }{          &        &       &            &            & \once }
}


Hændelsestabellen er det sidste dokument i analyse af problemområdet, som nu kan antages som modelleret. 

\todo{Vær mere beskrivende\ldots}
             %hændelser, hændelsestabel, tilstandsdiagram

