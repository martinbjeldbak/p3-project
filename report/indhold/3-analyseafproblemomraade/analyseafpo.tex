\chapter{Analyse af problemområdet}
\label{chap:analyseafpo}

I følgendende kapitel vil vi analysere problemområdet. Analysemetoden tager udgangspunkt metoden beskrevet i bogen Objektorienteret Analyse \& Design (OOA\&D)\cite[s. ~43]{ooad}. Problemområdet er den del af omgivelserne, som skal administreres, styres og overvåges af vores system, Foodl. I vores tilfælde er problemområdet: madresterne og madlavningen i de danske hustande. Formålet med analysen, er at skabe et overblik over, hvilke klasser og hændelser Foodl skal have, for at kunne udføre de funktioner vi ønsker, og for at systemet bliver så brugbart som muligt for dets fremtidige brugere. Udover at identificere klasser og hændelser, vil vi i starten af kapitlet undersøge og analysere nogle allerede eksisterende systemer, som tilbyder lignende funktioner. Dette gør vi for at finde ud af hvilke kvaliteter og mangler disse har, så vi derved opnår en bedre forståelse for, hvad Foodl skal kunne tilbyde brugerne, for at blive så attraktiv som mulig.



% Problemområde
\section{Klasser}
\label{sec:klasser}

En klasse er en beskrivelse af en samling af objekter med samme struktur, adfærdsmønster og attributter \cite[s. ~51]{ooad}. To klasser kan have en association mellem dem, men det er ikke altid helt klart, hvad denne association skal betyde. Vi har løst dette problem ved at navngive nogle af associationerne mellem klasser. Ser vi på \figref{fig:klassediagram} i \secref{sec:struktur}, så ser vi eksempler på navngivne associationer. Her er en ``Vare'' associeret med en ``Person'', og associationen er navngivet ``Indkøbsliste''. 

Overblikket over hvilke klasser systemet skal have, kan skabes ved først at finde så mange klasserkandidater som muligt, og dernæst at afgrænse disse, så kun de mest relevante klasser er tilbage. I vores tilfælde anvendte vi en kombination af rigebilleder, som kan ses i \figref{fig:rigbillede1} og \figref{fig:rigbillede2}, samt systemdefinitionen som hjælpemidler til at finde frem til de klasser, som vi vil modellere i problemområdet.

Grundet den evolutionære arbejdsmetode, som vi har arbejdet ud fra, er klasser undervejs i forløbet blevet tilføjet og fjernet. De klasser som er blevet fravalgt, kan ses i \apref{ap:fravalgteklasser}. Herunder ses de valgte klasser, samt beskrivelser og begrundelser for, hvorfor de er med i vores problemområde.

\begin{description}
\item[Person] \hfill \\
En person er en, der er ansvarlig eller hjælper til med madlavningen i husstanden. En person kan være i besiddelse af en indkøbsliste, der bruges til at håndtere fremtidige indkøb af varer. Derudover kan en person have bogmærket nogle opskrifter, som vedkommende synes godt om.

\item[Opskrift] \hfill \\
En opskrift er den centrale klasse i problemområdet. En opskrift kan findes i en opskriftsamling, i en kogebog eller på nettet. Opskrifter indeholder en ingrediensliste. En opskrift kan være ikke-bogmærket eller bogmærket, hvis en person ønsker, at opskriften skal være let tilgængelig eller ej.

\item[Råvaretype] \hfill \\
En råvaretype findes i køleskabene, på madhylderne i husstanden og i supermarkederne. Råvaretype adskiller sig fra klassen ingrediens, ved at en råvaretype ikke har nogen enhed eller mængde. Et eksempel på råvaretyper kunne være ``gulerødder'', ``mælk'', ``hakket oksekød'' osv. 

\item[Ingrediens] \hfill \\ 
En ingrediens tilhører en råvaretype, og består af en mængde og en enhed. En ingrediens kan befinde sig på en ingrediensliste på en opskrift. Klassen ingrediens skiller sig ud fra råvaretype, netop fordi den udover at være en råvaretype, også har en mængde og en enhed. Eksempler på ingredienser er: ``3 styk gulerødder'', ``4 liter mælk'', ``500 gram hakket oksekød'' osv.

\item[Vare] \hfill \\
En vare er en vare som man kender det fra supermarkeder. En vare, i modsætning til ingredienser, tilhører ikke en råvaretype. En vare kan \fx være toiletpapir, vaskepulver eller blyanter. En vare kan befinde sig på en persons indkøbsliste.

\item[Fejl] \hfill \\
Fejl opstår som \fx en mængdefejl i en opskrift eller en manglende instruktion i fremgangsmåden. Fejl er uventede, og det er svært at sige, hvor de opstår og i hvilket omfang.

\end{description}

Ud fra de valgte klasser, skal vi have formuleret nogle hændelser, som beskriver klassernes adfærd. For klassen ``opskrift'', kan en hændelse \fx være ``opskrift fundet''. En hændelse er en bestemt adfærd for en klasse, som beskrives med udsagnsord. En hændelse kan involvere en til flere klasser. Som i eksemplet før, så involverer hændelsen ``opskrift fundet'', både ``opskrift'' men også ``ingrediens'', da vi vurderer, at man skal finde en opskrift før man kan arbejde med ingredienser, fordi en opskrift består af ingredienser. Ligesom med klasser, så har vi gennem forløbet, pga. den evolutionære arbejdsmetode, tilføjet og fjernet hændelser, igen og igen. De hændelser, som er blevet fravalgt, kan ses i \apref{ap:fravalgteklasseroghaendelser}.

I \tableref{table:haendelsestabel} ses de valgte hændelser, og hvilke klasser, disse hændelser, har en indflydelse på. Hver hændelse forårsager et tilstandsskift, og disse tilstande kan ses i tilstandsdiagrammerne i \secref{sec:adfaerd}. Hændelsestabellen er resultatet af analysen af problemområdet, men vi præsenterer den allerede nu, da den giver et godt overblik over de hændelser og klasser, vi er endt op med. 

I hændelsestabellen benytter vi \iter-symbolet til at illustrere, at den tilhørende hændelse kan forekomme flere gange i den pågældende klasse. Det betyder, at denne hændelse kan ses som en iteration i tilstandsdiagrammerne i \secref{sec:adfaerd}. Derudover benytter vi \once-symbolet til at illustrere, at den tilhørende hændelse blot kan forekomme en gang i klassen. Det betyder, at hændelsen kan ses som en selektion eller en sekvens i tilstandsdiagrammerne.

\ourtable{haendelsestabel}{5}{Hændelsestabel for klasserne person, opskrift, ingrediens, råvaretype, vare og fejl}
                                                             {Klasser}
       {Hændelser             	}{ Opskrift & Person & Vare  & Ingrediens & Råvaretype & Fejl  }{
\ourrow{Opskrift smidt ud     	}{ \once    &        &       & \once      &            &       }
\ourrow{Opskrift fundet         }{ \once    &        &       & \once      &            &       }
\ourrow{Bogmærke sat ind      	}{ \iter    & \iter  &       &            &            &       }
\ourrow{Bogmærke fjernet      	}{ \iter    & \iter  &       &            &            &       }
\ourrow{Skrevet på indkøbsliste	}{ \iter    & \iter  & \once & \iter      &            &       }
\ourrow{Fjernet fra indkøbsliste}{          & \iter  & \once & \iter      &            &       }
\ourrow{Råvare opbrugt        	}{          &        &       &            & \iter      &       }
\ourrow{Råvare købt           	}{          &        &       &            & \iter      &       }
\ourrow{Fejl rapport\'{e}ret    }{ \iter    & \iter  &       &            &            & \once }
\ourrow{Tilbagemelding modtaget }{          & \once  &       &            &            & \once }
\ourrow{Fejl set bort fra       }{          &        &       &            &            & \once }
}


Problemområdet består af madrester og madlavning i husstanden, derfor er det vigtigt, at problemområdet bliver repræsenteret ordentligt, i form af klasser. Madlavning, skal ikke forståes, som at vi ønsker at overvåge, om der bliver lavet mad i husstanden, da det vi ønsker blot er at udvikle et system, der giver mulighed for at genbruge madrester, og give inspiration til madlavningen.
 
\section{Struktur}
\label{sec:struktur}

Der kan opbygges en struktur imellem de forskellige klasser. Dette kan ses i \figref{fig:klassediagram}. 

\begin{figure}
  \centering
  \input{billeder/klasseDiagram.pdf_tex}
  \capt{Klassediagram for problemområdet.}
  \label{fig:klassediagram}
\end{figure}


Klassediagrammet ovenover er bygget op af aggregeringer og associationer imellem klasserne i diagrammet. Hierakimønsteret benyttes idet at en indkøbsliste består af 0 til mange ingredienser, som hvor især består af en råvare. Hver enkelt klasses forbindelse forklares her nærmere:

\begin{description}
  \item[Bogmærke] \hfill \\
    Et bogmærke er associeret med netop én opskrift. Multipliciteten er valgt på baggrund af at et bogmærke kan sættes på én og kun én side i en kogebog.

  \item[Opskrift] \hfill \\
    En opskrift kan være associeret med 0 eller 1 bogmærke. Man kan argumentere for, at der godt kan placeres flere bogmærker på samme opskrift, men vi ønsker kun at overvåge hvorvidt en opskrift er bogmærket eller ej, og det er derfor kun nødvendigt at skelne mellem 0 og 1 bogmærker. En opskrift består af 1 til flere ingredienser. Hvis opskriften ikke bestod af ingredienser, ville det ikke være en opskrift.

\item[Indkøbsliste] \hfill \\
  En indkøbslisten består af 0 til flere ingredienser. 0 er muligt, da indkøbslisten er tom inden man tilføjer ingredienser til den. Det kan også være at man starter med bare at skrive ``Jeg handler ind kl. 17, skriv hvad jeg skal købe - Farmand'', og lader resten af familien udfylde sedlen.

\item[Ingrediens] \hfill \\
  En ingrediens består af netop én råvare og aggregeres altid af kun én opskrift. Selvom flere opskrifter indeholder oksekød, kan mængden være forskellig. Samtidig er det logisk, at hvis to opskrifter begge indeholder en ingrediens \textit{400 g oksekød}, så kan den ene opskrift ændres til \textit{500 g oksekød}, uden at den anden opskrift skal ændres.

\item[Råvare] \hfill \\
  En råvare er en dekomponering af en ingrediens. En råvare er blot en ingrediens uden nogen form for information om mængde eller enhed. En ingrediens kunne for eksempel være \textit{400 g oksekød}, som består af råvaren \textit{oksekød}.
\end{description}

            
\section{Adfærd}
\label{sec:adfaerd}

Den sidste aktivitet i analyse af problemområdet, består i at analysere og beskrive adfærdsmønstre for klassers objekter. Formålet med dette, er at få en bred for forståelse af, hvordan objekter kan opføre sig, hvilket vi vil bruge senere, når vi skal beskrive de funktioner \Foodl{} skal have. Dermed opnåes der også en mere glidende overgang over mod analyse af anvendelsesområdet \chapref{chap:analyseafao}. Desuden har vi anvendt adfærdmønstrene til at få overblik over, om de hændelser vi har valgt, er fyldestgørende, og ydermere, til at få inspiration til nye hændelser. Adfærdsmønstre har vi valgt at beskrive ved hjælp af tilstandsdiagrammer. For hver klasse følger der et tilstandsdiagram, som illusterer adfærden for objekter af den pågældende klasse. I tilstandsdiagrammerne skal de afrundede rektangulære bokse med tekst, anses som tilstande, som den pågældende klasse kan have. Pilene der fører til en tilstand, skal anses som hændelser, som kan være skyld i et tilstandsskift for objektet. Som eksempel, kan et objekt af klassen råvare, skifte mellem tilstandene ``eksisterer'' og ``brugbar'', ved hjælp af hændelserne ``Råvare købt'' og ``Råvare opbrugt''. Går en pil med en hændelse til og fra, den samme tilstand, er der tale om en løkke. En løkke er en hændelse som kan ske igen og igen, uden at objektet ændre tilstand. Eksempelvis kan hændelsen ``vare tilføjet'' ske igen og igen, for et indkøbsliste-objekt. Som oftest vil en klasse have en start hændelse, som ``sætter gang'' i  

\subsection{Opskrift}
En opskrift kan befinde sig i en kogebog, på et stykke papir eller på en hjemmesiden. I \figref{fig:opskrift-adfaerd}, ses adfærden for klassen ``Opskrift''. En opskrift kommer til live, ved hændelsen \textit{opskrift fundet}. Her befinder den sig i tilstanden \textit{findes i opskriftssamlingen}. Hvis man er rigtig glad for en opskrift, kan man sætte et bogmærke på opskriften, så man hurtigt kan finde den igen. Dette kan \fx gøres ved at sætte en post-it note på en bestemt side i kogebogen. Sådan en hændelse kaldes for \textit{bogmærke tilføjet}. Hvis man ombestemmer sig og fjerner bogmærket, indtræffer hændelsen \textit{bogmærke fjernet}. Derudover kan man være nødsaget til at handle ind til en specifik opskrift, fordi man mangler nogle ingredienser. Det betyder, at man skal tilføje nogle opskrifter til en indkøbsliste. Denne hændelser kaldes for \textit{skrevet på indkøbsliste} med den modsignende hændelse \textit{fjernet fra indkøbsliste}. Der kan også eksistere en fejl i opskriften, som man muligvis lægger mærke til. En sådan fejl kunne være et manglende billede til opskriften, eller at der står 20 minutter i ovnen i stedet for 40. Dette er beskrevet ved hjælp af hændelsen ``fejl fundet''. Disse fire hændelser medfører ikke et tilstandsskift. Opskriften ryger kun ud af tilstanden \textit{findes i opskriftssamlingen}, når opskriften fjernes, vha. hændelsen \textit{opskrift smidt ud}..

\pdffig[0.8]{tilstandsdiagrammer/opskrift}
  {Tilstandsdiagram for klassen opskrift. Klassen har én tilstand (findes i opskriftssamlingen) og en afsluttende hændelse ``opskrift fjernet'', der fører klassen ud i en sluttilstand.}
  {fig:opskrift-adfaerd}

\subsection{Bogmærke}
Som nævnt kan man vælge at tilføje et bogmærke til en opskrift, man kan lide. Denne hændelse kaldes \textit{bogmærke tilføjet}, og bringer klassen i tilstanden \textit{aktiv}. Bogmærket er aktiv, indtil den kommer i sluttilstanden efter, at hændelsen \textit{bogmærke fjernet} indtræffer. Se \figref{fig:bogmaerke-adfaerd}.

\begin{figure}[H]
	\centering
	\scalebox{0.8}{
	\subsection{Bogmærke}
Som nævnt kan man vælge at tilføje et bogmærke til en opskrift, man kan lide. Denne hændelse kaldes \textit{bogmærke tilføjet}, og bringer klassen i tilstanden \textit{aktiv}. Bogmærket er aktiv, indtil den kommer i sluttilstanden efter, at hændelsen \textit{bogmærke fjernet} indtræffer. Se \figref{fig:bogmaerke-adfaerd}.

\begin{figure}[H]
	\centering
	\scalebox{0.8}{
	\input{billeder/tilstandsdiagrammer/bogmaerke.pdf_tex}}
	\capt{Tilstandsdiagram for klassen bogmærke. De afrundede rektangulære bokse med tekst, skal anses som tilstande, som klassen kan have. De pile, der fører til en tilstand, skal anses som hændelser, som kan være skyld i et tilstandsskift. I dette tilfælde har klassen én tilstand (aktiv) og en afsluttende hændelse, der fører klassen ud i en sluttilstand (den sorte prik i den sorte cirkel).}
	\label{fig:bogmaerke-adfaerd}
\end{figure}}
	\capt{Tilstandsdiagram for klassen bogmærke. De afrundede rektangulære bokse med tekst, skal anses som tilstande, som klassen kan have. De pile, der fører til en tilstand, skal anses som hændelser, som kan være skyld i et tilstandsskift. I dette tilfælde har klassen én tilstand (aktiv) og en afsluttende hændelse, der fører klassen ud i en sluttilstand (den sorte prik i den sorte cirkel).}
	\label{fig:bogmaerke-adfaerd}
\end{figure}
\begin{figure}[htp]
\centering
\scalebox{0.6}{
\begin{figure}[htp]
\centering
\scalebox{0.6}{
\input{billeder/tilstandsdiagrammer/ingrediens.pdf_tex}}
\capt{Tilstandsdiagram for Ingrediens-klassens adfærdsmønstre}\label{fig:ingrediens-adfaerd}
\end{figure}}
\capt{Tilstandsdiagram for Ingrediens-klassens adfærdsmønstre}\label{fig:ingrediens-adfaerd}
\end{figure}
\paragraph{Råvare}
Når nogen i sin husstand køber en råvare, \fx i et supermarked, indtræffer hændelsen \textit{råvare købt} netop. I den forbindelse kommer et råvare-objekt til verdenen. Denne råvare er klar til at blive brugt, hvilket vil sige at råvaren har tilstanden \textit{brugbar}, indtil den havner i sin sluttilstande ved at man enten smider råvaren ud eller er har opbrugt den helt. Disse to hændelser kaldes i nævnte rækkefølge \textit{råvare smidt ud} og \textit{råvare opbrugt}.
\begin{figure}[htp]
\centering
\scalebox{0.6}{
\paragraph{Råvare}
Når nogen i sin husstand køber en råvare, \fx i et supermarked, indtræffer hændelsen \textit{råvare købt} netop. I den forbindelse kommer et råvare-objekt til verdenen. Denne råvare er klar til at blive brugt, hvilket vil sige at råvaren har tilstanden \textit{brugbar}, indtil den havner i sin sluttilstande ved at man enten smider råvaren ud eller er har opbrugt den helt. Disse to hændelser kaldes i nævnte rækkefølge \textit{råvare smidt ud} og \textit{råvare opbrugt}.
\begin{figure}[htp]
\centering
\scalebox{0.6}{
\input{billeder/tilstandsdiagrammer/raavare.pdf_tex}}
\capt{Tilstandsdiagram for Råvare-klassens adfærdsmønstre}\label{fig:raavare-adfaerd}
\end{figure}}
\capt{Tilstandsdiagram for Råvare-klassens adfærdsmønstre}\label{fig:raavare-adfaerd}
\end{figure}
\subsection{Indkøbsliste}
En indkøbsliste kommer til verden ved, at man i husstanden beslutter sig for at benytte en eller anden form for huskeliste for ting, der skal handles ind. Dette kan være i form af et papir, der ligger et fast sted på bordet eller hænger på opslagstavlen. Denne initierende hændelse kaldes \textit{indkøbsliste oprettet}. Se \figref{fig:indkoebsliste-adfaerd}.

Så længe indkøbslisten ligger på bordet eller et andet sted, kan mange personer komme forbi og tilføje eller fjerne varer på den. Denne tilstand kaldes for \textit{redigeres}. I denne tilstand er det muligt for alle, der kan komme til denne indkøbsliste, at tilføje eller fjerne de varer, der skal handles ind. Man kan \fx fjerne en vare ved at slå en streg over den og give et klart signal om, at denne ikke skal købes. Disse to hændelser hedder \textit{vare fjernet} og \textit{vare tilføjet}. Personen, der tilføjer tekst til indkøbslistene, er herre over, om der skal stå en ingrediens, der består af en mængde, en enhed og en råvare (\fx 1 ltr skummetmælk) eller der blot skal stå råvaren (\fx skummetmælk).

Indkøbslisten kan også indeholde bemærkninger, såsom ``hvis det er på tilbud''. Denne bemærkning kan selvfølgelig også fjernes igen på samme måde som en vare kan fjernes. Disse bemærkninger tilhører en eller flere varer på indkøbslisten, og derfor er hændelsen den samme som ved tilføjelse eller fjernelse af en vare.

Når en person beslutter sig for at tage indkøbslisten med på indkøb, anses indkøbslisten for at være færdig. Det er nu ikke længere muligt at redigere i indkøbslisten, og den afsluttende hændelse indtræffer. 

Ude i supermarkedet kan man købe mange forskellige råvarer. Vi ønsker ikke at overvåge, hvor meget folk har af en ingrediens, og benytter derfor istedet objektet råvare, der ikke indeholder nogen mængde eller enhed. Denne beslutning er taget på baggrund af møde 2 med vores informanter.\todo{Argumenter lidt bedre for valget.}

\begin{figure}[H]
	\centering
	\scalebox{0.8}{
		\subsection{Indkøbsliste}
En indkøbsliste kommer til verden ved, at man i husstanden beslutter sig for at benytte en eller anden form for huskeliste for ting, der skal handles ind. Dette kan være i form af et papir, der ligger et fast sted på bordet eller hænger på opslagstavlen. Denne initierende hændelse kaldes \textit{indkøbsliste oprettet}. Se \figref{fig:indkoebsliste-adfaerd}.

Så længe indkøbslisten ligger på bordet eller et andet sted, kan mange personer komme forbi og tilføje eller fjerne varer på den. Denne tilstand kaldes for \textit{redigeres}. I denne tilstand er det muligt for alle, der kan komme til denne indkøbsliste, at tilføje eller fjerne de varer, der skal handles ind. Man kan \fx fjerne en vare ved at slå en streg over den og give et klart signal om, at denne ikke skal købes. Disse to hændelser hedder \textit{vare fjernet} og \textit{vare tilføjet}. Personen, der tilføjer tekst til indkøbslistene, er herre over, om der skal stå en ingrediens, der består af en mængde, en enhed og en råvare (\fx 1 ltr skummetmælk) eller der blot skal stå råvaren (\fx skummetmælk).

Indkøbslisten kan også indeholde bemærkninger, såsom ``hvis det er på tilbud''. Denne bemærkning kan selvfølgelig også fjernes igen på samme måde som en vare kan fjernes. Disse bemærkninger tilhører en eller flere varer på indkøbslisten, og derfor er hændelsen den samme som ved tilføjelse eller fjernelse af en vare.

Når en person beslutter sig for at tage indkøbslisten med på indkøb, anses indkøbslisten for at være færdig. Det er nu ikke længere muligt at redigere i indkøbslisten, og den afsluttende hændelse indtræffer. 

Ude i supermarkedet kan man købe mange forskellige råvarer. Vi ønsker ikke at overvåge, hvor meget folk har af en ingrediens, og benytter derfor istedet objektet råvare, der ikke indeholder nogen mængde eller enhed. Denne beslutning er taget på baggrund af møde 2 med vores informanter.\todo{Argumenter lidt bedre for valget.}

\begin{figure}[H]
	\centering
	\scalebox{0.8}{
		\input{billeder/tilstandsdiagrammer/indkoebsliste.pdf_tex}}
		\capt{Tilstandsdiagram for klassen indkøbsliste. De afrundede rektangulære bokse med tekst, skal anses som tilstande, som klassen kan have. De pile, der fører til en tilstand, skal anses som hændelser, som kan være skyld i et tilstandsskift. I dette tilfælde har klassen én tilstand (redigeres) og en afsluttende hændelse, der fører klassen ud i en sluttilstand (den sorte prik i den sorte cirkel).}
		\label{fig:indkoebsliste-adfaerd}
\end{figure}}
		\capt{Tilstandsdiagram for klassen indkøbsliste. De afrundede rektangulære bokse med tekst, skal anses som tilstande, som klassen kan have. De pile, der fører til en tilstand, skal anses som hændelser, som kan være skyld i et tilstandsskift. I dette tilfælde har klassen én tilstand (redigeres) og en afsluttende hændelse, der fører klassen ud i en sluttilstand (den sorte prik i den sorte cirkel).}
		\label{fig:indkoebsliste-adfaerd}
\end{figure}





        

