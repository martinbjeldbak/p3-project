\chapter{Analyse af problemområde}
\label{chap:analyseafpo}

I følgendende kapitel vil vi analysere problemområdet. Analysemetoden tager udgangspunkt i bogen \cite[s. ~43]{ooad}. Problemområdet er den del af omgivelserne, som skal administreres, styres og overvåges af vores system, Foodl. I vores tilfælde er problemområdet: madresterne og madlavningen i de danske hustande. Formålet med analysen, er at skabe et overblik over, hvilke klasser og hændelser Foodl skal have, for at kunne udføre de funktioner vi ønsker, og for at systemet bliver så brugbart som muligt for dets fremtidige brugere. Udover at identificere klasser og hændelser, vil vi i starten af kapitlet undersøge og analysere nogle allerede eksisterende systemer, som tilbyder lignende funktioner. Dette gør vi for at finde ud af hvilke kvaliteter og mangler disse har, så vi derved opnår en bedre forståelse for, hvad Foodl skal kunne tilbyde brugerne, for at blive så attraktiv som mulig.



% Problemområde
\input{indhold/3-analyseafproblemomraade/problemomraade/eksisterendeSystemer/eksisterendeSystemer}
\input{indhold/3-analyseafproblemomraade/problemomraade/eksisterendeSystemer/forresten}
\input{indhold/3-analyseafproblemomraade/problemomraade/eksisterendeSystemer/dk-kogebogen}
\input{indhold/3-analyseafproblemomraade/problemomraade/eksisterendeSystemer/opskrifterdk}
\input{indhold/3-analyseafproblemomraade/problemomraade/eksisterendeSystemer/sammendrag}
\subsection{For Resten}
For Resten er en gratis mobilapp, der er udgivet af Forbrugerrådet som en del af en kampagne mod madspild. App’en findes til Android og iOS og kan installeres fra henholdsvis Google Play og App Store. App’en fungerer ved, at man på to ``hjul'' vælger kategori (``Kornprodukter'', ``Mejeriprodukter'', ``Kød og æg'' og fem andre kategorier) og rest (\fx ``Mørbrad'', ``Kylling'', ``Kødsovs'' osv.), hvorefter brugeren præsenteres for en række opskrifter, som inkluderer den valgte rest.

\begin{figure}[H]
\centering
\includegraphics[scale=0.7]{billeder/forbilleder/forresten.png}
\capt{Brugergrænseflade i For Resten. Valg af rest (t.v.) og visning af opskrift (t.h.)}
\label{fig:forresten}
\end{figure}

For hver rest er der ca. 4 - 5 opskrifter, hvilket med 124 rester giver et samlet antal opskrifter på omkring 500 - 600 (det faktiske antal opskrifter er ikke oplyst nogen steder). For opskrifterne vises der kun en fremgangsmåde, og altså ikke en liste over ingredienser. Det er derfor heller ikke muligt at op- eller nedskalere opskrifsportionerne. Derudover er der ingen mulighed for at favorisere opskrifter eller på anden måde gemme resultater af en søgning. I forhold til søgningen så kan man, som sagt, kun søge på én enkelt rest, og ikke sammensætte rester, som ved andre løsninger. Samtidig har man kun mulighed for at vælge sin rest på hjulene, og har altså ikke muligheden for at skrive i et felt. Under afprøvningen var det derfor i nogle tilfælde svært at finde den ønskede ingrediens. Det er de samme problemstillinger, der kommer til udtryk i brugernes anmeldelser af app’en på Google Play. \Fx skriver brugeren \textit{TJA} \cite{tja}:

\begin{quote}
  ``Fin ide, men den burde være blevet kælet lidt mere for, før den røg i play. Hvem laver restemad af én rest og så en masse ting der skal købes? Jeg har brug for at kunne søge på tre-fire rester for at se, hvordan de kan kombineres til noget spændende. Og hvis jeg finder en opskrift, jeg vil prøve, så kan jeg ikke gemme den i appen, men skal søge den frem igen når jeg står i netto og skal købe de ting, der skal til - og søge den frem igen, når jeg skal lave maden. Alt for besværligt.''
\end{quote}

Sammenlagt har app’en på Google Play bedømmelsen 2,4 stjerner ud af 5, baseret på 69 bedømmelser, hvoraf næsten halvdelen kun har givet app’en 1 stjerne. Et andet kritikpunkt, der kommer til udtryk i flere anmeldelser, er, at app’en bruger for mange systemressourcer og er langsom til at starte op.

\subsection{DK-Kogebogen}
\label{subsec:dk-kogebogen}

DK-Kogebogen er en webapplikation, som tilbyder en lang række funktioner, hvoraf ``Tøm køleskabet'' er en af dem. Udover ``Tøm køleskabet'', tilbyder DK-kogebogen også en ugentlig madplan, en ekstern hjemmeside med fokus på viden omkring mad (energiindhold, vitaminindhold osv.), en kalorieberegner og meget meget mere. Sagt med andre ord, så prioriterer DK-Kogebogen ikke udelukkende sine ressourcer på ``Tøm køleskabet'', og dette kan påvirke kvaliteten af denne. Desuden kan det også være svært at skabe sig et overblik på hjemmesiden, da der er så mange funktioner pakket sammen på samme side. Siden har et omfang på over 36.000 opskrifter, hvilket er det klart største antal blandt de undersøgte systemer. De mange opskrifter er indsendt og lavet af brugere af hjemmesiden. Dette kan både være positivt og negativt. Det kan være negativt, da der kan forekomme en del fejl i opskrifterne. DK-Kogebogen skriver på deres hjemmeside, om de indsendte opskrifter:

%kilde: http://www.dk-kogebogen.dk/opskrifts-service/indsend-opskrift.php
\begin{quote}
``Opskriften kan ses med det samme, men der vil senere blive rettet lidt til i teksterne.'' \cite{dk-kog-indtastopskrift}
\end{quote}

Det er altså muligt for alle at indsende opskrifter, som er fulde af fejl, og disse vil alligevel være synlige på siden. Til gengæld skaber det mulighed for, at den enkelte bruger føler sig mere knyttet til siden, da han/hun har et personligt engagement i den, fordi de indsender deres egne opskrifter; og dette vil resultere i en overordnet højere brugeraktivitet. Det skal noteres, at de 36.000 opskrifter ikke er unikke. Det vil sige, at der godt kan eksistere flere forskellige varianter af den samme opskrift; \fx er der 17 forskellige varianter af Boller i Karry, hvilket kan virke uoverskueligt.

Når brugeren vil anvende ``Tøm køleskabs''-funktionen, mødes vedkommende af følgende brugergrænseflade \ref{fig:dk-kogebogen1}, som er taget fra toppen af DK-Kogebogens forside. Her kan brugeren indtaste så mange ingredienser der er brug for, inden for en grænse på 55 karakter. For at systemet kan identificere ingredienser, skal der mellem hver ingrediens, som brugeren indtaster, være et mellemrumstegn. Begrænsningen på 55 tegn betyder, at der maksimalt kan søges på ca. 8-10 ingredienser af gangen (Forudsat at ingrediensnavne er 5-7 bogstaver lange).

\begin{figure}[H]
\centering
\includegraphics[scale=0.7]{billeder/forbilleder/dk-kogebogen.png}
\capt{DK-Kogebogens ``Tøm køleskabet''-funktion. Funktionen er tilgængelig i toppen af forsiden, og enhver anden underside.}
\label{fig:dk-kogebogen1}
\end{figure}

Efter brugeren har søgt på opskrifter med en eller flere specifikke ingredienser, viser DK-Kogebogen en liste med alle de opskrifter, som indeholder ingredienserne. På \figref{fig:dk-kogebogen2} ses en del af den liste, som er resultatet ved søgning på ``Tomat, Paprika og Kartoffel''. Ud fra opskrifterne kan der findes et kamera-symbol og/eller et dokument-symbol, som respektivt fortæller brugeren, om der er et billede af den pågældende opskrift, og/eller at opskriftens næringsindhold er beregnet og vist.


\begin{figure}[H]
\centering
\includegraphics[scale=0.7]{billeder/forbilleder/dk-kogebogen2.png}
\capt{Liste af opskrifter, som indeholder ingredienserne Tomat, Paprika og Kartoffel.}
\label{fig:dk-kogebogen2}
\end{figure}

Brugeren vælger derefter en af opskrifterne han/hun finder mest interessant. Der er ikke overensstemmelse med, hvad \fx portionerne skal angives i. Inde på hjemmesiden er det i nogle tilfælde muligt at op- eller nedskalere portionsstørrelsen, mens der i andre tilfælde skaleres på antal personer. Derudover er der også nogle opskrifter, hvor det slet ikke er muligt at skalere. Her er brugeren i stedet for tvunget til selv at finde ud af, hvor stor en portion opskriften ca. passer til. Denne mangel på konsistens er endnu en af problematikkerne ved, at det er brugerne selv, som indsender opskrifterne. En endelig vurdering af DK-Kogebogen, findes i \secref{subsec:eksisterende.sammendrag}.

\subsection{Opskrifter.dk}
Opskriftshjemmesiden Opskrifter.dk tilbyder en ``Tøm køleskabet''-funktion, som kan bruges til at finde opskrifter i deres samling af opskrifter. Ligesom i For Resten kan ingredienser kun vælges fra kategorier, dog er der i dette system hele 624 ingredienser, fordelt over 27 kategorier, at vælge mellem. Modsat For Resten kan man her vælge mere end én ingrediens. Dette gøres ved først at vælge en kategori og derefter finde og vælge en ingrediens og klikke på knappen ``Tilføj >''. Man kan ligeledes fjerne allerede valgte ingredienser ved at markere dem og klikke på knappen ``< Fjern'', eller fjerne alle valgte ingredienser ved at klikke på knappen ``< Fjern alle''. Når man har valgt de ingredienser, man ønsker at inkludere, kan man foretage sin søgning ved at klike på knappen ``Søg''. 

\begin{figure}[H]
\centering
\includegraphics[scale=0.7]{billeder/forbilleder/opskrifterdk.png}
\capt{Brugergrænsefladen for Opskrifter.dk’s ``Tøm køleskabet''-funktion.}
\label{fig:opskrifterdk1}
\end{figure}

Søgningen foretages blandt de ca. 2700 opskrifter, som er tilgængelige på Opskrifter.dk. Resultaterne vælges ud fra, om de inkluderer minimum én af de valgte ingredienser. Under hver opskriftnavn vises antallet af valgte ingredienser, som opskriften inkluderer, men det er umiddelbart ikke muligt at sortere resultaterne efter dette tal. Klikker man på et resultat, åbnes den valgte opskrift til højre for resultaterne, og altså ikke på en ny side eller i et nyt vindue/faneblad. Opskrifterne viser informationer såsom tilberedningstid samt alle ingredienserne og deres mængder. Det er på alle opskrifter muligt at skalere opskriften til et bestemt antal personer. Kvaliteten af opskrifterne er forholdsvis høj og ca. 40 \% af alle opskrifter er med billede. Dette skyldes sandsynligvis, at det ikke er muligt for almindelige brugere direkte at indsende opskrifter. Almindelige brugere kan derimod indsende opskriftforslag, som først skal gennemlæses og tilføjes af en administrator.

\begin{figure}[H]
\centering
\includegraphics[scale=0.7]{billeder/forbilleder/opskrifterdk2.png}
\capt{Resultatsiden vist ved søgning med Opskrifter.dk’s ``Tøm køleskabet''-funktion.}
\label{fig:opskrifterdk2}
\end{figure}

Opskrifter.dk har også et brugersystem, der tillader brugere at registrere sig og logge ind. Dette giver mulighed for, at man bl.a. kan gemme de opskrifter, man har fundet (ved at klikke på ``Læg i min kogebog''). Derudover husker ``Tøm køleskabet''-funktionen de ingredienser, man har indtastet, til næste gang man besøger siden.
Under ``Tøm køleskabet''-funktionen er det også muligt for brugeren at skrive kommentar til funktionen. Disse kommentar givet et indblik i, hvad Opskrifter.dk’s brugere synes om funktionaliteten. Nogle kommentarer går på manglende ingredienser, mens en stor del kommentarer går på, at funktionen finder opskrifter, som man ikke kan lave uden at skulle købe en masse ind. \Fx skriver brugeren Jytte Hasselriis:

%kilde: http://opskrifter.dk/Toem-koeleskabet.149.0.html
\begin{quote}
``Hvordan skulle jeg kunne lave fasan i flødesovs, når jeg ikke har en fasan i køleskabet????''
\end{quote}

Dette synes at udtrykke en vis utilfredshed med den måde, hvorpå Opskrifter.dk’s ``Tøm køleskabet''-funktion vælger resultater på.
\subsection{Sammendrag}
\label{sec:forbilleder:sammendrag}
De tre forbilleder: For Resten, DK-Kogebogen og Opskrifter.dk er i foregående afsnit blevet undersøgt og analyseret. Vi har i vores undersøgelse og analyse lagt vægt på fire hovedpunkter: antallet af opskrifter i systemet, Kvaliteten af opskrifterne, systemets fleksibilitet og opskriftssøgningsfunktionen (også kaldet ``Tøm køleskabet''-funktionen). De fire hovedpunkter er specificeret i detaljer i (REFERER TIL AFSNIT FORBILLEDER). For at skabe et samlet overblik, har vi valgt at samle de vigtigste og mest karakteristiske dele fra hver af forbillederne i dette afsnit.

\begin{table}
\centering
\begin{tabular}{| c | c | c | c |}
\hline
\textbf{Forbillede} & For Resten & DK-kogebogen & Opskrifter.dk \\ \hline
\textbf{Antal opskrifter} & 550 & 36.500 & 2.700 \\ \hline
\end{tabular}
\capt{Antallet af opskrifter i de tre forbilleder: For Resten, DK-Kogebogen og Opskrifter.dk. Antallet er angivet i cirkatal, da det præcise antal ikke er væsentligt.}
\label{table:forbilledeantal}
\end{table}

Som det ses i tabel (REFERER TIL TABEL HER), har DK-Kogebogen det langt største antal opskrifter, mens Opskrifter.dk har ca. fem gange flere opskrifter end For Resten. Dermed har DK-Kogenbogens ``Tøm køleskabet''-funktion også langt bedre chance for at give brugeren et resultat, når han/hun søger på opskrifter med specifikke ingredienser. Til gengæld er kvaliteten af opskrifterne på DK-Kogebogen meget varierende, og derfor kan brugeren risikere at støde på opskrifter, som er dårlige eller ubrugelige. Kvaliteten af opskrifterne på For Resten er sammenlagt dårlig. Opskrifterne er udelukkende lavet eller tilføjet af folkene bag app’en, og derfor er opskrifternes opbygning og design konsistent, hvilket naturligvis havde været en god egenskab, hvis det ikke var for det faktum, at opbygningen er uoverskuelig. Der er nemlig ingen ingrediensliste på opskriften og beskrivelsen af fremgangsmåden er også kortfattet. Derimod er kvaliteten af opskrifterne på Opskrifter.dk høj. Dette skyldes, at opskrifterne bliver gennemgået af en administrator, inden de bliver tilgængelige på Opskrifter.dk’s side, hvilket er modsat af DK-Kogebogen, hvor opskrifterne bliver tilgængelige med det samme. Desuden er Opskrifter.dk’s opskriftopbygning konsekvent i alle opskrifter, hvilket også er i modsætning til DK-Kogebogens opskrifter.

Der er stor forskel i fleksibiliteten fra forbillede til forbillede. I For Restens app, er det slet ikke muligt at op- og nedskalere portionsstørrelse, på DK-Kogebogens side er det kun muligt med nogle opskrifter, mens det på Opskrifter.dk er muligt at op- og nedskalere portionsstørrelse hver eneste opskrift. Det er en funktion som er meget brugbar, da man som bruger ikke ønsker at bruge en masse tid på selv at beregne en passende portionsstørrelse. Dette vil vi også bestræbe os på at implementere i de opskrifter der er tilgængelige på Foodl. Af sorteringsmuligheder af opskriftresultaterne, er det kun Opskrifter.dk som tilbyder denne mulighed. Her kan der sorteres efter alfabetisk orden, opskrifter med billeder, opskrifter med kød samt flere. Som en bruger på Opskrifter.dk dog pointere, mangler den sorteringsmulighed, som sortere efter de opskrifter som indeholder flest af de ingredienser som brugeren har indtastet. Denne sorteringsmulighed anses for os, som værende den mest relevante, da man som bruger er interesseret i at få anvendt så mange af ens madrester som muligt. 

De tre løsninger er vidt forskellige i deres måde at håndtere søgning på. Ud fra vores afprøvninger og observationer af løsningernes fordele og ulemper, kan vi uddrage hvilke egenskaber vi ønsker at benytte i vores eget projekt. \Fx viser den generelle utilfredshed med Forbrugerstyrelsens mobilapp For Resten, at det er vigtigt med mange opskrifter og muligheden for at vælge mere end en rest. Observationerne af For Resten og Opskrifter.dk viser også at brugergrænseflade er et vigtigt element. I disse to løsninger skal man vælge ingredienser ved at lede rundt i kategorier og i For Resten endda bevæge fingeren rundt i en cirkel for at rotere hjulene for kategorier og rester. Dette føles meget ineffektivt i forhold til at skrive navnet på ingrediensen på et tastatur. Derudover er der mellem DK-kogebogen og Opskrifter.dk en markant forskel på hvordan resultater findes. I DK-kogebogen findes kun opskrifter som inkluderer alle de indtastede ingredienser, mens Opskrifter.dk finder alle opskrifter som indeholder bare én af de valgte ingredienser. Dvs. at man med DK-kogebogen får færre resultater jo flere ingredienser man skriver, mens det med Opskrifter.dk er direkte modsat, idet antallet af resultater stiger voldsomt med antallet af ingredienser man skriver. Opskrifter.dk’s måde at gøre det på, kombineret med deres manglende sortering, giver et stor uoverskuelig mængde af resultater, hvor en stor del af disse måske kun indeholder en af de valgte ingredienser. Af dette kan man aflede at en kombination af de to må være den optimale løsning. Har man valgt få ingredienser, er man sandsynligvis interesseret i at få vist resultater som indeholder alle de ingredienser man har valgt. Har man derimod valgt mange ingredienser, er man interesseret i at få vist resultater som indeholder flest muligt af de ingredienser man har valgt.

\subsection{For Resten}
For Resten er en gratis mobilapp, der er udgivet af Forbrugerrådet som en del af en kampagne mod madspild. App’en findes til Android og iOS og kan installeres fra henholdsvis Google Play og App Store. App’en fungerer ved, at man på to ``hjul'' vælger kategori (``Kornprodukter'', ``Mejeriprodukter'', ``Kød og æg'' og fem andre kategorier) og rest (\fx ``Mørbrad'', ``Kylling'', ``Kødsovs'' osv.), hvorefter brugeren præsenteres for en række opskrifter, som inkluderer den valgte rest.

\begin{figure}[H]
\centering
\includegraphics[scale=0.7]{billeder/forbilleder/forresten.png}
\capt{Brugergrænseflade i For Resten. Valg af rest (t.v.) og visning af opskrift (t.h.)}
\label{fig:forresten}
\end{figure}

For hver rest er der ca. 4 - 5 opskrifter, hvilket med 124 rester giver et samlet antal opskrifter på omkring 500 - 600 (det faktiske antal opskrifter er ikke oplyst nogen steder). For opskrifterne vises der kun en fremgangsmåde, og altså ikke en liste over ingredienser. Det er derfor heller ikke muligt at op- eller nedskalere opskrifsportionerne. Derudover er der ingen mulighed for at favorisere opskrifter eller på anden måde gemme resultater af en søgning. I forhold til søgningen så kan man, som sagt, kun søge på én enkelt rest, og ikke sammensætte rester, som ved andre løsninger. Samtidig har man kun mulighed for at vælge sin rest på hjulene, og har altså ikke muligheden for at skrive i et felt. Under afprøvningen var det derfor i nogle tilfælde svært at finde den ønskede ingrediens. Det er de samme problemstillinger, der kommer til udtryk i brugernes anmeldelser af app’en på Google Play. \Fx skriver brugeren \textit{TJA} \cite{tja}:

\begin{quote}
  ``Fin ide, men den burde være blevet kælet lidt mere for, før den røg i play. Hvem laver restemad af én rest og så en masse ting der skal købes? Jeg har brug for at kunne søge på tre-fire rester for at se, hvordan de kan kombineres til noget spændende. Og hvis jeg finder en opskrift, jeg vil prøve, så kan jeg ikke gemme den i appen, men skal søge den frem igen når jeg står i netto og skal købe de ting, der skal til - og søge den frem igen, når jeg skal lave maden. Alt for besværligt.''
\end{quote}

Sammenlagt har app’en på Google Play bedømmelsen 2,4 stjerner ud af 5, baseret på 69 bedømmelser, hvoraf næsten halvdelen kun har givet app’en 1 stjerne. Et andet kritikpunkt, der kommer til udtryk i flere anmeldelser, er, at app’en bruger for mange systemressourcer og er langsom til at starte op.

\subsection{DK-Kogebogen}
\label{subsec:dk-kogebogen}

DK-Kogebogen er en webapplikation, som tilbyder en lang række funktioner, hvoraf ``Tøm køleskabet'' er en af dem. Udover ``Tøm køleskabet'', tilbyder DK-kogebogen også en ugentlig madplan, en ekstern hjemmeside med fokus på viden omkring mad (energiindhold, vitaminindhold osv.), en kalorieberegner og meget meget mere. Sagt med andre ord, så prioriterer DK-Kogebogen ikke udelukkende sine ressourcer på ``Tøm køleskabet'', og dette kan påvirke kvaliteten af denne. Desuden kan det også være svært at skabe sig et overblik på hjemmesiden, da der er så mange funktioner pakket sammen på samme side. Siden har et omfang på over 36.000 opskrifter, hvilket er det klart største antal blandt de undersøgte systemer. De mange opskrifter er indsendt og lavet af brugere af hjemmesiden. Dette kan både være positivt og negativt. Det kan være negativt, da der kan forekomme en del fejl i opskrifterne. DK-Kogebogen skriver på deres hjemmeside, om de indsendte opskrifter:

%kilde: http://www.dk-kogebogen.dk/opskrifts-service/indsend-opskrift.php
\begin{quote}
``Opskriften kan ses med det samme, men der vil senere blive rettet lidt til i teksterne.'' \cite{dk-kog-indtastopskrift}
\end{quote}

Det er altså muligt for alle at indsende opskrifter, som er fulde af fejl, og disse vil alligevel være synlige på siden. Til gengæld skaber det mulighed for, at den enkelte bruger føler sig mere knyttet til siden, da han/hun har et personligt engagement i den, fordi de indsender deres egne opskrifter; og dette vil resultere i en overordnet højere brugeraktivitet. Det skal noteres, at de 36.000 opskrifter ikke er unikke. Det vil sige, at der godt kan eksistere flere forskellige varianter af den samme opskrift; \fx er der 17 forskellige varianter af Boller i Karry, hvilket kan virke uoverskueligt.

Når brugeren vil anvende ``Tøm køleskabs''-funktionen, mødes vedkommende af følgende brugergrænseflade \ref{fig:dk-kogebogen1}, som er taget fra toppen af DK-Kogebogens forside. Her kan brugeren indtaste så mange ingredienser der er brug for, inden for en grænse på 55 karakter. For at systemet kan identificere ingredienser, skal der mellem hver ingrediens, som brugeren indtaster, være et mellemrumstegn. Begrænsningen på 55 tegn betyder, at der maksimalt kan søges på ca. 8-10 ingredienser af gangen (Forudsat at ingrediensnavne er 5-7 bogstaver lange).

\begin{figure}[H]
\centering
\includegraphics[scale=0.7]{billeder/forbilleder/dk-kogebogen.png}
\capt{DK-Kogebogens ``Tøm køleskabet''-funktion. Funktionen er tilgængelig i toppen af forsiden, og enhver anden underside.}
\label{fig:dk-kogebogen1}
\end{figure}

Efter brugeren har søgt på opskrifter med en eller flere specifikke ingredienser, viser DK-Kogebogen en liste med alle de opskrifter, som indeholder ingredienserne. På \figref{fig:dk-kogebogen2} ses en del af den liste, som er resultatet ved søgning på ``Tomat, Paprika og Kartoffel''. Ud fra opskrifterne kan der findes et kamera-symbol og/eller et dokument-symbol, som respektivt fortæller brugeren, om der er et billede af den pågældende opskrift, og/eller at opskriftens næringsindhold er beregnet og vist.


\begin{figure}[H]
\centering
\includegraphics[scale=0.7]{billeder/forbilleder/dk-kogebogen2.png}
\capt{Liste af opskrifter, som indeholder ingredienserne Tomat, Paprika og Kartoffel.}
\label{fig:dk-kogebogen2}
\end{figure}

Brugeren vælger derefter en af opskrifterne han/hun finder mest interessant. Der er ikke overensstemmelse med, hvad \fx portionerne skal angives i. Inde på hjemmesiden er det i nogle tilfælde muligt at op- eller nedskalere portionsstørrelsen, mens der i andre tilfælde skaleres på antal personer. Derudover er der også nogle opskrifter, hvor det slet ikke er muligt at skalere. Her er brugeren i stedet for tvunget til selv at finde ud af, hvor stor en portion opskriften ca. passer til. Denne mangel på konsistens er endnu en af problematikkerne ved, at det er brugerne selv, som indsender opskrifterne. En endelig vurdering af DK-Kogebogen, findes i \secref{subsec:eksisterende.sammendrag}.

\subsection{Opskrifter.dk}
Opskriftshjemmesiden Opskrifter.dk tilbyder en ``Tøm køleskabet''-funktion, som kan bruges til at finde opskrifter i deres samling af opskrifter. Ligesom i For Resten kan ingredienser kun vælges fra kategorier, dog er der i dette system hele 624 ingredienser, fordelt over 27 kategorier, at vælge mellem. Modsat For Resten kan man her vælge mere end én ingrediens. Dette gøres ved først at vælge en kategori og derefter finde og vælge en ingrediens og klikke på knappen ``Tilføj >''. Man kan ligeledes fjerne allerede valgte ingredienser ved at markere dem og klikke på knappen ``< Fjern'', eller fjerne alle valgte ingredienser ved at klikke på knappen ``< Fjern alle''. Når man har valgt de ingredienser, man ønsker at inkludere, kan man foretage sin søgning ved at klike på knappen ``Søg''. 

\begin{figure}[H]
\centering
\includegraphics[scale=0.7]{billeder/forbilleder/opskrifterdk.png}
\capt{Brugergrænsefladen for Opskrifter.dk’s ``Tøm køleskabet''-funktion.}
\label{fig:opskrifterdk1}
\end{figure}

Søgningen foretages blandt de ca. 2700 opskrifter, som er tilgængelige på Opskrifter.dk. Resultaterne vælges ud fra, om de inkluderer minimum én af de valgte ingredienser. Under hver opskriftnavn vises antallet af valgte ingredienser, som opskriften inkluderer, men det er umiddelbart ikke muligt at sortere resultaterne efter dette tal. Klikker man på et resultat, åbnes den valgte opskrift til højre for resultaterne, og altså ikke på en ny side eller i et nyt vindue/faneblad. Opskrifterne viser informationer såsom tilberedningstid samt alle ingredienserne og deres mængder. Det er på alle opskrifter muligt at skalere opskriften til et bestemt antal personer. Kvaliteten af opskrifterne er forholdsvis høj og ca. 40 \% af alle opskrifter er med billede. Dette skyldes sandsynligvis, at det ikke er muligt for almindelige brugere direkte at indsende opskrifter. Almindelige brugere kan derimod indsende opskriftforslag, som først skal gennemlæses og tilføjes af en administrator.

\begin{figure}[H]
\centering
\includegraphics[scale=0.7]{billeder/forbilleder/opskrifterdk2.png}
\capt{Resultatsiden vist ved søgning med Opskrifter.dk’s ``Tøm køleskabet''-funktion.}
\label{fig:opskrifterdk2}
\end{figure}

Opskrifter.dk har også et brugersystem, der tillader brugere at registrere sig og logge ind. Dette giver mulighed for, at man bl.a. kan gemme de opskrifter, man har fundet (ved at klikke på ``Læg i min kogebog''). Derudover husker ``Tøm køleskabet''-funktionen de ingredienser, man har indtastet, til næste gang man besøger siden.
Under ``Tøm køleskabet''-funktionen er det også muligt for brugeren at skrive kommentar til funktionen. Disse kommentar givet et indblik i, hvad Opskrifter.dk’s brugere synes om funktionaliteten. Nogle kommentarer går på manglende ingredienser, mens en stor del kommentarer går på, at funktionen finder opskrifter, som man ikke kan lave uden at skulle købe en masse ind. \Fx skriver brugeren Jytte Hasselriis:

%kilde: http://opskrifter.dk/Toem-koeleskabet.149.0.html
\begin{quote}
``Hvordan skulle jeg kunne lave fasan i flødesovs, når jeg ikke har en fasan i køleskabet????''
\end{quote}

Dette synes at udtrykke en vis utilfredshed med den måde, hvorpå Opskrifter.dk’s ``Tøm køleskabet''-funktion vælger resultater på.
\subsection{Sammendrag}
\label{sec:forbilleder:sammendrag}
De tre forbilleder: For Resten, DK-Kogebogen og Opskrifter.dk er i foregående afsnit blevet undersøgt og analyseret. Vi har i vores undersøgelse og analyse lagt vægt på fire hovedpunkter: antallet af opskrifter i systemet, Kvaliteten af opskrifterne, systemets fleksibilitet og opskriftssøgningsfunktionen (også kaldet ``Tøm køleskabet''-funktionen). De fire hovedpunkter er specificeret i detaljer i (REFERER TIL AFSNIT FORBILLEDER). For at skabe et samlet overblik, har vi valgt at samle de vigtigste og mest karakteristiske dele fra hver af forbillederne i dette afsnit.

\begin{table}
\centering
\begin{tabular}{| c | c | c | c |}
\hline
\textbf{Forbillede} & For Resten & DK-kogebogen & Opskrifter.dk \\ \hline
\textbf{Antal opskrifter} & 550 & 36.500 & 2.700 \\ \hline
\end{tabular}
\capt{Antallet af opskrifter i de tre forbilleder: For Resten, DK-Kogebogen og Opskrifter.dk. Antallet er angivet i cirkatal, da det præcise antal ikke er væsentligt.}
\label{table:forbilledeantal}
\end{table}

Som det ses i tabel (REFERER TIL TABEL HER), har DK-Kogebogen det langt største antal opskrifter, mens Opskrifter.dk har ca. fem gange flere opskrifter end For Resten. Dermed har DK-Kogenbogens ``Tøm køleskabet''-funktion også langt bedre chance for at give brugeren et resultat, når han/hun søger på opskrifter med specifikke ingredienser. Til gengæld er kvaliteten af opskrifterne på DK-Kogebogen meget varierende, og derfor kan brugeren risikere at støde på opskrifter, som er dårlige eller ubrugelige. Kvaliteten af opskrifterne på For Resten er sammenlagt dårlig. Opskrifterne er udelukkende lavet eller tilføjet af folkene bag app’en, og derfor er opskrifternes opbygning og design konsistent, hvilket naturligvis havde været en god egenskab, hvis det ikke var for det faktum, at opbygningen er uoverskuelig. Der er nemlig ingen ingrediensliste på opskriften og beskrivelsen af fremgangsmåden er også kortfattet. Derimod er kvaliteten af opskrifterne på Opskrifter.dk høj. Dette skyldes, at opskrifterne bliver gennemgået af en administrator, inden de bliver tilgængelige på Opskrifter.dk’s side, hvilket er modsat af DK-Kogebogen, hvor opskrifterne bliver tilgængelige med det samme. Desuden er Opskrifter.dk’s opskriftopbygning konsekvent i alle opskrifter, hvilket også er i modsætning til DK-Kogebogens opskrifter.

Der er stor forskel i fleksibiliteten fra forbillede til forbillede. I For Restens app, er det slet ikke muligt at op- og nedskalere portionsstørrelse, på DK-Kogebogens side er det kun muligt med nogle opskrifter, mens det på Opskrifter.dk er muligt at op- og nedskalere portionsstørrelse hver eneste opskrift. Det er en funktion som er meget brugbar, da man som bruger ikke ønsker at bruge en masse tid på selv at beregne en passende portionsstørrelse. Dette vil vi også bestræbe os på at implementere i de opskrifter der er tilgængelige på Foodl. Af sorteringsmuligheder af opskriftresultaterne, er det kun Opskrifter.dk som tilbyder denne mulighed. Her kan der sorteres efter alfabetisk orden, opskrifter med billeder, opskrifter med kød samt flere. Som en bruger på Opskrifter.dk dog pointere, mangler den sorteringsmulighed, som sortere efter de opskrifter som indeholder flest af de ingredienser som brugeren har indtastet. Denne sorteringsmulighed anses for os, som værende den mest relevante, da man som bruger er interesseret i at få anvendt så mange af ens madrester som muligt. 

De tre løsninger er vidt forskellige i deres måde at håndtere søgning på. Ud fra vores afprøvninger og observationer af løsningernes fordele og ulemper, kan vi uddrage hvilke egenskaber vi ønsker at benytte i vores eget projekt. \Fx viser den generelle utilfredshed med Forbrugerstyrelsens mobilapp For Resten, at det er vigtigt med mange opskrifter og muligheden for at vælge mere end en rest. Observationerne af For Resten og Opskrifter.dk viser også at brugergrænseflade er et vigtigt element. I disse to løsninger skal man vælge ingredienser ved at lede rundt i kategorier og i For Resten endda bevæge fingeren rundt i en cirkel for at rotere hjulene for kategorier og rester. Dette føles meget ineffektivt i forhold til at skrive navnet på ingrediensen på et tastatur. Derudover er der mellem DK-kogebogen og Opskrifter.dk en markant forskel på hvordan resultater findes. I DK-kogebogen findes kun opskrifter som inkluderer alle de indtastede ingredienser, mens Opskrifter.dk finder alle opskrifter som indeholder bare én af de valgte ingredienser. Dvs. at man med DK-kogebogen får færre resultater jo flere ingredienser man skriver, mens det med Opskrifter.dk er direkte modsat, idet antallet af resultater stiger voldsomt med antallet af ingredienser man skriver. Opskrifter.dk’s måde at gøre det på, kombineret med deres manglende sortering, giver et stor uoverskuelig mængde af resultater, hvor en stor del af disse måske kun indeholder en af de valgte ingredienser. Af dette kan man aflede at en kombination af de to må være den optimale løsning. Har man valgt få ingredienser, er man sandsynligvis interesseret i at få vist resultater som indeholder alle de ingredienser man har valgt. Har man derimod valgt mange ingredienser, er man interesseret i at få vist resultater som indeholder flest muligt af de ingredienser man har valgt.
 
\section{Klasser}
\label{sec:klasser}
Vi ønsker nu at vælge de bestanddele, som vi vil modellere i problemområdet. Disse vælges
ved at kigge på systemdefinitionen og ved at fremstille rige billeder (se \todo{bilag}).

Grundet den iterative arbejdsproces, er klasser undervejs blevet tilføjet og fjerent, og vi ønsker nu at kaste lys over baggrunden bag de valgte klasser. I gennem processen har vi også fravalgt klasser. Disse, med beskrivelse, kan findes i \apref{ap:fravalgteklasser}.

\subsection{Valgte klasser}
Herunder ses de valgte klasser og hvorfor vi vælger at have dem med i vores model af problemområdet.
Vi mener at disse klasser samler de objekter og hændelser, som er relevant for denne model af problemområdet.

\begin{description}
\item[Ingrediens] \hfill \\ 
I opskrifter bruges der flere ingredienser. En ingrediens består af en råvare og en mængde af denne. Det er et problem at finde opskrifter, der indeholder ingredienser svarende til de råvarer man har til rådig. Det er et problem for informanterne at maden laves i for store portioner, altså er det et problem hvis en opskrifts ingredienser indeholder store mængder.

\item[Bogmærke] \hfill \\
Det er en del af problemområdet, for brugere at huske de gode opskrifter. Der vil til tider blive benyttet en opskrift, der er så god, at den er værd at gemme til en anden gang. Derfor beholder vi denne klasse.

\item[Råvare] \hfill \\
En råvare findes i køleskabene og på madhylderne i husholdningerne. Det er et problem at finde opskrifter, der kun indeholder disse råvarer, derfor skelnes der mellem ingredieser og råvarer.

\item[Indkøbsliste] \hfill \\
Vi vurderer, at der i en husholdning ofte bliver skrevet en indkøbsliste med de ting man mangler. Indkøbslisten kan være skrevet på baggrund af en opskrift man gerne vil lave, eller en hel madplan man gerne vil følge over en længere periode.

\item[Opskrift] \hfill \\
En opskrift er det centrale i problemområdet. Opskrifterne indeholder forskellige ingredienser. Det er nødvendigt at have råvarer nok til at matche ingredienserne i opskriften, før denne kan laves. Man må gå ud fra at den typiske private madlaver har mange opskrifter i kogebogen, som han/hun reelt ikke har råvarerne til at kunne lave.
\end{description}

\subsection{Struktur}

De valgte klasser giver anledning til et klassediagram. Dette kan ses i \figref{fig:klassediagram}. 

\begin{figure}
  \centering
  \input{billeder/klassediagrampo.pdf_tex}
  \capt{Klassediagram for problemområdet.}
  \label{fig:klassediagram}
\end{figure}


Klassediagrammet ovenover er bygget op af aggregeringer og associationer imellem klasserne i diagrammet. For konkrete beskrivelser af de forskellige klasser, konsulteres hændelses- og klassebeskrivelserne. Her beskrives relationen mellem klasserne.

\begin{description}
  \item[Bogmærke] \hfill \\
    En bogmærke kan repræsentere opskrifter.

  \item[Opskrift] \hfill \\
    Består udelukkende af ingredienser og har fremgangsmåden som attribut. En opskrift kan bogmærkes en gang pr. bruger. Som minimum består en opskrift af en ingrediens.
Attributter: Titel, Billede, Fremgangsmåde

\item[Indkøbsliste] \hfill \\
  Indkøbslisten består udelukkende af en til flere ingredienser, der tilføjes ud fra opskrifterne. Derudover er det muligt at yderligere tilføje elementer på indkøbslisten i form af arbitrære tekststrenge.

\item[Ingrediens] \hfill \\
  En ingrediens består af en råvare.
Attributter: Mængde, Enhed

\item[Råvare] \hfill \\
  En råvare er en dekomponering af en ingrediens. En råvare er blot en ingrediens uden nogen form for information om mængde eller enhed. En ingrediens kunne for eksempel være 200g oksekød.
Attributter: Navn
\end{description}

                %klasser + klassediagram
\section{Hændelser}
\label{sec:haendelser}
Ved hjælp af klassekandidaterne, er det nu muligt at finde hændelseskandidater. Hændelserne er kommet til verden ud fra diverse forskellige forløb, der kan påvirke klasser. For at opretholde en konsistens mellem hændelserne, er de formuleret i datid. Det er igen vigtigt at nævne, at følgende er \emph{kandidater} og kan ændres under den iterative arbejdsproces. 

\subsection{Valgte hændelser}
Følgende hændelser er blevet skabt ud fra de valgte klassekandidater, som gruppen kom frem til i \secref{sec:klasser}. Klassekandidaterne er karaktiseret ved hjælp af følgende hændelseskandidater: 

\begin{itemize} [noitemsep]
\item Råvare opbrugt
\item Råvare smidt ud
\item Råvare købt
\item Opskrift fundet
\item Opskrift valgt
\item Opskrift smidt ud
\item Bogmærke tilføjet
\item Bogmærke fjernet
\item Indkøbsliste oprettet
\item Indkøbsliste færdig
\item Indkøbsliste smidt ud
\item Ingrediens tilføjet
\item Ingrediens fjernet
\item Tekst tilføjet
\item Tekst fjernet
\end{itemize}

\subsection{Hændelsestabel}
Når de valgte klasser og hændelser er kommet på plads, giver det mulighed at fremstille et hændelsestabel, der danner overblik over sammenhæng mellem klasser og fælles hændelser. Herunder ses hændelsestabellen:

\ourtable{haendelsestabel}{5}{Hændelsestabel for klasserne bogmærke, opskrift, indkøbsliste, ingrediens og råvare. \iter-symbolet illustrerer, at den tilhørende hændelse kan forekomme flere gange i samme klasse. Det betyder, at denne hændelse kan ses som en iteration i tilstandsdiagrammerne i \secref{sec:adfaerd}. \once-symbolet illustrerer, at den tilhørende hændelse forekommer en gang i samme klasse. Det betyder, at hændelsen kan ses som en selektion eller en sekvens i tilstandsdiagrammerne.}
                                                             {Klasser}
       {Hændelser             	}{Bogmærke   & Opskrift   & Indkøbsliste & Ingrediens  & Råvare  & Fejl       }{
\ourrow{Opskrift smidt ud     	}{           & \once      &              &  \once      &         &            }
\ourrow{Opskrift fundet         }{           & \once      &              &  \once      &         &            }
\ourrow{Bogmærke sat ind      	}{\once      & \iter      &              &             &         &            }
\ourrow{Bogmærke fjernet      	}{\once      & \iter      &              &             &         &            }
\ourrow{Indkøbsliste oprettet   }{           &            & \once        &             &         &            }
\ourrow{Indkøbsliste smidt ud   }{           &            & \once        &             &         &            }
\ourrow{Skrevet på indkøbsliste	}{           & \iter      &  \iter       &  \iter      &         &            }
\ourrow{Fjernet fra indkøbsliste}{           &            &  \iter       &  \iter      &         &            }
\ourrow{Råvare opbrugt        	}{           &            &              &             & \iter   &            }
\ourrow{Råvare købt           	}{           &            &              &             & \iter   &            }
\ourrow{Fejl fundet             }{           & \iter      &              &             &         & \once      }
\ourrow{Fejl rapport\'{e}ret    }{           &            &              &             &         & \once      }
\ourrow{Tilbagemelding modtaget }{           &            &              &             &         & \iter / \once}
\ourrow{Fejl bortset            }{           &            &              &             &         & \once      }
}


             %hændelser, hændelsestabel, tilstandsdiagram

