\section{Klasser}
\label{sec:klasser}

En klasse er en beskrivelse af en samling af objekter med samme struktur, adfærdsmønster og attributter \cite[s. ~51]{ooad}. Som nævnt i indledningen til kapitlet, er et af målene med analysen af problemområdet, at skabe et overblik over hvilke klasser og hændelser systemet skal have. Hændelser vil vi dog først komme nærmere ind på, i kommende section \ref{sec:haendelser}, men overblikket over hvilke klasser systemet skal have, kan skabes ved først at finde så mange klasserkandidater, som muligt, og dernæst at afgrænse disse, så kun de mest relevante klasser er tilbage. I vores tilfælde anvendte vi rigebilleder, som kan ses i \ref{ap:rigebilleder}, samt systemdefinitionen, til at hente inspiration til de klasser, som vi vil modellere i problemområdet.

Grundet den iterative arbejdsproces, er klasser undervejs blevet tilføjet og fjerent, og vi ønsker nu at kaste lys over baggrunden bag de valgte klasser. I gennem processen har vi også fravalgt klasser. Disse, med beskrivelse, kan findes i \apref{ap:fravalgteklasser}.

Herunder ses de valgte klasser og hvorfor vi vælger at have dem med i vores model af problemområdet.
Vi mener at disse klasser samler de objekter og hændelser, som er relevant for denne model af problemområdet.

\begin{description}
\item[Ingrediens] \hfill \\ 
I opskrifter bruges der flere ingredienser. En ingrediens består af en råvare og en mængde af denne. Det er et problem at finde opskrifter, der indeholder ingredienser svarende til de råvarer man har til rådig. Det er et problem for informanterne at maden laves i for store portioner, altså er det et problem hvis en opskrifts ingredienser indeholder store mængder.

\item[Bogmærke] \hfill \\
Det er en del af problemområdet, for brugere at huske de gode opskrifter. Der vil til tider blive benyttet en opskrift, der er så god, at den er værd at gemme til en anden gang. Derfor beholder vi denne klasse.

\item[Råvare] \hfill \\
En råvare findes i køleskabene og på madhylderne i husholdningerne. Det er et problem at finde opskrifter, der kun indeholder disse råvarer, derfor skelnes der mellem ingredieser og råvarer.

\item[Indkøbsliste] \hfill \\
Vi vurderer, at der i en husholdning ofte bliver skrevet en indkøbsliste med de ting man mangler. Indkøbslisten kan være skrevet på baggrund af en opskrift man gerne vil lave, eller en hel madplan man gerne vil følge over en længere periode.

\item[Opskrift] \hfill \\
En opskrift er det centrale i problemområdet. Opskrifterne indeholder forskellige ingredienser. Det er nødvendigt at have råvarer nok til at matche ingredienserne i opskriften, før denne kan laves. Man må gå ud fra at den typiske private madlaver har mange opskrifter i kogebogen, som han/hun reelt ikke har råvarerne til at kunne lave.
\end{description}

\subsection{Struktur}

De valgte klasser giver anledning til et klassediagram. Dette kan ses i \figref{fig:klassediagram}. 

\begin{figure}
  \centering
  \input{billeder/klassediagrampo.pdf_tex}
  \capt{Klassediagram for problemområdet.}
  \label{fig:klassediagram}
\end{figure}


Klassediagrammet ovenover er bygget op af aggregeringer og associationer imellem klasserne i diagrammet. Disse relationer er ligeledes beskrevet tekstuelt herunder.

\begin{description}
  \item[Bogmærke] \hfill \\
    Et bogmærke peger på netop en opskrift.

  \item[Opskrift] \hfill \\
    En opskrift består af en eller flere ingredienser og kan være tilknyttet nul til mange bogmærker.

\item[Indkøbsliste] \hfill \\
  En indkøbslisten består af flere ingredienser, den kan dog også være tom.

\item[Ingrediens] \hfill \\
  En ingrediens består af netop en råvare og kan tilhøre nul til mange opskrifter. Ligeledes kan en ingrediens tilhøre en indkøbsliste.

\item[Råvare] \hfill \\
  En råvare er en dekomponering af en ingrediens. En råvare er blot en ingrediens uden nogen form for information om mængde eller enhed. En ingrediens kunne for eksempel være "200g oksekød", hvormed den underliggende råvare ville være "oksekød".
\end{description}

