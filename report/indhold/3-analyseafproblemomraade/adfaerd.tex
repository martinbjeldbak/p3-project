\section{Adfærd}
\label{sec:adfaerd}

Den sidste aktivitet i analyse af problemområdet, består i at analysere og beskrive adfærdsmønstre for klassers objekter. Formålet med dette, er at få en bred for forståelse af, hvordan objekter kan opføre sig, hvilket vi vil bruge senere, når vi skal beskrive de funktioner \Foodl{} skal have. Dermed opnåes der også en mere glidende overgang over mod analyse af anvendelsesområdet \chapref{chap:analyseafao}. Desuden har vi anvendt adfærdmønstrene til at få overblik over, om de hændelser vi har valgt, er fyldestgørende, og ydermere, til at få inspiration til nye hændelser. Adfærdsmønstre har vi valgt at beskrive ved hjælp af tilstandsdiagrammer. For hver klasse følger der et tilstandsdiagram, som illusterer adfærden for objekter af den pågældende klasse. I tilstandsdiagrammerne skal de afrundede rektangulære bokse med tekst, anses som tilstande, som den pågældende klasse kan have. Pilene der fører til en tilstand, skal anses som hændelser, som kan være skyld i et tilstandsskift for objektet. Som eksempel, kan et objekt af klassen råvare, skifte mellem tilstandene ``eksisterer'' og ``brugbar'', ved hjælp af hændelserne ``Råvare købt'' og ``Råvare opbrugt''. Går en pil med en hændelse til og fra, den samme tilstand, er der tale om en løkke. En løkke er en hændelse som kan ske igen og igen, uden at objektet ændre tilstand. Eksempelvis kan hændelsen ``vare tilføjet'' ske igen og igen, for et indkøbsliste-objekt. Som oftest vil en klasse have en starthændelse, som ``sætter gang'' i klassen, og en sluthændelse, som ``slutter'' klassen. En starthændelse er markeret med en fyldt sort cirkel, og en sluthændelse er markeret med en fyldt sort cirkel, som ligger inde i en anden sort cirkel.

\subsection{Opskrift}
En opskrift kan befinde sig i en kogebog, på et stykke papir eller på en hjemmesiden. I \figref{fig:opskrift-adfaerd}, ses adfærden for klassen ``Opskrift''. En opskrift kommer til live, ved hændelsen \textit{opskrift fundet}. Her befinder den sig i tilstanden \textit{findes i opskriftssamlingen}. Hvis man er rigtig glad for en opskrift, kan man sætte et bogmærke på opskriften, så man hurtigt kan finde den igen. Dette kan \fx gøres ved at sætte en post-it note på en bestemt side i kogebogen. Sådan en hændelse kaldes for \textit{bogmærke tilføjet}. Hvis man ombestemmer sig og fjerner bogmærket, indtræffer hændelsen \textit{bogmærke fjernet}. Derudover kan man være nødsaget til at handle ind til en specifik opskrift, fordi man mangler nogle ingredienser. Det betyder, at man skal tilføje nogle opskrifter til en indkøbsliste. Denne hændelser kaldes for \textit{skrevet på indkøbsliste} med den modsignende hændelse \textit{fjernet fra indkøbsliste}. Der kan også eksistere en fejl i opskriften, som man muligvis lægger mærke til. En sådan fejl kunne være et manglende billede til opskriften, eller at der står 20 minutter i ovnen i stedet for 40. Dette er beskrevet ved hjælp af hændelsen ``fejl fundet''. Disse fire hændelser medfører ikke et tilstandsskift. Opskriften ryger kun ud af tilstanden \textit{findes i opskriftssamlingen}, når opskriften fjernes, vha. hændelsen \textit{opskrift smidt ud}..

\pdffig[0.8]{tilstandsdiagrammer/opskrift}
  {Tilstandsdiagram for klassen opskrift. Klassen har én tilstand (findes i opskriftssamlingen) og en afsluttende hændelse ``opskrift fjernet'', der fører klassen ud i en sluttilstand.}
  {fig:opskrift-adfaerd}

\subsection{Bogmærke}
Som nævnt kan man vælge at tilføje et bogmærke til en opskrift, man kan lide. Denne hændelse kaldes \textit{bogmærke tilføjet}, og bringer klassen i tilstanden \textit{aktiv}. Bogmærket er aktiv, indtil den kommer i sluttilstanden efter, at hændelsen \textit{bogmærke fjernet} indtræffer. Se \figref{fig:bogmaerke-adfaerd}.

\begin{figure}[H]
	\centering
	\scalebox{0.8}{
	\subsection{Bogmærke}
Som nævnt kan man vælge at tilføje et bogmærke til en opskrift, man kan lide. Denne hændelse kaldes \textit{bogmærke tilføjet}, og bringer klassen i tilstanden \textit{aktiv}. Bogmærket er aktiv, indtil den kommer i sluttilstanden efter, at hændelsen \textit{bogmærke fjernet} indtræffer. Se \figref{fig:bogmaerke-adfaerd}.

\begin{figure}[H]
	\centering
	\scalebox{0.8}{
	\subsection{Bogmærke}
Som nævnt kan man vælge at tilføje et bogmærke til en opskrift, man kan lide. Denne hændelse kaldes \textit{bogmærke tilføjet}, og bringer klassen i tilstanden \textit{aktiv}. Bogmærket er aktiv, indtil den kommer i sluttilstanden efter, at hændelsen \textit{bogmærke fjernet} indtræffer. Se \figref{fig:bogmaerke-adfaerd}.

\begin{figure}[H]
	\centering
	\scalebox{0.8}{
	\input{billeder/tilstandsdiagrammer/bogmaerke.pdf_tex}}
	\capt{Tilstandsdiagram for klassen bogmærke. De afrundede rektangulære bokse med tekst, skal anses som tilstande, som klassen kan have. De pile, der fører til en tilstand, skal anses som hændelser, som kan være skyld i et tilstandsskift. I dette tilfælde har klassen én tilstand (aktiv) og en afsluttende hændelse, der fører klassen ud i en sluttilstand (den sorte prik i den sorte cirkel).}
	\label{fig:bogmaerke-adfaerd}
\end{figure}}
	\capt{Tilstandsdiagram for klassen bogmærke. De afrundede rektangulære bokse med tekst, skal anses som tilstande, som klassen kan have. De pile, der fører til en tilstand, skal anses som hændelser, som kan være skyld i et tilstandsskift. I dette tilfælde har klassen én tilstand (aktiv) og en afsluttende hændelse, der fører klassen ud i en sluttilstand (den sorte prik i den sorte cirkel).}
	\label{fig:bogmaerke-adfaerd}
\end{figure}}
	\capt{Tilstandsdiagram for klassen bogmærke. De afrundede rektangulære bokse med tekst, skal anses som tilstande, som klassen kan have. De pile, der fører til en tilstand, skal anses som hændelser, som kan være skyld i et tilstandsskift. I dette tilfælde har klassen én tilstand (aktiv) og en afsluttende hændelse, der fører klassen ud i en sluttilstand (den sorte prik i den sorte cirkel).}
	\label{fig:bogmaerke-adfaerd}
\end{figure}
\begin{figure}[htp]
\centering
\scalebox{0.6}{
\begin{figure}[htp]
\centering
\scalebox{0.6}{
\begin{figure}[htp]
\centering
\scalebox{0.6}{
\input{billeder/tilstandsdiagrammer/ingrediens.pdf_tex}}
\capt{Tilstandsdiagram for Ingrediens-klassens adfærdsmønstre}\label{fig:ingrediens-adfaerd}
\end{figure}}
\capt{Tilstandsdiagram for Ingrediens-klassens adfærdsmønstre}\label{fig:ingrediens-adfaerd}
\end{figure}}
\capt{Tilstandsdiagram for Ingrediens-klassens adfærdsmønstre}\label{fig:ingrediens-adfaerd}
\end{figure}
\paragraph{Råvare}
Når nogen i sin husstand køber en råvare, \fx i et supermarked, indtræffer hændelsen \textit{råvare købt} netop. I den forbindelse kommer et råvare-objekt til verdenen. Denne råvare er klar til at blive brugt, hvilket vil sige at råvaren har tilstanden \textit{brugbar}, indtil den havner i sin sluttilstande ved at man enten smider råvaren ud eller er har opbrugt den helt. Disse to hændelser kaldes i nævnte rækkefølge \textit{råvare smidt ud} og \textit{råvare opbrugt}.
\begin{figure}[htp]
\centering
\scalebox{0.6}{
\paragraph{Råvare}
Når nogen i sin husstand køber en råvare, \fx i et supermarked, indtræffer hændelsen \textit{råvare købt} netop. I den forbindelse kommer et råvare-objekt til verdenen. Denne råvare er klar til at blive brugt, hvilket vil sige at råvaren har tilstanden \textit{brugbar}, indtil den havner i sin sluttilstande ved at man enten smider råvaren ud eller er har opbrugt den helt. Disse to hændelser kaldes i nævnte rækkefølge \textit{råvare smidt ud} og \textit{råvare opbrugt}.
\begin{figure}[htp]
\centering
\scalebox{0.6}{
\paragraph{Råvare}
Når nogen i sin husstand køber en råvare, \fx i et supermarked, indtræffer hændelsen \textit{råvare købt} netop. I den forbindelse kommer et råvare-objekt til verdenen. Denne råvare er klar til at blive brugt, hvilket vil sige at råvaren har tilstanden \textit{brugbar}, indtil den havner i sin sluttilstande ved at man enten smider råvaren ud eller er har opbrugt den helt. Disse to hændelser kaldes i nævnte rækkefølge \textit{råvare smidt ud} og \textit{råvare opbrugt}.
\begin{figure}[htp]
\centering
\scalebox{0.6}{
\input{billeder/tilstandsdiagrammer/raavare.pdf_tex}}
\capt{Tilstandsdiagram for Råvare-klassens adfærdsmønstre}\label{fig:raavare-adfaerd}
\end{figure}}
\capt{Tilstandsdiagram for Råvare-klassens adfærdsmønstre}\label{fig:raavare-adfaerd}
\end{figure}}
\capt{Tilstandsdiagram for Råvare-klassens adfærdsmønstre}\label{fig:raavare-adfaerd}
\end{figure}
\subsection{Indkøbsliste}
En indkøbsliste kommer til verden ved, at man i husstanden beslutter sig for at benytte en eller anden form for huskeliste for ting, der skal handles ind. Dette kan være i form af et papir, der ligger et fast sted på bordet eller hænger på opslagstavlen. Denne initierende hændelse kaldes \textit{indkøbsliste oprettet}. Se \figref{fig:indkoebsliste-adfaerd}.

Så længe indkøbslisten ligger på bordet eller et andet sted, kan mange personer komme forbi og tilføje eller fjerne varer på den. Denne tilstand kaldes for \textit{redigeres}. I denne tilstand er det muligt for alle, der kan komme til denne indkøbsliste, at tilføje eller fjerne de varer, der skal handles ind. Man kan \fx fjerne en vare ved at slå en streg over den og give et klart signal om, at denne ikke skal købes. Disse to hændelser hedder \textit{vare fjernet} og \textit{vare tilføjet}. Personen, der tilføjer tekst til indkøbslistene, er herre over, om der skal stå en ingrediens, der består af en mængde, en enhed og en råvare (\fx 1 ltr skummetmælk) eller der blot skal stå råvaren (\fx skummetmælk).

Indkøbslisten kan også indeholde bemærkninger, såsom ``hvis det er på tilbud''. Denne bemærkning kan selvfølgelig også fjernes igen på samme måde som en vare kan fjernes. Disse bemærkninger tilhører en eller flere varer på indkøbslisten, og derfor er hændelsen den samme som ved tilføjelse eller fjernelse af en vare.

Når en person beslutter sig for at tage indkøbslisten med på indkøb, anses indkøbslisten for at være færdig. Det er nu ikke længere muligt at redigere i indkøbslisten, og den afsluttende hændelse indtræffer. 

Ude i supermarkedet kan man købe mange forskellige råvarer. Vi ønsker ikke at overvåge, hvor meget folk har af en ingrediens, og benytter derfor istedet objektet råvare, der ikke indeholder nogen mængde eller enhed. Denne beslutning er taget på baggrund af møde 2 med vores informanter.\todo{Argumenter lidt bedre for valget.}

\begin{figure}[H]
	\centering
	\scalebox{0.8}{
		\subsection{Indkøbsliste}
En indkøbsliste kommer til verden ved, at man i husstanden beslutter sig for at benytte en eller anden form for huskeliste for ting, der skal handles ind. Dette kan være i form af et papir, der ligger et fast sted på bordet eller hænger på opslagstavlen. Denne initierende hændelse kaldes \textit{indkøbsliste oprettet}. Se \figref{fig:indkoebsliste-adfaerd}.

Så længe indkøbslisten ligger på bordet eller et andet sted, kan mange personer komme forbi og tilføje eller fjerne varer på den. Denne tilstand kaldes for \textit{redigeres}. I denne tilstand er det muligt for alle, der kan komme til denne indkøbsliste, at tilføje eller fjerne de varer, der skal handles ind. Man kan \fx fjerne en vare ved at slå en streg over den og give et klart signal om, at denne ikke skal købes. Disse to hændelser hedder \textit{vare fjernet} og \textit{vare tilføjet}. Personen, der tilføjer tekst til indkøbslistene, er herre over, om der skal stå en ingrediens, der består af en mængde, en enhed og en råvare (\fx 1 ltr skummetmælk) eller der blot skal stå råvaren (\fx skummetmælk).

Indkøbslisten kan også indeholde bemærkninger, såsom ``hvis det er på tilbud''. Denne bemærkning kan selvfølgelig også fjernes igen på samme måde som en vare kan fjernes. Disse bemærkninger tilhører en eller flere varer på indkøbslisten, og derfor er hændelsen den samme som ved tilføjelse eller fjernelse af en vare.

Når en person beslutter sig for at tage indkøbslisten med på indkøb, anses indkøbslisten for at være færdig. Det er nu ikke længere muligt at redigere i indkøbslisten, og den afsluttende hændelse indtræffer. 

Ude i supermarkedet kan man købe mange forskellige råvarer. Vi ønsker ikke at overvåge, hvor meget folk har af en ingrediens, og benytter derfor istedet objektet råvare, der ikke indeholder nogen mængde eller enhed. Denne beslutning er taget på baggrund af møde 2 med vores informanter.\todo{Argumenter lidt bedre for valget.}

\begin{figure}[H]
	\centering
	\scalebox{0.8}{
		\subsection{Indkøbsliste}
En indkøbsliste kommer til verden ved, at man i husstanden beslutter sig for at benytte en eller anden form for huskeliste for ting, der skal handles ind. Dette kan være i form af et papir, der ligger et fast sted på bordet eller hænger på opslagstavlen. Denne initierende hændelse kaldes \textit{indkøbsliste oprettet}. Se \figref{fig:indkoebsliste-adfaerd}.

Så længe indkøbslisten ligger på bordet eller et andet sted, kan mange personer komme forbi og tilføje eller fjerne varer på den. Denne tilstand kaldes for \textit{redigeres}. I denne tilstand er det muligt for alle, der kan komme til denne indkøbsliste, at tilføje eller fjerne de varer, der skal handles ind. Man kan \fx fjerne en vare ved at slå en streg over den og give et klart signal om, at denne ikke skal købes. Disse to hændelser hedder \textit{vare fjernet} og \textit{vare tilføjet}. Personen, der tilføjer tekst til indkøbslistene, er herre over, om der skal stå en ingrediens, der består af en mængde, en enhed og en råvare (\fx 1 ltr skummetmælk) eller der blot skal stå råvaren (\fx skummetmælk).

Indkøbslisten kan også indeholde bemærkninger, såsom ``hvis det er på tilbud''. Denne bemærkning kan selvfølgelig også fjernes igen på samme måde som en vare kan fjernes. Disse bemærkninger tilhører en eller flere varer på indkøbslisten, og derfor er hændelsen den samme som ved tilføjelse eller fjernelse af en vare.

Når en person beslutter sig for at tage indkøbslisten med på indkøb, anses indkøbslisten for at være færdig. Det er nu ikke længere muligt at redigere i indkøbslisten, og den afsluttende hændelse indtræffer. 

Ude i supermarkedet kan man købe mange forskellige råvarer. Vi ønsker ikke at overvåge, hvor meget folk har af en ingrediens, og benytter derfor istedet objektet råvare, der ikke indeholder nogen mængde eller enhed. Denne beslutning er taget på baggrund af møde 2 med vores informanter.\todo{Argumenter lidt bedre for valget.}

\begin{figure}[H]
	\centering
	\scalebox{0.8}{
		\input{billeder/tilstandsdiagrammer/indkoebsliste.pdf_tex}}
		\capt{Tilstandsdiagram for klassen indkøbsliste. De afrundede rektangulære bokse med tekst, skal anses som tilstande, som klassen kan have. De pile, der fører til en tilstand, skal anses som hændelser, som kan være skyld i et tilstandsskift. I dette tilfælde har klassen én tilstand (redigeres) og en afsluttende hændelse, der fører klassen ud i en sluttilstand (den sorte prik i den sorte cirkel).}
		\label{fig:indkoebsliste-adfaerd}
\end{figure}}
		\capt{Tilstandsdiagram for klassen indkøbsliste. De afrundede rektangulære bokse med tekst, skal anses som tilstande, som klassen kan have. De pile, der fører til en tilstand, skal anses som hændelser, som kan være skyld i et tilstandsskift. I dette tilfælde har klassen én tilstand (redigeres) og en afsluttende hændelse, der fører klassen ud i en sluttilstand (den sorte prik i den sorte cirkel).}
		\label{fig:indkoebsliste-adfaerd}
\end{figure}}
		\capt{Tilstandsdiagram for klassen indkøbsliste. De afrundede rektangulære bokse med tekst, skal anses som tilstande, som klassen kan have. De pile, der fører til en tilstand, skal anses som hændelser, som kan være skyld i et tilstandsskift. I dette tilfælde har klassen én tilstand (redigeres) og en afsluttende hændelse, der fører klassen ud i en sluttilstand (den sorte prik i den sorte cirkel).}
		\label{fig:indkoebsliste-adfaerd}
\end{figure}
\subsection{Fejl}
Fejl opstår ved, at opskriftsskribenten glemmer at tilføje nogle vigtige ingredienser i en opskrift, eller \fx mangler et trin i fremgangsmåden. Det skal derfor være muligt for personer at indrapportere deres input, så de kan hjælpe forelaget med at korrigere de fejl, der måtte eksistere i opskriften. Dette kan være med til at forbedre kvaliteten af udgivelsen. Tilstandsdiagrammet for klassen ``fejl'' ses i \figref{fig:fejl}.

Starthændelsen \textit{fejl fundet} opstår, når læseren opdager en fejl i en opskrift. Derefter eksisterer der en selektion, hvor brugeren har valget mellem at rapportere fejlen til opskriftsskribenten (hændelsen \textit{fejl rapporteret}) eller blot springe den over og gå videre til en anden opskrift (hændelsen \textit{fejl set bort fra}, hvilket fører til henholdsvis en ny tilstand ``fejl rapporteret'' eller sluttilstanden. Hvordan fejlen bliver rapporteret er ikke relevant i forhold til denne modellering af problemområdet, vi er bare interesseret i, at det bliver gjort. Når fejlen er rapporteret, er der mulighed for både en iterativ og en tilstandsændrende (til sluttilstanden) form for tilbagemelding (hændelsen \textit{tilbagemelding modtaget}). Dette afhænger af konteksten mellem personen og opskriftskribenten, om kontakten kun sker envejs, eller om der er løbende kontakt mellem de to parter.

Klassen er, ligesom vare-klassen, kommet til verden ud fra gruppediskussioner under komponentarkitekturdesignet. Dette er grundet, at vi mener forskellige opskriftshjemmesider har varierende kvalitet af opskrifter. Derfor skal man, som forbrugere, kunne gøre andre opmærksomme på diverse fejl, der skulle befinde sig på en opskrift. Så har modtageren mulighed for at kigge på fejlene og eventuelt rette dem, så brugere får en endnu bedre oplevelse.

\pdffig[0.6]{tilstandsdiagrammer/fejl}
{Tilstandsdiagram for klassen ``fejl''. Klassen har to tilstande, som begge kan føre til sluttilstanden i forhold til de valg, som der bliver foretaget}
  {fig:fejl}



%\aktortabelEx{Bruger}
{En person, der ønsker at bruge systemet foodl.dk til at finde opskrifter, der er mulige at lave med de råvarer, som personen er i besiddelse af.}
{Systemets brugere inkluderer mange personer i forskellige aldersgrupper med vidt forskellig erfaring inden for computerbrug.}
{Bruger A er en 23-årig universitetsstuderende, der føler sig sikker med at navigere rundt på internettet og gør det flere gange dagligt. A kan godt lide at afprøve de forskellige funktioner som hjemmesider stiller til rådighed, for at undersøge hvad de gør. A er meget lærenem, når det kommer til at benytte funktioner på hjemmesider. A bor alene, og har pga. supermarkedernes familietilpassede portioner, ofte madrester til overs, som A ønsker at bruge. Systemet bliver brugt til at få madresterne med i aftenens aftensmad.
 
Bruger B er en 45-årig familiemor eller -far, der hovedsagligt bruger computeren til arbejdsrelaterede opgaver og til at holde sig opdateret ved at læse nyheder på diverse nyhedshjemmesider. Bruger B benytter systemet til blandt andet at få brugt madrester fra den foregående dags aftensmad eller til at få inspiration til den kommende aftensmad. Bruger B vil være interesseret i at være i stand til at dele f.eks. indkøbsliste med ægtefællen, på tværs af enheder.}

