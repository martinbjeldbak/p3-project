\section{Adfærd}
\label{sec:adfaerd}

De forskellige klasser og hændelser i problemområdet er nu blevet analyseret og beskrevet. Alle hændelsesforløb, der er mulige for alle objekter i en klasse, kan beskrives ved hjælp af adfærdsmønstre\cite[s.~90]{ooad}. Den objektorienterede tilgang er baseret på, at der i vores systems model skal være et objekt for hvert objekt i problemområdet\cite[s.~91]{ooad}. Hvert objekt skal registrere og huske adfærden af det tilsvarende objekt i problemområdet, som det ændrer sig over tid.

For hver klasse følger er der en tilhørende figur, som illusterer klassens adfærd. I figurene er der et start  der bokse og pile. De afrundede rektangulære bokse med tekst, skal anses som tilstande, som den pågældende klasse kan have. Pilene der fører til en tilstand, skal anses som hændelser, som kan være skyld i et tilstandsskift.

\subsection{Indkøbsliste}
En indkøbsliste kommer til verden ved, at man i husstanden beslutter sig for at benytte en eller anden form for huskeliste for ting, der skal handles ind. Dette kan være i form af et papir, der ligger et fast sted på bordet eller hænger på opslagstavlen. Denne initierende hændelse kaldes \textit{indkøbsliste oprettet}. Se \figref{fig:indkoebsliste-adfaerd}.

Så længe indkøbslisten ligger på bordet eller et andet sted, kan mange personer komme forbi og tilføje eller fjerne varer på den. Denne tilstand kaldes for \textit{redigeres}. I denne tilstand er det muligt for alle, der kan komme til denne indkøbsliste, at tilføje eller fjerne de varer, der skal handles ind. Man kan \fx fjerne en vare ved at slå en streg over den og give et klart signal om, at denne ikke skal købes. Disse to hændelser hedder \textit{vare fjernet} og \textit{vare tilføjet}. Personen, der tilføjer tekst til indkøbslistene, er herre over, om der skal stå en ingrediens, der består af en mængde, en enhed og en råvare (\fx 1 ltr skummetmælk) eller der blot skal stå råvaren (\fx skummetmælk).

Indkøbslisten kan også indeholde bemærkninger, såsom ``hvis det er på tilbud''. Denne bemærkning kan selvfølgelig også fjernes igen på samme måde som en vare kan fjernes. Disse bemærkninger tilhører en eller flere varer på indkøbslisten, og derfor er hændelsen den samme som ved tilføjelse eller fjernelse af en vare.

Når en person beslutter sig for at tage indkøbslisten med på indkøb, anses indkøbslisten for at være færdig. Det er nu ikke længere muligt at redigere i indkøbslisten, og den afsluttende hændelse indtræffer. 

Ude i supermarkedet kan man købe mange forskellige råvarer. Vi ønsker ikke at overvåge, hvor meget folk har af en ingrediens, og benytter derfor istedet objektet råvare, der ikke indeholder nogen mængde eller enhed. Denne beslutning er taget på baggrund af møde 2 med vores informanter.\todo{Argumenter lidt bedre for valget.}

\begin{figure}[H]
	\centering
	\scalebox{0.8}{
		\subsection{Indkøbsliste}
En indkøbsliste kommer til verden ved, at man i husstanden beslutter sig for at benytte en eller anden form for huskeliste for ting, der skal handles ind. Dette kan være i form af et papir, der ligger et fast sted på bordet eller hænger på opslagstavlen. Denne initierende hændelse kaldes \textit{indkøbsliste oprettet}. Se \figref{fig:indkoebsliste-adfaerd}.

Så længe indkøbslisten ligger på bordet eller et andet sted, kan mange personer komme forbi og tilføje eller fjerne varer på den. Denne tilstand kaldes for \textit{redigeres}. I denne tilstand er det muligt for alle, der kan komme til denne indkøbsliste, at tilføje eller fjerne de varer, der skal handles ind. Man kan \fx fjerne en vare ved at slå en streg over den og give et klart signal om, at denne ikke skal købes. Disse to hændelser hedder \textit{vare fjernet} og \textit{vare tilføjet}. Personen, der tilføjer tekst til indkøbslistene, er herre over, om der skal stå en ingrediens, der består af en mængde, en enhed og en råvare (\fx 1 ltr skummetmælk) eller der blot skal stå råvaren (\fx skummetmælk).

Indkøbslisten kan også indeholde bemærkninger, såsom ``hvis det er på tilbud''. Denne bemærkning kan selvfølgelig også fjernes igen på samme måde som en vare kan fjernes. Disse bemærkninger tilhører en eller flere varer på indkøbslisten, og derfor er hændelsen den samme som ved tilføjelse eller fjernelse af en vare.

Når en person beslutter sig for at tage indkøbslisten med på indkøb, anses indkøbslisten for at være færdig. Det er nu ikke længere muligt at redigere i indkøbslisten, og den afsluttende hændelse indtræffer. 

Ude i supermarkedet kan man købe mange forskellige råvarer. Vi ønsker ikke at overvåge, hvor meget folk har af en ingrediens, og benytter derfor istedet objektet råvare, der ikke indeholder nogen mængde eller enhed. Denne beslutning er taget på baggrund af møde 2 med vores informanter.\todo{Argumenter lidt bedre for valget.}

\begin{figure}[H]
	\centering
	\scalebox{0.8}{
		\subsection{Indkøbsliste}
En indkøbsliste kommer til verden ved, at man i husstanden beslutter sig for at benytte en eller anden form for huskeliste for ting, der skal handles ind. Dette kan være i form af et papir, der ligger et fast sted på bordet eller hænger på opslagstavlen. Denne initierende hændelse kaldes \textit{indkøbsliste oprettet}. Se \figref{fig:indkoebsliste-adfaerd}.

Så længe indkøbslisten ligger på bordet eller et andet sted, kan mange personer komme forbi og tilføje eller fjerne varer på den. Denne tilstand kaldes for \textit{redigeres}. I denne tilstand er det muligt for alle, der kan komme til denne indkøbsliste, at tilføje eller fjerne de varer, der skal handles ind. Man kan \fx fjerne en vare ved at slå en streg over den og give et klart signal om, at denne ikke skal købes. Disse to hændelser hedder \textit{vare fjernet} og \textit{vare tilføjet}. Personen, der tilføjer tekst til indkøbslistene, er herre over, om der skal stå en ingrediens, der består af en mængde, en enhed og en råvare (\fx 1 ltr skummetmælk) eller der blot skal stå råvaren (\fx skummetmælk).

Indkøbslisten kan også indeholde bemærkninger, såsom ``hvis det er på tilbud''. Denne bemærkning kan selvfølgelig også fjernes igen på samme måde som en vare kan fjernes. Disse bemærkninger tilhører en eller flere varer på indkøbslisten, og derfor er hændelsen den samme som ved tilføjelse eller fjernelse af en vare.

Når en person beslutter sig for at tage indkøbslisten med på indkøb, anses indkøbslisten for at være færdig. Det er nu ikke længere muligt at redigere i indkøbslisten, og den afsluttende hændelse indtræffer. 

Ude i supermarkedet kan man købe mange forskellige råvarer. Vi ønsker ikke at overvåge, hvor meget folk har af en ingrediens, og benytter derfor istedet objektet råvare, der ikke indeholder nogen mængde eller enhed. Denne beslutning er taget på baggrund af møde 2 med vores informanter.\todo{Argumenter lidt bedre for valget.}

\begin{figure}[H]
	\centering
	\scalebox{0.8}{
		\input{billeder/tilstandsdiagrammer/indkoebsliste.pdf_tex}}
		\capt{Tilstandsdiagram for klassen indkøbsliste. De afrundede rektangulære bokse med tekst, skal anses som tilstande, som klassen kan have. De pile, der fører til en tilstand, skal anses som hændelser, som kan være skyld i et tilstandsskift. I dette tilfælde har klassen én tilstand (redigeres) og en afsluttende hændelse, der fører klassen ud i en sluttilstand (den sorte prik i den sorte cirkel).}
		\label{fig:indkoebsliste-adfaerd}
\end{figure}}
		\capt{Tilstandsdiagram for klassen indkøbsliste. De afrundede rektangulære bokse med tekst, skal anses som tilstande, som klassen kan have. De pile, der fører til en tilstand, skal anses som hændelser, som kan være skyld i et tilstandsskift. I dette tilfælde har klassen én tilstand (redigeres) og en afsluttende hændelse, der fører klassen ud i en sluttilstand (den sorte prik i den sorte cirkel).}
		\label{fig:indkoebsliste-adfaerd}
\end{figure}}
		\capt{Tilstandsdiagram for klassen indkøbsliste. De afrundede rektangulære bokse med tekst, skal anses som tilstande, som klassen kan have. De pile, der fører til en tilstand, skal anses som hændelser, som kan være skyld i et tilstandsskift. I dette tilfælde har klassen én tilstand (redigeres) og en afsluttende hændelse, der fører klassen ud i en sluttilstand (den sorte prik i den sorte cirkel).}
		\label{fig:indkoebsliste-adfaerd}
\end{figure}
\paragraph{Råvare}
Når nogen i sin husstand køber en råvare, \fx i et supermarked, indtræffer hændelsen \textit{råvare købt} netop. I den forbindelse kommer et råvare-objekt til verdenen. Denne råvare er klar til at blive brugt, hvilket vil sige at råvaren har tilstanden \textit{brugbar}, indtil den havner i sin sluttilstande ved at man enten smider råvaren ud eller er har opbrugt den helt. Disse to hændelser kaldes i nævnte rækkefølge \textit{råvare smidt ud} og \textit{råvare opbrugt}.
\begin{figure}[htp]
\centering
\scalebox{0.6}{
\paragraph{Råvare}
Når nogen i sin husstand køber en råvare, \fx i et supermarked, indtræffer hændelsen \textit{råvare købt} netop. I den forbindelse kommer et råvare-objekt til verdenen. Denne råvare er klar til at blive brugt, hvilket vil sige at råvaren har tilstanden \textit{brugbar}, indtil den havner i sin sluttilstande ved at man enten smider råvaren ud eller er har opbrugt den helt. Disse to hændelser kaldes i nævnte rækkefølge \textit{råvare smidt ud} og \textit{råvare opbrugt}.
\begin{figure}[htp]
\centering
\scalebox{0.6}{
\paragraph{Råvare}
Når nogen i sin husstand køber en råvare, \fx i et supermarked, indtræffer hændelsen \textit{råvare købt} netop. I den forbindelse kommer et råvare-objekt til verdenen. Denne råvare er klar til at blive brugt, hvilket vil sige at råvaren har tilstanden \textit{brugbar}, indtil den havner i sin sluttilstande ved at man enten smider råvaren ud eller er har opbrugt den helt. Disse to hændelser kaldes i nævnte rækkefølge \textit{råvare smidt ud} og \textit{råvare opbrugt}.
\begin{figure}[htp]
\centering
\scalebox{0.6}{
\input{billeder/tilstandsdiagrammer/raavare.pdf_tex}}
\capt{Tilstandsdiagram for Råvare-klassens adfærdsmønstre}\label{fig:raavare-adfaerd}
\end{figure}}
\capt{Tilstandsdiagram for Råvare-klassens adfærdsmønstre}\label{fig:raavare-adfaerd}
\end{figure}}
\capt{Tilstandsdiagram for Råvare-klassens adfærdsmønstre}\label{fig:raavare-adfaerd}
\end{figure}
\subsection{Opskrift}
En opskrift kan befinde sig i en kogebog, på et stykke papir eller på en hjemmesiden. I \figref{fig:opskrift-adfaerd}, ses adfærden for klassen ``Opskrift''. En opskrift kommer til live, ved hændelsen \textit{opskrift fundet}. Her befinder den sig i tilstanden \textit{findes i opskriftssamlingen}. Hvis man er rigtig glad for en opskrift, kan man sætte et bogmærke på opskriften, så man hurtigt kan finde den igen. Dette kan \fx gøres ved at sætte en post-it note på en bestemt side i kogebogen. Sådan en hændelse kaldes for \textit{bogmærke tilføjet}. Hvis man ombestemmer sig og fjerner bogmærket, indtræffer hændelsen \textit{bogmærke fjernet}. Derudover kan man være nødsaget til at handle ind til en specifik opskrift, fordi man mangler nogle ingredienser. Det betyder, at man skal tilføje nogle opskrifter til en indkøbsliste. Denne hændelser kaldes for \textit{skrevet på indkøbsliste} med den modsignende hændelse \textit{fjernet fra indkøbsliste}. Der kan også eksistere en fejl i opskriften, som man muligvis lægger mærke til. En sådan fejl kunne være et manglende billede til opskriften, eller at der står 20 minutter i ovnen i stedet for 40. Dette er beskrevet ved hjælp af hændelsen ``fejl fundet''. Disse fire hændelser medfører ikke et tilstandsskift. Opskriften ryger kun ud af tilstanden \textit{findes i opskriftssamlingen}, når opskriften fjernes, vha. hændelsen \textit{opskrift smidt ud}..

\pdffig[0.8]{tilstandsdiagrammer/opskrift}
  {Tilstandsdiagram for klassen opskrift. Klassen har én tilstand (findes i opskriftssamlingen) og en afsluttende hændelse ``opskrift fjernet'', der fører klassen ud i en sluttilstand.}
  {fig:opskrift-adfaerd}

\subsection{Bogmærke}
Som nævnt kan man vælge at tilføje et bogmærke til en opskrift, man kan lide. Denne hændelse kaldes \textit{bogmærke tilføjet}, og bringer klassen i tilstanden \textit{aktiv}. Bogmærket er aktiv, indtil den kommer i sluttilstanden efter, at hændelsen \textit{bogmærke fjernet} indtræffer. Se \figref{fig:bogmaerke-adfaerd}.

\begin{figure}[H]
	\centering
	\scalebox{0.8}{
	\subsection{Bogmærke}
Som nævnt kan man vælge at tilføje et bogmærke til en opskrift, man kan lide. Denne hændelse kaldes \textit{bogmærke tilføjet}, og bringer klassen i tilstanden \textit{aktiv}. Bogmærket er aktiv, indtil den kommer i sluttilstanden efter, at hændelsen \textit{bogmærke fjernet} indtræffer. Se \figref{fig:bogmaerke-adfaerd}.

\begin{figure}[H]
	\centering
	\scalebox{0.8}{
	\subsection{Bogmærke}
Som nævnt kan man vælge at tilføje et bogmærke til en opskrift, man kan lide. Denne hændelse kaldes \textit{bogmærke tilføjet}, og bringer klassen i tilstanden \textit{aktiv}. Bogmærket er aktiv, indtil den kommer i sluttilstanden efter, at hændelsen \textit{bogmærke fjernet} indtræffer. Se \figref{fig:bogmaerke-adfaerd}.

\begin{figure}[H]
	\centering
	\scalebox{0.8}{
	\input{billeder/tilstandsdiagrammer/bogmaerke.pdf_tex}}
	\capt{Tilstandsdiagram for klassen bogmærke. De afrundede rektangulære bokse med tekst, skal anses som tilstande, som klassen kan have. De pile, der fører til en tilstand, skal anses som hændelser, som kan være skyld i et tilstandsskift. I dette tilfælde har klassen én tilstand (aktiv) og en afsluttende hændelse, der fører klassen ud i en sluttilstand (den sorte prik i den sorte cirkel).}
	\label{fig:bogmaerke-adfaerd}
\end{figure}}
	\capt{Tilstandsdiagram for klassen bogmærke. De afrundede rektangulære bokse med tekst, skal anses som tilstande, som klassen kan have. De pile, der fører til en tilstand, skal anses som hændelser, som kan være skyld i et tilstandsskift. I dette tilfælde har klassen én tilstand (aktiv) og en afsluttende hændelse, der fører klassen ud i en sluttilstand (den sorte prik i den sorte cirkel).}
	\label{fig:bogmaerke-adfaerd}
\end{figure}}
	\capt{Tilstandsdiagram for klassen bogmærke. De afrundede rektangulære bokse med tekst, skal anses som tilstande, som klassen kan have. De pile, der fører til en tilstand, skal anses som hændelser, som kan være skyld i et tilstandsskift. I dette tilfælde har klassen én tilstand (aktiv) og en afsluttende hændelse, der fører klassen ud i en sluttilstand (den sorte prik i den sorte cirkel).}
	\label{fig:bogmaerke-adfaerd}
\end{figure}
\begin{figure}[htp]
\centering
\scalebox{0.6}{
\begin{figure}[htp]
\centering
\scalebox{0.6}{
\begin{figure}[htp]
\centering
\scalebox{0.6}{
\input{billeder/tilstandsdiagrammer/ingrediens.pdf_tex}}
\capt{Tilstandsdiagram for Ingrediens-klassens adfærdsmønstre}\label{fig:ingrediens-adfaerd}
\end{figure}}
\capt{Tilstandsdiagram for Ingrediens-klassens adfærdsmønstre}\label{fig:ingrediens-adfaerd}
\end{figure}}
\capt{Tilstandsdiagram for Ingrediens-klassens adfærdsmønstre}\label{fig:ingrediens-adfaerd}
\end{figure}

