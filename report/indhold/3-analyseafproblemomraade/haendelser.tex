\section{Hændelser}
\label{sec:haendelser}
Ved hjælp af klassekandidaterne, er det nu muligt at finde hændelseskandidater. Hændelserne er kommet til verden ud fra diverse forskellige forløb, der kan påvirke klasser. For at opretholde en konsistens mellem hændelserne, er de formuleret i datid. Det er igen vigtigt at nævne, at følgende er \emph{kandidater} og kan ændres under den iterative arbejdsproces. 

\subsection{Valgte hændelser}
Følgende hændelser er blevet skabt ud fra de valgte klassekandidater, som gruppen kom frem til i \secref{sec:klasser}. Klassekandidaterne er karaktiseret ved hjælp af følgende hændelseskandidater: 

\begin{itemize} [noitemsep]
\item Råvare opbrugt
\item Råvare smidt ud
\item Råvare købt
\item Opskrift fundet
\item Opskrift valgt
\item Opskrift smidt ud
\item Bogmærke tilføjet
\item Bogmærke fjernet
\item Indkøbsliste oprettet
\item Indkøbsliste færdig
\item Indkøbsliste smidt ud
\item Ingrediens tilføjet
\item Ingrediens fjernet
\item Tekst tilføjet
\item Tekst fjernet
\end{itemize}

\subsection{Hændelsestabel}
Når de valgte klasser og hændelser er kommet på plads, giver det mulighed at fremstille et hændelsestabel, der danner overblik over sammenhæng mellem klasser og fælles hændelser. Herunder ses hændelsestabellen:

\ourtable{haendelsestabel}{5}{Hændelsestabel for klasserne person, opskrift, ingrediens, råvaretype, vare og fejl}
                                                             {Klasser}
       {Hændelser             	}{ Opskrift & Person & Vare  & Ingrediens & Råvaretype & Fejl  }{
\ourrow{Opskrift smidt ud     	}{ \once    &        &       & \once      &            &       }
\ourrow{Opskrift fundet         }{ \once    &        &       & \once      &            &       }
\ourrow{Bogmærke sat ind      	}{ \iter    & \iter  &       &            &            &       }
\ourrow{Bogmærke fjernet      	}{ \iter    & \iter  &       &            &            &       }
\ourrow{Skrevet på indkøbsliste	}{ \iter    & \iter  & \once & \iter      &            &       }
\ourrow{Fjernet fra indkøbsliste}{          & \iter  & \once & \iter      &            &       }
\ourrow{Råvare opbrugt        	}{          &        &       &            & \iter      &       }
\ourrow{Råvare købt           	}{          &        &       &            & \iter      &       }
\ourrow{Fejl rapport\'{e}ret    }{ \iter    & \iter  &       &            &            & \once }
\ourrow{Tilbagemelding modtaget }{          & \once  &       &            &            & \once }
\ourrow{Fejl set bort fra       }{          &        &       &            &            & \once }
}


Hændelsestabellen er det sidste dokument i analyse af problemområdet, som nu kan antages som modelleret. 

\todo{Vær mere beskrivende\ldots}
