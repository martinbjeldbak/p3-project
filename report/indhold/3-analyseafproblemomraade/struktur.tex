\section{Struktur}
\label{sec:struktur}

Sturkturen på vores klasser, beskrives ved hjælp af et klassediagram. Med et klassediagram, skabes der et overblik over relationen mellem de forskellige klasser og objekter. \Foodl består af 7 klasser, nemlig ``Opskrift'', ``Bogmærke'', ``Råvare'', ``Ingrediens'', ``vare'', ``Indkøbsliste'' og ``Fejl''. Relationen mellem klasserne kan ses i \figref{fig:klassediagram}. En relation mellem to klasser markeret med en rumbe-pil, er en aggregerings-struktur, hvilket er en relation som beskrives ved udsagnet ``Har-en'' og ``indgår-i. Eksempelvis, så er en ingrediens en aggregering af indkøbsliste, hvilket betyder at en indkøbsliste ``har-en'' ingrediens, og en ingrediens ``ingår-i'' en indkøbsliste. En relation mellem to klasser markeret med en streg, er en associeringsstruktur, som betyder at de to klasser har en sammenhæng. I vores struktur, associerer klasserne bogmærke og opskrifter med hinanden. En associering er ikke, ligesom en aggregering, nem at knytte et udsagn til. Men i vores tilfælde kan en associering mellem bogmærke og opskrift beskrives som et ``hører-til''-forhold. Et bogmærke ``hører-til'' en opskrift, og en opskrift ``hører-til'' et bogmærke. Til relationerne mellem klasserne, høre der sig desuden et tilhørsforhold til. Dette beskriver forholdet mellem klasserne. Det kan eksempelvis være et et-til-et-forhold, eller et 0-til-mange-forhold. På \figref{fig:klassediagram} ses det, at forholdet mellem klassen opskrift, og klassen ingrediens er et 1-til-mange-forhold (*-tegnet betyder mange), mens forholdet mellem ingrediens og opskrift er et 1-forhold. Det betyder altså at der på hver opskrift, er minimum én ingrediens og maksimum mange, og at hver ingrediens hører til præcis én opskrift. Klassediagrammet indeholder et hierarkimønster, hvilket kan ses i forholdet mellem ``Indkøbsliste'', ``Opskrift'', ``Ingrediens'', ``vare'' og ``Råvare''. Hver enkelt klasses relationer og forhold, forklares herunder nærmere:

% Troede altså, at hierarki kun forekommer ved nedarvning? Men okay, jeg ved jo heller ikke noget som helst om noget som helst :b - Martin

\begin{figure}
  \centering
  \input{billeder/klasseDiagram.pdf_tex}
  \capt{Klassediagram for problemområdet.}
  \label{fig:klassediagram}
\end{figure}


\begin{description}
\item[Opskrift] \hfill \\
En opskrift kan være associeret med 0-til-1 bogmærke. Man kan argumentere for, at der godt kan placeres flere bogmærker på samme opskrift, men vi ønsker kun at overvåge hvorvidt en opskrift er bogmærket eller ej, og det er derfor kun nødvendigt at skelne mellem 0 og 1 bogmærke. En opskrift består af 1-til-mange ingredienser. Hvis opskriften ikke bestod af nogle ingredienser, ville det ikke være en opskrift.
    
\item[Bogmærke] \hfill \\
Et bogmærke er associeret med netop én opskrift. Forholdet er valgt på baggrund af, at et bogmærke kan sættes på én og kun én side i en kogebog.

\item[Råvare] \hfill \\
En råvare er en dekomponering af en ingrediens. En råvarer kan være 0-til-mange ingredienser. En råvare som \fx appelsin, kan være både appelsin skræl, appelsin saft, appelsinkød osv. Alle disse ingredienser er lavet ud fra den samme råvare, nemlig en appelsin. 

\item[Ingrediens] \hfill \\
En ingrediens består af netop én råvare og aggregeres altid af kun én opskrift. Selvom flere opskrifter indeholder oksekød, kan mængden være forskellig. Samtidig er det logisk, at hvis to opskrifter begge indeholder en ingrediens \fx 400 g oksekød, så kan den ene opskrift ændres til \textit{500 g oksekød}, uden at den anden opskrift skal ændres. En ingrediens er desuden en aggregering af indkøbsliste. Den kan tilhøre 0-til-1 indkøbsliste.  

\item[Vare] \hfill \\
En vare aggregeres af indkøbsliste. En vare kan tilhøre 0-til-1 indkøbsliste.  
  
\item[Indkøbsliste] \hfill \\
En indkøbslisten består af 0-til-mange ingredienser og 0-til-mange varer. 0 er muligt, da indkøbslisten er tom inden man tilføjer ingredienser og varer til den.

\item[Fejl] \hfill \\
En opskrift kan bestå af 0-til-mange fejl. Hvis opskriften er dårligt fremstillet eller kommer fra en hjemmeside, hvor brugere selv kan tilføje og dele opskrifter, kan der \fx være mange fejl i en enkelt opskrift. Hvis opskriften derimod stammer fra en kilde med revidering før udgivelse, \fx en opskriftsamling eller Arlas Karolines Køkken, vil der højst sandsynligt ikke være nogle fejl.

\end{description}

