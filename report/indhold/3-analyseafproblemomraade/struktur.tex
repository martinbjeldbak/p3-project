\section{Struktur}
\label{sec:struktur}

Sturkturen på vores klasser beskrives ved hjælp af et klassediagram. Med et klassediagram, skabes der et overblik over relationen mellem de forskellige klasser og objekter. \Foodl{} består af 6 klasser, nemlig ``Opskrift'', ``Vare'', ``Råvare'', ``Ingrediens'', ``Fejl'' og ``Person''. Relationen mellem klasserne kan ses i klassediagrammet i \figref{fig:klassediagram}. En relation mellem to klasser markeret med en rumbe-pil, er en aggregerings-struktur, hvilket er en relation som beskrives ved udsagnet ``har-en'' og ``indgår-i''. Eksempelvis, så er råvare en aggregering af ingrediens, hvilket betyder at en ingrediens ``har-en'' råvare, og en råvare ``ingår-i'' og er en dekomponering af en ingrediens. Dette indgår samtidig i et hierarkimønster, idet opskrifter aggregerer ingredienser, som aggregerer en råvare. En relation mellem to klasser markeret med en streg, er en associeringsstruktur, som betyder at de to klasser har en sammenhæng. I vores struktur, associerer klasserne fejl og opskrifter med hinanden. En associering er ikke, ligesom en aggregering, nem at knytte et udsagn til. Men i vores tilfælde kan en associering mellem fejl og opskrift beskrives som et ``hører-til''-forhold. Fejl ``hører-til'' en opskrift, og en opskrift kan muligvis ``høre-til'' fejl. Hver relation har desuden multipliciteter tilknyttet. Dette beskriver forholdet mellem klasserne. Det kan eksempelvis være et et-til-et-forhold, eller et 0-til-mange-forhold. På \figref{fig:klassediagram} ses det, at forholdet mellem klassen opskrift, og klassen ingrediens er et 1-til-mange-forhold (*-tegnet betyder mange), mens forholdet mellem ingrediens og opskrift er et 1-forhold. Det betyder altså at der på hver opskrift er minimum én ingrediens, og at hver ingrediens hører til præcis én opskrift. Hver enkelt klasses relationer, forhold og multiplicitet, forklares herunder nærmere:

\begin{figure}
  \centering
  \input{billeder/klassediagrampo.pdf_tex}
  \capt{Klassediagram for problemområdet.}
  \label{fig:klassediagram}
\end{figure}



\begin{description}

\item[Associering mellem person og vare (indkøbsliste)] \hfill \\
En person associerer 1-til-mange varer, modelleret på baggrund af indkøbslisten. En indkøbsliste kan benyttes af en person til at holde styr på hvilke varer, der skal handles ind.  Det er muligt for en person være relateret til mange vare. Hvert vareobjekt er unikt for hvert personobjekt, dvs. at det altså ikke muligt for personer, at dele varer mellem sig. En indkøbsliste kunne have været en klasse for sig selv, men fordi hver person kun har netop en indkøbsliste, er dette overflødigt.

\item[Associering mellem person og fejl] \hfill \\
Fejl associerer med bruger-klassen, da det er brugeren, der opdager fejl. En bruger kan opdage flere fejl i den samme, eller forskellige opskrifter i løbet af brugerobjektets levetid.

\item[Associering mellem opskrift og fejl] \hfill \\
En opskrift kan bestå af 0-til-mange fejl. Hvis opskriften er dårligt fremstillet eller kommer fra en hjemmeside, hvor brugere selv kan tilføje og dele opskrifter, kan der \fx være mange fejl i en enkelt opskrift. Hvis opskriften derimod stammer fra en kilde med revidering før udgivelse, \fx en opskriftsamling eller Arlas Karolines Køkken, vil der højst sandsynligt ikke være nogle fejl. 

\item[Associering mellem opskrift og person (bogmærke)] \hfill \\
Personer kan associeres med flere opskrifter i et forhold udtrykt som bogmærker. Støder personen på en opskrift, som personen synes er god, og ønsker at gøre den let tilgængelig til en anden gang, så skal det være muligt at bogmærke denne. Et bogmærke er dog ikke udtrykt i modellen som en klasse, da hvert bogmærke ikke har nogen identitet, udover associeringen med en opskrift og en person. Hver opskrift kan derudover også associeres med flere personer, dvs. at flere forskellige personer godt kan have et bogmærke på den samme opskrift. 

\item[Aggregering mellem opskrift og ingrediens] \hfill \\
En opskrift består af 1-til-mange ingredienser. Hvis opskriften ikke bestod af nogle ingredienser, ville det ikke være en opskrift og skulle derfor ikke modelleres. Hver ingrediens tilhører netop en opskrift.

\item[Aggregering mellem ingrediens og råvare] \hfill \\
Hver opskrift består af netop en råvare, mens en råvare kan tilhøre mange forskellige ingredienser. Ingrediensen ``300 g hvedemel'' består \fx af råvaren ``Hvedemel''.

\end{description}

