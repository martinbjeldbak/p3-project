\section{Struktur}
\label{sec:struktur}


\subsection{Struktur}

Der kan opbygges en struktur imellem de forskellige klasser. Dette kan ses i \figref{fig:klassediagram}. 

\begin{figure}
  \centering
  \input{billeder/klasseDiagram.pdf_tex}
  \capt{Klassediagram for problemområdet.}
  \label{fig:klassediagram}
\end{figure}


Klassediagrammet ovenover er bygget op af aggregeringer og associationer imellem klasserne i diagrammet. Disse relationer er ligeledes beskrevet tekstuelt herunder.

\begin{description}
  \item[Bogmærke] \hfill \\
    Et bogmærke peger på netop en opskrift.

  \item[Opskrift] \hfill \\
    En opskrift består af en eller flere ingredienser og kan være tilknyttet nul til mange bogmærker.

\item[Indkøbsliste] \hfill \\
  En indkøbslisten består af flere ingredienser, den kan dog også være tom.

\item[Ingrediens] \hfill \\
  En ingrediens består af netop en råvare og kan tilhøre nul til mange opskrifter. Ligeledes kan en ingrediens tilhøre en indkøbsliste.

\item[Råvare] \hfill \\
  En råvare er en dekomponering af en ingrediens. En råvare er blot en ingrediens uden nogen form for information om mængde eller enhed. En ingrediens kunne for eksempel være "200g oksekød", hvormed den underliggende råvare ville være "oksekød".
\end{description}

