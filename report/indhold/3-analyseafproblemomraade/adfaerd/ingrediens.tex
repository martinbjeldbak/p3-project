\subsection{Ingrediens}

\todo{Revider hele dette afsnit!}
En ingrediens kommer til verdenen sammen med en opskrift. Dette er fordi, at en ingrediens ikke er en del af problemområdet før den optræder i en opskrift. Det er nemlig først der, at man i en husstand behøver at bekymre sig om ingredienser. Hvis de ikke fandtes i nogle opskrifter, var der ikke nogen grund til at indkøbe en råvare svarende til ingrediensen, og ingrediensen ville derved ikke være en del af problemområdet. Denne hændelse kaldes \textit{Opskrift fundet}. Mens ingrediensen ekisister, er den i tilstanden af samme navn. Når man kender til en ingrediens, kan man skrive den på sin indkøbsliste. Dette gør man hvis man vil være sikre på at huske at købe en råvare svarende til ingrediensen, når man er ude at handle ind. Man kan selvfølgelig også fjerne ingrediensen fra sin indkøbsliste, hvis man har ombestemt sig og ikke ønsker at huskes på at købe den tilsvarende råvare. Disse to hændelser kaldes \textit{ingrediens tilføjet} og \textit{ingrediens fjernet}. Når opskriften, der indeholder en given ingrediens, smides ud, er hændelsen \textit{Opskrift smidt ud} netop indtruffet, og ingrediensen bringes i sin sluttilstand. Det bør bemærkes, at 2 forskellige opskrifter, der begge indeholder pasta, anses for at indeholde hver sin ingrediens. Dette er en vigtig adskillelse, da ingredienser består af en mængde og enhed, \fx 200 g pasta.

\begin{figure}[H]
	\centering
	\scalebox{0.6}{
	\begin{figure}[htp]
\centering
\scalebox{0.6}{
\begin{figure}[htp]
\centering
\scalebox{0.6}{
\begin{figure}[htp]
\centering
\scalebox{0.6}{
\input{billeder/tilstandsdiagrammer/ingrediens.pdf_tex}}
\capt{Tilstandsdiagram for Ingrediens-klassens adfærdsmønstre}\label{fig:ingrediens-adfaerd}
\end{figure}}
\capt{Tilstandsdiagram for Ingrediens-klassens adfærdsmønstre}\label{fig:ingrediens-adfaerd}
\end{figure}}
\capt{Tilstandsdiagram for Ingrediens-klassens adfærdsmønstre}\label{fig:ingrediens-adfaerd}
\end{figure}}
	\capt{Tilstandsdiagram for klassen ingrediens. I dette tilfælde har klassen én tilstand, nemlig eksisterer, og en afsluttende hændelse, der fører klassen ud i en sluttilstand.}
	\label{fig:ingrediens-adfaerd}
\end{figure}