\subsection{Råvaretype}
Når nogen køber en råvaretype, indtræffer hændelsen \textit{råvare købt}. Denne hændelse fører råvaretype-klassen i en ny tilstand, der hedder \textit{brugbar}. Denne råvaretype er klar til at blive benyttet til madlavning. Inden der bliver handlet ind, og inden der sker et tilstandsskift, så er råvaretypen i en tidligere tilstand, der hedder \textit{eksisterer}. Der eksisterer ingen starthændelse på klassen, da råvaretyper kan ses som altid eksisterende objekter i problemområdet. Dette tilstandsdiagram har tilsvarende ingen afsluttende hændelse, fordi vi vurderer, at en råvaretype altid vil eksistere, selvom man ikke er i besiddelse af den. Men den er først brugbar, når man har købt den. 

Indehaverne af en råvare har nu to muligheder. Man kan enten bruge hele råvaren, eller smide den ud. Disse to hændelser samler vi  under en fælles hændelse, som vi kalder \textit{råvare opbrugt}. Denne hændelse fører klassen i den forrige tilstand, der hedder \textit{eksisterer}, da den ikke længere er brugbar i husstanden. 

\pdffig[0.8]{tilstandsdiagrammer/raavare}
  {Tilstandsdiagram for klassen råvaretype. Klassen har to tilstande ``eksisterer'' og ``brugbar'. Råvaretype har ikke en igangsættende eller afsluttende hændelse, da der altid vil eksistere en råvaretype, uanset om man har den i sin husstand eller ej.}
  {fig:raavare-adfaerd}
