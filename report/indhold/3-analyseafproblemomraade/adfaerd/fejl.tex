\subsection{Fejl}
Fejl opstår ved, at opskriftsskribenten glemmer eller overser d\'{e}t, at tilføje nogle vigtige ingredienser i en opskrift, eller mangler et trin i fremgangsmåden. Skarpe hjemmekokke kan nemt opdage sådan nogle fejl og ville godt kunne tænke sig, at give deres input til at hjælpe forelaget korrigere de fejl, der måtte eksistere i opskriften. Tilstandsdiagrammet for klassen ``fejl'' ses i \figref{fig:fejl}.

Starthændelsen \textit{fejl fundet} sætter klassen i gang, ved at brugeren opdager en fejl i en opskrift.

Denne klasse er kommet til verden ud fra gruppediskussioner under komponentarkitekturdesignet, da vi mener, at forskellige opskriftshjemmesider har varierende kvalitet af opskrifter. Derfor skal man, som forbrugere, kunne gøre andre opmærksom på diverse forskellige fejl. Så har modtageren mulighed for at kigge på fejlene og eventuelt rette dem, så brugere får en endnu bedre oplevelse.

\pdffig[0.6]{tilstandsdiagrammer/fejl}
  {Tilstandsdiagram for klassen fejl.}
  {fig:fejl}
