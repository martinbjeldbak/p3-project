\subsection{Bogmærke}
Som nævnt kan man vælge at tilføje et bogmærke til en opskrift, man kan lide. Denne hændelse kaldes \textit{bogmærke tilføjet}, og bringer klassen i tilstanden \textit{aktiv}. Bogmærket er aktiv, indtil den kommer i sluttilstanden efter, at hændelsen \textit{bogmærke fjernet} indtræffer. Se \figref{fig:bogmaerke-adfaerd}.

\begin{figure}[H]
	\centering
	\scalebox{0.8}{
	\subsection{Bogmærke}
Som nævnt kan man vælge at tilføje et bogmærke til en opskrift, man kan lide. Denne hændelse kaldes \textit{bogmærke sat ind}, og bringer klassen i tilstanden \textit{aktiv}. Bogmærket er aktiv, indtil den kommer i sluttilstanden efter, at hændelsen \textit{bogmærke fjernet} indtræffer. Se \figref{fig:bogmaerke-adfaerd} for tilstandsdiagrammet over denne klasse.

\pdffig[0.8]{tilstandsdiagrammer/bogmaerke}
  {Tilstandsdiagram for klassen bogmærke. Klassen har én tilstand (aktiv) og en afsluttende hændelse ``bogmærke fjernet'', der fører klassen ud i sluttilstanden.}
  {fig:bogmaerke-adfaerd}
}
	\capt{Tilstandsdiagram for klassen bogmærke. De afrundede rektangulære bokse med tekst, skal anses som tilstande, som klassen kan have. De pile, der fører til en tilstand, skal anses som hændelser, som kan være skyld i et tilstandsskift. I dette tilfælde har klassen én tilstand (aktiv) og en afsluttende hændelse, der fører klassen ud i en sluttilstand (den sorte prik i den sorte cirkel).}
	\label{fig:bogmaerke-adfaerd}
\end{figure}