\subsection{Opskrift}
En opskrift kan befinde sig i en kogebog, på et stykke papir eller på en hjemmesiden. I \figref{fig:opskrift-adfaerd} ses adfærden for klassen ``Opskrift''. En opskrift eksisterer ved hændelsen \textit{opskrift fundet}. Her befinder den sig i tilstanden \textit{findes i opskriftssamlingen}. Hvis man er rigtig glad for en opskrift, kan man sætte et bogmærke på opskriften, så man hurtigt kan finde den igen. Sådan en hændelse kaldes for \textit{bogmærke tilføjet}. Hvis man bestemmer sig for at fjerne bogmærket, indtræffer hændelsen \textit{bogmærke fjernet}. Derudover kan man være nødsaget til at handle ind til en specifik opskrift, fordi man mangler nogle ingredienser. Det betyder, at man skal tilføje nogle opskrifter til en indkøbsliste. Denne hændelse kaldes for \textit{skrevet på indkøbsliste}, hvor der også findes den modsatte hændelse \textit{fjernet fra indkøbsliste}. Der kan også eksistere en fejl i opskriften. En fejl kunne være et manglende billede til opskriften, eller at der står 20 minutter i ovnen i stedet for 40. Dette er beskrevet ved hjælp af hændelsen ``fejl fundet''. Disse fire hændelser medfører ikke et tilstandsskift. Opskriften ryger kun ud af tilstanden \textit{findes i opskriftssamlingen}, når opskriften fjernes vha. hændelsen \textit{opskrift smidt ud}..

\pdffig[0.8]{tilstandsdiagrammer/opskrift}
  {Tilstandsdiagram for klassen opskrift. Klassen har én tilstand (findes i opskriftssamlingen) og en afsluttende hændelse ``opskrift fjernet'', der fører klassen ud i en sluttilstand.}
  {fig:opskrift-adfaerd}
