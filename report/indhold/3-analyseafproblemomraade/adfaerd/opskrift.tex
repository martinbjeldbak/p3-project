\subsection{Opskrift}
En opskrift kan befinde sig i en kogebog eller blot et stykke papir, som man en gang har skrevet ned, fordi man synes om opskriften. Det er ikke relevant, hvor denne opskrift befinder sig, men at man har den i husstanden. Det er ikke sikkert, at man ønsker at bevare alle opskrifter, man har på nuværende tidspunkt. Derfor er det muligt at fjerne opskrifter, man ikke ønsker at gemme. Denne afsluttende hændelse kaldes for \textit{opskrift fjernet}. Denne hændelse fører opskrift-klassen i sin sluttilstand. Se \figref{fig:opskrift-adfaerd}.

\todo{skal vi ikke have en starttilstand?} Dette gøres igennem tilstanden \textit{opskrift fundet}.

Hvis man er rigtig glad for en opskrift, kan man sætte et bogmærke på opskriften, så man hurtigt kan finde den igen. Dette kan \fx gøres ved at sætte en post-it note på en bestemt side i kogebogen. Sådan en hændelse kaldes for \textit{bogmærke tilføjet}. Hvis man ombestemmer sig og fjerner bogmærket, kaldes hændelsen \textit{bogmærke fjernet}. Derudover kan man være nødsaget til at handle ind til en specifik opskrift, fordi man mangler nogle ingredienser. Det betyder, at man skal tilføje nogle opskrifter til en indkøbsliste. Denne hændelse kaldes for \textit{tilføj til indkøbsliste}. Disse tre hændelser medfører ikke et tilstandsskift. Opskriften ryger kun ud af tilstanden \textit{findes i opskriftssamlingen}, når opskriften fjernes.

\begin{figure}[H]
	\centering
	\scalebox{0.8}{
	\subsection{Opskrift}
En opskrift kan befinde sig i en kogebog, på et stykke papir eller på en hjemmesiden. I \figref{fig:opskrift-adfaerd}, ses adfærden for klassen ``Opskrift''. En opskrift kommer til live, ved hændelsen \textit{opskrift fundet}. Her befinder den sig i tilstanden \textit{findes i opskriftssamlingen}. Hvis man er rigtig glad for en opskrift, kan man sætte et bogmærke på opskriften, så man hurtigt kan finde den igen. Dette kan \fx gøres ved at sætte en post-it note på en bestemt side i kogebogen. Sådan en hændelse kaldes for \textit{bogmærke tilføjet}. Hvis man ombestemmer sig og fjerner bogmærket, indtræffer hændelsen \textit{bogmærke fjernet}. Derudover kan man være nødsaget til at handle ind til en specifik opskrift, fordi man mangler nogle ingredienser. Det betyder, at man skal tilføje nogle opskrifter til en indkøbsliste. Denne hændelser kaldes for \textit{skrevet på indkøbsliste} med den modsignende hændelse \textit{fjernet fra indkøbsliste}. Der kan også eksistere en fejl i opskriften, som man muligvis lægger mærke til. En sådan fejl kunne være et manglende billede til opskriften, eller at der står 20 minutter i ovnen i stedet for 40. Dette er beskrevet ved hjælp af hændelsen ``fejl fundet''. Disse fire hændelser medfører ikke et tilstandsskift. Opskriften ryger kun ud af tilstanden \textit{findes i opskriftssamlingen}, når opskriften fjernes, vha. hændelsen \textit{opskrift smidt ud}..

\pdffig[0.8]{tilstandsdiagrammer/opskrift}
  {Tilstandsdiagram for klassen opskrift. Klassen har én tilstand (findes i opskriftssamlingen) og en afsluttende hændelse ``opskrift fjernet'', der fører klassen ud i en sluttilstand.}
  {fig:opskrift-adfaerd}
}
	\capt{Tilstandsdiagram for klassen opskrift. Klassen har én tilstand (findes i opskriftssamlingen) og en afsluttende hændelse ``opskrift fjernet'', der fører klassen ud i en sluttilstand.}
	\label{fig:opskrift-adfaerd}
\end{figure}	