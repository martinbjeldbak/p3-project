\subsection{Indkøbsliste}
En indkøbsliste kommer til verden ved, at man i husstanden beslutter sig for at benytte en eller anden form for huskeliste for ting, der skal handles ind. Dette kan være i form af et papir, der ligger et fast sted på bordet eller hænger på opslagstavlen. Denne initierende hændelse kaldes \textit{indkøbsliste oprettet}. Se \figref{fig:indkoebsliste-adfaerd}.

Så længe indkøbslisten ligger på bordet eller et andet sted, kan mange personer komme forbi og tilføje eller fjerne varer på den. Denne tilstand kaldes for \textit{redigeres}, mens det at tilføje eller fjerne noget fra indkøbslisten er hændelser ved navn \textit{skrevet fra indkøbsliste} og \textit{fjernet fra indkøbsliste}. Personen, der tilføjer tekst til indkøbslistene, er herre over, om der skal stå en ingrediens, der består af en mængde, en enhed og en råvare (\fx 1 ltr skummetmælk) eller der blot skal stå råvaren (\fx skummetmælk).

Indkøbslisten kan også indeholde bemærkninger, såsom ``hvis det er på tilbud''. Denne bemærkning kan selvfølgelig også fjernes igen på samme måde som en vare kan fjernes. Disse bemærkninger tilhører en eller flere varer på indkøbslisten, og derfor er hændelsen den samme som ved tilføjelse eller fjernelse af en vare.

Når en person beslutter sig for at tage indkøbslisten med på indkøb, anses indkøbslisten for at være færdig. Det er nu ikke længere muligt at redigere i indkøbslisten, og den afsluttende hændelse indtræffer, da man nu bortskaffer indkøbslisten. 

Ude i supermarkedet kan man købe mange forskellige råvarer. Vi ønsker ikke at overvåge, hvor meget folk har af en ingrediens, og benytter derfor i stedet objektet råvare, der ikke indeholder nogen mængde eller enhed. Denne beslutning er taget på baggrund af møde 2 med vores informanter.\todo{Argumenter lidt bedre for valget.}

\pdffig[0.8]{tilstandsdiagrammer/indkoebsliste}
  {Tilstandsdiagram for klassen indkøbsliste. Klassen har én tilstand (redigeres) og en afsluttende hændelse ``indkøbsliste smidt ud'', der fører klassen ud i sluttilstanden.}
  {fig:indkoebsliste-adfaerd}
