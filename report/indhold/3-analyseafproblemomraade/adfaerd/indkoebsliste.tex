\subsection{Indkøbsliste}
En indkøbsliste kommer til verdenen ved at man i husstanden beslutter sig for at benytte en indkøbsliste. Dette kan være i form af et papir der ligger et fast sted på bordet eller hænger på opslagstavlen. Denne hændelse kaldes \textit{indkøbsliste oprettet}. Så længe den ligger der på bordet eller et andet sted, kan mange personer komme forbi og tilføje eller fjerne ting på den. Denne tilstand kaldes for \textit{redigeres}. I denne tilstand er det muligt at tilføje ingredienser fra en opskrift man mangler ingrediensen for at lave. Man kan også fjerne en ingrediens ved at slå en streg over den, og give et klart signal om at denne ikke skal købes. Disse to hændelser hedder \textit{ingrediens tilføjet} og \textit{ingrediens fjernet}. Man kan også skrive andre ting på en indkøbsliste end ingredienser. Det kan være en bemærkning, \fx ``hvis det er på tilbud''. Denne bemærkning kan selvfølgelig også fjerne igen på samme måde som en ingrediens. Disse to hændelser hedder \textit{tekst tilføjet} og \textit{tekst fjernet}. Når en person beslutter sig for at tage indkøbslisten med på indkøb, anses indkøbslisten for at være færdig. Denne hændelse kaldes \textit{indkøbsliste færdig}. Hele husstanden har altså ikke mulighed for at redigere denne mere, og indkøbslisten får derfor tilstanden \textit{aktiv}. Ude i supermarkedet kan man købe mange forskellige råvarer. Dette overvåges med hændelsen \textit{råvare købt}. Det bør bemærkes, at indkøbslisten indeholder ingredienser, hvilke består af en råvare, en mængde og en enhed. Vi ønsker ikke at overvåge hvor meget folk har af en ingrediens, og benytter derfor istedet objektet råvare, der ikke indeholder nogen mængde eller enhed. Denne beslutning er taget på baggrund af møde 2 med vores informanter. 
\begin{figure}[htp]
\centering
\scalebox{0.6}{
\subsection{Indkøbsliste}
En indkøbsliste kommer til verden ved, at man i husstanden beslutter sig for at benytte en eller anden form for huskeliste for ting, der skal handles ind. Dette kan være i form af et papir, der ligger et fast sted på bordet eller hænger på opslagstavlen. Denne initierende hændelse kaldes \textit{indkøbsliste oprettet}. Se \figref{fig:indkoebsliste-adfaerd}.

Så længe indkøbslisten ligger på bordet eller et andet sted, kan mange personer komme forbi og tilføje eller fjerne varer på den. Denne tilstand kaldes for \textit{redigeres}. I denne tilstand er det muligt for alle, der kan komme til denne indkøbsliste, at tilføje eller fjerne de varer, der skal handles ind. Man kan \fx fjerne en vare ved at slå en streg over den og give et klart signal om, at denne ikke skal købes. Disse to hændelser hedder \textit{vare fjernet} og \textit{vare tilføjet}. Personen, der tilføjer tekst til indkøbslistene, er herre over, om der skal stå en ingrediens, der består af en mængde, en enhed og en råvare (\fx 1 ltr skummetmælk) eller der blot skal stå råvaren (\fx skummetmælk).

Indkøbslisten kan også indeholde bemærkninger, såsom ``hvis det er på tilbud''. Denne bemærkning kan selvfølgelig også fjernes igen på samme måde som en vare kan fjernes. Disse bemærkninger tilhører en eller flere varer på indkøbslisten, og derfor er hændelsen den samme som ved tilføjelse eller fjernelse af en vare.

Når en person beslutter sig for at tage indkøbslisten med på indkøb, anses indkøbslisten for at være færdig. Det er nu ikke længere muligt at redigere i indkøbslisten, og den afsluttende hændelse indtræffer. 

Ude i supermarkedet kan man købe mange forskellige råvarer. Vi ønsker ikke at overvåge, hvor meget folk har af en ingrediens, og benytter derfor istedet objektet råvare, der ikke indeholder nogen mængde eller enhed. Denne beslutning er taget på baggrund af møde 2 med vores informanter.\todo{Argumenter lidt bedre for valget.}

\begin{figure}[H]
	\centering
	\scalebox{0.8}{
		\subsection{Indkøbsliste}
En indkøbsliste kommer til verden ved, at man i husstanden beslutter sig for at benytte en eller anden form for huskeliste for ting, der skal handles ind. Dette kan være i form af et papir, der ligger et fast sted på bordet eller hænger på opslagstavlen. Denne initierende hændelse kaldes \textit{indkøbsliste oprettet}. Se \figref{fig:indkoebsliste-adfaerd}.

Så længe indkøbslisten ligger på bordet eller et andet sted, kan mange personer komme forbi og tilføje eller fjerne varer på den. Denne tilstand kaldes for \textit{redigeres}. I denne tilstand er det muligt for alle, der kan komme til denne indkøbsliste, at tilføje eller fjerne de varer, der skal handles ind. Man kan \fx fjerne en vare ved at slå en streg over den og give et klart signal om, at denne ikke skal købes. Disse to hændelser hedder \textit{vare fjernet} og \textit{vare tilføjet}. Personen, der tilføjer tekst til indkøbslistene, er herre over, om der skal stå en ingrediens, der består af en mængde, en enhed og en råvare (\fx 1 ltr skummetmælk) eller der blot skal stå råvaren (\fx skummetmælk).

Indkøbslisten kan også indeholde bemærkninger, såsom ``hvis det er på tilbud''. Denne bemærkning kan selvfølgelig også fjernes igen på samme måde som en vare kan fjernes. Disse bemærkninger tilhører en eller flere varer på indkøbslisten, og derfor er hændelsen den samme som ved tilføjelse eller fjernelse af en vare.

Når en person beslutter sig for at tage indkøbslisten med på indkøb, anses indkøbslisten for at være færdig. Det er nu ikke længere muligt at redigere i indkøbslisten, og den afsluttende hændelse indtræffer. 

Ude i supermarkedet kan man købe mange forskellige råvarer. Vi ønsker ikke at overvåge, hvor meget folk har af en ingrediens, og benytter derfor istedet objektet råvare, der ikke indeholder nogen mængde eller enhed. Denne beslutning er taget på baggrund af møde 2 med vores informanter.\todo{Argumenter lidt bedre for valget.}

\begin{figure}[H]
	\centering
	\scalebox{0.8}{
		\subsection{Indkøbsliste}
En indkøbsliste kommer til verden ved, at man i husstanden beslutter sig for at benytte en eller anden form for huskeliste for ting, der skal handles ind. Dette kan være i form af et papir, der ligger et fast sted på bordet eller hænger på opslagstavlen. Denne initierende hændelse kaldes \textit{indkøbsliste oprettet}. Se \figref{fig:indkoebsliste-adfaerd}.

Så længe indkøbslisten ligger på bordet eller et andet sted, kan mange personer komme forbi og tilføje eller fjerne varer på den. Denne tilstand kaldes for \textit{redigeres}. I denne tilstand er det muligt for alle, der kan komme til denne indkøbsliste, at tilføje eller fjerne de varer, der skal handles ind. Man kan \fx fjerne en vare ved at slå en streg over den og give et klart signal om, at denne ikke skal købes. Disse to hændelser hedder \textit{vare fjernet} og \textit{vare tilføjet}. Personen, der tilføjer tekst til indkøbslistene, er herre over, om der skal stå en ingrediens, der består af en mængde, en enhed og en råvare (\fx 1 ltr skummetmælk) eller der blot skal stå råvaren (\fx skummetmælk).

Indkøbslisten kan også indeholde bemærkninger, såsom ``hvis det er på tilbud''. Denne bemærkning kan selvfølgelig også fjernes igen på samme måde som en vare kan fjernes. Disse bemærkninger tilhører en eller flere varer på indkøbslisten, og derfor er hændelsen den samme som ved tilføjelse eller fjernelse af en vare.

Når en person beslutter sig for at tage indkøbslisten med på indkøb, anses indkøbslisten for at være færdig. Det er nu ikke længere muligt at redigere i indkøbslisten, og den afsluttende hændelse indtræffer. 

Ude i supermarkedet kan man købe mange forskellige råvarer. Vi ønsker ikke at overvåge, hvor meget folk har af en ingrediens, og benytter derfor istedet objektet råvare, der ikke indeholder nogen mængde eller enhed. Denne beslutning er taget på baggrund af møde 2 med vores informanter.\todo{Argumenter lidt bedre for valget.}

\begin{figure}[H]
	\centering
	\scalebox{0.8}{
		\input{billeder/tilstandsdiagrammer/indkoebsliste.pdf_tex}}
		\capt{Tilstandsdiagram for klassen indkøbsliste. De afrundede rektangulære bokse med tekst, skal anses som tilstande, som klassen kan have. De pile, der fører til en tilstand, skal anses som hændelser, som kan være skyld i et tilstandsskift. I dette tilfælde har klassen én tilstand (redigeres) og en afsluttende hændelse, der fører klassen ud i en sluttilstand (den sorte prik i den sorte cirkel).}
		\label{fig:indkoebsliste-adfaerd}
\end{figure}}
		\capt{Tilstandsdiagram for klassen indkøbsliste. De afrundede rektangulære bokse med tekst, skal anses som tilstande, som klassen kan have. De pile, der fører til en tilstand, skal anses som hændelser, som kan være skyld i et tilstandsskift. I dette tilfælde har klassen én tilstand (redigeres) og en afsluttende hændelse, der fører klassen ud i en sluttilstand (den sorte prik i den sorte cirkel).}
		\label{fig:indkoebsliste-adfaerd}
\end{figure}}
		\capt{Tilstandsdiagram for klassen indkøbsliste. De afrundede rektangulære bokse med tekst, skal anses som tilstande, som klassen kan have. De pile, der fører til en tilstand, skal anses som hændelser, som kan være skyld i et tilstandsskift. I dette tilfælde har klassen én tilstand (redigeres) og en afsluttende hændelse, der fører klassen ud i en sluttilstand (den sorte prik i den sorte cirkel).}
		\label{fig:indkoebsliste-adfaerd}
\end{figure}}
\capt{Tilstandsdiagram for Indkøbsliste-klassens adfærdsmønstre}\label{fig:indkoebsliste-adfaerd}
\end{figure}