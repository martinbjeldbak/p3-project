\section{Klasser}
\label{sec:klasser}

En klasse er en beskrivelse af en samling af objekter med samme struktur, adfærdsmønster og attributter \cite[s. ~51]{ooad}. Som nævnt i indledningen til kapitlet, er et af målene med analysen af problemområdet, at skabe et overblik over hvilke klasser og hændelser systemet skal have. Hændelser vil vi dog først komme nærmere ind på, i kommende section \ref{sec:haendelser}, men overblikket over hvilke klasser systemet skal have, kan skabes ved først at finde så mange klasserkandidater, som muligt, og dernæst at afgrænse disse, så kun de mest relevante klasser er tilbage. I vores tilfælde anvendte vi rigebilleder, som kan ses i \ref{ap:rigebilleder}, samt systemdefinitionen, til at hente inspiration til de klasser, som vi vil modellere i problemområdet.

Grundet den iterative arbejdsproces, er klasser undervejs blevet tilføjet og fjerent, og vi ønsker nu at kaste lys over baggrunden bag de valgte klasser. I gennem processen har vi også fravalgt klasser. Disse, med beskrivelse, kan findes i \apref{ap:fravalgteklasser}.

Herunder ses de valgte klasser og hvorfor vi vælger at have dem med i vores model af problemområdet.
Vi mener at disse klasser samler de objekter og hændelser, som er relevant for denne model af problemområdet.

\begin{description}
\item[Ingrediens] \hfill \\ 
I opskrifter bruges der flere ingredienser. En ingrediens består af en råvare og en mængde af denne. Det er et problem at finde opskrifter, der indeholder ingredienser svarende til de råvarer man har til rådig. Det er et problem for informanterne at maden laves i for store portioner, altså er det et problem hvis en opskrifts ingredienser indeholder store mængder.

\item[Bogmærke] \hfill \\
Det er en del af problemområdet, for brugere at huske de gode opskrifter. Der vil til tider blive benyttet en opskrift, der er så god, at den er værd at gemme til en anden gang. Derfor beholder vi denne klasse.

\item[Råvare] \hfill \\
En råvare findes i køleskabene og på madhylderne i husholdningerne. Det er et problem at finde opskrifter, der kun indeholder disse råvarer, derfor skelnes der mellem ingredieser og råvarer.

\item[Indkøbsliste] \hfill \\
Vi vurderer, at der i en husholdning ofte bliver skrevet en indkøbsliste med de ting man mangler. Indkøbslisten kan være skrevet på baggrund af en opskrift man gerne vil lave, eller en hel madplan man gerne vil følge over en længere periode.

\item[Opskrift] \hfill \\
En opskrift er det centrale i problemområdet. Opskrifterne indeholder forskellige ingredienser. Det er nødvendigt at have råvarer nok til at matche ingredienserne i opskriften, før denne kan laves. Man må gå ud fra at den typiske private madlaver har mange opskrifter i kogebogen, som han/hun reelt ikke har råvarerne til at kunne lave.
\end{description}

\subsection{Valgte hændelser}
\label{sec:haendelser}
Ved hjælp af klassekandidaterne, er det nu muligt at finde hændelseskandidater. Hændelserne er kommet til verden ud fra diverse forskellige forløb, der kan påvirke klasser. For at opretholde en konsistens mellem hændelserne, er de formuleret i datid. Det er igen vigtigt at nævne, at følgende er \emph{kandidater} og kan ændres under den iterative arbejdsproces. 

\subsection{Valgte hændelser}
Følgende hændelser er blevet skabt ud fra de valgte klassekandidater, som gruppen kom frem til i \secref{sec:klasser}. Klassekandidaterne er karaktiseret ved hjælp af følgende hændelseskandidater: 

\begin{itemize} [noitemsep]
\item Råvare opbrugt
\item Råvare smidt ud
\item Råvare købt
\item Opskrift fundet
\item Opskrift valgt
\item Opskrift smidt ud
\item Bogmærke tilføjet
\item Bogmærke fjernet
\item Indkøbsliste oprettet
\item Indkøbsliste færdig
\item Indkøbsliste smidt ud
\item Ingrediens tilføjet
\item Ingrediens fjernet
\item Tekst tilføjet
\item Tekst fjernet
\end{itemize}

\subsection{Hændelsestabel}
Når de valgte klasser og hændelser er kommet på plads, giver det mulighed at fremstille et hændelsestabel, der danner overblik over sammenhæng mellem klasser og fælles hændelser. Herunder ses hændelsestabellen:

\ourtable{haendelsestabel}{5}{Hændelsestabel for klasserne person, opskrift, ingrediens, råvaretype, vare og fejl}
                                                             {Klasser}
       {Hændelser             	}{ Opskrift & Person & Vare  & Ingrediens & Råvaretype & Fejl  }{
\ourrow{Opskrift smidt ud     	}{ \once    &        &       & \once      &            &       }
\ourrow{Opskrift fundet         }{ \once    &        &       & \once      &            &       }
\ourrow{Bogmærke sat ind      	}{ \iter    & \iter  &       &            &            &       }
\ourrow{Bogmærke fjernet      	}{ \iter    & \iter  &       &            &            &       }
\ourrow{Skrevet på indkøbsliste	}{ \iter    & \iter  & \once & \iter      &            &       }
\ourrow{Fjernet fra indkøbsliste}{          & \iter  & \once & \iter      &            &       }
\ourrow{Råvare opbrugt        	}{          &        &       &            & \iter      &       }
\ourrow{Råvare købt           	}{          &        &       &            & \iter      &       }
\ourrow{Fejl rapport\'{e}ret    }{ \iter    & \iter  &       &            &            & \once }
\ourrow{Tilbagemelding modtaget }{          & \once  &       &            &            & \once }
\ourrow{Fejl set bort fra       }{          &        &       &            &            & \once }
}



