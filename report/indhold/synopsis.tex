Denne rapport gør rede for udviklingsprocessen bag produktionen af systemet \Foodl{}, der er en webapplikation. 
Undersøgelser viser, at personer, der bor i parcelhuse, smider i gennemsnit 42 kg mad ud om året.
\Foodl{} har til formål at gøre det lettere for de madansvarlige i de danske husstande at genbruge madrester fra bl.a. gårsdagens aftensmad for at mindske spildet af mad.

Igennem en iterativ udviklingsmetode har vi analyseret, designet, implementeret og testet systemet.
Vi benyttede en objektorienteret metode, der er blevet præsenteret i bogen Objektorienteret Analyse \& Design \cite{ooad}.
Til udvikling af systemet havde vi et tæt samarbejde med to informanter, der hjalp os med at fortolke og revidere problemstillingen. Informanterne var en stor del af analysen, designet og kvalitetssikringen af systemet.

I konklusionen har vi konkluderet, at \Foodl{} gør det lettere for de madansvarlige at få inspiration til at genbruge madrester.
Vi har, i perspektiveringen, reflekteret over, hvad der kan gøre systemet endnu bedre.