\subsection{Valg af systemdefinition}
\label{subsec:valgafsystemdefinition}

%SYSTEMVALG
I fællesskab med informanterne, valgte vi Systemdefinition 1, det online opskriftregister. Denne beslutning satte nogle fastere rammer for projektets videre forløb. Vi ønskede at høre om informanterne ville være i stand til bruge lignende systemer. Vi holdte os meget åbne for nye løsningsforslag, da vi gerne ville gøre det muligt for informanterne at komme med nye idéer, også selvom de var markant anderledes end vores systemdefinitioner. Derfor foregik møderne som semistrukturerede interviews. Det var vigtigt for os at få informanternes idéer til hvilke funktioner et sådan system skulle have og hvilke krav, de stiller.

Der blev taget højde for informanternes respons (Al dokumentation og alle referater fra de afholdte møder med informanterne kan findes i \apref{ap:informant}.) på de udarbejdede systemdefinitioner, og der blev truffet et valg, som alle parter kan se som en mulig løsning på problemerne mht. madlavning og madspild i danske husstande. Efter møderne var det helt klart, at informanterne så det online opskriftsregister (S1), der kan ses i \secref{subsec:alternativesystemdefinitioner}, som et meget brugbart system. Det mente simpelthen ikke, at et planlægningsværktøj var noget, de ville komme til at bruge. De var meget mere interesserede i at have et værktøj, der gjorde det muligt for dem at slå madrester op i et system, der ville være i stand til at give opskriter, der indeholder de indtastede ingredienser.

%AFSLUTNING
Systemdefinitionen er blevet valgt. Det er nu på tide at overveje, hvordan systemet skal fungere, hvordan det skal designes, og hvilke funktioner, der skal inkluderes i systemet.
