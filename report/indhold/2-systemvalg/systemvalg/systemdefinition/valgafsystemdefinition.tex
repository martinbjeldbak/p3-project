\subsection{Valg af systemdefinition}
\label{subsec:valgafsystemdefinition}

%SYSTEMVALG
Efter at have fremstillet de to systemdefinitioner præsenterede vi disse for informanter for at få feedback på projektets retning og få valgt en attraktivt systemdefinition. Formålet med mødet er at fortælle informanterne om gruppens initierende idé om en opskriftssøgemaskine, der kun finder opskrifter man kan lave ud fra de råvarer, man har til rådighed. Alt efter hvilken systemdefinition, som vi vælger i fællesskab med informanterne, så vil projektet tage en specifik retning. Vi vil høre om informanterne ville være i stand til bruge et sådan system. Vi holder os meget åbne for nye løsningsforslag, da vi gerne vil gøre det muligt for informanterne at komme med nye idéer, også selvom de er markant anderledes fra vores initierende problemstilling og systemdefinitioner. Derfor foregår møderne som et semistruktureret interview. Det er vigtigt for os at få informanternes idéer til hvilke funktioner et sådan system skal have og hvilke krav, de stiller.

Der er blevet taget højde for informanternes respons (Al dokumentation og alle referater fra de afholdte møder med informanterne kan findes i \todo{referer til bilag}.) på de udarbejdede systemdefinitioner, og der er nu blevet truffet et valg, som alle parter kan se som en mulig løsning på problemerne mht. madlavning og madspild i danske husstande. Efter møderne var det helt klart, at informanterne ser S1 som et meget brugbart system. 

Den valgte systemdefinition S1, som skal fungere som et online opskriftsregister, er defineret på følgende måde:

\begin{quote}
Systemet skal fungere som et online opskriftsregister, der giver brugeren idéer til opskrifter som kan lave ud fra de madvarer brugeren har. Systemet fokuserer på at mindske madspild, da forbrugere smider mad ud på grund af et manglende formål med anvendelsen af resterne. Brugerne af systemet er en del af en husholdning og vil have meget varierende erfaringer inden for brug af internettet. Udviklerne af systemet er ulønnede studerende. Deadline for det færdige system kan ikke ændres. Systemet skal køre på en server, der kan tilgås via en webapplikation fra en internetbrowser på enhver type computer. På baggrund af en mængde fødevarer som input, findes forskellige opskrifter, der bedst muligt matcher disse fødevarer. Opskrifterne skal kunne sorteres på flere måder, og ingredienser skal kunne huskes til næste gang, hvis ønsket.
\end{quote}

I danske husstande smides der ofte mad væk. En del af madspildet kan \fx ske efter endt madlavning, hvor man kan risikere at stå tilbage med nogle fødevarer, der helst skal bruges hurtigst muligt for ikke at blive for gamle. Man kunne \fx have lavet hele kyllingbryst med persille, og fået en masse af begge dele tilovers. Næste aften har man ikke lyst til at få samme ret igen, men maden kan ikke holde sig meget længere, så man fristes derfor til at smide maden væk. Dette problem forsøger vi at løse ved at udvikle et system, der stemmer overens med den valgte systemdefinition. Systemet vælger vi at kalde Foodl.

%AFSLUTNING
Der er nu blevet valgt en retning, som projektet skal følge. Systemdefinitionen skal fungere som retningslinjer, som systemet skal overholde. Det er nu på tide at begynde at overveje, hvordan systemet skal fungere og hvilke funktioner, der skal inkluderes i systemet.
