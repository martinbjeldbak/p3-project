\subsection{Alternative systemdefinitioner}
\label{subsec:alternativesystemdefinitioner}

%SYSTEMDEFINITION
Vi starter med at definere selve systemdefinitionerne og derefter sikrer vi os, at systemdefinitionerne stemmer overens med BATOFF-kriterier. Når definitionerne er blevet udarbejdet og kontrolleret, så skal de præsenteres for informanter. 

Efter møder med vores informanter, har vi fået forståelse for at madspild er et reelt problem for de to familier. Det er blevet forklaret, hvad der ofte er grunden til madspildet, og på baggrund af møderne og det rige billede, der kan ses i \figref{fig:rigbillede1}, er to systemdefinitioner blevet udarbejdet. Informanterne er blevet præsenteret med hele systemdefinitionerne, da de skulle træffe et valg. Den valgte systemdefinition bliver vist og forklaret i \secref{subsec:valgafsystemdefinition}.

\begin{description}
\item[Systemdefinition 1 (S1)] 
Systemet skal fungere som et online opskriftsregister, der giver brugeren idéer til opskrifter som kan lave ud fra de madvarer brugeren har. Systemet fokuserer på at mindske madspild, da forbrugere smider mad ud på grund af et manglende formål med anvendelsen af resterne. Brugerne af systemet er en del af en husholdning og vil have meget varierende erfaringer inden for brug af internettet. Udviklerne af systemet er ulønnede studerende. Deadline for det færdige system kan ikke ændres. Systemet skal køre på en server, der kan tilgås via en webapplikation fra en internetbrowser på enhver type computer. På baggrund af en mængde fødevarer som input, findes forskellige opskrifter, der bedst muligt matcher disse fødevarer. Opskrifterne skal kunne sorteres på flere måder, og ingredienser skal kunne tilføjes til en indkøbsliste. Ligeledes skal man kunne gemme favoritopskrifter til senere brug.
\item[Systemdefinition 2 (S2)] 
Systemet fungerer som et planlægningsværktøj, som har til formål at planlægge madlavningen over en given tidsperiode (\fx dage, uger, måneder) ud fra de opskrifter, der er blevet lavet over de sidste par dage. Formålet med planlægningsværktøjet er at sikre, at brugeren får en sund og varieret kost igennem udvalgte opskrifter, hvor der også tages højde for ingrediensers vitaminindhold. Brugerne af programmet vil være husstande, der har varierende erfaringer inden for brug af internettet. Udviklerne af systemet er ulønnede studerende. Systemet skal køre på en server, der kan tilgås via en webapplikation fra en internetbrowser på enhver type computer.
\end{description}

Efter udarbejdelsen af systemdefinitionerne har vi undersøgt, om de overholder BATOFF-kriterierne. Dette har vi gjort ved at skrive de forskellige kriterier ind i \tableref{table:batoff}, hvor vi har indtastet de relevante forklaringer for de seks kriterier. 

\ourtable{batoff}{2}{Tabel med forklaringer over BATOFF-kriterier for systemdefinitionerne S1 og S2. Systemdefinitionerne vil blive refereret til som S1 og S2 fremover.}
                                                 {Systemdefinitioner}
       {Kriterier        }{ Online opskriftsregister (S1) & Madplanlægger (S2)                   }{
\ourrow{Betingelser      }{ Ulønnede udviklere.           & Ulønnede udviklere.                  }
\ourrow{Avendelsesområde }{ Dele af husholdninger,        & Dele af husholdninger,               }
\ourrow{                 }{ \fx forældre eller lignende.  & \fx forældre eller lignende.         }
\ourrow{Teknologier      }{ Internetforbindelse.          & Internetforbindelse.                 }
\ourrow{                 }{ PC. Tablet. Mobiltelefon.     & PC. Tablet. Mobiltelefon.            }
\ourrow{Objekter         }{ Opskrifter. Ingredienser.     & Opskrifter. Ingredienser. Vitaminer. }
\ourrow{Funktioner       }{ Søgningsværktøj. Find         & Planlægningsværktøj. Planlægge       }
\ourrow{                 }{ opskrifter indeholdende       & madlavning ud fra                    }
\ourrow{                 }{ valgte ingredienser           & tidligere dages retter.              }
\ourrow{Filosofi         }{ Folk smider mad ud, fordi de  & Folk spiser ikke varieret nok,       }
\ourrow{                 }{ mangler et sted at            & hvilket er en trussel                }
\ourrow{                 }{ bruge deres rester.           & mod folkesundheden.                  }
}

Tabel \ref{table:batoff} giver et overblik over systemdefinitionerne. Overblikket gør det lettere for informanterne at træffe et kvalificeret valg mht. hvilken definition, der ville løse problemet med madspild.
