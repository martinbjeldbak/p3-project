\subsection{Alternative systemdefinitioner}
\label{subsec:alternativesystemdefinitioner}

%SYSTEMDEFINITION
Vi starter med at definere selve systemdefinitionerne og derefter sikrer vi os, at systemdefinitionerne stemmer overens med BATOFF-kriterier. Når definitionerne er blevet udarbejdet og kontrolleret, så skal de præsenteres for informanter. 

Efter møder med vores informanter, har vi fået forståelse for at madspild er et reelt problem for de to familier. Det er blevet forklaret, hvad der ofte er grunden til madspildet, og på baggrund af møderne er to systemdefinitioner blevet konstrueret. Vi vælger at forklare systemdefinitionernes grundlæggende idé i grove træk, så læseren ikke behøver at læse en masse unødvendig tekst igennem. De fravalgte systemdefinition kan læses i \todo{referer til S2 i bilag}. Informanterne er blevet præsenteret med hele systemdefinitionerne, da de skulle træffe et valg. Den valgte systemdefinition bliver vist og forklaret herefter.

\begin{description}
\item[Systemdefinition 1 (S1)] bygger på en idé om, at en bruger af systemet kan indtaste ingredienser, som brugeren ønsker at bruge i madlavningen. Systemet skal give forslag til hvilke retter, der kan laves ud af disse ingredienser.
\item[Systemdefinition 2 (S2)] bygger på en idé om, at systemet skal være i stand til at udarbejde en madplan til brugeren. Systemet skal gøre dette ved at holde styr på, hvad der er blevet lavet til aftensmad de foregående dage og bruge rester fra disse dage.
\end{description}

Efter udarbejdelsen af systemdefinitionerne har vi undersøgt, om de overholder BATOFF-kriterierne. Dette har vi gjort ved at skrive de forskellige kriterier ind i \tableref{table:batoff}, hvor vi har indtastet de relevante forklaringer for de seks kriterier. På denne måde har vi skabt os et overblik over kriterierne, og de kan meget lettere sammenlignes. Dette er en fordel, da vi mindsker det arbejde, der skal til for at få et overblik over systemdefinitionerne, og dermed gjort det lettere for informanterne at tage et kvalificeret valg mht. hvilken definition, der ville løse problemet med madspild i husstanden.

\ourtable{batoff}{2}{Tabel med forklaringer over BATOFF-kriterier for systemdefinitionerne S1 og S2. Systemdefinitionerne vil blive refereret til som S1 og S2 fremover.}
                                                 {Systemdefinitioner}
       {Kriterier        }{ Online opskriftsregister (S1) & Madplanlægger (S2)                   }{
\ourrow{Betingelser      }{ Ulønnede udviklere.           & Ulønnede udviklere.                  }
\ourrow{Avendelsesområde }{ Dele af husholdninger,        & Dele af husholdninger,               }
\ourrow{                 }{ \fx forældre eller lignende.  & \fx forældre eller lignende.         }
\ourrow{Teknologier      }{ Internetforbindelse.          & Internetforbindelse.                 }
\ourrow{                 }{ PC. Tablet. Mobiltelefon.     & PC. Tablet. Mobiltelefon.            }
\ourrow{Objekter         }{ Opskrifter. Ingredienser.     & Opskrifter. Ingredienser. Vitaminer. }
\ourrow{Funktioner       }{ Søgningsværktøj. Find         & Planlægningsværktøj. Planlægge       }
\ourrow{                 }{ opskrifter indeholdende       & madlavning ud fra                    }
\ourrow{                 }{ valgte ingredienser           & tidligere dages retter.              }
\ourrow{Filosofi         }{ Folk smider mad ud, fordi de  & Folk spiser ikke varieret nok,       }
\ourrow{                 }{ mangler et sted at            & hvilket er en trussel                }
\ourrow{                 }{ bruge deres rester.           & mod folkesundheden.                  }
}