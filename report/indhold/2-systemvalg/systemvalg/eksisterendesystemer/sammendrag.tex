\subsection{Sammendrag}
\label{subsec:eksisterende.sammendrag}

De tre systemer; For Resten, DK-Kogebogen og Opskrifter.dk er i foregående afsnit blevet undersøgt og analyseret. Der blev lagt vægt på fire hovedpunkter: antallet af opskrifter i systemet, kvaliteten af opskrifterne, systemets fleksibilitet og opskriftssøgningsfunktionen (også kaldet ``Tøm køleskabet''-funktionen). De fire hovedpunkter er specificeret i detaljer i \secref{sec:eksisterendesystemer}. For at skabe et samlet overblik, har vi valgt at samle de vigtigste og mest karakteristiske dele fra hver af systemerne i \tableref{table:sammentabel}.

\ourtable{sammentabel}{3}{Oversigt over opfyldelse af kriterier af de eksisterende systemer.}
                                            {System}
       { Funktioner                }{ For Resten   & DK-kogebogen   & Opskrifter.dk }{
\ourrow{ Kvalitet af opskrifter    }{ ringe        & svingende      & god           }
\ourrow{ Antal opskrifter          }{ 550          & 36.500         & 2.700         }
\ourrow{ Fleksibilitet             }{ meget ringe  & middel         & god           }
\ourrow{ Opskriftssøgningsfunktion }{ meget ringe  & middel         & middel        }
}

\begin{description}
\item[Kvalitet af opskrifter] \hfill \\
Kvaliteten af opskrifterne på DK-Kogebogen er meget varierende. Brugeren risikerer at støde på opskrifter, som er ubrugelige. 

Kvaliteten af opskrifterne på For Resten kunne være meget bedre. Opskrifterne er udelukkende lavet eller tilføjet af folkene bag app’en, og derfor er opskrifternes opbygning og design konsistent, hvilket naturligvis havde været en god egenskab, hvis det ikke var for det faktum, at opbygningen er uoverskuelig, og at der ingen billeder eksisterer af opskriften. Der er ingen ingrediensliste på opskriften og beskrivelsen af fremgangsmåden er kortfattet. 

Kvaliteten af opskrifterne på Opskrifter.dk er høj. Dette skyldes, at opskrifterne bliver gennemgået af en administrator, inden de bliver tilgængelige på Opskrifter.dk’s side, hvilket er modsat af DK-Kogebogen, hvor opskrifterne bliver tilgængelige med det samme. Desuden er Opskrifter.dk’s opskriftsopbygning konsekvent i alle opskrifter, hvilket også er i modsætning til DK-Kogebogens opskrifter. Der mangler dog billeder på nogle opskrifter.

\item[Antal opskrifter] \hfill \\
Som det ses i \tableref{table:sammentabel}, har DK-Kogebogen det langt største antal opskrifter, mens Opskrifter.dk har ca. fem gange flere opskrifter end For Resten. Dermed har DK-Kogebogens ``Tøm køleskabet''-funktion også langt bedre chance for at give brugeren et resultat, når der søges på opskrifter med specifikke ingredienser.

\item[Fleksibilitet] \hfill \\
Der er stor forskel på fleksibiliteten fra system til system. I For Restens app, er det slet ikke muligt at skalere portionsstørrelse. På DK-Kogebogens side er det kun muligt med nogle opskrifter, mens det på Opskrifter.dk er muligt at skalere portionsstørrelse i alle opskrifter. Denne relativ simpel funktion ses som meget brugbar, da brugeren undgår selv at skulle gange ingrediensmængder op.

Kun Opskrifter.dk tilbyder brugeren muligheden for at sortere i resultaterne af en opskriftssøgning. Her kan der sorteres efter alfabetisk orden, opskrifter med billeder, opskrifter med kød samt flere. Som en bruger på Opskrifter.dk dog pointere, mangler den sorteringsmulighed, som sortere efter de opskrifter som indeholder flest af de ingredienser som brugeren har indtastet. Denne sorteringsmulighed anses for os, som værende den mest relevante, da man som bruger er interesseret i at få anvendt så mange af ens madrester som muligt.

\item[Opskriftssøgningsfunktion] \hfill \\
De tre systemer er vidt forskellige i deres måde at håndtere søgning på. Der er mellem DK-kogebogen og Opskrifter.dk en markant forskel på, hvordan resultater findes. I DK-Kogebogen findes kun opskrifter, som inkluderer alle de indtastede ingredienser, mens Opskrifter.dk finder alle opskrifter, som indeholder bare én af de valgte ingredienser. Dvs., at man med DK-Kogebogen får færre resultater jo flere ingredienser man skriver, mens det med Opskrifter.dk er direkte modsat, idet antallet af resultater stiger voldsomt med antallet af ingredienser, man skriver. Opskrifter.dk’s måde at gøre det på, kombineret med deres manglende sortering, giver en uoverskuelig mængde af resultater, hvor en stor del af disse måske kun indeholder en af de valgte ingredienser.
\end{description}

%Ud fra vores observationer af løsningernes fordele og ulemper, kan vi uddrage hvilke egenskaber vi ønsker at benytte i vores eget projekt. \Fx viser den generelle utilfredshed med Forbrugerstyrelsens mobilapp For Resten, at det er vigtigt med mange opskrifter og muligheden for at vælge mere end én rest. Observationerne af For Resten og Opskrifter.dk viser også, at brugergrænseflade er et vigtigt element. I disse to løsninger skal man vælge ingredienser ved at lede rundt i kategorier. Dette føles meget ineffektivt i forhold til at skrive navnet på ingrediensen på et tastatur.

%Af dette kan man aflede, at en kombination af de to må være den optimale løsning. Har man valgt få ingredienser, er man sandsynligvis interesseret i at få vist resultater som indeholder alle de ingredienser man har valgt. Har man derimod valgt mange ingredienser, er man interesseret i at få vist resultater som indeholder flest muligt af de ingredienser man har valgt.

%Som det ses i \tableref{table:sammentabel}, har DK-Kogebogen det langt største antal opskrifter, mens Opskrifter.dk har ca. fem gange flere opskrifter end For Resten. Dermed har DK-Kogebogens ``Tøm køleskabet''-funktion også langt bedre chance for at give brugeren et resultat, når han/hun søger på opskrifter med specifikke ingredienser. Til gengæld er kvaliteten af opskrifterne på DK-Kogebogen meget varierende, og derfor kan brugeren risikere at støde på opskrifter, som er dårlige eller ubrugelige. Kvaliteten af opskrifterne på For Resten er sammenlagt dårlig. Opskrifterne er udelukkende lavet eller tilføjet af folkene bag app’en, og derfor er opskrifternes opbygning og design konsistent, hvilket naturligvis havde været en god egenskab, hvis det ikke var for det faktum, at opbygningen er uoverskuelig. Der er ingen ingrediensliste på opskriften og beskrivelsen af fremgangsmåden er kortfattet. Kvaliteten af opskrifterne på Opskrifter.dk er høj. Dette skyldes, at opskrifterne bliver gennemgået af en administrator, inden de bliver tilgængelige på Opskrifter.dk’s side, hvilket er modsat af DK-Kogebogen, hvor opskrifterne bliver tilgængelige med det samme. Desuden er Opskrifter.dk’s opskriftopbygning konsekvent i alle opskrifter, hvilket også er i modsætning til DK-Kogebogens opskrifter.

%Der er stor forskel på fleksibiliteten fra system til system. I For Restens app, er det slet ikke muligt at skalere portionsstørrelse. På DK-Kogebogens side er det kun muligt med nogle opskrifter, mens det på Opskrifter.dk er muligt at skalere portionsstørrelse alle opskrifter. Det er en funktion, som er meget brugbar, da man som bruger ikke ønsker at bruge en masse tid på selv at beregne en passende portionsstørrelse. Af sorteringsmuligheder af opskriftresultaterne, er det kun Opskrifter.dk som tilbyder denne mulighed. Her kan der sorteres efter alfabetisk orden, opskrifter med billeder, opskrifter med kød samt flere. Som en bruger på Opskrifter.dk dog pointere, mangler den sorteringsmulighed, som sortere efter de opskrifter som indeholder flest af de ingredienser som brugeren har indtastet. Denne sorteringsmulighed anses for os, som værende den mest relevante, da man som bruger er interesseret i at få anvendt så mange af ens madrester som muligt. 

%De tre løsninger er vidt forskellige i deres måde at håndtere søgning på. Ud fra vores afprøvninger og observationer af løsningernes fordele og ulemper, kan vi uddrage hvilke egenskaber vi ønsker at benytte i vores eget projekt. \Fx viser den generelle utilfredshed med Forbrugerstyrelsens mobilapp For Resten, at det er vigtigt med mange opskrifter og muligheden for at vælge mere end en rest. Observationerne af For Resten og Opskrifter.dk viser også at brugergrænseflade er et vigtigt element. I disse to løsninger skal man vælge ingredienser ved at lede rundt i kategorier og i For Resten endda bevæge fingeren rundt i en cirkel for at rotere hjulene for kategorier og rester. Dette føles meget ineffektivt i forhold til at skrive navnet på ingrediensen på et tastatur. Derudover er der mellem DK-kogebogen og Opskrifter.dk en markant forskel på hvordan resultater findes. I DK-kogebogen findes kun opskrifter som inkluderer alle de indtastede ingredienser, mens Opskrifter.dk finder alle opskrifter som indeholder bare én af de valgte ingredienser. Dvs. at man med DK-kogebogen får færre resultater jo flere ingredienser man skriver, mens det med Opskrifter.dk er direkte modsat, idet antallet af resultater stiger voldsomt med antallet af ingredienser man skriver. Opskrifter.dk’s måde at gøre det på, kombineret med deres manglende sortering, giver et stor uoverskuelig mængde af resultater, hvor en stor del af disse måske kun indeholder en af de valgte ingredienser. Af dette kan man aflede at en kombination af de to må være den optimale løsning. Har man valgt få ingredienser, er man sandsynligvis interesseret i at få vist resultater som indeholder alle de ingredienser man har valgt. Har man derimod valgt mange ingredienser, er man interesseret i at få vist resultater som indeholder flest muligt af de ingredienser man har valgt.
