\section{Brug}
\label{sec:brug}
Denne analyse af anvendelsesområdets brug, har til formål at gøre det klart, hvilke aktører, der benytter Foodl. Resultatet af analysen af aktører er en mængde aktørbeskrivelser. Derudover analyseres hvilke mønstre, der er for aktørernes brug af Foodl. Resultatet af denne akvitet er en mængde brugsmønstre, der beskrives både i form af en brugsmønsterspecifikation og derefter et tilstandsdiagram. Hver enkelt brugsmønstre vedrører en eller flere aktører. Denne relation vises med en aktørtabel, som afslutter aktiviteten brug.

\subsection{Aktører}
Vi har fundet 3 aktører, der vil kunne interagere med Foodl. Disse aktører kan ses i \tableref{table:aktoerbeskrivelser}.

\aktortabelEx{Bruger}
{En person, der ønsker at bruge systemet foodl.dk til at finde opskrifter, der er mulige at lave med de råvarer, som personen er i besiddelse af.}
{Systemets brugere inkluderer mange personer i forskellige aldersgrupper med vidt forskellig erfaring inden for computerbrug.}
{Bruger A er en 23-årig universitetsstuderende, der føler sig sikker med at navigere rundt på internettet og gør det flere gange dagligt. A kan godt lide at afprøve de forskellige funktioner som hjemmesider stiller til rådighed, for at undersøge hvad de gør. A er meget lærenem, når det kommer til at benytte funktioner på hjemmesider. A bor alene, og har pga. supermarkedernes familietilpassede portioner, ofte madrester til overs, som A ønsker at bruge. Systemet bliver brugt til at få madresterne med i aftenens aftensmad.
 
Bruger B er en 45-årig familiemor eller -far, der hovedsagligt bruger computeren til arbejdsrelaterede opgaver og til at holde sig opdateret ved at læse nyheder på diverse nyhedshjemmesider. Bruger B benytter systemet til blandt andet at få brugt madrester fra den foregående dags aftensmad eller til at få inspiration til den kommende aftensmad. Bruger B vil være interesseret i at være i stand til at dele f.eks. indkøbsliste med ægtefællen, på tværs af enheder.}


% Allerede copy/pasted før bruger
%Aktøren ``bruger'' vil være den vigtigste aktør i forhold til brugen af systemet. Det er meningen, at der skal være rigtig mange brugere, der kan få gavn af systemet. Disse brugeres kompetencer mht. computerbrug vil variere utroligt meget, og det skal vi tage højde for.

%Herunder beskrives de to sidste aktører:

\aktortabelEx{Administrator}{ak-administrator}
{En person, der har til formål at administrere og håndtere eventuelle fejl i systemet, der rapporteres af systemets brugere.}
{Systemets administrator har et højt erfaringsniveau med hensyn til systemet. De har også kontakt til systemets udviklere, der kan rette eventuelle seriøse og systemkritiske fejl.}
{Administrator A er en ubetalt studerende, som håndterer fejl på Foodl i sin fritid. A gennemgår fejlrapporter og vurderer om en fejl er så kritisk at han bør kontakte en systemudvikler, eller om han selv kan udbedre fejlen, for eksempel ved at slette et dødt link.}
{}

\aktortabel{Crawler}
{Et system, der skal besøge foruddefinerede hjemmesider indeholdende opskrifter, som crawleren skal analysere, oversætte og gemme oplysningerne i en database.}
{Crawleren fungerer systematisk ud fra nogle foruddefinerede parametre. Der findes én crawler til hver opskriftshjemmeside, fordi hjemmesiderne ikke nødvendigvis er opbygget på samme måde. Dette betyder, at de samme parametre ikke nødvendigvis vil gælde for flere opskriftshjemmesider.}
{}

% flyttet til 5.1.2. brugmøstre
%Som vi nævnte før, så er de centrale formål for dette kapitel at undersøge, hvilke aktører, som findes for systemet, og hvordan de aktører kan bruge systemet. I dette ovenstående afsnit har vi identificeret og beskrevet de tre aktører, der findes for systemet. Herefter beskriver vi, hvordan systemet kan bruges vha. brugsmønstre. Disse brugsmønstre stemmer overens med \tableref{table:aktoertabel}.



\subsection{Brugsmønstre}
I forbindelse med modelleringen af brugsmønstre, har vi fokuseret på fra starten af, at identificere så mange brugsmønsterkandidater som muligt. Vi har benyttet denne teknik i et forsøg på at sikre os, ikke at have overset et vigtigt brugsmønstre. Kandidaterne er valgt i forbindelse med en brainstorming, og er hver især blevet analyseret for relevans efter brainstormen. Dette har resulteret i at nogle af kandidaterne er blevet fravalgt. Der har været flere grunde til at vi har fravalgt kandidater:
\begin{itemize}
\item Brugsmønstret har ikke været en del af anvendelsesområdet
\item Brugsmønstret har været for simpelt
\item Brugsmønstret har været været en del af eller magen til et andet brugsmønstre
\end{itemize}
De fravalgte kandidater, samt begrundelsen fra fravælgelsen fremgår af \ref{ap:fravalgtebrugsmoenstre}.

Efter fravælgelsen, står vi tilbage med en række brugsmønstre, som beskriver hændelserne, der vedører en given aktør. Brugmønstrene præsenteres først i form af et tilstandsdiagram, der hurtigt giver et godt visuelt overblik over brugsmønsteret. Hvis der er noget man har brug for en mere detaljeret beskrivelse af, så kan man finde denne beskrivelse i den efterfølgende brugsmønsterspecifikation.
Brugsmønstrene kan ses i \tableref{table:brugmoenstre}
\brugtabel{Favorisering}{favorisering}
{Favorisering igangsættes af \textit{brugeren}. Når \textit{brugeren} har lavet en søgning, og flere forskellige opskrifter vises som søgeresultater, kan \textit{brugeren} klikke på favorisér-knappen tilhørende den enkelte opskrift, for at favorisere denne. Når en opskrift er blevet favoriseret, tilføjes denne til en liste af favoritopskrifter. Det er muligt at fjerne opskriften fra favoritlisten på to måder. Enten fra samme sted, som favoriseringen blev tilføjet, eller direkte i favoritlisten.}
{}
{}
{Brugmønsteret favorisering}

\brugtabel{Rapportering}{rapportering}
{Rapportering igangsættes af \textit{brugeren}, når denne opdager en fejl på hjemmesiden. Hvis \textit{brugeren} opdager en fejl, der har med et søgningsresultat (opskrift) at gøre, så klikkes der på en rapporteringsknap, der er ved det enkelte søgningsresultatet. Når der skal rapporteres en fejl vedrørende opskrifter, så åbnes en dialogboks på siden, hvor \textit{brugeren} herefter skal vælge en fejltype. Der skelnes mellem beskrivelige og ubeskrivelige fejltyper. Et eksempel på en ubeskrivelig fejltype er bl.a., hvis et link ikke fungerer, som hører under fejltypen “dødt link”. De ubeskrivelige fejltyper behøver ingen beskrivelse, da fejltypen er beskrivelse nok i sig selv. Vælger \textit{brugeren} derimod en beskrivelig fejltype, så præsenteres en beskrivelsesboks for \textit{brugeren}, hvor fejlen beskrives med tekst. Derefter er rapporten klar, og \textit{brugeren} skal nu godkende rapporten, inden den bliver sendt til \textit{administratoren}.

Derudover er der en generel rapporteringsknap, der vedrører andre, generelle fejl på siden. Når denne knap benyttes, så dirigeres brugeren direkte hen til en beskrivelsesboks, hvor fejlen beskrives med tekst. Til slut skal rapporten godkendes af \textit{brugeren}, inden den bliver sendt til \textit{administratoren}. Det er altid muligt at annullere rapporteringen under alle tilstande i brugsmønstret.}
{}
{Opret fejlrapport}
{Brugmønsteret rapportering}

\brugtabel{Fejlhåndtering}{fejlhaandtering}
{Fejlhåndering igangsættes af \textit{administratoren}. \textit{Administratoren} logger ind på hjemmesiden, og bevæger sig ind på fejlhåndteringssiden. Her præsenteres en liste af  fejlrapporter, der, via \textit{brugeren}, er blevet rapporteret og dokumenteret i systemet. \textit{Administratoren} kan derefter klikke på en fejlrapport i listen for at se en detaljeret beskrivelse af den givne fejl, som derefter kan håndteres.}
{}
{}
{Brugmønsteret fejlhåndtering}

\brugtabel{Indkøbslistehåndtering}{indkoebslistehaandtering}
{Håndteringen af indkøbslisten følger materialemønsteret\cite[p.~128]{ooad}. I materialemønstret er det muligt for aktøreren, at lave flere handlinger i en vilkårlig rækkefølge, da der i brugmønstret typisk ikke vil være særlig mange tilstande i forhold til handlinger. Håndteringen igangsættes af \textit{brugeren}, se \figref{fig:bm-indkoebslistehaandtering}. \textit{Brugeren} kan tilgå indkøbslisten fra en vilkårlig underside på \Foodl. Når \textit{brugeren} har tilgået indkøbslisten, så er redigeringen igangsat, og det er muligt at tilføje/fjerne varer fra indkøbslisten. Ingredienser bliver ``omdannet'' til varer, når de tilføjes på indkøbslisten. Alle elementer på indkøbslisten er af vare-klassen, hvilket gør det muligt for \textit{brugeren} at indtaste vilkårlige varer, ikke blot ingredienser, der er relateret til \fx en opskrift. \textit{Brugeren} har også mulighed for at tilføje varer eller alle opskriftens ingredienser direkte fra søgeresultatet, der er en liste af opskrifter. Når \textit{brugeren} forlader søgeresultatet, så er indkøbslisten gemt, og den kan stadig tilgås fra en vilkårlig underside på \Foodl. 

Der kan altid kun være én indkøbsliste ad gangen pr. bruger. Fra indkøbslistevisningen er det muligt at printe indkøbslisten samt tømme indkøbslisten for alt indhold. Håndteringen af indkøbslisten afsluttes når \textit{brugeren} lukker ned for redigering - altså forlader indkøbsliste-siden. Redigering startes igen, når man tilgår indkøbslisten, og alle de tilføjede var, der ikke er blevet slettet, vil stadig være tilgængelige.}
{Vare, Opskrift, Person}
{Håndter indkøbsliste}
{Brugsmønster for håndteringen af indkøbslisten, hvilket er relationen mellem ``person''- og ``vare''-klasserne i problemområdet.}


\brugtabel{Indlogning}
{Indlogning igangsættes af \textit{brugeren}. Der er tre startmuligheder, og de ender alle tre i samme tilstand “Login aktiv”. Den første mulighed er, hvis brugeren var logget ind fra en tidligere session, så hentes oplysningerne fra denne session automatisk. Den anden mulighed er, at brugeren trykker på “Login / Registrer”, som kan tilgås fra en vilkårlig Foodl-underside. Systemet venter nu på \textit{brugerens} loginoplysninger. \textit{Brugeren} indtaster oplysningerne og systemet påbegynder godkendelsesprocessen. Hvis oplysningerne bliver afvist, så skal \textit{brugeren} genindtaste oplysnignerne. Hvis de bliver godkendt, så bliver \textit{brugeren} logget ind på siden. Den tredje mulighed er, hvis \textit{brugeren} ønsker at lave en ny konto i systemet. \textit{Brugeren} skal nu indtaste brugernavn og adgangskode i systemet (det er ikke obligatorisk at indtaste e-mail). Hvis der opstår en fejl i oplysningerne, så skal \textit{brugeren} genindtaste oplysningerne. Når oplysningerne bliver godkendt, så bliver \textit{brugeren} logget ind med det brugernavn og adgangskode, som er blevet indtastet. På hvilket som helst tidspunkt, har \textit{brugeren} mulighed for at annullere indlogningsprocessen undervejs.

Hvis \textit{brugeren} er logget ind på en konto, så bliver \textit{brugerens} indkøbsliste og favoritter indlæst. Disse kan tilgås fra en vilkårlig Foodl-underside. \textit{Brugeren} kan nu oprette eller fjerne favoritter og tilgå og/eller håndtere indkøbslisten.}
{}
{}
\begin{figure}
\centering
\scalebox{0.7}{
\brugtabel{Indlogning}
{Indlogning igangsættes af \textit{brugeren}. Der er tre startmuligheder, og de ender alle tre i samme tilstand “Login aktiv”. Den første mulighed er, hvis brugeren var logget ind fra en tidligere session, så hentes oplysningerne fra denne session automatisk. Den anden mulighed er, at brugeren trykker på “Login / Registrer”, som kan tilgås fra en vilkårlig Foodl-underside. Systemet venter nu på \textit{brugerens} loginoplysninger. \textit{Brugeren} indtaster oplysningerne og systemet påbegynder godkendelsesprocessen. Hvis oplysningerne bliver afvist, så skal \textit{brugeren} genindtaste oplysnignerne. Hvis de bliver godkendt, så bliver \textit{brugeren} logget ind på siden. Den tredje mulighed er, hvis \textit{brugeren} ønsker at lave en ny konto i systemet. \textit{Brugeren} skal nu indtaste brugernavn og adgangskode i systemet (det er ikke obligatorisk at indtaste e-mail). Hvis der opstår en fejl i oplysningerne, så skal \textit{brugeren} genindtaste oplysningerne. Når oplysningerne bliver godkendt, så bliver \textit{brugeren} logget ind med det brugernavn og adgangskode, som er blevet indtastet. På hvilket som helst tidspunkt, har \textit{brugeren} mulighed for at annullere indlogningsprocessen undervejs.

Hvis \textit{brugeren} er logget ind på en konto, så bliver \textit{brugerens} indkøbsliste og favoritter indlæst. Disse kan tilgås fra en vilkårlig Foodl-underside. \textit{Brugeren} kan nu oprette eller fjerne favoritter og tilgå og/eller håndtere indkøbslisten.}
{}
{}
\begin{figure}
\centering
\scalebox{0.7}{
\input{billeder/brugsmoenstre/indlogning.pdf_tex}}
\capt{Brugmønsteret indlogning}\label{fig:bm-indlogning}
\end{figure}
}
\capt{Brugmønsteret indlogning}\label{fig:bm-indlogning}
\end{figure}


\brugtabel{Søgning}{soegning}
{En søgning igangsættes af \textit{brugeren}. \textit{Brugeren} bliver præsenteret for et tom søgefelt, der illustrerer, at man kan indtaste nogle søgekriterier i form af råvaretyper i det tomme felt. En søgning fuldføres, når man taster på knappen ``søg'', hvilket kun er muligt, når der mindst er én råvaretype som søgekriterie. Det er muligt at indtaste og fjerne så mange råvaretyper fra søgefeltet, som \textit{brugeren} ønsker. Se \figref{fig:bm-soegning}.

Når der er blevet foretaget en søgning, så bliver der vist et søgeresultat, der er en liste af opskrifter, der indeholder en eller flere af de indtastede råvaretyper. \textit{Brugeren} har nu endnu en gang mulighed for at fjerne eller indtaste nye råvaretyper fra denne resultatside. \textit{Brugeren} kan søge på de nye søgekriterier direkte fra denne side uden at genlæse visningen i sin internetbrowser.

Opskrifterne sorteres i første omgang efter, hvor godt deres ingredienser matcher de valgte råvaretyper. \textit{Brugeren} kan vælge to andre sorteringsmuligheder. Udover den primære sortering, så kan opskrifterne sorteres efter tilberedningstid og alfabetisk orden. Det er også muligt at begrænse søgeresultatet efter tilberedningstid. Her har \textit{brugeren} mulighed for at bestemme om, der skal bruges lidt eller meget tid på madlavningen. Når \textit{brugeren} har valgt, så bliver listen dynamisk opdateret.

Søgningen kan under alle tilstande afsluttes ved at lukke siden. Brugsmøntret skiller sig ud fra de andre brugsmønstre, ved at det har et brugsmønter integreret i sig, nemlig indkøbslistehåndtering. Det er markeret med en *-tegn, for at symbolisere, at det er et brugsmønster.}
{Opskrift, Råvaretype}
{Søg efter opskrifter, Skaler søgeresultat, Sorter søgeresultat, Begræns søgeresultat}
{Brugsmønster for søgning.}


\brugtabel{Crawling}
{Crawling igangsættes af \textit{crawler}. Under en crawling besøges opskrifter på en opskriftsside én af gangen. \textit{Crawleren} analyserer opskrifterne for at kunne skelne hvilke dele af den der er ingredienser, fremgangsmåde, serveringsstørrelse, billede af retten, m.m. Alle opskrifter, der bliver fundet, bliver omsat så de passer til systemets model af en opskrift, der tilføjes til systemets indeks.}
{}
{}
%\brugtabel{Crawling}
{Crawling igangsættes af \textit{crawler}. Under en crawling besøges opskrifter på en opskriftsside én af gangen. \textit{Crawleren} analyserer opskrifterne for at kunne skelne hvilke dele af den der er ingredienser, fremgangsmåde, serveringsstørrelse, billede af retten, m.m. Alle opskrifter, der bliver fundet, bliver omsat så de passer til systemets model af en opskrift, der tilføjes til systemets indeks.}
{}
{}
%\input{billeder/brugsmoenstre/crawling.pdf_tex}
%\caption{Brugmønsteret crawling}

%\caption{Brugmønsteret crawling}



\subsection{Aktørtabel}
\tableref{table:aktoertabel} viser hvilke hændelser, der vedrører en given aktør. Et flueben betyder at brugsmønstret i rækken vedrører aktøren i kolonnen.
\ourtable{aktoertabel}{3}{Aktørtabel for Foodl.}
                                                  {Brugsmønstre}
       {Aktører                }{ Bruger        & Administrator & Crawler       }{
\ourrow{Søgning                }{ \checkmark    &               &               }
\ourrow{Favorisering           }{ \checkmark    &               &               }
\ourrow{Indkøbslistehåndtering }{ \checkmark    &               &               }
\ourrow{Login                  }{ \checkmark    & \checkmark    &               }
\ourrow{Fejlhåndtering         }{               & \checkmark    &               }
\ourrow{Rapportering           }{ \checkmark    & \checkmark    &               }
\ourrow{Crawling               }{               &               & \checkmark    }
}





