\brugtabel{Indlogning}
{Indlogning igangsættes af \textit{brugeren}. Der er tre startmuligheder, og de ender alle tre i samme tilstand “Login aktiv”. Den første mulighed er, hvis brugeren var logget ind fra en tidligere session, så hentes oplysningerne fra denne session automatisk. Den anden mulighed er, at brugeren trykker på “Login / Registrer”, som kan tilgås fra en vilkårlig Foodl-underside. Systemet venter nu på \textit{brugerens} loginoplysninger. \textit{Brugeren} indtaster oplysningerne og systemet påbegynder godkendelsesprocessen. Hvis oplysningerne bliver afvist, så skal \textit{brugeren} genindtaste oplysnignerne. Hvis de bliver godkendt, så bliver \textit{brugeren} logget ind på siden. Den tredje mulighed er, hvis \textit{brugeren} ønsker at lave en ny konto i systemet. \textit{Brugeren} skal nu indtaste brugernavn og adgangskode i systemet (det er ikke obligatorisk at indtaste e-mail). Hvis der opstår en fejl i oplysningerne, så skal \textit{brugeren} genindtaste oplysningerne. Når oplysningerne bliver godkendt, så bliver \textit{brugeren} logget ind med det brugernavn og adgangskode, som er blevet indtastet. På hvilket som helst tidspunkt, har \textit{brugeren} mulighed for at annullere indlogningsprocessen undervejs.

Hvis \textit{brugeren} er logget ind på en konto, så bliver \textit{brugerens} indkøbsliste og favoritter indlæst. Disse kan tilgås fra en vilkårlig Foodl-underside. \textit{Brugeren} kan nu oprette eller fjerne favoritter og tilgå og/eller håndtere indkøbslisten.}
{}
{}
