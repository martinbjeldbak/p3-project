\begin{tabular}{p{\textwidth}}
    \hline
    \begin{center} 
    \textbf{\textit{Søgning}} 
    \end{center} \\ \hline
    \textbf{Brugsmønster:} En søgning igangsættes af brugeren, ved at indtaste et antal forskellige råvarer og derefter trykke på “søg”. En del af de indtastede råvarer kan også være gemte ingredienser fra tidligere søgninger. Efter en søgning, vises en mængde opskrifter baseret på de råvarer, der er blevet indtastet. Det er muligt at tilføje eller fjerne råvarer, der kan yderligere specificer søgningen.  Opskrifterne sorteres i første omgang efter, hvor godt deres ingredienser matcher de valgte råvarer. Brugeren kan vælge en sekundær sortering, hvor brugeren har mulighed for at sortere efter bedømmelse eller navn. Opskrifterne kan også filtreres på flere måder (også samtidig), så der kun vises opskrifter uden fx kød, svin og/eller gluten. I søgeresultatet er det muligt at tilføje samt gemme og fjerne begrænsninger og råvarer. Søgningen kan under alle tilstande afsluttes ved at lukke siden. \\
    \textbf{Objekter:}  \\
    \textbf{Funktioner:}  \\ \hline
\end{tabular}