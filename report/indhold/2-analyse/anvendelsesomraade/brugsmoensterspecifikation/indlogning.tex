\brugtabel{Indlogning}
{Indlogning igangsættes af brugeren. Der er tre startmuligheder, og de ender alle tre i samme tilstand “Login aktiv”. Den første mulighed er, hvis brugeren var logget ind fra en tidligere session, så hentes oplysningerne fra denne session automatisk. Den anden mulighed er, at brugeren trykker på “Login / Registrer”, som kan tilgås fra en vilkårlig Foodl-underside. Systemet venter nu på brugerens loginoplysninger. Brugeren indtaster oplysningerne og systemet påbegynder godkendelsesprocessen. Hvis oplysningerne bliver afvist, så skal brugeren genindtaste oplysnignerne. Hvis de bliver godkendt, så bliver brugeren logget ind på siden. Den tredje mulighed er, hvis brugeren ønsker at lave en ny bruger i systemet. Brugeren skal nu indtaste brugernavn og adgangskode i systemet (det er ikke obligatorisk at indtaste e-mail). Hvis der opstår en fejl i oplysningerne, så skal brugeren genindtaste oplysningerne. Når oplysningerne bliver godkendt, så bliver brugeren logget ind med det brugernavn og adgangskode, som er blevet indtastet. På hvilket som helst tidspunkt, har brugeren mulighed for at annullere indlogningsprocessen undervejs.
Hvis brugeren er logget ind på en konto, så bliver brugerens indkøbsliste og favoritter indlæst, som kan tilgås fra en vilkårlig Foodl-underside. Brugeren kan nu oprette eller fjerne favoritter og tilgå og/eller håndtere indkøbslisten.}
{}
{}
