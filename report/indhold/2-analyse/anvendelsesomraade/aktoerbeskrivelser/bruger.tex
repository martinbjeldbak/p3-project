\begin{tabular}{p{\textwidth}}
    \hline
    \begin{center} \textbf{\textit{Bruger}} \end{center} \\ \hline
    \textbf{Formål:} En person, der ønsker at bruge systemet foodl.dk til at finde opskrifter, der er mulige at lave med de råvarer, som personen er i besiddelse af. \\
    \textbf{Karakteristik:} Systemets brugere inkluderer mange personer i forskellige aldersgrupper med vidt forskellig erfaring inden for computerbrug. \\
    \textbf{Eksempler:} Bruger A er en 23-årig universitetsstuderende, der føler sig sikker med at navigere rundt på internettet og gør det flere gange dagligt. A kan godt lide at afprøve de forskellige funktioner som hjemmesider stiller til rådighed, for at undersøge hvad de gør. A er meget lærenem, når det kommer til at benytte funktioner på hjemmesider. A bor alene, og har pga. supermarkedernes familietilpassede portioner, ofte madrester til overs, som A ønsker at bruge. Systemet bliver brugt til at få madresterne med i aftenens aftensmad.
 
Bruger B er en 45-årig familiemor eller -far, der hovedsagligt bruger computeren til arbejdsrelaterede opgaver og til at holde sig opdateret ved at læse nyheder på diverse nyhedshjemmesider. Bruger B benytter systemet til blandt andet at få brugt madrester fra den foregående dags aftensmad eller til at få inspiration til den kommende aftensmad. Bruger B vil være interesseret i at være i stand til at dele f.eks. indkøbsliste med ægtefællen, på tværs af enheder. \\ \hline 
\end{tabular}
