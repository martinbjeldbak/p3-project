\section{Brugsspecifikation}
\label{sec:brugsspecifikation}
%\tjek{Det første led i analysen af anvendelsesområdet er en aktør- og brugsmønsterspecifikation, der skal sammen med funktioner (\secref{sec:funktioner}), danne krav til systemets brug.

%Aktørene er blevet identificeret ud fra de forskellige entiteter gruppen mener, vil benytte systemet. Hver aktør har en gradvis forskellig rolle end de andre aktører og er blevet navngivet baseret på den måde, aktøren vil interagere med systemet på. Følgende er aktørbeskrivelser over de valgte aktører, der vil benytte systemet:}


\aktortabelEx{Bruger}
{En person, der ønsker at bruge systemet foodl.dk til at finde opskrifter, der er mulige at lave med de råvarer, som personen er i besiddelse af.}
{Systemets brugere inkluderer mange personer i forskellige aldersgrupper med vidt forskellig erfaring inden for computerbrug.}
{Bruger A er en 23-årig universitetsstuderende, der føler sig sikker med at navigere rundt på internettet og gør det flere gange dagligt. A kan godt lide at afprøve de forskellige funktioner som hjemmesider stiller til rådighed, for at undersøge hvad de gør. A er meget lærenem, når det kommer til at benytte funktioner på hjemmesider. A bor alene, og har pga. supermarkedernes familietilpassede portioner, ofte madrester til overs, som A ønsker at bruge. Systemet bliver brugt til at få madresterne med i aftenens aftensmad.
 
Bruger B er en 45-årig familiemor eller -far, der hovedsagligt bruger computeren til arbejdsrelaterede opgaver og til at holde sig opdateret ved at læse nyheder på diverse nyhedshjemmesider. Bruger B benytter systemet til blandt andet at få brugt madrester fra den foregående dags aftensmad eller til at få inspiration til den kommende aftensmad. Bruger B vil være interesseret i at være i stand til at dele f.eks. indkøbsliste med ægtefællen, på tværs af enheder.}


% Allerede copy/pasted før bruger
%Aktøren ``bruger'' vil være den vigtigste aktør i forhold til brugen af systemet. Det er meningen, at der skal være rigtig mange brugere, der kan få gavn af systemet. Disse brugeres kompetencer mht. computerbrug vil variere utroligt meget, og det skal vi tage højde for.

%Herunder beskrives de to sidste aktører:

\aktortabelEx{Administrator}{ak-administrator}
{En person, der har til formål at administrere og håndtere eventuelle fejl i systemet, der rapporteres af systemets brugere.}
{Systemets administrator har et højt erfaringsniveau med hensyn til systemet. De har også kontakt til systemets udviklere, der kan rette eventuelle seriøse og systemkritiske fejl.}
{Administrator A er en ubetalt studerende, som håndterer fejl på Foodl i sin fritid. A gennemgår fejlrapporter og vurderer om en fejl er så kritisk at han bør kontakte en systemudvikler, eller om han selv kan udbedre fejlen, for eksempel ved at slette et dødt link.}
{}

\aktortabel{Crawler}
{Et system, der skal besøge foruddefinerede hjemmesider indeholdende opskrifter, som crawleren skal analysere, oversætte og gemme oplysningerne i en database.}
{Crawleren fungerer systematisk ud fra nogle foruddefinerede parametre. Der findes én crawler til hver opskriftshjemmeside, fordi hjemmesiderne ikke nødvendigvis er opbygget på samme måde. Dette betyder, at de samme parametre ikke nødvendigvis vil gælde for flere opskriftshjemmesider.}
{}

% flyttet til 5.1.2. brugmøstre
%Som vi nævnte før, så er de centrale formål for dette kapitel at undersøge, hvilke aktører, som findes for systemet, og hvordan de aktører kan bruge systemet. I dette ovenstående afsnit har vi identificeret og beskrevet de tre aktører, der findes for systemet. Herefter beskriver vi, hvordan systemet kan bruges vha. brugsmønstre. Disse brugsmønstre stemmer overens med \tableref{table:aktoertabel}.


%\tjek{Følgende ses nogle kandidater for brugsmønstre. Brugsmønstrene består af tilstandsdiagrammer, der illustrerer hvad aktørerne vil kunne bruge systemet til. Som i klasse- og hændelsesaktiviteterne, er fravalgte brugsmønstre med tilhørende forklaring dokumenteret. De valgte ses kort herefter sammen med aktørtabellen, der viser sammenhænget mellem de valgte aktører og deres brugsmønstre.}

\subsection{Fravalgte brugsmønstre}
\begin{description}
  \item[Skalering] Skalering af ingredienser i opskrifter i forhold til antallet af personer
Fjernet, da skalering indgår i søgningen fordi der er ingen grund til at skalere opskriften i hver visning og kan dermed indgå som en slags “filter” i søgningen.

\item[Råvarehåndtering] Når man gemmer ingredienser i opskrifter... (?)
Fjernet, da råvarehåndtering indgår i søgningen

\item[Overvågning] Hvad skal vi egentligt overvåge?

\item[Begrænsning] \fx ingen kød, glutenfri
Fjernet, da begrænsning indgår i søgningen

\item[Sortering] Sortere opskrifter i forhold til forskellige filtre
Fjernet, da sortering indgår i søgningen

\item[Madplanlægning] Fjernet, da klassen ``madplan'' blev fravalgt i problemområdet

\item[Synkronisering] I stedet for et loginsystem, vil vi benytte cookies og derfor skal synkronisering mellem forskellige enheder håndteres
Vi vælger i stedet et loginsystem, da brugerne er allerede klar over, hvordan sådan et system virker
  
\end{description}

\subsection{Valgte brugsmønstre}

\begin{description}
  \item[Søgning] At søge på opskrifter i forhold til de indtastede ingredienser

  \item[Favorisering] At favorisere/bogmærke en opskrift

  \item[Indkøbslistehåndtering] Skal kunne tilføje ingredienser fra flere forskellige opskrifter på én fælles indkøbsliste

  \item[Indlogning] Håndterer brugerens session vha. et loginsystem

  \item[Fejlhåndtering]

  \item[Crawling] Når crawleren besøger opskriftshjemmesider og tilføjer/opdaterer/sletter opskifter til/fra Foodl’s indeks
\end{description}

\ourtable{aktoertabel}{3}{Aktørtabel for Foodl.}
                                                  {Brugsmønstre}
       {Aktører                }{ Bruger        & Administrator & Crawler       }{
\ourrow{Søgning                }{ \checkmark    &               &               }
\ourrow{Favorisering           }{ \checkmark    &               &               }
\ourrow{Indkøbslistehåndtering }{ \checkmark    &               &               }
\ourrow{Login                  }{ \checkmark    & \checkmark    &               }
\ourrow{Fejlhåndtering         }{               & \checkmark    &               }
\ourrow{Rapportering           }{ \checkmark    & \checkmark    &               }
\ourrow{Crawling               }{               &               & \checkmark    }
}

