\section{Systemvalg}
I det private køkken, sker det til tider, at der smides mad væk. En del af madspildet kan for eksempel ske efter endt madlavning, hvor man kan risikere at stå tilbage med nogle fødevarer, der helst skal bruges hurtigst muligt for ikke at blive for gamle. Man kunne for eksempel have lavet hele kyllingbryst med persille, og fået en masse af begge dele tilovers. Næste aften har man ikke lyst til at få samme ret igen, men maden kan ikke holde sig meget længere, så man fristes derfor til at smide maden væk. Dette problem forsøger vi at løse ved at udvikle et system. Systemet vælger vi at kalde Foodl.

\subsection{Møde 1 med informanter}
Formålet med møde 1 er at opnå en større indsigt i informantens mad- og indkøbsvaner og på den måde få en forståelse for hvilke problemer der er mulighed for at løse ved udviklingen af et system.

Det er på baggrund af møde 1 er det blevet gjort klart, at der blandt informanterne findes flere forskellige problemer vedrørende madlavning, hvoraf nogle er prioriterede højere end andre blandt informanterne. Ved at bruge BATOFF-modellen, er 2 systemdefinitioner blevet lavet. Systemdefinitionerne beskriver kort 2 forskellige systemer, der hver især løser nogle af de, af informanterne, rejste problemstillinger .
