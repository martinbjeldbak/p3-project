\section{Klasser}
\label{sec:klasser}
Når systemdefinitionen gradvist er på plads, brainstormer gruppen over forskellige klasser. Det er vigtigt, at \emph{identitet}, \emph{tilstand} og \emph{adfærd} tages i betragtning for hver enkelt klassekandidat. De er blevet dannet ud fra systemdefinitionen og diverse rige billeder og metaforer, som hver gruppemedlem lavede (se bilag XX), samt diskussion over elementerne i problemområdet. Herunder er en liste af de ``vigtigste'' klasser, som blev diskuteret og modelleret. Argumenterne for, om klassen skal indgå/ikke indgå i systemet ses under den specifikke klasse. Det skal siges, at klasserne er kandidater og er blevet tilføjet/fjernet under hele projektforløbet pga. den iterative arbejdsproces, som gruppen vælger at benytte.

\subsection{Fravalgte klassekandidater}
Herunder ses de fravalgte klasser, som gruppen ikke finder en del af problemområdet. De listes her, da der har været meget diskussion, om hvorvidt klasserne skulle med i systemet eller ej. 

\begin{description}
\item[Person] \hfill \\
Vi vælger at fjerne person-klassen, fordi vi mener, at vores løsning ikke skal være noget socialt media, og derfor mener vi, at det er selve husholdningen, der er fælles om madlavningen selvom det måske blot er en person, der laver mad og står for indkøb og lignende. Brugeren af programmet er ikke en del af problemområdet og skal derfor ikke modelleres som klasse.

\item[Køkken] \hfill \\
Køkken og husholdning dækker over samme del af problemområdet, og vi fjerner derfor køkken og vurderer bagefter om husholdning skal være en klasse.

\item[Husholdning] \hfill \\
En husholdning repræsenterer et hjem, som indeholder én til flere personer. Det er ikke en del af problemområdet at holde styr på eller at kommunikere med andre husstande.

\item[Køleskab/skab/opbevaringsskab] \hfill \\
Om råvarene befinder sig i et køleskab eller i en skuffe er, for os, uinteressant, derfor er vi ikke interesseret i at modellere disse skabe som en klasse.

\item[Køkkenredskab/komfur] \hfill \\
Ligesom ved køleskab/anden opbevaring er vi ikke interesseret i at modellere hvilke redskaber husholdningen har adgang til.

\item[Service] \hfill \\ 
Vi har valgt at fjerne klassen service, fordi denne ikke findes i vores problemområde. Service er noget, der bliver brugt, når man er i færd med at spise det færdige mad, og ikke under selve madlavningen.

\item[Butik] \hfill \\
Vi har valgt at fjerne klassen Butik, da vores system fokuserer på madspild og varieret kost. En modellering af butikker ville være relevant, hvis vores fokus lå på at begrænse udgifter på mad, men da dette imidlertidig ikke er situationen i problemområdet, bliver den udeladt.

\item[Typisk/atypisk ingrediens] \hfill \\
Det er ikke en del af problemområdet at folk ikke er klar over hvilke ingredienser der er normale/unormale at have.

\item[Enhed/Mængde] \hfill \\
Enhed og mængde er ikke klasser, men attributter til en ingrediens.

\item[Madplan] \hfill \\
Informanterne syntes ikke at en madplan var særlig nødvendig. Den indgik i systemdefinition S2, som informanterne fravalgte. Madplanen anses derfor som overflødig, hvorfor denne fjernes som klasse.
\end{description}

\subsection{Valgte klassekandidater}
Herunder ses de valgte klasser og hvorfor gruppen vælger, at have dem med i det færdige system. Gruppen mener, at de forekommer som vigtige objekter i problemområdet og skal derfor modelleres.

\begin{description}
\item[Ingrediens] \hfill \\ 
I opskrifter bruges der flere ingredienser. En ingrediens består af en råvare og en mængde af denne. Det er et problem at finde opskrifter, der indeholder ingredienser svarende til de råvarer man har til rådig. Det er et problem for informanterne at maden laves i for store portioner, altså er det et problem hvis en opskrifts ingredienser indeholder store mængder.

\item[Bogmærke] \hfill \\
Det er en del af problemområdet, for brugere at huske de gode opskrifter. Der vil til tider blive benyttet en opskrift, der er så god, at den er værd at gemme til en anden gang. Derfor beholder vi denne klasse.

\item[Råvare] \hfill \\
En råvare findes i køleskabene og på madhylderne i husholdningerne. Det er et problem at finde opskrifter, der kun indeholder disse råvarer, derfor skelnes der mellem ingredieser og råvarer.

\item[Indkøbsliste] \hfill \\
Vi vurderer, at der i en husholdning ofte bliver skrevet en indkøbsliste med de ting man mangler. Indkøbslisten kan være skrevet på baggrund af en opskrift man gerne vil lave, eller en hel madplan man gerne vil følge over en længere periode.

\item[Opskrift] \hfill \\
En opskrift er det centrale i problemområdet. Opskrifterne indeholder forskellige ingredienser. Det er nødvendigt at have råvarer nok til at matche ingredienserne i opskriften, før denne kan laves. Man må gå ud fra at den typiske private madlaver har mange opskrifter i kogebogen, som han/hun reelt ikke har råvarerne til at kunne lave.
\end{description}

\subsection{Klassediagram}

De valgte klasser giver anledning til et klassediagram. Under diagrammet ses en kort beskrivelse af relationerne mellem klasserne samt klassens attributter. 

\begin{figure}
  \centering
  \input{billeder/klasseDiagram.pdf_tex}
  \capt{Klassediagram for problemområdet.}
  \label{fig:klassediagram}
\end{figure}


Klassediagrammet ovenover er bygget op af aggregeringer og associationer imellem klasserne i diagrammet. For konkrete beskrivelser af de forskellige klasser, konsulteres hændelses- og klassebeskrivelserne. Her beskrives relationen mellem klasserne.

\begin{description}
  \item[Bogmærke] \hfill \\
    En bogmærke kan repræsentere opskrifter.

  \item[Opskrift] \hfill \\
    Består udelukkende af ingredienser og har fremgangsmåden som attribut. En opskrift kan bogmærkes en gang pr. bruger. Som minimum består en opskrift af en ingrediens.
Attributter: Titel, Billede, Fremgangsmåde

\item[Indkøbsliste] \hfill \\
  Indkøbslisten består udelukkende af en til flere ingredienser, der tilføjes ud fra opskrifterne. Derudover er det muligt at yderligere tilføje elementer på indkøbslisten i form af arbitrære tekststrenge.

\item[Ingrediens] \hfill \\
  En ingrediens består af en råvare.
Attributter: Mængde, Enhed

\item[Råvare] \hfill \\
  En råvare er en dekomponering af en ingrediens. En råvare er blot en ingrediens uden nogen form for information om mængde eller enhed. En ingrediens kunne for eksempel være 200g oksekød.
Attributter: Navn
\end{description}

