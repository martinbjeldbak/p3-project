\section{Klasser}
\label{sec:klasser}
Vi ønsker nu at vælge de bestanddele, som vi vil modellere i problemområdet. Disse vælges
ved at kigge på systemdefinitionen og ved at fremstille rige billeder (se \todo{bilag}).

Grundet den iterative arbejdsproces, er klasser undervejs blevet tilføjet og fjerent, og vi ønsker nu at kaste lys over baggrunden bag de valgte klasser. I gennem processen har vi også fravalgt klasser. Disse, med beskrivelse, kan findes i \apref{ap:fravalgteklasser}.

\subsection{Valgte klasser}
Herunder ses de valgte klasser og hvorfor vi vælger at have dem med i vores model af problemområdet.
Vi mener at disse klasser samler de objekter og hændelser, som er relevant for denne model af problemområdet.

\begin{description}
\item[Ingrediens] \hfill \\ 
I opskrifter bruges der flere ingredienser. En ingrediens består af en råvare og en mængde af denne. Det er et problem at finde opskrifter, der indeholder ingredienser svarende til de råvarer man har til rådig. Det er et problem for informanterne at maden laves i for store portioner, altså er det et problem hvis en opskrifts ingredienser indeholder store mængder.

\item[Bogmærke] \hfill \\
Det er en del af problemområdet, for brugere at huske de gode opskrifter. Der vil til tider blive benyttet en opskrift, der er så god, at den er værd at gemme til en anden gang. Derfor beholder vi denne klasse.

\item[Råvare] \hfill \\
En råvare findes i køleskabene og på madhylderne i husholdningerne. Det er et problem at finde opskrifter, der kun indeholder disse råvarer, derfor skelnes der mellem ingredieser og råvarer.

\item[Indkøbsliste] \hfill \\
Vi vurderer, at der i en husholdning ofte bliver skrevet en indkøbsliste med de ting man mangler. Indkøbslisten kan være skrevet på baggrund af en opskrift man gerne vil lave, eller en hel madplan man gerne vil følge over en længere periode.

\item[Opskrift] \hfill \\
En opskrift er det centrale i problemområdet. Opskrifterne indeholder forskellige ingredienser. Det er nødvendigt at have råvarer nok til at matche ingredienserne i opskriften, før denne kan laves. Man må gå ud fra at den typiske private madlaver har mange opskrifter i kogebogen, som han/hun reelt ikke har råvarerne til at kunne lave.
\end{description}

\subsection{Struktur}

De valgte klasser giver anledning til et klassediagram. Under diagrammet ses en kort beskrivelse af relationerne mellem klasserne samt klassens attributter. 

\begin{figure}
  \centering
  \input{billeder/klasseDiagram.pdf_tex}
  \capt{Klassediagram for problemområdet.}
  \label{fig:klassediagram}
\end{figure}


Klassediagrammet ovenover er bygget op af aggregeringer og associationer imellem klasserne i diagrammet. For konkrete beskrivelser af de forskellige klasser, konsulteres hændelses- og klassebeskrivelserne. Her beskrives relationen mellem klasserne.

\begin{description}
  \item[Bogmærke] \hfill \\
    En bogmærke kan repræsentere opskrifter.

  \item[Opskrift] \hfill \\
    Består udelukkende af ingredienser og har fremgangsmåden som attribut. En opskrift kan bogmærkes en gang pr. bruger. Som minimum består en opskrift af en ingrediens.
Attributter: Titel, Billede, Fremgangsmåde

\item[Indkøbsliste] \hfill \\
  Indkøbslisten består udelukkende af en til flere ingredienser, der tilføjes ud fra opskrifterne. Derudover er det muligt at yderligere tilføje elementer på indkøbslisten i form af arbitrære tekststrenge.

\item[Ingrediens] \hfill \\
  En ingrediens består af en råvare.
Attributter: Mængde, Enhed

\item[Råvare] \hfill \\
  En råvare er en dekomponering af en ingrediens. En råvare er blot en ingrediens uden nogen form for information om mængde eller enhed. En ingrediens kunne for eksempel være 200g oksekød.
Attributter: Navn
\end{description}

