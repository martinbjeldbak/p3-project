\section{Systemdefinition}

Systemdefinitionen nedenunder er en naturlig, kortfattet beskrivelse af den løsning vi ønsker at fremstille. Den er baseret på gruppens egne ønsker om projektretning og interviews med gruppens to informanter. BATOFF-modellen er blevet brugt til at formulere systemdefinitionen. BATOFF indeholdholder følgende punkter: \emph{\textbf{b}etingelser}, \emph{\textbf{a}nvendelsesområde}, \emph{\textbf{t}eknologier}, \emph{\textbf{f}unktioner}, \emph{\textbf{f}ilosofi}. Gruppen benytter BATOFF-modellen fordi det fastsætter nogle rammer i forhold til opsætningen og indholdet af systemdefinitionen samt opretter en form for standard, der gør det muligt at sammenligne flere forskellige systemdefinitioner på en logisk måde.

Efter møde 1 (se bilag) med vores informanter, har gruppen fået forståelse for at madspild er et reelt problem for dem. Det er blevet forklaret hvad der ofte er grunden til madspildet, og på baggrund af møde 1 er to systemdefinitioner blevet konstrueret. Blandt de to systemdefinitioner valgte informanterne systemdefinition 1. Systemdefinition 2 kan ses i den akademiske rapport.

Systemdefinition er defineret som følgende:

\begin{quote}
Systemet skal fungere som et online opskriftsregister, der giver brugeren idéer til opskrifter han rent faktisk kan lave ud fra de madvarer han har. Systemet fokuserer på at mindske madspild, da forbrugere smider mad ud på grund af et manglende formål med anvendelsen af resterne. Brugerne af programmet er en del af en husholdning og vil have meget varierende erfaringer inden for brug af internettet. Udviklerne af systemet er ulønnede studerende. Deadline for det færdige system kan ikke ændres. Systemet skal køre på en server, der kan tilgås via en webapplikation fra en internetbrowser på enhver type computer. På baggrund af en mængde fødevarer som input, findes forskellige opskrifter, der bedst muligt matcher disse. Opskrifterne skal kunne sorteres på flere måder, og ingredienser skal kunne huskes til næste gang, hvis ønsket.
\end{quote}

Gruppen mødtes anden gang med informanterne efter at have fremstillet de to systemdefinitioner for at få feedback på projektets retning og få valgt et attraktivt systemdefinition. Formålet med mødet er at fortælle informanterne om gruppens idé om en opskriftssøgemaskine, der kun finder opskrifter man kan lave ud fra de råvarer, man har til rådighed. Gruppen vil høre om informanterne ville kunne bruge et sådan system. Gruppen holder sig på nuværende tidspunkt meget åbent, da gruppen vil gerne gøre det muligt for informanterne at komme med nye idéer, også selvom de er markant anderledes fra gruppens initierende problemstilling og systemdefinitioner. Derfor foregår mødet som et semistruktureret interview. Det er vigtigt for gruppen at få informanternes idéer til hvilke funktioner et sådan system skal have og hvilken krav, de stiller.

Møde 2 gjorde det klart, at informanterne ser systemdefinition 1 som et meget brugbart system. Vi er blevet præsenteret for en masse funktioner, som hver informant har ønsket til et sådan system.

Dokumentation og referater fra møde 1 og 2 samt alle efterfølgende møder med informanterne kan findes i bilag XX.
