\section{Eksisterende systemer}
Der eksisterer allerede en lang række systemer på internettet, som tilbyder en service, der ligner den, som vi ønsker Foodl skal kunne tilbyde. Disse systemer kalder vi for forbilleder. Heriblandt er Forbrugerrådets For Resten, DK-kogebogen, Opskrifter.dk samt flere. Der eksisterer desuden også en række engelske hjemmesider, som ligeledes tilbyder lignende service, som derfor også kunne være interessante at undersøge nærmere. Vi har dog valgt at fokusere på de tre førnævnte systemer: 

\begin{enumerate}[noitemsep]
  \item \href{https://play.google.com/store/apps/details?id=com.nodes.forresten}{For Resten} \cite{forresten}
  \item \href{http://www.dk-kogebogen.dk/}{DK-kogebogen} \cite{dkkogebogen}
  \item \href{http://opskrifter.dk/}{Opskrifter.dk} \cite{opskrifterdk}
\end{enumerate}

Dette er grundet, at de er dansksprogede, og derfor bør anses som værende direkte konkurrenter til Foodl. I følgende afsnit vil de tre systemer blive undersøgt og analyseret, for at finde ud af hvilke kvaliteter og mangler, systemerne har. Derved vil vi opnå en bedre forståelse for, hvad Foodl skal kunne tilbyde brugerne, for at gøre systemet bedre end allerede eksisterende systemer. DK-kogenbogen og Opskrifter.dk er begge web-applikationer, som kan tilgåes fra enhver web-browser, mens For Resten er en mobil-applikation udviklet til iOS- og Android-smartphones. Nogle kriterier er mere relevante at undersøge end andre.Vi vil undersøge og analysere de følgende features:

\begin{itemize}[noitemsep]
  \item Antal opskrifter
  \item Kvalitet af opskrifter
  \item Fleksibilitet
  \item Opskriftssøgningsfunktion
\end{itemize}

Det er relevant at undersøge, hvor mange opskrifter de tre forbilleder har i deres databaser. Jo færre opskrifter de har, jo større er risikoen nemlig for, at man, som bruger, ikke får nogle resultater, når man søger efter opskrifter med en specifik ingrediens. Kvaliteten af opskrifterne er også vigtig. Er hjemmesiden \fx et system, der tillader alle og enhver at uploade deres opskrifter til hjemmesiden, så er der en risiko for at nogle opskrifter vil være dårlige, eller ligefrem ubrugelige. Oplever brugeren, gang på gang, at han/hun, under en søgning, får dårlige eller ubrugelige opskrifter som søgningsresultater, så vil sidens troværdighed mindskes. 
Hjemmesidens fleksibilitet vurderes ud fra om brugeren har mulighed for \fx at op-skalére eller ned-skalére opskrifter således, at de er tilpasset flere eller færre personer; om det er muligt at sortere efter tilberedningstid eller andre ting, og om brugeren har mulighed for at sætte begrænsninger op for, hvilke opskrifter han/hun ønsker skal vises (\fx kun opskrifter uden svinekød, laktose, nødder osv.). Den sidste egenskab, som gruppen ønsker at analysere og undersøge, er opskriftssøgningsfunktionen. Dette er hovedfunktionen for systemet. Her vil vi undersøge, hvordan de tre forbillederne har valgt at bygge deres tøm-køleskabs-funktion op. \Fx hvordan man søger på ingredienser, hvor let tilgængelig funktionen er mm.
