\section{Unit test}
\label{sec:unittests}
For at verificere at programmet opfører sig som vi ønsker, har vi benyttet RSpec, hvilket er et Ruby unit test framework, der først og fremmest er minded mod ``behavior-driven development'', men også kan bruges til at skrive unit tests med. Vi har unit testet bruger-modellen, fordi den er central for systemet, og den er i direkte kontakt med brugeren og kan have kritiske fejl. Derudover har vi også unit testet enkelte forespørgelser (requests i RSpec), der simulerer en brugeres interaktion med systemet. \Fx testes der login-systemet samt det HTML, der vises, efter man har logget ind. Vi har ikke foretaget unit tests af hele systemet, hvilket er grundet nød af ressourcer. Idet pålidelighedskriteriet, specificeret i \chapref{chap:design}, er blevet vurderet som ``mindre vigtigt'', har vi valgt at nedprioritere dem. Eksempler på nogle af de unit test vi har foretaget på bruger-modellen, ved hjælp af RSpec, ses i \lstref{lst:password} og \lstref{lst:email}. For resten af tests refereres der til kildekoden.


Det skal nævnes, at ``@user'' er en instansvariabel defineret tidligere i unit test filen og er et objekt af klassen ``user'' (se \secref{sec:modelfunktion}), der indeholder de attributter og metoder, der hører med til alle instanser af klassen. Eksempel \ref{lst:password} viser en enkel unit test for brugeroprettelsesprocessen. Her skal kodeordfeltet og kodeordbekræftelsesfeltet være præcis ens, ellers kan objektet ikke være valid. Metoden \methodref{valid?} på bruger-objektet er en Active Record metode (bruger-modellen nedarver fra ActiveRecord-klassen) og returnerer blot sand eller falsk, alt afhængigt om objektet kan gemmes i databasen eller ej. Denne metode bliver kaldt på linje 3 i eksemplet.

\begin{lstlisting}[caption={Unit test af brugerens har indtastede password, når han/hun opretter sig som bruger},label=lst:password,language=Ruby]
describe "when password is not present" do
  before { @user.password = @user.password_confirmation = " " }
  it { should_not be_valid }
end
\end{lstlisting}

Eksempel \ref{lst:email} viser unit test af emailadressefeltet, også fundet i brugermodellen. Her testes der på validiteten af emailadressen. Emailadresser i bruger-modellen bliver tjekket af et regular expression, inden objektet gemmes i databasen. Derfor testes der på adresser, der strider imod det regular expression og burde derfor ikke godkendes ved brug af \methodref{@user.valid?}.

\begin{lstlisting}[caption={Unit test af en brugers email er valid, når han/hun opretter sig som bruger},label=lst:email,language=Ruby]
describe "when email format is invalid" do
  it "should be invalid" do
    addresses = %w[user@foo,com user_at_foo.org example.user@foo.
                   foo@bar_baz.com foo@bar+baz.com]
    addresses.each do |invalid_address|
      @user.email = invalid_address
      @user.should_not be_valid
    end
  end
end
\end{lstlisting}

Der er foretaget flere lignende unit test på bruger-modellen samt login-systemet. Der er ikke nogen konkret kodedækningsprocent, da det er svært at vurdere, hvorvidt de forskellige actions er blevet testet.

\subsection{Metode til at finde lighed mellem tekststrenge}
Vi benyttede en ret simpel fremgangsmåde til at unit teste metoden \methodref{CompareStrings}, der tester ligheden mellem to tekststrenge. Vi testede denne funktion i Ruby, ved at initialisere et array som vist i \lstref{lst:testcases}. Der er ikke brugt RSpec til unit tests af dette program.

\begin{lstlisting}[caption={Et eksempel på en række testcases til brug ved unit test.},label=lst:testcases,language=Ruby]
test_cases = [
  ["aa12345", "aa67890", "aa"],
  ["aa12345", "678aa90", "aa"],
  ["aa12345", "67890aa", "aa"]
]
\end{lstlisting}

Arrayet \texttt{test\_cases} fyldes med testdata og forventet værdier. Selve arrayet består af et sub-array, hvor hvert element i nævnte rækkefølge er: \lstinline{[input1, input2, forventet returværdi]}.
For hvert element i arrayet, blev det testet om \methodref{CompareStrings(input1, input2)} returnerede den forventede værdi. Eksemplet er kun et udpluk af 26 test-cases, der fremgår af kildekoden.
