\section{Unittests}
\label{sec:unittests}
For at verificere at programmet opfører sig som vi ønsker, bruger vi RSpec Rails hvilket er et framework for Rails til ``behaviour-driven development'' (også kendt som ``test-driven development''). Vi benytter ikke denne udviklingsmodel og frameworket benyttes her til at lave unittests.
Vi har lagt fokus på at lave unittest for controllers og helpers, fordi modellen er rene ActiveRecord moduler og al signifikant opførsel i views er implementeret i helpers.

\subsection{CompareStrings}
Dataudtrækkeren benyttede ikke Rails frameworket, hvorfor unittesten af dens funktion CompareStrings ikke blev lavet med Rspec Rails. Vi benyttede en ret simpel fremgangsmåde til at unitteste funktionen \methodref{CompareStrings}. Vi unittestede denne funktion I Ruby, ved at initialisere et array:
\begin{lstlisting}
test_cases = 
		[
		["aa12345", "aa67890", "aa"],
		["aa12345", "678aa90", "aa"],
		["aa12345", "67890aa", "aa"]
		]
\end{lstlisting}
I arrayet \textit{text\_cases}, består hvert element af et array, hvor hvert element i nævnte rækkefølge er: [input1, input2, forventet returværdi].
For hvert element i arrayet, blev det testet om \methodref{CompareStrings(input1, inpu2)} returnerede den forventede værdi. Eksemplet er kun et udpluk af 26 test-cases, der fremgår af kildekoden.