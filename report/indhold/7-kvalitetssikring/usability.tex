\section{Test af usability}
\label{sec:usability}

For at teste usability i forbindelse med vores system, har vi benyttet os af Instant Data Analysis-metoden\cite{debida}. Vi har dog kun haft 2 tænkte-højt sessions, hvor det anbefalede er 4-6. Først lavede vi en liste over alle de ting vi gerne ville have testpersonerne til at udforske. Det kunne \fx være at skifte sin adgangskode eller at favorisere en opskrift. Den fulde liste kan ses i \apref{ap:usabilitytest}. På baggrund af denne liste konstruerede vi en case bestående af opgaver, som testpersonerne skulle udføre i en bestemt rækkefølge, for netop at komme omkring alle de udvalgte ting vi ville have udforsket. Vi udførte de to tænkehøjt sessioner på informanterne Keld og Merete i nævnte rækkefølge. Vi havde ikke mulighed for at benytte os af Usability Lab på Cassiopaia, da mange andre grupper havde booket en tid, og vi kunne ikke få tiderne til at passe sammen med vores informanters ønsker. I stedet for at bruge Usability Lab til at overvåge processen, installerede vi programmet Teamwiever på den bærbare computere informanten udførte casen på. De 3 personer, der ikke var ude ved informanten kunne fra deres egen bærbar logge på Teamviewer og se hvad der foregik på skærmen, samtidig med at det indbyggede webcam og mikrofon gjorde det muligt at se informantens reaktion og høre ham tænke højt.

Dokumentation for de 2 tænke-højt sessions kan findes på url'erne:
\begin{url}
youtube.com
youtube.com
\end{url}

Efter de 2 tænke-højt sessions, samlede vi os i grupperummet og lavede en brainstorm over de usability-problemer informanterne var stødt på. Problemerne blev klassificeret enten som kritisk, seriøst eller kosmetisk.

\subsection{Usability-problemer}
 \begin{itemize}
 \item Indkøbsliste
  \begin{itemize}
  \item Han havde problemer med at finde indkøbslisten, der er placeret i toppen (kritisk)
  \item Opdatering af indholdet på indkøbslisten (headeren) skete ikke automatisk (kosmetisk)
  \item Han ønsker en form for bekræftelse, når en opskrift bliver tilføjet til indkøbslisten (kosmetisk)
  \item Nogle ingredienser kunne have mængden 0, fx “0 tsk salt”, “0 citronskal i strimler” (kosmetisk)
  \end{itemize}
 \item Søgning
  \begin{itemize}
  \item Han lavede en stavefejl, og dette førte til at, der ikke kom forslag til råvarer i søgefeltet (seriøs)
  \item Han kommaseparerede de råvarer han ville søge på, hvilket ikke gav nogle autocomplete-forslag (kritisk)
  \end{itemize}
 \item Forside
  \begin{itemize}
  \item Han prøvede at klikke tilbage vha. browserens tilbageknap (seriøs)
   \begin{itemize}
   \item Han klikkede dog på foodl-logoet, der navigerer tilbage til forsiden i andet forsøg 
   \item Han ønskede at man brugte konventionen hus-logo (hjem)
   \end{itemize}
  \end{itemize}
 \item Header og Toolbar
  \begin{itemize}
  \item Både headeren og toolbaren virkede usynlige for brugeren (seriøs)
  \item Funktionen af knapperne i toolbaren var ikke selvbeskrivende (seriøs)
   \begin{itemize}
   \item Han prøvede at få sorteret opskrifterne ved at højreklikke på siden
   \end{itemize}
  \item Han forstod ikke, at “Navn” betød, at der ville blive sorteret i alfabetisk orden (seriøs)
  \end{itemize}
 \item Favorit
  \begin{itemize}
  \item Favoriserede opskrifter forsvinder ikke med det samme, når de bliver fjernes fra favoritter (kosmetisk)
  \end{itemize}
\end{itemize}


\textbf{\ourtable{usabilitytable}{3}{Antallet af usabilityproblemer kategoriseret efter type}
           						 {Type}
       {Informant             	}{Kritiske   & Seriøse   & Kosmetiske}{
\ourrow{Kjeld     	}{2 & 5 & 4}
\ourrow{Merete     	}{0 & 0 & 0}
}
}