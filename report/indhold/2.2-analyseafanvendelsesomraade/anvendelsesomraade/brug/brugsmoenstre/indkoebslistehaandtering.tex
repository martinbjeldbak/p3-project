\brugtabel{Indkøbslistehåndtering}
{Indkøbslistehåndtering igangsættes af \textit{brugeren}. \textit{Brugeren} kan tilføje/fjerne ingrediense direkte fra søgeresultatet (en liste af opskrifter). Når \textit{brugeren} forlader søgeresultatet, så er indkøbslisten gemt, og den kan tilgås fra en vilgår underside på Foodl. Når indkøbslisten tilgås fra en af Foodls undersider, så er redigeringen igangsat. Der kan altid kun være én indkøbsliste af gangen. Når redigeringen er aktiv, så er det muligt at tilføje/fjerne ingredienser og tekststrenge. Tekststrenge bruges som en ekstra funktion for \textit{brugeren}, der giver \textit{brugeren} mulighed for at indtaste varer, der skal købes, men som ikke nødvendigvis er ingredienser, man skal bruge i en opskrift (\fx “toiletpapir”). Indkøbslisten kan fra samme side printes eller slettes. Håndteringen af indkøbslisten afsluttes når \textit{brugeren} lukker ned for redigering - altså forlader indkøbsliste-siden.}
{}
{}
%\brugtabel{Indkøbslistehåndtering}{indkoebslistehaandtering}
{Håndteringen af indkøbslisten følger materialemønsteret\cite[p.~128]{ooad}. I materialemønstret er det muligt for aktøreren, at lave flere handlinger i en vilkårlig rækkefølge, da der i brugmønstret typisk ikke vil være særlig mange tilstande i forhold til handlinger. Håndteringen igangsættes af \textit{brugeren}, se \figref{fig:bm-indkoebslistehaandtering}. \textit{Brugeren} kan tilgå indkøbslisten fra en vilkårlig underside på \Foodl. Når \textit{brugeren} har tilgået indkøbslisten, så er redigeringen igangsat, og det er muligt at tilføje/fjerne varer fra indkøbslisten. Ingredienser bliver ``omdannet'' til varer, når de tilføjes på indkøbslisten. Alle elementer på indkøbslisten er af vare-klassen, hvilket gør det muligt for \textit{brugeren} at indtaste vilkårlige varer, ikke blot ingredienser, der er relateret til \fx en opskrift. \textit{Brugeren} har også mulighed for at tilføje varer eller alle opskriftens ingredienser direkte fra søgeresultatet, der er en liste af opskrifter. Når \textit{brugeren} forlader søgeresultatet, så er indkøbslisten gemt, og den kan stadig tilgås fra en vilkårlig underside på \Foodl. 

Der kan altid kun være én indkøbsliste ad gangen pr. bruger. Fra indkøbslistevisningen er det muligt at printe indkøbslisten samt tømme indkøbslisten for alt indhold. Håndteringen af indkøbslisten afsluttes når \textit{brugeren} lukker ned for redigering - altså forlader indkøbsliste-siden. Redigering startes igen, når man tilgår indkøbslisten, og alle de tilføjede var, der ikke er blevet slettet, vil stadig være tilgængelige.}
{Vare, Opskrift, Person}
{Håndter indkøbsliste}
{Brugsmønster for håndteringen af indkøbslisten, hvilket er relationen mellem ``person''- og ``vare''-klasserne i problemområdet.}

%\caption{Brugmønsteret indkøbslistehåndtering}