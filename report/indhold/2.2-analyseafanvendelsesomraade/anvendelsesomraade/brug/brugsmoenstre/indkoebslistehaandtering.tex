\brugtabel{Indkøbslistehåndtering}
{Indkøbslistehåndtering igangsættes af \textit{brugeren}. \textit{Brugeren} kan tilføje/fjerne ingrediense direkte fra søgeresultatet (en liste af opskrifter). Når \textit{brugeren} forlader søgeresultatet, så er indkøbslisten gemt, og den kan tilgås fra en vilgår underside på Foodl. Når indkøbslisten tilgås fra en af Foodls undersider, så er redigeringen igangsat. Der kan altid kun være én indkøbsliste af gangen. Når redigeringen er aktiv, så er det muligt at tilføje/fjerne ingredienser og tekststrenge. Tekststrenge bruges som en ekstra funktion for \textit{brugeren}, der giver \textit{brugeren} mulighed for at indtaste varer, der skal købes, men som ikke nødvendigvis er ingredienser, man skal bruge i en opskrift (\fx “toiletpapir”). Indkøbslisten kan fra samme side printes eller slettes. Håndteringen af indkøbslisten afsluttes når \textit{brugeren} lukker ned for redigering - altså forlader indkøbsliste-siden.}
{}
{}
