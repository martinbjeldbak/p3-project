\section{Brug}
\label{sec:brug}
Denne analyse af anvendelsesområdets brug, har til formål at gøre det klart, hvilke aktører, der benytter Foodl. Resultatet af analysen af aktører er en mængde aktørbeskrivelser. Derudover analyseres hvilke mønstre, der er for aktørernes brug af Foodl. Resultatet af denne akvitet er en mængde brugsmønstre, der beskrives både i form af en brugsmønsterspecifikation og derefter et tilstandsdiagram. Hver enkelt brugsmønster vedrører en eller flere aktører. Denne relation vises med en aktørtabel, som afslutter aktiviteten brug.

\subsection{Aktører}
Vi har fundet 3 aktører, der vil kunne interagere med Foodl. Disse aktører kan ses i \tableref{table:aktoerbeskrivelser}.

\aktortabelEx{Bruger}
{En person, der ønsker at bruge systemet foodl.dk til at finde opskrifter, der er mulige at lave med de råvarer, som personen er i besiddelse af.}
{Systemets brugere inkluderer mange personer i forskellige aldersgrupper med vidt forskellig erfaring inden for computerbrug.}
{Bruger A er en 23-årig universitetsstuderende, der føler sig sikker med at navigere rundt på internettet og gør det flere gange dagligt. A kan godt lide at afprøve de forskellige funktioner som hjemmesider stiller til rådighed, for at undersøge hvad de gør. A er meget lærenem, når det kommer til at benytte funktioner på hjemmesider. A bor alene, og har pga. supermarkedernes familietilpassede portioner, ofte madrester til overs, som A ønsker at bruge. Systemet bliver brugt til at få madresterne med i aftenens aftensmad. 

Bruger B er en 45-årig familiemor eller -far, der hovedsagligt bruger computeren til arbejdsrelaterede opgaver og til at holde sig opdateret ved at læse nyheder på diverse nyhedshjemmesider. Bruger B benytter systemet til blandt andet at få brugt madrester fra den foregående dags aftensmad eller til at få inspiration til den kommende aftensmad. Bruger B vil være interesseret i at være i stand til at dele f.eks. indkøbsliste med ægtefællen, på tværs af enheder.}
{}

\aktortabelEx{Administrator}{ak-administrator}
{En person, der har til formål at administrere og håndtere eventuelle fejl i systemet, der rapporteres af systemets brugere.}
{Systemets administrator har et højt erfaringsniveau med hensyn til systemet. De har også kontakt til systemets udviklere, der kan rette eventuelle seriøse og systemkritiske fejl.}
{Administrator A er en ubetalt studerende, som håndterer fejl på \Foodl{} i sin fritid. A gennemgår fejlrapporter og vurderer om en fejl er så kritisk at han bør kontakte en systemudvikler, eller om han selv kan udbedre fejlen, \fx ved at slette et dødt link.}
{}


  \begin{tabular}{p{\textwidth}}
    \hline
    \begin{center} \textbf{\textit{Crawler}} \end{center} \\ \hline
    \textbf{Formål:} Et system, der skal besøge foruddefinerede hjemmesider indeholdende opskrifter, som crawleren skal analysere, oversætte og gemme oplysningerne i en database. \\ 
    \textbf{Karakteristik:} Crawleren fungerer systematisk ud fra nogle foruddefinerede parametre. Der findes én crawler til hver opskriftshjemmeside, fordi hjemmesiderne ikke nødvendigvis er opbygget på samme måde. Dette betyder, at de samme parametre ikke nødvendigvis vil gælde for flere opskriftshjemmesider. \\ \hline
  \end{tabular}





\subsection{Brugsmønstre}
I forbindelse med modelleringen af brugsmønstre, har vi fokuseret på fra starten af, at identificere så mange brugsmønsterkandidater som muligt. Vi har benyttet denne teknik i et forsøg på at sikre os, ikke at have overset et vigtigt brugsmønster. Kandidaterne er valgt i forbindelse med en brainstorming, og er hver især blevet analyseret for relevans efter brainstormen. Dette har resulteret i at nogle af kandidaterne er blevet fravalgt. Der har været flere grunde til at vi har fravalgt kandidater:
\begin{itemize}
\item Brugsmønstret har ikke været en del af anvendelsesområdet
\item Brugsmønstret har været for simpelt
\item Brugsmønstret har været været en del af eller magen til et andet brugsmønstre
\end{itemize}
De fravalgte kandidater, samt begrundelsen fra fravælgelsen fremgår af \apref{ap:fravalgtebrugsmoenstre}.

Efter fravælgelsen, står vi tilbage med en række brugsmønstre, som beskriver hændelserne, der vedører en given aktør. Brugmønstrene præsenteres først i form af et tilstandsdiagram, der hurtigt giver et godt visuelt overblik over brugsmønsteret. Hvis der er noget man har brug for en mere detaljeret beskrivelse af, så kan man finde denne beskrivelse i den efterfølgende brugsmønsterspecifikation.
Brugsmønstrene kan ses i \tableref{table:brugmoenstre}


\begin{table}[h]
\brugtabel{Favorisering}
{Favorisering igangsættes af brugeren. Når brugeren har lavet en søgning, og flere forskellige opskrifter vises som søgeresultater, kan brugeren klikke på favorisér-knappen tilhørende den enkelte opskrift, for at favorisere denne. Når en opskrift er blevet favoriseret, tilføjes denne til en liste af favoritopskrifter. Det er muligt at fjerne opskriften fra favoritlisten på to måder. Enten fra samme sted, som favoriseringen blev tilføjet, eller direkte i favoritlisten.}
{}
{}


%Noget er fucked up med brugsmoenstrene, der burde være konverteret. Der mangler x.pdf filerne
%\begin{figure}
%\centering
%\def \svgwidth{\columnwidth}
%\brugtabel{Favorisering}
{Favorisering igangsættes af brugeren. Når brugeren har lavet en søgning, og flere forskellige opskrifter vises som søgeresultater, kan brugeren klikke på favorisér-knappen tilhørende den enkelte opskrift, for at favorisere denne. Når en opskrift er blevet favoriseret, tilføjes denne til en liste af favoritopskrifter. Det er muligt at fjerne opskriften fra favoritlisten på to måder. Enten fra samme sted, som favoriseringen blev tilføjet, eller direkte i favoritlisten.}
{}
{}

%\end{figure}

\end{table}

\begin{table}[h]
\begin{tabular}{p{\textwidth}}
    \hline
    \begin{center} 
    \textbf{\textit{Rapportering}} 
    \end{center} \\ \hline
    \textbf{Brugsmønster:} Rapportering igangsættes af brugeren, når denne opdager en fejl på hjemmesiden. Hvis brugeren opdager en fejl, der har med et søgningsresultat (opskrift) at gøre, så klikkes der på en rapporteringsknap, der er ved det enkelte søgningsresultatet. Når der skal rapporteres en fejl vedrørende opskrifter, så åbnes en dialogboks på siden, hvor brugeren herefter skal vælge en fejltype. Der skelnes mellem beskrivelige og ubeskrivelige fejltyper. Et eksempel på en ubeskrivelig fejltype er bl.a., hvis et link ikke fungerer, som hører under fejltypen “dødt link”. De ubeskrivelige fejltyper behøver ingen beskrivelse, da fejltypen er beskrivelse nok i sig selv. Vælger brugeren derimod en beskrivelig fejltype, så præsenteres en beskrivelsesboks for brugeren, hvor fejlen beskrives med tekst. Derefter er rapporten klar, og brugeren skal nu godkende rapporten, inden den bliver sendt til administratoren.
Derudover er der en generel rapporteringsknap, der vedrører andre, generelle fejl på siden. Når denne knap benyttes, så dirigeres brugeren direkte hen til en beskrivelsesboks, hvor fejlen beskrives med tekst. Til slut skal rapporten godkendes af brugeren, inden den bliver sendt til administratoren. Det er altid muligt at annullere rapporteringen under alle tilstande i brugsmønstret. \\
    \textbf{Objekter:}  \\
    \textbf{Funktioner:}  \\ \hline
\end{tabular}
\todo{INDSÆT BILLEDE AF BRUGSMØNSTER}
\end{table}

\begin{table}[h]
\brugtabel{Fejlhåndtering}
{Fejlhåndering igangsættes af \textit{administratoren}. \textit{Administratoren} logger ind på hjemmesiden, og bevæger sig ind på fejlhåndteringssiden. Her præsenteres en liste af  fejlrapporter, der, via \textit{brugeren}, er blevet rapporteret og dokumenteret i systemet. \textit{Administratoren} kan derefter klikke på en fejlrapport i listen for at se en detaljeret beskrivelse af den givne fejl, som derefter kan håndteres.}
{}
{}

\begin{figure}
\centering
\scalebox{0.7}{
\brugtabel{Fejlhåndtering}
{Fejlhåndering igangsættes af \textit{administratoren}. \textit{Administratoren} logger ind på hjemmesiden, og bevæger sig ind på fejlhåndteringssiden. Her præsenteres en liste af  fejlrapporter, der, via \textit{brugeren}, er blevet rapporteret og dokumenteret i systemet. \textit{Administratoren} kan derefter klikke på en fejlrapport i listen for at se en detaljeret beskrivelse af den givne fejl, som derefter kan håndteres.}
{}
{}

\begin{figure}
\centering
\scalebox{0.7}{
\input{billeder/brugsmoenstre/fejlhaandtering.pdf_tex}
}
\capt{Brugmønsteret fejlhåndtering}\label{fig:bm-fejlhaandtering}
\end{figure}


}
\capt{Brugmønsteret fejlhåndtering}\label{fig:bm-fejlhaandtering}
\end{figure}


\todo{INDSÆT BILLEDE AF BRUGSMØNSTER}
\end{table}

\begin{table}[h]
\brugtabel{Indkøbslistehåndtering}{indkoebslistehaandtering}
{Håndteringen af indkøbslisten følger materialemønsteret\cite[p.~128]{ooad}. I materialemønstret er det muligt for aktøreren, at lave flere handlinger i en vilkårlig rækkefølge, da der i brugmønstret typisk ikke vil være særlig mange tilstande i forhold til handlinger. Håndteringen igangsættes af \textit{brugeren}. \textit{Brugeren} kan tilgå indkøbslisten fra en vilkårlig underside på \Foodl. Når \textit{brugeren} har tilgået indkøbslisten, så er redigeringen igangsat, og det er muligt at tilføje/fjerne varer fra indkøbslisten. Ingredienser bliver ``omdannet'' til varer, når de tilføjes på indkøbslisten. Alle elementer på indkøbslisten er af vare-klassen, hvilket gør det muligt for \textit{brugeren} at indtaste vilkårlige varer, ikke blot ingredienser, der er relateret til \fx en opskrift. \textit{Brugeren} har også mulighed for at tilføje varer eller alle opskriftens ingredienser direkte fra søgeresultatet, der er en liste af opskrifter. Når \textit{brugeren} forlader søgeresultatet, så er indkøbslisten gemt, og den kan stadig tilgås fra en vilkårlig underside på \Foodl. 

Der kan altid kun være én indkøbsliste ad gangen pr bruger. Fra indkøbslistevisningen er det muligt at printe indkøbslisten samt tømme indkøbslisten for alt indhold. Håndteringen af indkøbslisten afsluttes når \textit{brugeren} lukker ned for redigering - altså forlader indkøbsliste-siden. Redigering startes igen, når man tilgår indkøbslisten, og alle de tilføjede var, der ikke er blevet slettet, vil stadig være tilgængelige.

{Vare, Bruger}
{Håndter indkøbsliste}
{Brugsmønster for håndteringen af indkøbslisten, hvilket er relationen mellem ``bruger''- og ``vare''-klasserne i problemområdet.}

\todo{INDSÆT BILLEDE AF BRUGSMØNSTER}
\end{table}

\begin{table}[h]
\brugtabel{Indlogning}
{Indlogning igangsættes af brugeren. Der er tre startmuligheder, og de ender alle tre i samme tilstand “Login aktiv”. Den første mulighed er, hvis brugeren var logget ind fra en tidligere session, så hentes oplysningerne fra denne session automatisk. Den anden mulighed er, at brugeren trykker på “Login / Registrer”, som kan tilgås fra en vilkårlig Foodl-underside. Systemet venter nu på brugerens loginoplysninger. Brugeren indtaster oplysningerne og systemet påbegynder godkendelsesprocessen. Hvis oplysningerne bliver afvist, så skal brugeren genindtaste oplysnignerne. Hvis de bliver godkendt, så bliver brugeren logget ind på siden. Den tredje mulighed er, hvis brugeren ønsker at lave en ny bruger i systemet. Brugeren skal nu indtaste brugernavn og adgangskode i systemet (det er ikke obligatorisk at indtaste e-mail). Hvis der opstår en fejl i oplysningerne, så skal brugeren genindtaste oplysningerne. Når oplysningerne bliver godkendt, så bliver brugeren logget ind med det brugernavn og adgangskode, som er blevet indtastet. På hvilket som helst tidspunkt, har brugeren mulighed for at annullere indlogningsprocessen undervejs.
Hvis brugeren er logget ind på en konto, så bliver brugerens indkøbsliste og favoritter indlæst, som kan tilgås fra en vilkårlig Foodl-underside. Brugeren kan nu oprette eller fjerne favoritter og tilgå og/eller håndtere indkøbslisten.}
{}
{}

\todo{INDSÆT BILLEDE AF BRUGSMØNSTER}
\end{table}

\begin{table}[h]
\brugtabel{Søgning}
{En søgning igangsættes af \textit{brugeren}, ved at indtaste et antal forskellige råvarer og derefter trykke på “søg”. En del af de indtastede råvarer kan også være gemte ingredienser fra tidligere søgninger. Efter en søgning, vises en mængde opskrifter baseret på de råvarer, der er blevet indtastet. Det er muligt at tilføje eller fjerne råvarer, der kan yderligere specificer søgningen.  Opskrifterne sorteres i første omgang efter, hvor godt deres ingredienser matcher de valgte råvarer. \textit{Brugeren} kan vælge en sekundær sortering, hvor \textit{brugeren} har mulighed for at sortere efter bedømmelse eller navn. Opskrifterne kan også filtreres på flere måder (også samtidig), så der kun vises opskrifter uden fx kød, svin og/eller gluten. I søgeresultatet er det muligt at tilføje samt gemme og fjerne begrænsninger og råvarer. Søgningen kan under alle tilstande afsluttes ved at lukke siden.}
{}
{}
\begin{figure}
\centering
\scalebox{0.6}{
\brugtabel{Søgning}
{En søgning igangsættes af \textit{brugeren}, ved at indtaste et antal forskellige råvarer og derefter trykke på “søg”. En del af de indtastede råvarer kan også være gemte ingredienser fra tidligere søgninger. Efter en søgning, vises en mængde opskrifter baseret på de råvarer, der er blevet indtastet. Det er muligt at tilføje eller fjerne råvarer, der kan yderligere specificer søgningen.  Opskrifterne sorteres i første omgang efter, hvor godt deres ingredienser matcher de valgte råvarer. \textit{Brugeren} kan vælge en sekundær sortering, hvor \textit{brugeren} har mulighed for at sortere efter bedømmelse eller navn. Opskrifterne kan også filtreres på flere måder (også samtidig), så der kun vises opskrifter uden fx kød, svin og/eller gluten. I søgeresultatet er det muligt at tilføje samt gemme og fjerne begrænsninger og råvarer. Søgningen kan under alle tilstande afsluttes ved at lukke siden.}
{}
{}
\begin{figure}
\centering
\scalebox{0.6}{
\input{billeder/brugsmoenstre/soegning.pdf_tex}}
\capt{Brugmønsteret søgning}\label{fig:bm-soegning}
\end{figure}}
\capt{Brugmønsteret søgning}\label{fig:bm-soegning}
\end{figure}
\todo{INDSÆT BILLEDE AF BRUGSMØNSTER}
\end{table}

\begin{table}[h]
\brugtabel{Crawling}
{Crawling igangsættes af \textit{crawler}. Under en crawling besøges opskrifter på en opskriftsside én af gangen. \textit{Crawleren} analyserer opskrifterne for at kunne skelne hvilke dele af den der er ingredienser, fremgangsmåde, serveringsstørrelse, billede af retten, m.m. Alle opskrifter, der bliver fundet, bliver omsat så de passer til systemets model af en opskrift, der tilføjes til systemets indeks.}
{}
{}

\todo{INDSÆT BILLEDE AF BRUGSMØNSTER}
\label{table:brugmoenstre}
\end{table}


\subsection{Aktørtabel}
\tableref{table:aktoertabel} viser hvilke brugsmønstre, der vedrører en given aktør. Et flueben betyder at brugsmønstret i samme række vedrører aktøren i samme kollonne.
\begin{table}
  \centering
    \begin{tabular}{ r|c c c }
  \hline
                       &    \multicolumn{3}{c}{\textbf{Brugsmønstre}}   \\ 
\textbf{Aktører}       & Bruger     & Administrator & Crawler    \\ \hline 
Søgning                & \checkmark &               &            \\ 
Favorisering           & \checkmark &               &            \\ 
Indkøbslistehåndtering & \checkmark &               &            \\ 
Login                  & \checkmark & \checkmark    &            \\ 
Fejlhåndtering         &            & \checkmark    &            \\ 
Rapportering           & \checkmark & \checkmark    &            \\ 
Crawling               &            &               & \checkmark \\
    \hline
    \end{tabular}
    \capt{Aktørtabel for Foodl.}
    \label{table:aktoertabel}
\end{table}






