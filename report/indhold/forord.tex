\chapter*{Forord}
Denne rapport beskriver udviklingen af systemet \Foodl, der kan benyttes på hjemmesiden www.foodl.dk. \Foodl er et system, der gør det lettere at finde en anvendelse af sine madrester, i stedet for blot at smide dem ud. \Foodl søger blandt opskrifter, der er at finde på Arla's hjemmeside www.arla.dk, og præsenterer en opskrift på \Foodl, hvor fremgangsmåden er udeladt. Det er nødvendigt for en bruger at følge et link til Arla's hjemmeside for at få vist hele opskriften.

Udviklingsprocessen er baseret på objektorienterede analyse og design-teknikker\cite{ooad}, teknikker til design og evaluering af brugergrænseflader\cite{deb} og inddragelse af 2 informanter, Merete og Kjeld, der ved hjælp af deres viden og erfaringer i et køkken har været med til at udforme systemet og afprøve dette. Informanterne er blevet inddraget ved forskellige aktiviteter, herunder interviews, test af prototyper og en usability-test på det endelige system.

\section*{Læsevejledning}
