\chapter*{Forord}
Denne rapport beskriver udviklingen af systemet \Foodl{} - en webapplikation, som er tilgængelig på \url{http://www.foodl.dk}. \Foodl{} er et system, der gør det lettere at få inspiration til, hvordan man kan anvende sine madrester, så man kan undgå at smide dem ud. \Foodl{} søger blandt opskrifter, fra Arla Foods hjemmeside \url{http://www.arla.dk}, og præsentere de opskrifter, som indeholder flest af de madvarer, som brugeren har indtastet. Det er nødvendigt for brugeren at følge et link til Arla's hjemmeside, for at få vist hele opskriften.

Udviklingsprocessen er baseret på teknikker fra Objektorienterede Analyse og Design \cite{ooad}, teknikker til Design og Evaluering af Brugergrænseflader \cite{deb} og inddragelse af 2 informanter, Merete og Keld, der ved hjælp af deres viden og erfaringer i et køkken, har været med til at udforme systemet og afprøve dette. Informanterne er blevet inddraget ved forskellige aktiviteter, herunder interviews, test af prototyper og en usability-test af det endelige system.

