\chapter*{Forord}
Denne rapport beskriver udviklingen af systemet \Foodl{} - en webapplikation, som er tilgængelig på \url{http://foodl.dk}. \Foodl{} er et system, der gør det lettere at få inspiration til, hvordan man kan anvende sine madrester, så man kan undgå at smide dem ud. \Foodl{} søger blandt opskrifter, fra Arla Foods hjemmeside \url{http://arla.dk}, og præsenterer de opskrifter, som indeholder flest af de madvarer, som brugeren har indtastet. Det er nødvendigt for brugeren at følge et link til Arla's hjemmeside, for at få vist hele opskriften og dens fremgangsmåde.

Udviklingsprocessen er baseret på teknikker fra Objektorienteret Analyse \& Design \cite{ooad}, teknikker til Design og Evaluering af Brugergrænseflader \cite{deb} og inddragelse af to informanter, Merete og Keld, der ved hjælp af deres viden og erfaringer i et køkken, har været med til at udforme systemet og afprøve dette. Informanterne er blevet inddraget ved forskellige aktiviteter, herunder interviews, test af prototyper og en usability-test af det endelige system.

Projektet er delt op i to overordnede dele, der er i denn rapport er præsenteret som ``Del 1 - Udviklingsrapport'' og ``Del 2 - Akademisk rapport''. 

Udviklingsrapporten beskriver processen og de tanker og overvejelser, vi har gjort os i forhold til udviklingen af produktet. Det er her, vi dokumenterer hele processen af systemudviklingen af \Foodl{}. I den akademiske rapport beskriver, reflekterer og perpsektiverer vi i forhold til udviklingsmetoden i projektet.

Logoet på forsiden repræsenterer programmets navn, Foodl.

Systemets kildekode kan finde via følgende link: \url{https://bitbucket.org/Acolarh/p3-project}

\textbf{Tak til}
\begin{itemize}[noitemsep]
	\item Informanterne, Merete og Keld
	\begin{itemize}[noitemsep]
		\item Tak for samarbejdet igennem hele projektet
	\end{itemize}
	\item Arla
	\begin{itemize}[noitemsep]
		\item Tak for brug af opskrifter
	\end{itemize}
\end{itemize}