\subsection{Designovervejelser}
\label{subsec:designovervejelser}

Under designet af \Foodl{}'s brugergrænseflade har vi gjort os nogle designmæssige overvejelser for at øge kvaliteten af programmet. Vi har blandt andet taget højde for følgende principper for god usability\cite[s.~90]{deb}:

\begin{description}
\item[Visibility] \hfill \\
Vi har valgt, at de vigtigste funktioner altid skal være synlige for brugeren. Det omfatter links til at navigere til siderne \textit{indkøbsliste}, \textit{favoritter}, \textit{log ind / opret bruger} og \textit{forside}. De nævnte links er altid synlige allerøverst på siden i sidehovedet.
 
\item[Consistency] \hfill \\
Vi er konsistente med måden hvorpå knapper og lignende, hvorfra brugeren kan interagere med systemet, ser ud. Bl.a.\ bliver knappergrønne, når musen føres over dem. Knapper har den samme firkantede form med svagt afrundede hjørner, grålig baggrund og mørkere grå kanter. Har en knap tekst på sig, er teksten som udgangspunkt grøn. Hvis knappen er blevet trykket i bund og altså er aktiveret, er skriften orange. Hvis en knap er inaktiv, og man ikke kan trykke på den, så er kontrasten mellem knappens baggrund, tekst og kant meget lav. Dette kaldes normalt \textit{grayed out}, og er måden man typisk viser, at en knap ikke kan trykkes på.

Logoet linker til forsiden, hvilket er ret normalt for de fleste hjemmesider. I og med at det er normalt og velkendt af mange personer, så er det menneskers evne til at genkende ting, vi prøver at ramme ved at lave logoet som en ``hjem''-knap.

En opskrifts vurdering kan læses som 5 stjerner, hvoraf nogle af stjernerne bliver farvet fra venstre mod højre, ligesom folk kender det fra \fx avisers anmeldelser af film.

\item[Familiarity] \hfill \\
Vi benytter symboler, som folk i forvejen kender, som \fx krydset man trykker på for at slette en råvaretype, man har indtastet, eller når man vil slette en ingrediens fra indkøbslisten. Når man skal tilføje en ingrediens til indkøbslisten trykker man på et plus-symbol, som ofte forbindes med at tilføje noget.

Når en søgning er sat i gang, så vises en cirkulær animation, som ofte bruges, når en proces, der kan tage et stykke tid, er i gang.   

\item[Affordance] \hfill \\
Vi har designet \Foodl{}'s logo, samt forsiden hvor logoet præsenteres, på en sådan måde at folk med det samme får et indtryk af, hvad systemet kan. Logoets tekst, \textit{Foodl}, får tankerne over på mad, hvilket yderligere forstærkes af, at det ene ``O'' er udskiftet med en ananas. Navnet Foodl minder en smule om Google, og har til formål at lede folks tanker i retning af en søgemaskine, hvilket yderligere forstærkes af sidehovedet og forsidens design, der minder en hel del om Googles søgemaskine.

Når man skal favorisere en opskrift, altså tilkendegive at man synes om denne, så skal man trykke på en knap med et hjertesymbol. Hjertet skal gerne give folk en ide om, at man altså skal trykke på knappen, hvis man synes om opskriften.

Indkøbslisten ligner et rigtig linieret A4-papir. Varer på listen vises med en skrifttype, der ligner håndskrift, dog ikke i så høj grad, at det bliver svært at læse teksten. Vi benytter altså en metafor her, ved at tage et kendt objekt og mappe det til noget virtuelt, der ser ud og fungerer på samme måde.

En stjerne bliver associeret med noget godt. Derfor viser vi et antal stjerner ved en opskrift, for at folk skal forbinde det med, hvor god en opskrift er.

\item[Navigation] \hfill \\
Vi har forsøgt at gøre navigation imellem \Foodl's sider mere enkel ved at have et sidehoved, der sørger for, at de overordnede navigeringsmuligheder imellem undersider altid er synlige.

Hvis man ikke er logget ind, kommer der en sort firkantet boks frem med teksten ``Husk at oprette en bruger for at gemme din indkøbsliste og favoritter''. I boksen er der en pil, der peger lige netop på det link i sidehovedet, man skal trykke på for at oprette en bruger.

Når man har udført en søgning, får man allerførst vist alle resultater. Derefter kan man sortere og begrænse søgningen. Dette opfylder designprincippet \textit{Overblik først, derefter detaljer}.

\item[Control] \hfill \\
Når musen er over en knap, der kan interageres med, bliver knappens baggrund grøn. Så ved man, at man kan trykke på knappen, og på samme måde finder man hurtigt ud af, at en slider kan trækkes i.
  
Når der trækkes i slideren, der skalerer opskrifterne i et søgeresultat, så ændres antal personer og ingrediensernes mængder sig med det samme, og ikke først når slideren er sluppet. På den måde får brugeren bedre kontrol og kan hurtigt ramme den rigtige skalering.

\item[Feedback] \hfill \\
Systemet giver brugeren feedback, så man er klar over, at systemet har reageret på interaktionen. Brugeren bliver informeret, når noget er gået galt, eller hvis han har lavet en fejl.

En animation af en cirkulær bevægelse viser, at en søgning er i gang. På den måde tror brugeren ikke, at han har klikket ved siden af en knap, hvis søgning blot tager lidt længere tid end forventet.

En besked popper op på siden, når brugeren skal have feedback. Det kan \fx være, når brugeren har oprettet en bruger, eller har indtastet en forkert kode i et forsøg på at logge ind.

\item[Revocery] \hfill \\
Hvis en bruger har glemt sin kode, kan man få den tilsendt igen.

\item[Style] \hfill \\
Vi har også gjort os nogle overvejelser, vedrørende hvilke farver vi benytter i systemet. Igennem hele systemet har vi benyttet nogle få farver, hvilket stemmer overens med regler for godt design af grænseflader \cite[s. ~344]{deb}. Vi har forsøgt at følge vestlige farvekonventioner. 

Den lidt kølige og grønne baggrundsfarve, som kan ses igennem hele systemet, fungerer godt som en rolig baggrundsfarve, som også er indbydende for brugeren. I følge farvekonventionen så er den kølige farve god til baggrundsinformation. Den grønne farve er beskrevet som en tryg og klar farve \cite[s. ~344]{deb}. Der bliver ikke signaleret noget hastende. Dvs., at brugeren skal tage sig tid til at kigge lidt på siden, og se hvilke funktioner der er på forsiden inden der bliver foretaget nogle søgninger.

Vi har brugt nogle mere advarende farver til \fx fejlmeddelelser eller knapper, der skal rapportere en fejl i systemet. Fejlmeddelelser bliver præsenteret i en rød boks, som, ifølge de vestlige farvekonventioner, signalerer farer, med hvid tekst, som signalerer neutralitet. Vi vurderede, at disse to farver var en god kombination, fordi den hvide farve neutraliserer den farlige røde en smule. Derudover har vi en rapporteringsknap, som er farvet gul, som signalerer forsigtighed. Man skal ikke bare klikke på denne knap hele tiden. Den skal bruges til nogle alvorlige rapporteringer.

\item[Conviviality] \hfill \\
Hvis en bruger laver en fejl, så forholder vi os relativt roligt og viser brugeren en lille tekstmeddelelse med feedback. Vi afspiller ikke en lyd, der får folk til at spilde kaffen.

Når noget er gået galt, giver vi ikke brugeren skylden for dette direkte. Hvis en person forsøger at logge ind uden at have indtastet gyldig email-adrasse, så skriver vi blot \textit{Angiv en email-adresse}, i stedet for \textit{Du har indtastet en forkert email-adresse}. På samme måde med alle fejlmeddelelser, bebrejder vi ikke brugeren direkte.
\end{description}

