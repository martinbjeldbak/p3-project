\subsection{Designovervejelser}
\label{subsec:designovervejelser}

%Igennem hele systemet benytter vi os af nogle få farver. Vi har forsøgt at følge vestlige farvekonventioner\cite[p. ~344]{deb} (når vi beskriver de brugte farver, så forklarer vi også, hvad konventionen ved farven er). Den lidt kølige og grønne baggrundsfarve, som bl.a. kan ses på \figref{fig:overblik-forside}, fungerer godt som en rolig baggrundsfarve, som også er indbydende for brugeren. I følge farvekonventionen så er den kølige farve god til baggrundsinformation. Den grønne farve er beskrevet som en tryg og klar farve. Der bliver ikke signaleret noget hastende. Dvs., at brugeren skal tage sig tid til at kigge lidt på siden, og se hvilke funktioner der er på forsiden inden der bliver foretaget nogle søgninger.

%Det er meget deb-agtigt, at vi er konsistente med knappers farve når der klikkes, en knaps runde form og så videre

%Kan man ikke sige, at vi forsøger at skabe lidt affordance med vores 
%logo. Altså allerede med navnet og billedet af ananasen forsøger vi 
%at fortælle bruger en smule om hvad systemet kan


\url{https://docs.google.com/document/d/1VlVwRcCxbYr7QAJfe6z3doNCTkKKlDAsf9Cw7ng4L4E/edit}