\paragraph{Brugbart} Systemet skal være håndgribeligt for brugeren. Det skal være nemt at bruge og nemt at gå til uden nogen form for oplæring, fordi vi vurderer ud fra interviews med informanterne, at der ikke er nogen, der ønsker at bruge meget tid på at skulle sætte sig ind i vores system, som ifølge vurdering af brugbart, burde være meget ligetil. Dette medfører, at systemet skal være intuitivt, og vi mener, at vi opnår dette ved at designe systemet på en lignende måde som andre velkendte systemer, såsom Googles søgefunktion og udseende. Hvorfor tages der udgangspunkt i præcis Google? Google er den mest brugte søgemaskine i verden, efterfulgt af Bing og Yahoo.\cite{googlesoeg} \cite{ebizmba} Derudover er Google også den meste besøgte hjemmeside på internettet, efterfulgt af Facebook og Youtube. \cite{alexadk} Vi ønsker at nå op på et højt abstraktionsniveau, ved at holde systemet simpelt og intuitivt som Google/Facebook/Youtube gør. Det gør det nemmere for brugeren at forstå systemet. Brugeren opnår en høj forståelighed for systemets brug som følge af, at vi ønsker at gøre systemet så brugbart som muligt. Vi ønsker at brugeren skal være i stand til at kigge på systemet og genkende funktioner fra lignende systemer, såsom Google og Facebook. På den måde benytter vi os af menneskers evne til at genkende ting, og det gør det mere intuitivt, da de i forvejen kender funktionerne fra de velkendte systemer og skal ikke til at lære en hel ny måde, at gøre tingene på.

\paragraph{Sikkert} Systemet behandler ikke personfølsomme oplysninger og håndterer ikke noget, liv afhænger af. Derfor vurderes sikkerhedskriteriet som irrelevant for systemets endelige design.

\paragraph{Effektivt} Det er vigtigt, at søgninger og navigation i systemet foregår hurtigt og responsivt. Alle ved, at det er utroligt irriterende, hvis man bruger et system, hvor søgefunktioner og navigation ikke reagerer i løbet af relativt kort tid, bliver man hurtigt træt af det pågældende system og kan finde på at benytte et andet system. Hvis der går for mange sekunder, før der sker noget på skærmen, så kan brugeren begynde at klikke på de samme funktioner flere gange, fordi man tror, at de måske ikke har klikket korrekt i første forsøg, og det er trættende i længden og er ikke hvad systemet er beregnet til. Marissa Mayer, tidligere leder hos Google, nu præsident og direktør for Yahoo udtalte følgende: \cite{googlespeed}
\begin{quote}
``When the Google Maps home page was put on a diet, shrunk from 100K to about 70K to 80K, traffic was up 10 percent the first week and in the following three weeks, 25 percent more.''
\end{quote}  

Man kan i sidste ende vælge at forlade systemet, fordi det er langsomt og ineffektivt, og dette viser statistikken for Google Maps hjemmeside også, da hjemmesidens trafik steg med 25 \% på tre uger efter hjemmesiden blev komprimeret og effektiviseret bare en lille smule. Dette medvirkede også til, at hjemmesiden blev hurtigere til at reagere på brugerinteraktionen. Både Google og deres brugere får glæde af dette.

\paragraph{Korrekt} Korrektheden kategoriseres som mindre vigtigt, fordi gruppens fokus ligger på brugbarhed og effektivitet. Èt forkert søgeresultatet her og der, eller en gemt opskrift, der alligevel ikke blev bogmærket, kan måske skræmme nogle brugere væk, men det er ikke livsnødvendigt, at dette punkt i systemet skal konstant være 100 \% korrekt.

\paragraph{Pålideligt} Pålideligheden vurderes som mindre vigtig, fordi konsistens og nøjagtighed ikke er vigtige aspekter at tage hensyn til, da det ikke er vurderet som et stort problem, hvis systemet giver ét forkert søgeresultat. Dog mener vi, at fejltolerance kan vurderes en smule højere end de andre underkriterier. Det er i øvrigt utroligt ineffiktivt, hvis et system går ned, fordi brugeren har udført en handling, der ikke er taget højde for; hvilket kan medfør negative konsekvenser eller at hele systemet går ned. Systemet skal være i stand til at melde tilbage til brugeren, at der er sket en fejl uden, at systemet går ned. Derudover mener vi, at det er en god idé at have en beskrivende fejlbesked, når der opstår en fejl, så brugeren kan læse sig til, hvad der er gået galt så brugeren ikke føler sig helt forvirret, hvis der sker noget, der ikke må ske.

\paragraph{Vedligeholdbart og fleksibelt} Det er vigtigt for implementeringen af systemet, at det er nemt at udbygge, videreudvikle og vedligeholde. I og med at gruppen benytter sig af en iterativ arbejdsproces, så er det vigtigt, at gruppen kan vende tilbage til koden efter \fx en måned og, uden problemer eller forsinkelser, være i stand til at læse og modificere i systemets funktioner. Endnu en grund til, at vedligeholdelse og fleksibilitet er vigtige kriterier, er at muligheden for fremtidige udvidelse til \fx andre sprog skal være overskueligt.

\paragraph{Testbart} Testbarhed hænger meget sammen med kriterierne vedligeholdbarhed og fleksibilitet, og derfor skal underkriterierne vurderes på samme niveau. Det er relevant at holde implementeringen simpel og modulær samt selvbeskrivende, men hvis man vurderer tests højt og laver systemet meget testbart, stider det imod effektivitet af programmet, da der skal laves nogle ekstra tilføjelser, der kan sløve programmet ned. Testbarhed spiller dermed en væsentlig rolle i systemet, og vurderes som mindre vigtigt. Dvs., at systemet stadig skal testes, bare ikke i så høj en grad, som hvis sikkerhed og korrekt var vurderet som mere vigtig, end de er nu.

\paragraph{Forståeligt} Forståelighed er vigtigt, fordi hvis systemet skulle lanceres i virkeligheden, kunne man umuligt vide, hvem der i fremtiden ville komme til at benytte systemet. Vi vil genbruge velkendte mønstre og designprincipper samt forsøge at gruppèr ansvaret af komponenter og klasser. Vi vil standardisere designsproget således at uanset hvor man er i systemet, vil det være muligt at genkende nogle elementer. Det leder til en forbedret konsistens mellem de forskellige views i grænsefladen. Alt dette gøre vi for at nå et højere abstraktionsniveau, der øger forståeligheden af systemet og derfor vurderes forståelighed som vigtigt. 
 
\paragraph{Genbrugbart} Gruppen ønsker ikke at tage højde for, at andre systemer skal være i stand til at benytte foodl, da det ikke er beskrevet i systemdefinitionen fordi vi er ikke interesseret i at bruge kræfter på at gøre det nemt for andre systemer at benytte vores før vi har et fungerende brugergrænseflade. Derfor er dette kriterie irrelevant for projektet, men ville dog være mere relevant, hvis dette projekt skulle fortsætte i fremtiden. Med forståelighed vurderet som vigtig skulle dette stykke arbejde være meget håndgribeligt.

\paragraph{Flytbart} Systemet skal kunne flyttes frit mellem forskellige tekniske platforme, såsom Windows og Unix hvis det bliver nødvendigt. For eksempel skal det være muligt at køre systemet på andre webhostingsservicer eller skalere systemet op hvis behov overstiger det tekniske systems ydelse. Derfor skal gruppen tænke over hvilken eksterne systemer, der bliver benyttet i det færdige produkt. Nogle eksterne systemer har platformsafhængigheder, der begrænser valget af platforme, som produktet kan køre på. Derfor skal gruppen være i stand til at konsekvent benytte cross-platform biblioteker og implementere selve systemet, sådan at det er let at porte over til andre platforme.

\paragraph{Integrerbart} Da modularitet og standardisering af datarepræsentationer er en væsentlig del af implementationen af projektet og kriterierne fleksibilitet samt vedligeholdbarhed er vurderet som vigtige, vil vi fortage nogle overvejelder i forhold til integrerbarhed. Det vurderes som mindre vigtig fordi systemet ikke skal benyttes i andre systemer (se ``genbrugbart''). I forhold til at koble på andre systemer refereres til ``flytbart'' kriteriet ovenover.
