\subsection{Begrundelser for hvert kriterie}
\label{sec:begrundelsekriterier}

%\begin{table}[H]
%  \centering
%    \begin{tabular}{ r | c  c  c  c  c }
%  \hline
%  & \multicolumn{5}{c}{\textbf{Vigtighed}} \\
%   \textbf{Kriterium}    & Meget vigtig &  Vigtig  & Mindre vigtig  & Irrelevant  & Trivielt \\ \hline
%         Brugbart        & \checkmark            &                    &                         &                      &                   \\ 
%         Sikkert         &                       &                    &                         &   \checkmark         &                   \\ 
%         Effektivt       &                       &    \checkmark      &                         &                      &                   \\ 
%         Korrekt         &                       &                    &    \checkmark           &                      &                   \\
%         Pålideligt      &                       &                    &   \checkmark            &                      &                   \\ 
%         Vedligeholdbart &                       &   \checkmark       &                         &                      &                   \\
%         Testbart        &                       &                    &  \checkmark             &                      &                   \\ 
%         Fleksibelt      &                       &  \checkmark        &                         &                      &                   \\ 
%         Forståeligt     &                       &   \checkmark       &                         &                      &                   \\ 
%         Genbrugbart     &                       &                    &                         &       \checkmark     &                   \\ 
%         Flytbart        &                       &                    &   \checkmark            &                      &                   \\ 
%         Integrerbart    &                       &                    &   \checkmark            &                      &                   \\
%    \hline
%    \end{tabular}
%    \capt{Oversigt over vigtigheden af designkriterierne for projektet.}
%    \label{table:kriterietabel}
%\end{table}

\ourtable{kriterietabel}{5}{Oversigt over vigtigheden af designkriterierne for projektet.}
                                                            {Vigtighed}
       {Kriterium      }{ Meget vigtig   & Vigtig         & Mindre vigtig  & Irrelevant     & Trivielt       }{
\ourrow{Brugbart       }{ \checkmark     &                &                &                &                }
\ourrow{Sikkert        }{                &                &                & \checkmark     &                }
\ourrow{Effektivt      }{                & \checkmark     &                &                &                }
\ourrow{Korrekt        }{                &                & \checkmark     &                &                }
\ourrow{Pålideligt     }{                &                & \checkmark     &                &                }
\ourrow{Vedligeholdbart}{                & \checkmark     &                &                &                }
\ourrow{Testbart       }{                &                & \checkmark     &                &                }
\ourrow{Fleksibelt     }{                & \checkmark     &                &                &                }
\ourrow{Forståeligt    }{                & \checkmark     &                &                &                }
\ourrow{Genbrugbart    }{                &                &                & \checkmark     &                }
\ourrow{Flytbart       }{                &                & \checkmark     &                &                }
\ourrow{Integrerbart   }{                &                & \checkmark     &                &                }
}


Gruppen er i sammenråd med vores informanter, blevet enige om, hvad der rent designmæssigt er væsentligt for at \Foodl{} bliver mest attraktivit. Her under en liste over kriterierne og en vurdering af kriteriernes betydning for \Foodl{}. Vurderingerne er ligeledes opsummeret i \tableref{table:kriterietabel}.

\paragraph{Brugbart (meget vigtig)} 
Systemet skal være håndgribeligt for brugeren. Det skal være nemt at bruge og nemt at gå til uden nogen form for oplæring, fordi vi har vurderet, ud fra interviews med informanterne, at de fleste personer, ikke ønsker at skulle bruge meget tid på at skulle sætte sig ind i vores system. 

\paragraph{Sikkert (irrelevant)} 
Systemet behandler ikke personfølsomme oplysninger, som \fx kreditkortnumre. Folks sikkerhed afhænger på ingen måde af systemet er sikkert, som \fx systemer der styrer lyskryds gør. Sikkerhedskriteriet vurderes som irrelevant for systemets endelige design.

\paragraph{Effektivt (vigtig)} 
Generelt er det vigtigt, at søgninger og navigation i en webapplikation foregår hurtigt og responsivt. Hvis man bruger en webapplikation, hvor søgefunktioner og navigation ikke reagerer i løbet af relativt kort tid, bliver man hurtigt træt af det pågældende system og kan finde på at benytte et andet system. Marissa Mayer, tidligere leder hos Google, nu præsident og direktør for Yahoo, udtalte følgende: \cite{googlespeed}

\begin{quote}
``When the Google Maps home page was put on a diet, shrunk from 100K to about 70K to 80K, traffic was up 10 percent the first week and in the following three weeks, 25 percent more.''
\end{quote}  

Det skal være hurtigt at bruge \Foodl{} til at finde en anvendelse af sin madrest i en opskrift. Det er utrolig hurtigt blot at smide madresten i skraldespanden, så hvis folk skal bruge \Foodl{}, så er det vigtigt at de ikke mister tålmodigheden fordi de hele tiden skal vente på at en proces på \Foodl{} bliver færdig. 

\paragraph{Korrekt (mindre vigtig)} 
Korrektheden kategoriseres som mindre vigtigt, fordi gruppens fokus ligger på brugbarhed og effektivitet. Et par forkerte søgeresultater eller en bogmærket opskrift, der alligevel ikke blev bogmærket, kan måske skræmme nogle brugere væk, men det er ikke et livsnødvendigt kriterie for systemet. Brugere vil stadig kunne få inspiration, til hvordan de kan bruge deres madrest, selvom 1 ud af 10 opskrifter måske ikke passer særlig godt på den madrest de vil bruge.

\paragraph{Pålideligt (mindre vigtigt)} 
Pålideligheden vurderes som mindre vigtig, fordi det indebærer konsistens og nøjagtighed, hvilket vi anser som indre vigtigt. Hvis mængden af en ingrediens er en smule forkert er systemet stadig brugbart. Brugere kan endda klikke ind på den originale opskrift og se de helt korrekte mængder. Hvis søgemaskinen ikke er konsistent og ikke altid finder de samme opskrifter med den samme søgning, så gør det heller ikke det store. Det vil faktisk give brugeren endnu bedre inspiration til hvordan man kan anvende sine madrester, end ved blot at vise samme rækkefølge af søgeresultaterne hver gang samme søgning foretages.

Det ville dog være et problem, hvis et system går ned, fordi brugeren har udført en handling, der ikke er taget højde for. Systemet skal være i stand til at melde tilbage til brugeren, at der er sket en fejl uden at systemet går ned.

\paragraph{Vedligeholdbart og fleksibelt (vigtig)} 
Det er vigtigt for implementeringen af systemet, at det er nemt at udbygge, videreudvikle og vedligeholde. I og med at gruppen benytter sig af en iterativ arbejdsproces, så er det vigtigt, at gruppen kan vende tilbage til koden efter \fx en måned og, uden problemer eller forsinkelser, være i stand til at læse og modificere i systemets funktioner. For at kunne udvide systemet til andre sprog eller tilføje opskrifter fra mange forskellige sider, er der god grund til at vedligeholdelse og fleksibilitet er vigtige kriterier.

\paragraph{Testbart (mindre vigtig)} 
Systemet kan gøres testbart ved at dele koden op i mange små funktioner, så hver af disse kan testes individuelt. Men opdelingen i for mange små stumper kode kan medføre en lavere effektivitet. Det er vigtigt at \Foodl{} er effektivt, så vi vurderer testbarheden mindre vigtigt, men har det i baghovedet at vi til en vis grad selvfølgelig skal kunne teste om systemet lever op til de funktioner, der kræves.

\paragraph{Forståeligt (vigtig)} 
\Foodl{} er ikke et system, der vil have et fast antal brugere. Derfor er det svært at forudsige behovet for antallet af udviklere tilknyttet systemet. Hvis \Foodl{} bliver meget populært kan der være behov for at få hjælp fra flere udviklere, og derfor er det vigtigt at systemet er forståeligt, så nye udviklere forholdsvist hurtigt kan komme i gang.

\paragraph{Genbrugbart (irrelevant)} 
Da systemdefinitionen ikke stiller krav om at \Foodl{} skal kunne bruges af andre systemer vurderes dette som irrelevant.

\paragraph{Flytbart (mindre vigtig)} 
Systemet skal kunne flyttes frit mellem forskellige tekniske platforme, såsom Windows og Unix, hvis det bliver nødvendigt. \Fx skal det være muligt at køre systemet på andre webhostingsservices eller skalere systemet op, hvis behovet overstiger det tekniske systems ydelse. Derfor skal vi tænke over hvilke eksterne systemer, der bliver benyttet i det færdige produkt. Nogle eksterne systemer har platformsafhængigheder, der begrænser valget af platforme, som produktet kan køre på. Derfor skal gruppen være i stand til konsekvent at benytte cross-platform-biblioteker og implementere selve systemet, sådan at det er let at flytte over til andre platforme. Det er ikke sikkert at \Foodl{} bliver så populært at webhosten ikke ydelsesmæssigt kan følge med og det er også uvist om en anden teknisk platform vil klare den belastning bedre. Derfor vurderes kriteriet som mindre vigtigt.

\paragraph{Integrerbart (mindre vigtig)} 
Da modularitet og standardisering af datarepræsentationer er en væsentlig del af implementationen af projektet, og kriterierne fleksibilitet samt vedligeholdbarhed er vurderet som vigtige, vil vi foretage nogle overvejelser i forhold til integrerbarhed. Det vurderes som mindre vigtig fordi systemet ikke skal benyttes i andre systemer (se ``genbrugbart''). I forhold til at koble på andre systemer refereres til ``flytbart''-kriteriet ovenover.
