\subsection{Model-View-Controller}
\label{subsec:mvc}
3-lag arkitekturen minder meget om Model-View-Controller mønstret, som vi benytter og kort har beskrevet i \secref{subsec:mvc}. Dog skal det nævnes, at de ikke er helt ens. I en 3-lags komponentarkitektur er det ikke muligt for præsentationslaget at tilgå modellen. MVC-mønstret er mere trekantet idet, at views kan tilgå både funktionskomponenten og modelkomponenten og dermed blive opdateret direkte fra modellen\cite{designpatterns}. MVC-arkitekturen kan ses i \figref{fig:railsmvc}. Der findes mange forskellige frameworks, der bygger på MVC-arkitekturen. Fordelen ved at benytte et eksisterende og anerkendt mønster, er dets mange funktioner og brugervenlighed, der gør det muligt at hurtigt kunne implementere de nødvendige dele af systemet.

Arkitekturen består af tre hovedkomponenter: model, view og controller. Modelkomponenten bærer de data, som typisk er repræsenteret som tabeller, i en database, der skal kunne manipuleres. Controllerkomponenten fungerer som et slags bindeled mellem modellen og views ved at sende forespørgsler til modellen eller views, alt afhængig af brugerinputtet fra de tilsluttede views. Derudover sender controlleren også forespørgsler i form af tilstandsændringer til modellen, hvis brugeren \fx opdaterer noget i et view. Et view præsenterer brugeren for data i modellen og muliggør ændringer i tilstanden af modellen.
