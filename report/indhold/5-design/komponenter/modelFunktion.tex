\subsection{Model- og funktionskomponent}
\label{sec:modelfunktion}

Model- og funktionskomponenterne repræsenterer objekterne og deres funktionalitet i problem- og anvendelsesområderne. Vi opdaterer klassediagrammet fra problemområdet i \secref{sec:struktur} ved at benytte klassernes adfærd beskrevet i \secref{sec:adfaerd} til at danne attributter og nye klasser. Operationerne på klasserne er blevet til ud fra aktørernes brugsmønstre i anvendelsesområdet, som er beskrevet i \chapref{chap:analyseafao}. Klasserne problemområdet, som er blevet opdateret eller ændret på anden vis, er beskrevet nederfor. De opdaterede klasser kan ses i \figref{fig:klasseDes}.


\pdffig{klassediagramDes}
	{Det endelige klassediagram.}
	{fig:klasseDes}


Klassen \classref{Bruger} er en abstrahering fra de personer, der kan bruge systemet. En bruger har et adgangsniveau, som kan være en bruger, der har / ikke har administratorrettighedher (AR). Dette er modulleret som en note ved adgangsniveau i \figref{fig:klasseDes}. En bruger med AR kan læse og rette fejl, der er indberettet af en bruger uden AR. Derudover kan en bruger med AR slette andre brugere af systemet.

En bruger uden AR har nogle få funktioner til rådighed, som \fx \methodref{ret stamdata()}, \methodref{håndter indkøbsliste()}, \methodref{håndter favoritter()} og \methodref{fejlrapport()}, som går ud på at indberette en fejl vedr. siden eller en opskrift.

Det er ikke obligatorisk at være logget ind på systemet for at bruge det, men i funktionslaget, er der en klasse, vi kalder for \classref{Brugerhåndtering}, hvorfra klassen \classref{Bruger} kan udføre operationerne \methodref{log ind()} og \methodref{log ud()}. Bortset fra \classref{Brugerhåndtering}, så har \classref{Bruger} fortsat de samme associationer til \classref{Fejlrapport}, \classref{Opskrift} og \classref{Vare}. 

Klassen \classref{Vare} er associeret til klassen \classref{Bruger} i form af en indkøbsliste, hvorpå brugerens varer er skrevet på. Vi har ikke modulleret indkøbslisten som en klasse, fordi hver bruger blot har én enkelt indkøbsliste af gangen. En bruger kan altså være tilkoblet mange varer (tilføjet manuelt eller gennem en opskrift) og denne liste af varer bliver er dermed indkøbslisten. Det samme gælder for associationen mellem \classref{Bruger} og \classref{Opskrift}, er en association i form af en favoritliste. Vi har ikke modulleret favoritter som en klasse, fordi vi har vurderet, at det er blot et forhold mellem brugeren og opskriften. Men derfor skal \classref{Opskrift} være en klasse for sig.

Klasserne \classref{Fejlrapport} og \classref{Fejlhåndtering} stammer fra klassen ``fejl'' i problemområdet. Her er den private hændelse ``få tilbagemelding'' iterativ, hvilket vil sige, at der kræves en ny klasse eller en omstrukturering. Vi valgte at omstrukturere klassen for at danne overblik over de mange forskellige fejltyper, der kan opstå. Derfor konstruerede vi klasserne \classref{fejlkategori} og \classref{fejlrapport} til at beskrive de fejl, der eksisterer. Fejlene er blevet udvidet, så vi overvåger ikke kun de fejl i opskrifter, men denne konstruktion gør det også muligt for brugere, at beskrive andre, mere generelle fejl i systemet. Klassen \classref{fejlrapport} er blevet dannet ud fra hændelsen ``rapport\'{e}r'' fejl. Klassen \classref{fejlkategori} bruges derimod til at håndtere de mange \classref{fejlrapport}er, som brugerne kan oprette fra forskellige steder i systemet. Hendelsen ``fejl fundet'' på klassen ``opskrift'' har \fx ledt til et instans af \classref{fejlkategori} for hændelsen.
