\subsection{Den overordnede komponentarkitektur for Foodl}
\label{sec:overordnetkomponent}
En komponentarkitektur har til formål at skabe en fleksibel og forståelig strukturering af sysem. Med en god komponentarkitetktur gør vi systemet let at forstå og organiserer designarbejdet. Igennem modelarkitekturen bliver det samlede design reduceret til en række mindre komplekse opgaver\cite[s.~185]{ooad}. Der findes mange forskellige frameworks i forskellige programmeringssprog, der gør det lettere at opbygge en god komponentarkitektur. Vi valgte at benytte programmeringssproget Ruby med frameworket Ruby on Rails, der bygger på MVC-arkitekturen. Vi syntes Rails virkede hurtigt at komme i gang med og vi kunne konstatere at der fandtes mange gode tutorials, der fik os godt startet. Så snart en ny Railsapplikation bliver genereret, bliver der dannet mapperne ``models'', ``views'' og ``controllers'' indeholdende tilsvarende filer til de komponenter, så på den måde er man tvunget til at implementere MVC-arkitekturen i sin løsning. Igen ses princippet ``konvention over konfiguration'' på et større plan. Et diagram over den konkrete måde, hvorpå Ruby on Rails implementerer MVC, ses i \figref{fig:railsmvc}.

Som det blev nævnt i tidligere \secref{sec:komponenter}, så startede vi med en meget simpelt lagdelt struktur for at få et godt startpunkt for det videre arbejde med komponentarkitekturen. Herefter har vi arbejdet på at identificere de forskellige komponenter, der er i vores system og få dem placeret i de korrekte forhold til hinanden. Resultatet af bearbejdningen kan ses i den nye komponentarkitektur vist i \figref{fig:komponenter}.

\pdffig[0.7]{komponenter2}
	{Den generelle komponenetarkitektur for \Foodl{}.}
	{fig:komponenter}

I forhold til den generiske lagdelt komponentarkitektur set i \figref{fig:simpellag}, er der kommet et lag mere på i bunden af \figref{fig:komponenter}. Dette er et lag, der specificerer systemets tekniske platform, hvilket vi kalder for Ruby on Rails-laget og MySQL, som er eksterne systemer, vi udnytter til brug af systemet. Viewkomponenten er også nyt og stammer fra blandingen med MVC-mønstret. Den tillader nemme måder, at fremstille skabeloner for de forskellige sider på hjemmesiden. Controllerkomponenten har overtaget funktionskomponentens plads men fungerer ligeledes som logik-laget.

Komponentarkitekturen i \figref{fig:komponenter} viser tydeligt, hvordan de forskellige komponenter har forskellige ansvarsområder. Brugergrænsefladen kommunikerer bl.a. med controllerkomponenten ved at videresende brugerens interaktion til controlleren. Controlleren kan ud fra den interaktion enten sende et forespørgsel for at få vist et specifikt View, eller kommunikere med modelkomponenten, hvis der sker noget relevant for problemområdets objekter. Brugergrænsefladen har også direkte kontakt med JavaScript, som også tilhører klientkomponenten.  JavaScript er et eksisterende system, som vi benytter til klient-side-scripting. JavaScript består af yderligere systemer, som er frameworks for JavaScript, som vi også vil benytte os af.

Både Controller, View og Model har direkte kontakt med den tekniske platform, hvorfra der er kontakt til databasen.Active Record-komponenten indbygget i Ruby on Rails giver muligheden for at kommunikere med databasen, uanset hvilket system den underliggende database består af. I softwareudvikling er Active Record et udviklingsmønster, der findes i diverse software, som gemmer data i relationelle databaser. Ruby on Rails har en implementering af Active Record, som vi benytter med MySQL-databasen. Controlleren, hvori et objekt er i overenstemmelse med dette mønster, vil omfatte funktioner som ``Insert'', ``Update'' og ``Delete'', samt egenskaber, der ca. svarer til kolonnerne i den underliggende databasetabel.\cite{activerecordwiki} Derudover består den tekniske platform af BCrypt, som er et eksternt system, der vil blive brugt til hashing af kodeord, så disse ikke blive gemt som rentekst i modellen.

Forskellen på viewkomponenten og brugergrænsefladekomponenten er, at et view er en slags skabelon for det, der skal sendes til brugergrænsefladen. Et view bliver kaldt af controllerkomponenten og bliver udfyldt med indhold fra modellen, igen gennem controllerkomponenten før den sendes til brugergrænsefladen.Brugergrænsefladens opgave er så muliggøre interaktionen mellem bruger og system ved at vise indholdet i et view. Typisk vil brugergrænsefladen være en webbrowser på klientens side, der tilslutter sig webapplikationen ved at pege på webapplikationens webadresse. Et eksempel kunne være, at der skal en specifik visning af noget data i modellen til forskellige typer brugere. Normale brugere skal præsenteres for et cirkeldiagram og administratorbrugere skal præsenteres for et søjlediagram. Viewet har det dynamiske indhold og henter ting fra modellen, alt afhængigt af brugerens adgangsniveau og sender det relevante HTML og JavaScript op til brugergrænsefladen. 

I og med at systemet skal være en webapplikation, så er det ikke tilfældigt, at komponentarkitekturen i \figref{fig:komponenter} minder meget om en klient-server-arkitektur beskrevet i \secref{subsec:mvc}. En klient-server-arkitektur beskriver, at der er en eller flere klienter, der kommunikerer med en server. 

Klienten vil bestå af Brugergrænsefladen og JavaScript, som er grænsefladekomponenten og lidt af funktionskomponenten. Det betyder, at resten af funktionskomponenten og modelkomponenten vil køre på selve serveren. Denne ansvarsfordeling mht. klient og server, hedder distribueret funktionalitet netop fordi funktionaliten er distribueret i begge områder.
