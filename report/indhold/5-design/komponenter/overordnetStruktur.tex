\subsection{Den overordnede komponentarkitektur}
\label{sec:overordnetkomponent}
En komponentarkitektur har til formål at give et struktureret overblik over systemets ansvarsfordeling. En god komponentarkitektur gør systemet lettere forståeligt, organiserer designmæssige arbejdsopgaver og giver god mulighed for reflektion over systemets kontekstuelle stabilitet. Altså at komponentarkitekturen skal afspejle de stabile forhold fra problem- og anvendelsesområdet og derved gøre systemet mere brugbart og fremtidssikkert. Ydermere deles designarbejdet ud i flere mindre komplekse arbejdsopgaver. 

Som det blev nævnt i tidligere \secref{sec:komponenter}, så startede vi med en meget simpelt lagdelt struktur for at få et godt startpunkt for det videre arbejde med komponentarkitekturen. Herefter har vi arbejdet på at identificere de forskellige komponenter, der er i vores system og få dem placeret i de korrekte forhold til hinanden. Vores overordnede komponentarkitektur kan ses i \figref{fig:komponenter}.

\pdffig[0.7]{komponenter2}
	{Den generelle komponenetarkitektur for \Foodl{}.}
	{fig:komponenter}

I forhold til \figref{fig:simpellag}, så er der kommet et lag mere på i bunden af \figref{fig:komponenter}. Dette er et lag, der specificerer systemets tekniske platform, hvilket vi kalder for Ruby on Rails-laget og MySQL, som er eksterne systemer, som vi vil udnytte til udvikling af systemet.

Komponentarkitekturen i \figref{fig:komponenter} viser tydeligt, hvordan de forskellige komponenter har forskellige ansvarsområder. 
Brugergrænsefladen kommunikerer bl.a. med Controller, ved at videresende brugerens interaktion til controlleren. Controlleren kan ud fra den interaktion enten sende et forespørgsel for at få vist et specifikt View, eller kommunikere med modelkomponenten, hvis der sker noget relevant for problemområdets objekter. Brugergrænsefladen har også direkte kontakt med JavaScript, hvilken er på ca. samme lag som Controller, View og Model. JavaScript er et eksisterende system, som vi benytter til klient-side-scripting. JavaScript består af yderligere systemer, som er frameworks for JavaScript, som vi også vil benytte os af.

Både Controller, View og Model har direkte kontakt med den tekniske platform, hvorfra der er kontakt til databasen, MySQL. Vi har moduleret det som, at det er Active Record, der giver os mulighed for at kommunikere med databasen. I softwareudvikling er Active Record et udviklingsmønster, der findes i diverse software, som gemmer data i relationelle databaser. Grænsefladen, hvori et objekt er i overenstemmelse med dette mønster, vil omfatte funktioner som ``Insert'', ``Update'' og ``Delete'', samt egenskaber, der ca. svarer til kolonnerne i den underliggende databasetabel.\cite{activerecordwiki} Derudover består den tekniske platform af BCrypt, som er et eksternet system, der vil blive brugt til hashing af kodeord, så disse ikke blive gemt i almindelige tekstfiler.

I og med at systemet skal være en webapplikation, så giver det meget god mening, at komponentarkitekturen i \figref{fig:komponenter} minder meget om en klient-server-arkitektur. En klient-server-arkitektur beskriver, at der er en eller flere klienter, der kommunikerer med en server. 

Klienten vil bestå af Brugergrænsefladen og JavaScript, som er grænsefladekomponenten og lidt af funktionskomponenten. Det betyder, at resten af funktionskomponenten og modelkomponenten vil køre på selve serveren. Denne ansvarsfordeling mht. klient og server, hedder distribueret funktionalitet netop fordi funktionaliten er distribueret i begge områder.