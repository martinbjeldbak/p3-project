\section{Komponenter}
\label{sec:komponenter}
Hvad er en komponent for en størrelse? Når der hentydes til komponenter i sammenhæng med systemudvikling, så er en komponent en samling af programdele, som udgør en helhed og har veldefinerede ansvarsområder. En programdel kan også være et helt delsystem, men i og med at platformen for vores system er objektorienteret, så vil de fleste programdele være de klasser, der findes i systemet.

For at identificere de forskellige komponenter i vores system, startede vi med at opstille en meget simpel lagdelt arkitektur, som er visualiseret i \figref{fig:simpellag}. 
Komponenten ``brugergrænseflade'' har ansvaret for at aflæse brugerens interaktion med systemets knapper og ud fra dette opdatere, hvad brugeren får vist på skærmen. 
Funktionskomponenten er ansvarlig for systemets funktionalitet. Funktionaliteten bliver leveret til brugergrænsefladen som et set operationer på de offentlige klasser i funktionskomponenten. 
Modelkomponentens primære ansvarsområde er at holde styr på objekterne, der repræsenterer problemområdet. Sker der noget relevant i problemområdet, så skal modelkomponentens objekter skifte tilstand i overenstemmelse hermed. 
Dette mønster minder meget om Model-View-Controller mønstret, som vi kort har beskrevet i \secref{subsec:mvc}.

\pdffig[0.7]{simpellag}
	{Meget simpel lagdelt komponentarkitektur.}
	{fig:simpellag}

Ud fra den simple lagdelte struktur, som er vist i \figref{fig:simpellag}, er der blevet udviklet en overordnet komponentarkitektur, som kan ses i \secref{sec:overordnetkomponent}.

\subsection{Den overordnede komponentarkitektur for Foodl}
\label{sec:overordnetkomponent}
En komponentarkitektur har til formål at skabe en fleksibel og forståelig strukturering af sysem. Med en god komponentarkitetktur gør vi systemet let at forstå og organiserer designarbejdet. Igennem modelarkitekturen bliver det samlede design reduceret til en række mindre komplekse opgaver\cite[s.~185]{ooad}. Der findes mange forskellige frameworks i forskellige programmeringssprog, der gør det lettere at opbygge en god komponentarkitektur. Vi valgte at benytte programmeringssproget Ruby med frameworket Ruby on Rails, der bygger på MVC-arkitekturen. Ruby on rails beskrives i \secref{sec:teknologier}. Vi synes Rails var let tilgængeligt, og vi kunne konstatere at der fandtes mange gode tutorials, der fik os godt startet. Så snart en ny Railsapplikation bliver genereret, bliver der dannet mapperne ``models'', ``views'' og ``controllers'', så på den måde er man tvunget til at implementere MVC-arkitekturen i sit system. Det er et eksempel på princippet ``konvention over konfiguration'', som Ruby on rails tager udgangspunkt i. Et diagram over den konkrete måde, hvorpå Ruby on Rails implementerer MVC, ses i \figref{fig:railsmvc}.

\Foodl{} benytter klient-server arkitekturmønstret se \figref{fig:railsmvc}. Det er oplagt af flere grund. Et vigtigt kriterie er, at systemet skal være vedligeholdbart. Det opnås bedst ved at lade systemet fungere som en webapplikation på en server. Ved at lade alle brugere af webapplikationen være klienter, kan vi hurtigt ændre eller tilføje noget til systemet, uden at brugeren skal have opdateret hele systemet, de skal blot genindlæse siden, fordi systemet ikke befinder sig lokalt hos den enkelte bruger. Ved blot at lade dette ansvar forblive på serveren, får brugerne et mere vedligeholdbart system. Vi lader derfor webbrowserkomponenten fungere som klient, der tilkobler sig serveren for at sende forespørgsler. Det er controller-komponenten, der sørger for at sende data frem og tilbage mellem klienterne og serveren. Der er altså kun én server, som integrerer med mange forskellige klienter igennem et netværk og deler fælles ressourcer med alle de klienter, der er tilkoblet. Serverkomponenten stiller diverse funktioner tilgængelig for klienterne, \fx at muliggøre søgning eller lagring af oplysninger i modelkomponenten. Serveren, i dette tilfælde, og i modsætning til klient-servermønstret beskrevet i OOA\&D\cite{ooad}, kan godt kende noget til klienten. \Fx er det muligt, at vise andre views, alt afhængigt af hvilken webbrowser, der bliver tilkoblet serverkomponenten.

\pdffig[0.7]{railsMVC}{Et tilpasset generiske Model-View-Controller mønster indkapslet i et klient-server arkitekturmønster\cite{railsmvc}.}{fig:railsmvc}

Forskellen på viewkomponenten, set i \figref{fig:railsmvc}, og brugergrænsefladekomponenten, set i \figref{fig:simpellag}, er at et view er en slags skabelon for det, der skal sendes til brugergrænsefladen. Et view bliver kaldt af controllerkomponenten og bliver udfyldt med indhold fra modellen, igen gennem controllerkomponenten før den sendes til brugergrænsefladen.Brugergrænsefladens opgave er så muliggøre interaktionen mellem bruger og system ved at vise indholdet i et view. Typisk vil brugergrænsefladen være en webbrowser på klientens side, der tilslutter sig webapplikationen ved at pege på webapplikationens webadresse. Et eksempel kunne være, at der skal en specifik visning af noget data i modellen til forskellige typer brugere, hvor førstegangsbrugere burde blive budt velkommen på side, burde en administrator måske blive fortalt om brugere har indrapporteret fejl. Et view gør dette muligt, da det har et dynamiske indhold og henter ting fra modellen, alt afhængigt af brugerens adgangsniveau og sender det relevante layout op til brugergrænsefladen. 

Som det blev nævnt i tidligere \secref{sec:komponenter}, så startede vi med en meget simpelt lagdelt struktur for at få et godt udgangspunkt for det videre arbejde med komponentarkitekturen. Herefter har vi arbejdet på at identificere de forskellige komponenter, der er i vores system og få dem placeret i de korrekte forhold til hinanden. Resultatet af bearbejdningen kan ses i den nye komponentarkitektur vist i \figref{fig:komponenter}.

I forhold til den generiske lagdelt komponentarkitektur set i \figref{fig:simpellag}, er der kommet to lag mere på i bunden af \figref{fig:komponenter}, der specificerer systemets tekniske platform. Det første lag, lige under \textit{model}, er et Ruby on Rails-lag. Ruby on Rails laget tilgår og ændrer data i en database. I vores system benytter vi en MySQL database, der kræver at man bruger MySQL-queryes til at tilgå og ændre i databasen. Dette illustreres som det nederste lag i komponentarkitekturen. Viewkomponenten er også ny og er en del af MVC-mønstret. Den tillader nemme måder, at fremstille skabeloner for de forskellige sider på hjemmesiden. Controllerkomponenten har overtaget funktionskomponentens plads men fungerer ligeledes som laget, hvorfra handlinger udføres, ligsom det gjorde i funktionskomponenten.

Komponentarkitekturen i \figref{fig:komponenter} viser tydeligt, hvordan de forskellige komponenter har forskellige ansvarsområder. Brugergrænsefladen kommunikerer bl.a. med controllerkomponenten ved at videresende brugerens interaktion til controlleren. Controlleren kan ud fra den interaktion enten sende et forespørgsel for at få vist et specifikt View, eller kommunikere med modelkomponenten, hvis brugerens interaktion kræver en ændring eller visning af objekter fra modellen af problemområdet. For at gøre brugergrænsefladen dynamisk, benytter vi JavaScript, som denne har direkte kontakt til. JavaScript er et eksisterende system, som vi benytter til klient-side-scripting. Sammen med JavaScript benytter vi jQuiry og jQuiryUI, som er frameworks til JavaScript, der giver os mulighed for lettere at tilgå forskellige elementer i brugergrænsefladen og bruge nogle indbyggede grafiske elementer og symboler.


Både Controller, View og Model har direkte kontakt med den tekniske platform, hvorfra der er kontakt til databasen. Active Record-komponenten indbygget i Ruby on Rails giver muligheden for at kommunikere med databasen, uanset hvilket system den underliggende database består af. I softwareudvikling er Active Record et udviklingsmønster, der findes i diverse software, som gemmer data i relationelle databaser. Ruby on Rails har en implementering af Active Record, som vi benytter med MySQL-databasen. Controlleren, hvori et objekt er i overenstemmelse med dette mønster, vil omfatte funktioner som ``Insert'', ``Update'' og ``Delete'', samt egenskaber, der ca. svarer til kolonnerne i den underliggende databasetabel.\cite{activerecordwiki} Derudover består den tekniske platform af BCrypt, som er et eksternt system, der bliver brugt til hashing af kodeord, så disse ikke blive gemt som ren tekst i modellen.


\pdffig[0.7]{komponenter2}
	{Den overordnede komponenetarkitektur for \Foodl{}.}
	{fig:komponenter}


Klienten vil bestå af Brugergrænsefladen og JavaScript, som er grænsefladekomponenten. controllerkomponenten og modelkomponenten vil køre på selve serveren.
\subsection{Model- og funktionskomponent}
\label{sec:modelfunktion}
% model component: a part of a system that implements the              l
% problem-domian mode The purpose of the model component is to deliver l
% current and historical data to functions, interfaces, and to users   l
% and systems. The stored information is related to the system's       l
% problem domain, that is, the part of the world that the system is    l
% used to administrate, monitor or contro                              l
%

Klassen \classref{Bruger} introduceres i model- og funktionskomponenten.
Klassen eksisterer ikke i problemområdet, da vi ikke er interesseret
i at administrere, overvåge, eller styre brugere i så stor en omfang,
som problemområdet modellerer. Det er \fx ikke relevant, hvordan
eller hvornår brugeren tilføjer bogmærker eller finder opskrifter.
\classref{Bruger}-klassen finder dog sted i modelkomponenten, fordi
det gør det lettere for brugere (som aktører), til \fx at administrere
deres favoritter eller indholdet i indkøbslisten. Det muliggøre også et
loginsystem, der identificerer brugere for at tilpasse dele af systemet
til den enkelte besøgende.

Klasserne \classref{fejlrapport} og \classref{fejlhåndtering} stammer
fra klassen ``fejl'' i problemområdet. Her er den private hændelse
``få tilbagemelding'' iterativ, hvilket vil sige, at der kræves en ny
klasse eller en omstrukturering. Vi valgte at omstrukturere klassen
for at danne overblik over de mange forskellige fejltyper, der kan
opstå. Derfor konstruerede vi klasserne \classref{fejlkategori} og
\classref{fejlrapport} til at beskrive de fejl, der eksisterer. Fejlene
er blevet udvidet, så vi overvåger ikke kun de fejl i opskrifter,
men denne konstruktion gør det også muligt for brugere, at beskrive
andre, mere generelle fejl i systemet. Klassen \classref{fejlrapport}
er blevet dannet ud fra hændelsen ``rapport\'{e}r'' fejl. Klassen
\classref{fejlkategori} bruges derimod til at håndtere de mange
\classref{fejlrapport}er, som brugerne kan oprette fra forskellige
steder i systemet. Hendelsen ``fejl fundet'' på klassen ``opskrift'' har
\fx ledt til et instans af \classref{fejlkategori} for hændelsen.

``Indkøbsliste''-klassen fra problemområdet er ikke repræsenteret af en
klasse i denne model-funktionskomponentdiagram, da vi besluttede for, at
brugere kun har en fast indkøbsliste. Derfor er der ingen grund til at
benytte en hel klasse til dette formål, da der højst skulle instanseres
en af denne klasse for hver besøgende. Derfor benytter vi i stedet en
relation mellem \classref{bruger} og \classref{vare} til modellering
af indkøbslisten. En \classref{bruger} kan altså være tilkoblet mange
\classref{vare}r (tilføjet manuelt eller gennem en \classref{opskrift})
og denne liste af \classref{vare}r bliver er dermed indkøbslisten.

\todo{Noget med, at ingredienser og råvarer bliver til varer\ldots}

\pdffig{klassediagramDes}
	{Det endelige klassediagram.}
	{fig:klasseDes}

