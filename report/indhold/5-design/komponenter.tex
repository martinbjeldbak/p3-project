\section{Komponenter}
\label{sec:komponenter}
Hvad er en komponent for en størrelse? Når der hentydes til komponenter i sammenhæng med systemudvikling, så er en komponent en samling af programdele, som udgør en helhed og har veldefinerede ansvarsområder. En programdel kan også være et helt delsystem, men i og med at platformen for vores system er objektorienteret, så vil de fleste programdele være de klasser, der findes i systemet.

For at identificere de forskellige komponenter i vores system, startede vi med at opstille en meget simpel lagdelt arkitektur, som er visualiseret i \figref{fig:simpellag}. 
Komponenten ``brugergrænseflade'' har ansvaret for at aflæse brugerens interaktion med systemets knapper og ud fra dette opdatere, hvad brugeren får vist på skærmen. 
Funktionskomponenten er ansvarlig for systemets funktionalitet. Funktionaliteten bliver leveret til brugergrænsefladen som et set operationer på de offentlige klasser i funktionskomponenten. 
Modelkomponentens primære ansvarsområde er at holde styr på objekterne, der repræsenterer problemområdet. Sker der noget relevant i problemområdet, så skal modelkomponentens objekter skifte tilstand i overenstemmelse hermed. 
Dette mønster minder meget om Model-View-Controller mønstret, som vi kort har beskrevet i \secref{subsec:mvc}.

\pdffig[0.7]{simpellag}
	{Meget simpel lagdelt komponentarkitektur.}
	{fig:simpellag}

Ud fra den simple lagdelte struktur, som er vist i \figref{fig:simpellag}, er der blevet udviklet en overordnet komponentarkitektur, som kan ses i \secref{sec:overordnetkomponent}.

\subsection{Den overordnede komponentarkitektur}
\label{sec:overordnetkomponent}
% lagret mønster
%   - kunne også opdele til client-server (Brugergrænseflade + javascript på klientside)
% ingen systemgrænseflade: beskriv hvorfor (hvorfor beskrive det? Er der grund til det?)
% open-strict
% eksisterende komponenter
% beskriv vores komponenter (MVC)
% A component architecture is a structural system view that separates system concerns. A good component architecture makes a system easier to understand, organizing the design work and reflecting the stability of the system's context. It also transforms the design task into several less complex tasks.
% In our method, we call the program parts that structure the calsses a "component". 
% Component: A collection of program parts that constitutes a whole and has well-defined responsibilities.
% A component can also be a subsystem.
% On an object-oriented platform, most of these program parts will be classes.
% "user interface" is responsible for reading the buttons and updating displays that let users interact with the system.
% Component architecture reduces complexity by seperating concerns
% During analysis, we model the system's problem domain and application domain. These descriptions contain charatcteristic structures; we must strive to reflect the most stabel of these in the architecture.
% Layered Architecture Pattern + Generic Architecture Pattern

Vi starter med et lagret komponentmønstre bestående af model, funktion og grænseflade. 

\begin{figure}
  \centering
  \scalebox{0.7}{
    \input{billeder/komponenter2.pdf_tex}
  }
  \capt{Den generelle komponentarkitektur for \Foodl{}.}
  \label{fig:komponenter}
\end{figure}

\subsection{Model- og funktionskomponent}
\label{sec:modelfunktion}
% model component: a part of a system that implements the problem-domian model
% The purpose of the model component is to deliver current and historical data to functions, interfaces, and to users and systems. The stored information is related to the system's problem domain, that is, the part of the world that the system is used to administrate, monitor or control
% 

Klasserne \classref{fejlrapport} og \classref{fejlhåndtering} stammer fra klassen ``fejl'' i problemområdet. Her er den private hændelse ``få tilbagemelding'' iterativ, hvilket vil sige, at der kræves en ny klasse eller en omstrukturering. Vi valgte at omstrukturere klassen for at danne overblik over de mange forskellige fejltyper, der kan opstå. Derfor konstruerede vi klasserne \classref{fejlkategori} og \classref{fejlrapport} til at beskrive de fejl, der eksisterer. Fejlene er blevet udvidet, så vi overvåger ikke kun de fejl i opskrifter, men denne konstruktion gør det også muligt for brugere, at beskrive andre, mere generelle fejl i systemet. Klassen \classref{fejlrapport} er blevet dannet ud fra hændelsen ``rapport\'{e}r'' fejl. Klassen \classref{fejlkategori} bruges derimod til at håndtere de mange \classref{fejlrapport}er, som brugerne kan oprette fra forskellige steder i systemet. Hendelsen ``fejl fundet'' på klassen ``opskrift'' har \fx ledt til et instans af \classref{fejlkategori} for hændelsen. 

``Indkøbsliste''-klassen fra problemområdet er ikke repræsenteret af en klasse i denne model-funktionskomponentdiagram, da vi besluttede for, at brugere kun har en fast indkøbsliste. Derfor er der ingen grund til at benytte en hel klasse til dette formål, da der højst skulle instanseres en af denne klasse for hver besøgende. Derfor benytter vi i stedet en relation mellem \classref{bruger} og \classref{vare} til modellering af indkøbslisten. En \classref{bruger} kan altså være tilkoblet mange \classref{vare}r (tilføjet manuelt eller gennem en \classref{opskrift}) og denne liste af \classref{vare}r bliver er dermed indkøbslisten.

\pdffig{klassediagramDes}
	{Det endelige klassediagram.}
	{fig:klasseDes}

