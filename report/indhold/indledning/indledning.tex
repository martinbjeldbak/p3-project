\section{Indledning}
\label{sec:indledning}

Mange lande smider næsten halvdelen af deres mad ud i affaldet. Der er helt sikkert en meget god grund til, at spiseligt mad bliver smidt ud. Det er da det eneste, der ville være logisk? Men det er langt fra virkeligheden. Spiseligt mad bliver smidt ud bl.a. pga., at fødevaren ser forkert ud eller har den forkerte størrelse.\cite{tedmadspild}

I Danmark findes der omkring 2,6 millioner husstande \cite{husstande}, der dagligt skal få madlavningen til at gå op i en højere enhed. Der er nemlig mange ting at tage højde for under madlavningen. Der skal tænkes på sundhed for den enkelte, i form af selve kosten, men også sundhed for alle på længere sigt, hvilket opnås ved, at vi i samlet flok skåner miljøet ved at anvende madrester.

Under madlavningen kan miljøet skånes ved at undgå at smide for meget mad væk. Det er ikke realistisk at have et mål, der siger, at vi slet ingen ressourcespild skal have. Det kan ikke lade sige gøre. \cite{tedmadspild} Tal fra Politiken viser, at en parcelhusejer smider i gennemsnit 42 kilo spiseligt mad ud om året. \cite{madspildpol} Hvis man ikke vil benytte madresterne fra foregående aften i en lignende ret dagen efter, så kan det være svært at finde en ny ret, hvor man kan benytte de specifikke madrester. I kogebøger bliver det hurtigt uoverskueligt, at skulle søge igennem opskrifter for at finde en opskrift, hvor man benytter sig af de madrester, man har på det pågældende tidspunkt.

En almindelig person ved navn Dee Simmons kommenterer følgende\cite{madspildcomment} som en kommentar på Tristam Stuarts præsentation om den globale skandale vedr. madspild:

\begin{quote}
``I am a single person and I do waste a lot simply because of packaging. I don't know what \% of people are single but I either have to freeze stuff if I can or eat the same dish for 2-3 days.''
\end{quote}

Når man står i supermarkedet, så sælges alt i kæmpe portioner. Hakket oksekød findes typisk i pakker med 500 gram som det mindste. Til en enkelt person er dette ofte for meget, og derved risikerer man enten at stå med 100-200 gram hakket oksekød tilovers. Hvis man ikke vælger at lave en stor portion aftensmad, der skal spises over flere dage, så skal fødevaren fryses ned, så det ikke skal smides ud. 

Sådanne typer madspild koster desuden danske husholdninger 16 milliarder kroner om året, eller ca. 20 \% af madforbruget af en gennemsnitlig dansk børnefamilie. \cite{madspild16}
