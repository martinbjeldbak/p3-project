\section{Indledning}
\label{sec:indledning}

Madspild er et verdensomspændende problem. Tal viser at mange vestlige lande, smider næsten 50\% af alt det mad landet produceret, ud i affaldsspanden. Ofte er der tale om spiseligt mad, som slet ingenting fejler. Men hvad er egentligt grunden til, at så meget mad bliver smidt ud? En af grundene er eksempelvis, at hvis en fødevare ser forkert ud eller ikke har den rigtige størrelse, det kan eksempelvis være en agurk der er for skæv; så bliver fødevaren bare smidt ud \cite{tedmadspild}. 

En anden grund til det enorme madspild, finder sted i de private køkkener. I Danmark findes der omkring 2,6 millioner husstande \cite{husstande}, der dagligt skal få madlavningen til at gå op i en højere enhed. Dette kan ofte være en svær opgave, i en travl og stresset hverdag, hvor der sjældent er tid til at kredse om maden, eller skænke madspild en tanke. Det sker derfor ofte at danskerne, i deres iver på at få handlet hurtigt ind og få styr på dagens madlavning, køber for meget mad, og glemmer at få anvendt det mad de i forvejen har i køleskabet. Resultatet af dette, kan ses i tal fra Politiken, som viser at en personer, som bor i parcelhuse, i gennemsnit smider 42 kilo mad ud om året. \cite{madspildpol} Men der ligger også noget andet til grund for det store madspild blandt privatpersoner. Nemlig det problem, som eneboende står overfor, hvilket Dee Simmons beskriver her: 

\begin{quote}
``Jeg er single, og jeg spilder en masse mad, ene og alene på grund af madindpakningen (i supermarkederne). Jeg ved ikke, hvor stor en procentandel der er singler, men jeg må gang på gang konstatere, at jeg enten skal fryse ting ned, eller spise den samme ret 2-3 dage i træk.'' \cite{tedcomment}
\end{quote}
 
Når man står i supermarkedet, så sælges alt i store portioner. Hakket oksekød findes typisk kun i pakker med 400-500 gram som det mindste. Til en enkelt person er dette ofte for meget, og man risikerer derfor at stå tilbage med 100-200 gram hakket oksekød, som enten skal bruges dagen efter, eller fryses ned, hvis ikke det skal gå til spilde. Idet cirka 40 \% af alle husstande i Danmark, er beboet af en enkelt person \cite{madspild16}, er der tale om et større problem. 

Det er naturligvis ikke realistisk, slet ikke at have noget madspild. Dette ville kræve en enorm mængde tid og planlægning, som er urealistik for de fleste danskere. Størrelserne på supermarkedernes indpakning, styres af supermarkederne selv, og er modelleret efter, hvordan de kan få størst mulig indtjening. Det er ikke noget den enkelte kan styre. Tilgengæld må det være muligt at formindske den enkeltes madspild, ved at gøre madlavning og 
anvendelse af de madrester man allerede har i køleskabet, nemmere. Hvis man vil benytte madresterne og varene i sit køleskab, kan det være svært og 
tidskrævende at finde en opskrift, hvori disse indgår. En metode, kunne være at bladre sine kogebøger igennem, men dette ville hurtigt blive uoverskuelig 
og tidskrævende, da fremgangsmåden ved opslag i kogebøger er, 1. søge efter opskrift man har lyst til at spise, 2. kigge opskriftens ingrediensliste 
igennem, for at se om de madrester og vare man har i køleskabet indgår. Hvis det ikke er tilfældet så søg efter ny opskrift.    
