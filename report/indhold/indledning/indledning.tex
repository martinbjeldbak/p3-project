\chapter{Indledning}
\label{sec:indledning}

Madspild er et verdensomspændende problem\cite{gustavssonSpild}. Tal viser, at industrien (supermarkeder, restauranter o.l.) i mange vestlige lande smider over 30\%, af alt det mad landet producerer, ud i affaldsspanden\cite{gustavssonSpild}. Ofte er der tale om spiseligt mad. Hvad er årsagen til, at så meget mad bliver smidt ud? En af grundene er eksempelvis, at hvis en fødevare ser forkert ud eller ikke har den rigtige størrelse (eksempelvis en agurk, der er for skæv), så bliver fødevaren bare smidt ud, inden den når frem til \fx supermarkedet \cite{tedmadspild}. 

En anden grund til det enorme madspild, finder sted i de private husstande. I Danmark findes der omkring 2,6 millioner husstande \cite{husstande}, der dagligt skal få madlavningen til at gå op i en højere enhed. Dette kan ofte være en svær opgave i en travl og stresset hverdag, hvor der sjældent er tid til at kredse om maden, eller skænke madspild en tanke. Det sker derfor ofte, at danskerne, i deres iver på at få handlet hurtigt ind og få styr på dagens madlavning, køber for meget mad, og glemmer at få anvendt det mad, de i forvejen har i køleskabet. Resultatet af dette, kan ses i tal fra Politiken, som viser, at personer, der bor i parcelhuse, i gennemsnit smider 42 kilo mad ud om året \cite{madspildpol}. Der er flere årsager til det store madspild blandt privatpersoner. Til et foredrag vedrørende madspild ved TED-konferencen kommenterer en brugeer: 

\begin{quote}
``I am a single person and I do waste a lot simply because of packaging. I don't know what \% of people are single but I either have to freeze stuff if I can or eat the same dish for 2-3 days \cite{tedcomment}.'' - Dee Simmons
\end{quote}
 
Der ligger tydeligvis også et problem i størrelsen på de portioner, supermarkederne stiller til rådighed. Hakket oksekød findes typisk kun i pakker med 400-500 gram som det mindste. Til en enkelt person er dette ofte for meget, og man risikerer derfor at stå tilbage med halvdelen, som enten skal bruges dagen efter eller fryses ned, hvis ikke det skal gå til spilde. Idet cirka 40 \% af alle husstande i Danmark, er beboet af en enkelt person \cite{madspild16}, er der tale om et større problem.

Det er naturligvis ikke realistisk, at forestille sig en verden uden madspild. Dette ville kræve en enorm mængde tid og planlægning, som er urealistik for de fleste danskere. Størrelserne på supermarkedernes indpakning, styres af supermarkederne selv, og er modelleret efter, hvordan de kan få størst mulig indtjening. Det er ikke noget den enkelte kan styre. Til gengæld må det være muligt at mindske den enkeltes madspild, ved at gøre madlavning og anvendelse af de madrester, man allerede har i køleskabet, nemmere. Hvis man vil benytte madresterne og varerne i sit køleskab, kan det være svært og tidskrævende at finde en opskrift, hvori disse indgår. En metode, kunne være at bladre sine kogebøger igennem, men dette ville hurtigt blive uoverskuelig og tidskrævende, da fremgangsmåden ved opslag i kogebøger er: 1) søge efter opskrift, som man har lyst og tid til at lave; 2) kigge opskriftens ingrediensliste igennem for at se om de madrester og varer, man har i køleskabet, indgår. Hvis det er tilfældet, at der mangler for mange ingredienser, så skal man søge efter en ny opskrift.
