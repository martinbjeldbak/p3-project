\subsection{Model-View-Controller}
\label{subsec:mvc}
Komponentarkitekturen vi har arbejdet os frem til (set i \figref{fig:komponenter}), afspejler et velanerkendt designparadigme kendt som Model-View-Controller\cite{designpatterns}. Vi mener derfor, at det er oplagt, at bruge eksisterende systemer som allerede benytter sig af MVC designmønstret for at gøre implementeringen så overskuelig som muligt. Arkitekturen blev opfundet af Trygve Reenskaug i 1979, som en slags standardarkitektur for interaktive systemer (\fx webapplikationer). Arkitekturen består af tre hovedkomponenter, nemlig models, views og controllers, der hver især repræsenterer nogle diverse dele af systemet og har nogle specifikke egenskaber og opgaver.

Modelkomponenten bærer de data (typisk repræsenteret af en database), som skal kunne manipuleres. Controllerkomponenten fungerer som en slags lim mellem modellen og views ved at sende forespørgsler til modellen eller views alt afhængig af inputtet fra de tilsluttede views. Derudover sender controllere også forespørgsler i form af tilstandsændringer til modellen hvis brugeren \fx opdaterer noget i et view. Et view præsenterer brugeren for data i modellen og muliggør ændringer i tilstanden af modellen.

MVC er den arkitektur enhver applikation skrevet i Ruby on Rails tager udgangspunkt i. Så snart en ny Railsapplikation bliver genereret, bliver der dannet mapperne ``models'',``views'' og ``controllers'' indeholdende tilsvarende filer til de komponenter, så på den måde er man tvunget til at implementere MVC-arkitekturen i sin løsning. Igen ses princippet ``konvention over konfiguration'' på en større plan. Et diagram over den konkrete måde, at Ruby on Rails implementerer MVC ses i \figref{fig:railsmvc}.

\pdffig[0.7]{railsMVC}{Rails' måde at lave model-view-controller på.}{fig:railsmvc}

