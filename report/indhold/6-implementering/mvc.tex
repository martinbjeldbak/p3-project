\subsection{Model-View-Controller}
\label{subsec:mvc}

Når man opretter en ny Railsapplikation, så bliver skelettet som sagt genereret for hele applikationen. Heriblandt ``App'', hvori ``models'', ``views'' og ``controllers'' befinder sig; ``Config'', hvori ``routes'' befinder sig, som er det element, der forbinder controllers og views; og undermappen ``Spec'', hvori unit-testing af Models, Views og Controllers foregår; samt flere.

Model-View-Controller (MVC) er den arkitektur en hver Railsapplikation tager udgangspunkt i. Så snart en ny Railsapplikation bliver genereret, vil den indeholde mapperne ``models'',``views'' og ``controllers'', så på den måde er en Railsapplikation tvunget til at implementere MVC-arkitekturen. Arkitekturen blev opfundet af Trygve Reenskaug i 1979, som en slags standard arkitektur for interaktive applikationer (\fx webapplikationer). Arkitekturen består af tre komponenter, nemlig models, views og controllers, der hver især har nogle specifikke egenskaber og opgaver. Modellen er den komponent, der har ansvaret for, at applikationen \todo{Skriv om ansvarsområde for hver af komponenterne, model, view og kontroller.}