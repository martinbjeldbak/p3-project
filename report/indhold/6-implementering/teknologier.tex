\section{Teknologier}
\label{sec:teknologier}
Systemet er en webapplikation, hvor der er lagt vægt på enkelt og forståeligt design, effektivitet, fleksibilitet og brugbarhed. For at kunne opfylde alle disse krav og kriterier, som er beskrevet i \secref{sec:kriterier}, har vi gjort brug af nogle forskellige webudviklings- og programmeringsteknologier.

Vi har brugt følgende teknologier:

\begin{itemize}[noitemsep]
\item HTML og CSS
\item Javascript, jQuery UI, AJAX og JSON
\item Ruby on Rails
\item MySQL og phpMyAdmin
\end{itemize}

Vi har brugt HTML\cite{htmlwiki} til at opmærke og strukturere hjemmesiden med. HTML fungerer godt til opmærkning og strukturering af en hjemmeside, men det er ikke særlig flot at se på. For at designe hjemmesiden har vi brugt CSS\cite{csswiki}, der er et sprog, som bruges til at beskrive, hvordan man ønsker indholdet af bl.a. et HTML-dokument skal præsenteres i \fx en webbrowser.

Hvad angår databaser, så har vi har brugt MySQL\cite{mysqlwiki}, der er en flertrådet SQL-databaseserver, som understøtter flere samtidige brugere. I og med at vi består af en gruppe af individer med vidt forskellige kompetencer og erfaringer med web- og systemudvikling, har vi brugt det browserbaserede program, phpMyAdmin\cite{phpmyadminwiki}, til at administrere og opdatere MySQL-databasen. phpMyAdmin præsenterer en letforståelig brugergrænseflade til redigering af databasen. Udviklerne bag MySQL har også udviklet deres eget administrationsmodul til databasen, men den skal installers på en computer, hvorimod phpMyAdmin præsenteres direkte via webbrowseren, hvilket betyder, at vi ikke behøver installere et program på seks forskellige computere, men blot kan bruge det via webbrowseren.

For at gøre hjemmesiden dynamisk og brugervenlig, har vi brugt JavaScript\cite{javascriptwiki}, som er et objektorienteret scriptsprog, som de fleste moderne webbrowsere forstår. Derudover har vi også brugt jQuery UI\cite{jqueryuiwiki}, der bruges til at udvikle interaktive webapplikationer, og AJAX\cite{ajaxwiki} og JSON\cite{jsonwiki}, som bruges til at udvikle asynkrone webapplikationer. Figur \ref{fig:toolbar} illustrerer tre forskellige jQuery UI applikationer, som brugeren bliver præsenteret for, når der bliver udført en søgning på foodl-hjemmesiden. AJAX gør det muligt at udveksle data mellem hjemmeside og server uden at siden skal gennemgå en fuld sideopdatering hver gang, der sker en overførsel, for at vise det nye indhold. Alle dataoverførsler sker i baggrunden og brugeren præsenteres med disse ændringer med det samme.

\begin{figure}[H]
\centering
\includegraphics[scale=0.6]{billeder/jqueryuieksempel.png}
\capt{Figuren viser tre forskellige eksempler af interaktive applikationer, der er lavet vha. jQuery UI. Til venstre ses en samling af tre knapper, som fungerer som afkrydsningsbokse, hvilket betyder, at man kan markere flere af gangen. Disse knapper gør det muligt for brugeren at begrænse søgeresultatet efter disse tre kriterier for tilberedningstiden. Midt for ses en justerbar knap, der gør det muligt for brugeren at skalere opskrifterne. Til højre ses endnu en samling af tre knapper, der fungerer som radioknapper, hvilket betyder, at man kun kan vælge én knap af gangen. Disse knapper giver brugeren mulighed for at sortere resultaterne efter navn, relevans eller tilberedningstid.}
\label{fig:toolbar}
\end{figure}

Derudover har vi anvendt Ruby on Rails \cite{rubyonrailswiki}, der er et webudviklingsframework baseret på programmeringssproget Ruby, til at understøtte webudviklingen. Intentionerne bag Rails var at skabe et framework, der gjorde det lettere og hurtigere for webudviklere at udvikle webapplikationer, hvilket også er en af de væsentlige årsager til, at vi valgte at anvende Rails. 

Målet med Rails blev bl.a. opnået vha. af følgende filosofi:
\begin{quote}
``konventioner over konfigurationer''
\end{quote} 

Filosofien har både sine fordele og ulemper. Det er en ulempe, at webudvikleren skal have arbejdet meget med Rails i forvejen, for at kunne huske de mange konventionerne og kommandoer, eller bruge en masse tid, på at slå dem op, når de skal bruges. Det er på den anden siden en fordel, at webudviklere, vha. Rails' konventioner, kan lave applikationer, som kan udrette meget, ud fra få linjers kode. Rails viser allerede fra første møde, at der skal meget lidt arbejde til, for at få et stort afkast. Dette kan ses på \figref{fig:Rails-new-foodl}, hvor en enkelt linje i kommandoprompten, genererer en ny mappe, som indeholder en funktionsdygtig Railsapplikation, kaldet ``Foodl''.

\begin{figure}
	\centering
	\includegraphics[scale=0.6]{billeder/Rails-new-foodl.png}
	\capt{Railskommando, der indtastes i kommandopromten, hvorefter rails genererer en mappe med en fuldtfungerende web-applikation, kaldet ``Foodl''}
	\label{fig:Rails-new-foodl}
\end{figure}

Enhver Railsapplikation tager udgangspunkt i arkitekturen Model-View-Controller. Mappen med webapplikationen består af en lang række undermapper, hvori ``models'', ``views'' og ``controllers'' bl.a. befinder sig. Kort sagt genererer Rails, vha. \texttt{new}-kommandoen, hele skelettet for webapplikationen, og derefter er det ``bare'' at fylde det indhold, man ønsker i sin webapplikation, i de rigtige mapper. Denne Model-View-Controllerarkitektur vil blive forklaret yderligere i følgende \secref{subsec:mvc}.

\subsection{Model-View-Controller}
\label{subsec:mvc}

Når man opretter en ny Railsapplikation, så bliver skelettet som sagt genereret for hele applikationen. Heriblandt ``App'', hvori ``models'', ``views'' og ``controllers'' befinder sig; ``Config'', hvori ``routes'' befinder sig, som er det element, der forbinder controllers og views; og undermappen ``Spec'', hvori unit-testing af Models, Views og Controllers foregår; samt flere.

Model-View-Controller (MVC) er den arkitektur en hver Railsapplikation tager udgangspunkt i. Så snart en ny Railsapplikation bliver genereret, vil den indeholde mapperne ``models'',``views'' og ``controllers'', så på den måde er en Railsapplikation tvunget til at implementere MVC-arkitekturen. Arkitekturen blev opfundet af Trygve Reenskaug i 1979, som en slags standard arkitektur for interaktive applikationer (\fx webapplikationer). Arkitekturen består af tre komponenter, nemlig models, views og controllers, der hver især har nogle specifikke egenskaber og opgaver. Modellen er den komponent, der har ansvaret for, at applikationen \todo{Skriv om ansvarsområde for hver af komponenterne, model, view og kontroller.}
