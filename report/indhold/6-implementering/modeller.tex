Enhver af de nedenstående modeller er repræsenteret i klassediagrammet, beskrevet i \secref{sec:modelfunktion}. Disse modeller har en direkte kobling til klassediagrammet og skal ses som den endelige implementering af designfasen. Herunder er hver model beskrevet med, hvad de indeholder, og hvad deres funktion består af.

\classref{User} er en af de eneste modelrepræsentationer, der benytter sig af Active Record Callbacks for at sørge for, at der \fx ikke kan oprettes to brugere med samme emailadresse, og at emailen er ``gyldig'' (\fx at den skal indeholde et ``@''-tegn). Dette gør vi, da vi ikke mener, at andre modeller er i direkte kontakt med brugeren på samme måde. Hovedsageligt bliver alle de andre klasser håndteret af systemet og derfor behøver de \fx ikke gyldighedskontroller som \classref{User}. Hvis systemet skulle bruges i produktion og håndteres af flere mennesker, ville det være åbenlyst at tilføje mere kontrol på modellerne.

\begin{description}

  \item[FoodType] \hfill \\
    Denne model repræsenterer ``råvaretype''-klassen i klassediagrammet. \classref{FoodType} er en liste af råvarer, som hver ingrediens i alle af opskrifterne svarer til. Meningen er, at hver ingrediens i alle opskrifter, benytter sig af én \classref{FoodType}. Grunden til dette er, at nogle opskriftshjemmesider benytter varemærker i ingrediensnavne eller bruger forskellige ord for den samme type råvare, \fx havsalt og salt eller schalotteløg og skalotteløg. Dette kan skabe forvirring og gør søgning meget besværligt, hvis der ikke blev søgt på ``bagekartofler'' når man indtaster ``kartofler'' i søgningen. De første råvarer er taget fra en liste af ingredienser\cite{ingrediensliste}, hvorefter de manuelt er blevet indtastet, hvis en ingrediens fra en opskriftshjemmeside ikke eksisterer i forvejen. De forskellige instanser af \classref{FoodType} bliver \fx vist som søgeforslag, når man foretager en søgning på forsiden. Alle råvarer er skrevet i flertal for at få konsistens; og vi vurderer, at det er yderst sjældent, at man har lyst til \fx at bruge én enkelt gulerod som ingrediens i sin rester.

  \item[Ingredient] \hfill \\
    Ingrediensmodellen indgår udelukkende i opskrifter og indeholder navn, mængde og enhed. Hver \classref{Ingredient} har en tilsvarende \classref{FoodType}. \Fx svarer ``ostindisk karry'' til ``karry''. Navnfeltet er taget fra opskriftshjemmesiderne, da de typisk er mere beskrivende end blot en \classref{FoodType}. Det gør, at navnefeltet ikke har nogen betydning for søgningen, da der ikke bliver søgt på ingrediensnavnet. Enheden for hver \classref{Ingredient} er beskrevet vha. SI-systemet og er også taget fra ingredienslisten på opskriftshjemmesiderne. Mængdenfeltet beskriver mængden af ingrediensen skaleret ned til en enkelt person. Det muliggør en nemmere skaleringsfunktion på \Foodl{}, så man ikke behøver, at navigere videre til selve opskriftsiden, hvis man allerede er klar over, hvor mange personer, der skal laves mad til.

  \item[Recipe] \hfill \\
    Opskriftsmodellen indeholder al information om opskrifter: navn, billede, vurdering, url, tilberedningstid, samt flere instanser af klassen \classref{Ingredient}. Billedet er taget direkte fra opskriftshjemmesiden og er krævet, da vi i forhold til informanternes ønsker, ikke interesserer os for opskriftshjemmesider uden et billede af opskriften. I forhold til favoritter, har opskrifter også mulighed for at tilhøre brugere, men dette er repræsenteret af en anden tabel i databasen.  

  \item[IssueCategory] \hfill \\
    Fejltypemodellen beskriver de forskellige typer fejl, som brugeren kan rapportere til administratorerne. Modellen består af et navnefelt og en sandhedsværdi. Navnet på en fejlkategori kunne \fx være ``dødt link'' eller ``stavefejl''. Fejltyperne kan derudover være beskrivelige eller ubeskrivelige, da der nødvendigvis ikke behøves en forklaring af brugeren på \fx et dødt link, hvorimod andre mere generelle fejltyper kræver en form for beskrivelse, før de kan rapporteres.

  \item[Issue] \hfill \\
    En fejl tilslutter sig en \classref{IssueCategory} for, at administratorerne på siden har mulighed for at sortere og filtrere listen af fejl i forbindelse med deres kateogori. Derudover indeholder \classref{Issue} en bruger og/eller en beskrivelse, alt afhængingt af hvilken fejlkategori den indgår i. Dvs., at hvis brugeren er logget ind under oprettelse af en fejlrapport, så kan administratorerne tilgå brugerens mailadresse og eventuelle oplysninger, hvis de ønsker at kontakte brugeren for yderligere oplysninger eller lignende.

  \item[ListItem] \hfill \\
    Når ingredienser fra en opskrift eller en tekststreng bliver tilføjet til indkøbslisten, så bliver de repræsenteret som et ListItem. Dette gøres for at undgå, at indholdet i indkøbslisten ikke ændres, hvis der bliver fjernet eller ændret i en ingrediens  og vice versa. Et \classref{ListItem} har alle egenskaber som en \classref{Ingredient} har, hvor den derudover kræver en \classref{User} for at koble brugeren med de varer, der er tilføjet til indkøbslisten.

  \item[User] \hfill \\
    Brugermodellen repræsnterer de oprettede brugere på \Foodl{}. Brugere skal oprette sig med en emailadresse samt kodeord. Email er den unikke måde, hvorpå brugeren bliver identificeret. Der kan altså på intet tidspunkt være oprettet to brugere med samme emailadresse. Kodeord bliver gemt i databasen vha. et Ruby-bibliotek, der hedder BCrypt, der hasher kodeordet i stedet for at gemme det i plain-text. Brugere har i øvrigt mange \classref{ListItem}s, \classref{Issue}s og har også en tilslutning til favorit-tabellen, som er beskrevet i \classref{Recipe}-modellen.

\end{description}

