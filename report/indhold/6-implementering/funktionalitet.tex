\section{Funktionalitet}
\label{sec:funktionalitet}

Da {\Foodl} er en Rails-applikation består den, som andre MVC-applikationer, hovedsageligt af modeller, controllers og views. Modeller er klasser, der repræsenerer data, dvs. f.eks. tabeller i databasen. Controllers er klasser bestående af actions, eller handlinger, som er metoder, der har til formål at reagere på en aktørs interaktion med serveren. Den sidste grundlæggende del af MVC-designet er views, som er template-filer, som bliver udfyldt med data gennem controlleren og sendt tilbage til aktøren. I \Foodl{} er der flest HTML-templates, som har til formål at præsentere data i brugerens webbrowser.

Derudover består en Rails-applikation også af helpers og assets. Helpers er hjælpefunktioner, som skal være tilgængelige flere steder i applikationen, dvs. i forskellige controllers og views. Assets er de JavaScripts, stylesheets og billeder, som applikationen bruger på klientsiden. 

\subsection{Controllers}
\label{sec:controllers}

Det centrale i implementeringen af controllerkomponenten i vores webapplikation er dets controllers. Controllerkomponenten styrer, hvorpå man i en MVC-applikation fordeler ansvaret for de forskellige modelklasser mellem flere controller-klasser. Som hovedregel laver man en controller for hver modelklasse, med det formål at styre klientens interaktion med den pågældende modelklasse. Webapplikationen indeholder følgende controllers:

\begin{description}
  \item[ApplicationController] \hfill \\ 
  Den overordnede applikationscontroller, som alle andre controllers i \Foodl{} nedarver fra. Den kan f.eks. indeholde forskellige helper-metoder, som skal kunne tilgås overalt i applikationen.

  \item[FavoritesController] \hfill \\ 
  Denne controller håndterer brugerens favoritter, hvad enten denne er logget ind eller ej. Controlleren har tre actions, en til at liste favoritter, en til at tilføje favoritter og en til at fjerne favoritter.

  \item[HomeController] \hfill \\ 
  Denne controller håndterer applikationens to statiske sider, "Om foodl" og "Kontakt foodl".

  \item[IssuesController] \hfill \\ 
  Fejlhåndteringen foregår med denne controller. Dvs. oprettelsen af de fejl, der rapporteres og listen af fejl.

  \item[RecipesController] \hfill \\
  Kun én action er indeholdt i denne controller, en action til at servere det billede, der hører til en opskrift.

  \item[SearchController] \hfill \\
  Søgning håndteres i denne controller. Søgningen er beskrevet i detaljer i \secref{sec:funktionalitet-soegning}.

  \item[SessionsController] \hfill \\
  Denne controller håndterer brugersessions, dvs. når en bruger vil logge ind eller ud.

  \item[ShoppingListController] \hfill \\
  Indkøbslisten håndteres med denne controller. Tilføjelsen af elementer, hvad enten det er ingredienser, rå tekststrenge eller hele opskrifter. Sletning af elementer muliggøres også af denne controller.

  \item[UsersController] \hfill \\
  Denne controller håndterer alt i forbindelse med brugere, dvs. f.eks. opretning af bruger, opdatering af kodeord og funktionen "Glemt kodeord".

\end{description}

\subsection{Modeller}

Enhver af de nedenstående modeller er repræsenteret i klassediagrammet, beskrevet i \secref{sec:modelfunktion}. Disse modeller har en direkte kobling til klassediagrammet og skal ses som den endelige implementering af designfasen. Herunder er hver model beskrevet med, hvad de indeholder, og hvad deres funktion består af.

\classref{User} er en af de eneste modelrepræsentationer, der benytter sig af Active Record Callbacks for at sørge for, at der \fx ikke kan oprettes to brugere med samme emailadresse, og at emailen er ``gyldig'' (\fx at den skal indeholde et ``@''-tegn). Dette gør vi, da vi ikke mener, at andre modeller er i direkte kontakt med brugeren på samme måde. De andre klasser bliver hovedsagligt håndteret af systemet, og derfor behøver de \fx ikke gyldighedskontrolleres ligsom \classref{User}. Hvis systemet skulle bruges i produktion og håndteres af flere mennesker, ville det være åbenlyst at tilføje mere kontrol på modellerne.

\begin{description}

  \item[FoodType] \hfill \\
    Denne model repræsenterer ``råvaretype''-klassen i klassediagrammet. \classref{FoodType} er en liste af råvarer, som hver ingrediens i alle af opskrifterne svarer til. Meningen er, at hver ingrediens i alle opskrifter, benytter sig af én \classref{FoodType}. Grunden til dette er, at nogle opskriftshjemmesider benytter varemærker i ingrediensnavne eller bruger forskellige ord for den samme type råvare, \fx havsalt og salt eller schalotteløg og skalotteløg. Dette kan skabe forvirring og gør søgning meget besværligt, hvis der ikke blev søgt på ``bagekartofler'' når man indtaster ``kartofler'' i søgningen. De første råvarer er taget fra en liste af ingredienser\cite{ingrediensliste}, hvorefter de manuelt er blevet indtastet, hvis en ingrediens fra en opskriftshjemmeside ikke eksisterer i forvejen. De forskellige instanser af \classref{FoodType} bliver \fx vist som søgeforslag, når man foretager en søgning på forsiden. Alle råvarer er skrevet i flertal for at få konsistens; og vi vurderer, at det er yderst sjældent, at man har lyst til \fx at bruge én enkelt gulerod som ingrediens i sin rester.

  \item[Ingredient] \hfill \\
    Ingrediensmodellen indgår udelukkende i opskrifter og indeholder navn, mængde og enhed. Hver \classref{Ingredient} har en tilsvarende \classref{FoodType}. \Fx svarer ``ostindisk karry'' til ``karry''. Navnfeltet er taget fra opskriftshjemmesiderne, da de typisk er mere beskrivende end blot en \classref{FoodType}. Det gør, at navnefeltet ikke har nogen betydning for søgningen, da der ikke bliver søgt på ingrediensnavnet. Enheden for hver \classref{Ingredient} er beskrevet vha. SI-systemet og er også taget fra ingredienslisten på opskriftshjemmesiden (Arla). Mængdefeltet beskriver mængden af ingrediensen skaleret ned til en enkelt person. Det muliggør en nemmere skaleringsfunktion på \Foodl{}, så man ikke behøver, at navigere videre til selve opskriftsiden, hvis man allerede er klar over, hvor mange personer, der skal laves mad til.

  \item[Recipe] \hfill \\
    Opskriftsmodellen indeholder al information om opskrifter: navn, billede, vurdering, url, tilberedningstid, samt flere instanser af klassen \classref{Ingredient}. Billedet er taget direkte fra opskriftshjemmesiden og er krævet, da vi i forhold til informanternes ønsker, ikke interesserer os for opskriftshjemmesider uden et billede af opskriften. I forhold til favoritter, har opskrifter også mulighed for at tilhøre brugere, men dette er repræsenteret af en anden tabel i databasen.  

  \item[IssueCategory] \hfill \\
    Fejltypemodellen beskriver de forskellige typer fejl, som brugeren kan rapportere til administratorerne. Modellen består af et navnefelt og en sandhedsværdi. Navnet på en fejlkategori kunne \fx være ``dødt link'' eller ``stavefejl''. Fejltyperne kan derudover være beskrivelige eller ubeskrivelige, da der nødvendigvis ikke behøves en forklaring af brugeren på \fx et dødt link, hvorimod andre mere generelle fejltyper kræver en form for beskrivelse, før de kan rapporteres.

  \item[Issue] \hfill \\
    En fejl tilslutter sig en \classref{IssueCategory} for, at administratorerne på siden har mulighed for at sortere og filtrere listen af fejl i forbindelse med deres kategori. Derudover indeholder \classref{Issue} en bruger og/eller en beskrivelse, alt afhængigt af hvilken fejlkategori den indgår i. Dvs., at hvis brugeren er logget ind under oprettelse af en fejlrapport, så kan administratorerne tilgå brugerens mailadresse og eventuelle oplysninger, hvis de ønsker at kontakte brugeren for yderligere oplysninger eller lignende.

  \item[ListItem] \hfill \\
    Når ingredienser fra en opskrift eller en tekststreng bliver tilføjet til indkøbslisten, så bliver de repræsenteret som et ListItem. Dette gøres for at undgå, at indholdet i indkøbslisten ikke ændres, hvis der bliver fjernet eller ændret i en ingrediens  og vice versa. Et \classref{ListItem} har alle egenskaber som en \classref{Ingredient} har, hvor den derudover kræver en \classref{User} for at koble brugeren med de varer, der er tilføjet til indkøbslisten.

  \item[User] \hfill \\
    Brugermodellen repræsnterer de oprettede brugere på \Foodl{}. Brugere skal oprette sig med en emailadresse samt kodeord. Email er den unikke måde, hvorpå brugeren bliver identificeret. Der kan altså på intet tidspunkt være oprettet to brugere med samme emailadresse. Kodeord bliver gemt i databasen vha. et Ruby-bibliotek, der hedder BCrypt, der hasher kodeordet i stedet for at gemme det i plain-text. Brugere har i øvrigt mange \classref{ListItem}s, \classref{Issue}s og har også en tilslutning til favorit-tabellen, som er beskrevet i \classref{Recipe}-modellen.

\end{description}



\subsection{Tabeller}
\label{sec:tabeller}

Hver model svarer til en tabel i databasen. Vores tabelstruktur kan ses i \figref{fig:database}. Hver kasse repræsenterer en tabel i databasen, og hver pil repræsenterer en relation mellem to tabeller. Pilen vender således, at den peger væk fra den tabel, som har en fremmed nøgle. Det ses \fx mellem \dbtableref{ingredients} og \dbtableref{food\_types}. Her indeholder tabellen \dbtableref{ingredients} en kolonne \texttt{food\_type\_id}, der er en fremmed nøgle, som bruges til at pege på en række i \dbtableref{food\_types}-tabellen, derfor pilen fra \dbtableref{ingredients} til \dbtableref{food\_types}.

\dbtableref{sessions}-tabellen bruges af en intern Session-model i Rails, som er nødvendig for at gemme sessionsdata på serveren, dvs. \fx den besøgendes favoritter og indkøbsliste, når denne ikke er logget ind.

Ydermere er der tabellen \dbtableref{users\_recipes}, som er nødvendig for at udtrykke mange-til-mange-forholdet mellem brugere og opskrifter. Dvs., at hver relation mellem en bruger og en opskrift (en favorisering) svarer til en række i denne tabel, som indeholder både et id for en opskrift og et id for en bruger. Når tabellen først er oprettet, sørger Rails selv for at abstrahere denne relation, så længe en \methodref{has\_and\_belongs\_to\_many}-relation, der peger på \classref{User}-modellen, er sat op i \classref{Opskrift}-modellen og vice versa. Dette medvirker til, at man \fx nemt kan tilføje en opskrift til en brugers favoritter vha. en enkelt linje Ruby-kode, som er illustreret i \lstref{lst:rubymanytomany}.

\begin{lstlisting}[caption={Hvis man har et \classref{User}-objekt i \texttt{user} (som f.eks. returneret med \lstinline{User.find_by_id(42)}) og et \classref{Recipe}-object i \texttt{recipe}, kan opskriften associeres med brugeren med denne linje Ruby-kode.},label=lst:rubymanytomany,language=Ruby]
user.favorites << recipe
\end{lstlisting}

Som illustreret i \figref{fig:database}, så er der udover mange-til-mange-forholdene også mere simple relationer. Det ses blandt andet mellem opskrifter og ingredienser, og mellem ingredienser og råvaretyper. Dette betyder, at ser man på \classref{Ingredient}-modellen, så er der sat to \methodref{belongs\_to}-relationer op, som peger på henholdsvis \classref{Recipe} og \classref{FoodType}. Ligeledes er der i både \classref{Recipe}- og \classref{FoodType}-modellerne sat en \methodref{has\_many}-relation op, som peger på \classref{Ingredient}.

\pdffig[0.6]{database}
  {Databasetabelstrukturen og associeringerne mellem tabellerne.}
  {fig:database}

\subsection{Søgning}
\label{sec:funktionalitet-soegning}
Søgningen er den vigtigste del af \Foodl{}. Som beskrevet i \secref{sec:systemdefinition}, så er målet med projektet netop at kunne søge blandt mange opskrifter og returnere de opskrifter, som indeholder de råvarer, som brugeren har valgt.

Alt dette styres af klassen \classref{SearchController}, som består af tre actions. Ligeledes er aktiviteten spredt ud over to views; \methodref{index}, som er selve forsiden af \Foodl{} med mulighed for at indtaste råvarer, og \methodref{result}, som er resultatsiden, hvor de relevante opskrifter bliver præsenteret.

Den tredje action i controlleren er \methodref{autocomplete\_food\_types}, hvortil der ikke er knyttet et view. Denne action har til formål at returnere en liste over råvarer baseret på brugerens indtastning. Denne funktionalitet er beskrevet i detaljer i nedenstående afsnit.

\subsubsection{Søgeforslag}
En central del af forsiden og brugerens mulighed for at vælge råvarer er, at systemet kommer med en liste over forslag, som er baseret på brugerens indtastning.

Det eneste denne action gør er at generere en SQL-forespørgsel, som sendes til MySQL-databasen og returnerer et array bestående af fem råvareforslag. Dette array konverteres til JSON-format og returneres. Dermed kan man via JavaScript tilgå denne action med en AJAX-forespørgsel, mens brugeren indtaster, og næsten med det samme fremvises resultatet.

Et eksempel på en genereret SQL-forespørgsel kan ses i \lstref{lst:soegeforslag-sql}. SQL-forespørgslen er en \texttt{SELECT}-forespørgsel, hvilket vil sige, at vi ønsker at læse fra en tabel i databasen. I første linje vælges kolonnen \texttt{name}, således at kun indholdet af denne kolonne returneres. I linje 2 vælges hvilken tabel forespørgslen skal udføres på, i dette tilfælde \dbtableref{food\_types}-tabellen, altså tabellen med alle råvarer. I linje tre sættes en betingelse for hvilke rækker fra tabellen, der skal returneres, nemlig alle dem, hvor råvarenavnet indeholder ordet ``pølse''. ``\%''-tegnet er wilcard-karakter i SQL, og matcher nul eller flere karakterer. Dvs. ``\%pølse\%'' matcher både ``pølser'', ``pølsebrød'' og ``wienerpølse''. I fjerde linje påbegyndes sorteringsudtrykket. Der sorteres på baggrund af tre parametre. Først og fremmest sorteres resultatet efter en \texttt{CASE}, hvor rækkens navnefelt testes for lighed med det indtastede. Er dette tilfældet er udtrykket lig 1, ellers er det 0. Da der sorteres i faldende rækkefølge (\texttt{DESC}), vil den række (hvis den eksisterer), hvor navnet er lig med det indtastede, altså komme først. Den næste sorteringsparameter, har en lignende \texttt{CASE}, dog med en wildcard-karakter i enden. Dermed vil alle rækker, hvor navnet begynder med det indtastede, komme før de rækker, hvor det indtastede står inde midt i råvarenavnet. Den sidste sorteringsparameter, sorterer efter længden af råvarenavnet i stigende rækkefølge. Dette gøres for, at de mest relevante foreslag kommer øverst. Den sidste linje i forespørgslen (\texttt{LIMIT}), begrænser mængden af resultater til fem.

\begin{lstlisting}[caption={Hvis en bruger indtaster ``pølse'' udføres denne SQL-forespørgsel.},label=lst:soegeforslag-sql,language=SQL]
SELECT name
FROM food_types
WHERE name LIKE "%pølse%"
ORDER BY CASE
    WHEN name LIKE "pølse" THEN 1
    ELSE 0
  END DESC,
  CASE
    WHEN name LIKE "pølse%" THEN 1
    ELSE 0
  END DESC,
  LENGTH(name) ASC
LIMIT 5
\end{lstlisting}

Resultatet af SQL-forespørgslen i \lstref{lst:soegeforslag-sql} er derfor en liste med 5 forslag til råvarer. Det returnerede JSON-array for ``pølse'' kan ses i \lstref{lst:poelse-json}.

\begin{lstlisting}[caption={Et returneret JSON-array for ``pølse''.},label=lst:poelse-json,language=Ruby]
["Pølser","Pølsebrød","Spegepølse","Wienerpølse","Røde pølser"]
\end{lstlisting}

De valgte råvarer tilføjes med JavaScript til en liste, som sendes videre til \methodref{result}-siden, når brugeren klikker ``Søg''.

\subsubsection{Søgeresultat}
Resultatsiden håndteres af en action ved navn \methodref{result}, som foretager søgningen på opskrifterne i databasen. 

Den primære parameter er en liste af råvaretyper, som er en tekststreng separeret med ``|''-tegn. Det første, der bliver gjort med denne streng, er, at den bliver splittet op i et array bestående af råvarenavne. Derefter genereres en SQL-forespørgsel, som henter id'erne for de valgte råvaretyper fra \dbtableref{food\_types}-tabellen.

For søgestrengen ``\texttt{Pølser|Wasabi|Peber|}'' returneres derfor array'et \lstinline[language=Ruby]{[1116, 1113, 2]}. Disse id'er bruges til at generere endnu en SQL-forespørgsel, med det formål at finde opskrifter, som indeholder de valgte råvaretyper. Et eksempel på dette kan ses i \lstref{lst:soegeresultat-sql}.

\begin{lstlisting}[caption={Ved en søgning på ``Pølser'', ``Wasabi'' og ``Peber'' udføres denne SQL-forespørgsel.},label=lst:soegeresultat-sql,language=SQL]
SELECT recipes.*, COUNT(*) AS relevance
FROM ingredients
RIGHT JOIN recipes ON recipe_id = recipes.id
WHERE food_type_id IN ( 1116, 1113, 2 )
GROUP BY recipes.id
ORDER BY relevance DESC
LIMIT 0, 50
\end{lstlisting}

Resultatet grupperes efter opskrifternes id, således at man kan tælle, hvor mange af de valgte råvarer opskriften indeholder. Denne værdi gemmes i variablen \texttt{relevance}, og bliver (som standard) brugt til at sortere resultatet med. Den bliver ligeledes sendt videre til view'et sammen med hver opskrift, således at det fremgår af resultaterne, hvor mange af de valgte ingredienser, hver opskrift indeholder.

Opskrifterne returneres på to måder alt efter, hvordan siden forespørges. Ved en almindelig browser-forespørgsel returneres hele resultatsiden som forventet. Men foretages forespørgslen via AJAX, så returneres kun listen med opskrifter. Dette gør, at resultatsiden dynamisk kan hente en ny opskriftsliste fra serveren, hvis \fx søgeparametrene ændres.
