\section{Funktionalitet}
\label{sec:funktionalitet}

Da {\Foodl} er en Railsapplikation består den, som andre MVC-applikationer, hovedsageligt af modeller, controllers og views. Modeller er klasser, der repræsenerer data, dvs. f.eks. tabeller i databasen. Controllers er klasser bestående af actions, eller handlinger, som er metoder, der har til formål at reagere på en aktørs interaktion med serveren. Den sidste grundlæggende del af MVC-designet er views, som er template-filer, som bliver udfyldt med data gennem controlleren og sendt tilbage til aktøren. I {\Foodl} er der flest HTML-templates, som har til formål at præentere data i brugerens webbrowser.

Derudover blabla helpers.. assets.. osv\todo{}

\subsection{Controllers}
\label{sec:controllers}

{\Foodl} består af følgende controllers:

\begin{description}
\item[ApplicationController] \hfill \\ 
Den overordnede applikationscontroller, som alle andre controllers i Foodl nedarver fra. Den kan f.eks. indeholde forskellige helper-metoder, som skal kunne tilgås overalt i applikationen.

\item[FavoritesController] \hfill \\ 
Denne controller håndterer brugerens favoritter, hvad enten denne er logget ind eller ej. Controlleren har tre actions, en til at liste favoritter, en til at tilføje favoritter og en til at fjerne favoritter.

\item[HomeController] \hfill \\ 
Denne controller håndterer applikationens to statiske sider, "Om foodl" og "Kontakt foodl".

\item[IssuesController] \hfill \\ 
Fejlhåndteringen foregår med denne controller. Dvs. oprettelsen af fejlrapporter og listen af disse.

\item[RecipesController] \hfill \\
Kun én action er indeholdt i denne controller, en action til at servere det billede der hører til en opskrift.

\item[SearchController] \hfill \\
Søgning håndteres i denne controller. Søgningen er beskrevet i detaljer i \secref{sec:soegning}.

\item[SessionsController] \hfill \\
Denne controller håndtere brugersessioner, dvs. når en bruger vil logge ind eller ud.

\item[ShoppingListController] \hfill \\
Indkøbslisten håndteres med denne controller. Tilføjelsen af elementer, hvad enten det er ingredienser, rå tekststrenge eller hele opskrifter, samt sletning af elementer muliggøres af denne controller.

\item[UsersController] \hfill \\
Denne controller håndtere alt i forbindelse med brugere, dvs. f.eks. opretning af bruger, opdatering af kodeord og funktionen "Glemt kodeord".

\end{description}

\subsection{Modeller}

\begin{description}

  \item[FoodType] \hfill \\
  Råvare

  \item[Ingredient] \hfill \\
  Ingrediens

  \item[IssueCategory] \hfill \\
  Fejltype

  \item[Issue] \hfill \\
  Fejlrapport

  \item[ListItem] \hfill \\
  Vare

  \item[Recipe] \hfill \\
  Opskrift

  \item[User] \hfill \\
  Bruger

\end{description}

\subsection{Tabeller}
\label{sec:tabeller}

Hver model svarer til en tabel i databasen. Vores tabelstruktur kan ses i \figref{fig:database}.

\pdffig{database}
  {Databasetabelstrukturen og associeringerne mellem tabellerne.}
  {fig:database}

\subsection{Søgning}
\label{sec:soegning}



