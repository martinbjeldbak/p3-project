\section{Funktioner}
\label{sec:funktioner}

For at skabe overblik over systemets funktionailtet er en funktionsliste  blevet udarbejdet, hvilket beskriver, hvad systemet skal kunne. Dette skal som minimum dække over, hvordan aktørerne vil interagere med systemet.

Funktionslisten kan ses i \tableref{table:funktionsliste} og indeholder, ud over listen af funktioner, to parametre for hver funktion, funktionstype og kompleksitet. Funktionstype klassificerer generelle typer af funktioner på, hvordan de forbinder systemet med omgivelserne. Aflæsning- og opdaterings-funktioner aflæser og ændrer i modellen, hvor beregningsfunktioner aflæser modellen og udarbejder noget nyt data, som vises i anvendelesområdet. Kompleksitet er vores bedømmelse af, hvor svært det er at implementere funktionen, og kun de komplekse funktioner vil blive nærmere uddybet her.

\begin{table}[H]
  \centering
  \begin{tabular}{ r|c c }
    \hline
    & \multicolumn{2}{c}{\textbf{Egenskaber}} \\
    \textbf{Navn} & Funktionstype & Kompleksitet \\ \hline
    Håndter opskrifter & Beregning & Medium \\
    Håndter favoritter & Opdatering & Medium \\
    Håndter indkøbsliste & Opdatering & Medium \\
    Søg & Beregning & Kompleks \\
    Skaler søgeresultat & Beregning & Simpel \\
    Log ind & Opdatering  & Simpel  \\
    Log af & Opdatering & Simpel  \\
    Registrer bruger & Opdatering & Simpel  \\
    Vis favoritter & Aflæsning & Simpel  \\
    Opret fejlrapport & Opdatering & Medium \\
    Luk fejlrapport & Opdatering & Simpel  \\
    Udfør crawling & Beregning & Kompleks \\ \hline
  \end{tabular}
  \capt{Funktionsliste over systemet.}
  \label{table:funktionsliste}
\end{table}

\begin{table}[H]
  \centering
  \begin{tabular}{ r| l l }

    \multicolumn{3}{c}{\textbf{Udfør crawling}} \\ \hline
    & \multicolumn{2}{c}{\textbf{Egenskaber}} \\
    \textbf{Navn} & Funktionstype & Kompleksitet \\ \hline
    Find opskrift & Beregning & Medium \\
    Find ingredienser på opskrift & Beregning & Medium \\
    Gem opskrift  & Opdatering & Simpel \\ \hline
  \end{tabular}
  \capt{Funktionsliste over den komplekse funktion ``crawling''.}
  \label{table:crawlingliste}
\end{table}


Funktionerne i ovenstående tabel bliver udført af aktørene på systemet. De første fire funktioner i \tableref{table:funktionsliste} er funktioner, der skal blive udført af dataudtrækkeren. Denne har til formål at trække relevant data ud af specifikke hjemmesider, og evt.\ oprette, slette eller opdatere opskrifter alt efter hvilken funktion, der er relevant. Hvis der \fx er sket en ændring i en af opskrifterne, så skal denne opdateres som følge heraf. De øvrige funktioner bliver udført af brugeren eller administratoren og kan ses i brugsmønstrene.

\subsection{Specifikation af komplekse funktioner}
Den komplekse funktion ``udfør dataudtrækning'' er beskrevet ved at dekomponere funktionen i mindre komplekse funktioner. Funktionslisten for ``udfør dataudtrækning'' kan ses i \tableref{table:kompleksfunktion-dataud}. Under funktionen ``find opskrift'' udfører dataudtrækkeren dataudtrækning af en opskriftshjemmeside, og alle opskrifter identificeres. Funktionen ``find ingredienser'' finder alle ingredienser for opskrifterne, og funktionen ``gem opskrift'' tilføjer det hele til modellen.

\ourtable{kompleksfunktion-dataud}{2}{Funktionsliste over den komplekse funktion ``udfør dataudtrækning''.}
                                     {Egenskaber}
       {Navn              }{ Funktionstype & Kompleksitet  }{
\ourrow{Find opskrift     }{ Beregning     & Medium        }
\ourrow{Find ingredienser }{ Beregning     & Medium        }
\ourrow{Gem opskrift      }{ Opdatering    & Simpel        }
}

