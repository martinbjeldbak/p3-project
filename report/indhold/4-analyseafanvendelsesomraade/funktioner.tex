\section{Funktioner}
\label{sec:funktioner}

For at skabe overblik over systemets funktionailtet er en funktionsliste  blevet udarbejdet, hvilket beskriver, hvad systemet skal kunne. Dette skal som minimum dække over, hvordan aktørerne vil interagere med systemet.

Funktionslisten kan ses i \tableref{table:funktionsliste} og indeholder, ud over listen af funktioner, to parametre for hver funktion, funktionstype og kompleksitet. Funktionstype klassificerer generelle typer af funktioner på, hvordan de forbinder systemet med omgivelserne. Aflæsning- og opdaterings-funktioner aflæser og ændrer i modellen, hvor beregningsfunktioner aflæser modellen og udarbejder noget nyt data, som vises i anvendelesområdet. Kompleksitet er vores bedømmelse af, hvor svært det er at implementere funktionen, og kun de komplekse funktioner vil blive nærmere uddybet her.

\ourtable{funktionsliste}{2}{Funktionsliste over systemet.}
                                        {Egenskaber}
       {Navn                 }{ Funktionstype & Kompleksitet  }{
\ourrow{Opret opskrift       }{ Opdatering    & Simpel        }
\ourrow{Slet opskrift        }{ Opdatering    & Simpel        }
\ourrow{Opdater opskrift     }{ Opdatering    & Simpel        }
\ourrow{Opret favorit        }{ Opdatering    & Simpel        }
\ourrow{Slet favorit         }{ Opdatering    & Simpel        }
\ourrow{Slet bruger          }{ Opdatering    & Simpel        }
\ourrow{Håndter indkøbsliste }{ Opdatering    & Medium        }
\ourrow{Søg                  }{ Beregning     & Kompleks      }
\ourrow{Skaler søgeresultat  }{ Beregning     & Simpel        }
\ourrow{Ret stamdata         }{ Opdatering    & Medium        }
\ourrow{Log ind              }{ Opdatering    & Simpel        }
\ourrow{Log af               }{ Opdatering    & Simpel        }
\ourrow{Registrer bruger     }{ Opdatering    & Simpel        }
\ourrow{Vis favoritter       }{ Aflæsning     & Simpel        }
\ourrow{Opret fejlrapport    }{ Opdatering    & Medium        }
\ourrow{Luk fejlrapport      }{ Opdatering    & Simpel        }
\ourrow{Udfør crawling       }{ Beregning     & Kompleks      }
}

\ourtable{crawlingliste}{2}{Funktionsliste over den komplekse funktion ``udfør crawling''.}
                                     {Egenskaber}
       {Navn              }{ Funktionstype & Kompleksitet  }{
\ourrow{Find opskrift     }{ Beregning     & Medium        }
\ourrow{Find ingredienser }{ Beregning     & Medium        }
\ourrow{Gem opskrift      }{ Opdatering    & Simpel        }
}


Funktionerne i ovenstående tabel bliver udført af aktørene på systemet. De første fire funktioner i \tableref{table:funktionsliste} er funktioner, der skal blive udført af dataudtrækkeren. Denne har til formål at trække relevant data ud af specifikke hjemmesider, og evt.\ oprette, slette eller opdatere opskrifter alt efter hvilken funktion, der er relevant. Hvis der \fx er sket en ændring i en af opskrifterne, så skal denne opdateres som følge heraf. De øvrige funktioner bliver udført af brugeren eller administratoren og kan ses i brugsmønstrene.

\subsection{Specifikation af komplekse funktioner}
Den komplekse funktion ``udfør dataudtrækning'' er beskrevet ved at dekomponere funktionen i mindre komplekse funktioner. Funktionslisten for ``udfør dataudtrækning'' kan ses i \tableref{table:kompleksfunktion-dataud}. Under funktionen ``find opskrift'' udfører dataudtrækkeren dataudtrækning af en opskriftshjemmeside, og alle opskrifter identificeres. Funktionen ``find ingredienser'' finder alle ingredienser for opskrifterne, og funktionen ``gem opskrift'' tilføjer det hele til modellen.

\ourtable{kompleksfunktion-dataud}{2}{Funktionsliste over den komplekse funktion ``udfør dataudtrækning''.}
                                     {Egenskaber}
       {Navn              }{ Funktionstype & Kompleksitet  }{
\ourrow{Find opskrift     }{ Beregning     & Medium        }
\ourrow{Find ingredienser }{ Beregning     & Medium        }
\ourrow{Gem opskrift      }{ Opdatering    & Simpel        }
}


Søgningen efter opskrifter er en kompleks beregningsfunktion, som tager en liste over råvaretyper som input og returnerer en liste over relevante opskrifter. Listen skal bestå af de opskrifter, som bedst muligt udnytter de råvaretyper som funktionen har fået som input, og skal kunne sorteres efter forudbestemte kriterier. Vi mener ikke, at det ville være nødvendigt at lave en nedbrydning af denne beregningsfunktion, da det indebærer ikke så stor en diversitet. Dette er på grund af, at database-forespørgslen, der returnerer listen af opskrifter er kompleks, men indeholder ikke nogle mindre funktioner. Forespørgslen er beskrevet i \secref{sec:funktionalitet-soegning}.
