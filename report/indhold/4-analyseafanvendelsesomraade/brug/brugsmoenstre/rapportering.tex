\brugtabel{Rapportering}{rapportering}
{Rapportering igangsættes af \textit{brugeren}, når denne opdager en fejl på hjemmesiden. Hvis \textit{brugeren} opdager en fejl, der har med et søgningsresultat (opskrift) at gøre, så klikkes der på en rapporteringsknap, der er ved det enkelte søgningsresultatet. Når der skal rapporteres en fejl vedrørende opskrifter, så åbnes en dialogboks på siden, hvor \textit{brugeren} herefter skal vælge en fejltype. Der skelnes mellem beskrivelige og ubeskrivelige fejltyper. Et eksempel på en ubeskrivelig fejltype er bl.a., hvis et link ikke fungerer, som hører under fejltypen “dødt link”. De ubeskrivelige fejltyper behøver ingen beskrivelse, da fejltypen er beskrivelse nok i sig selv. Vælger \textit{brugeren} derimod en beskrivelig fejltype, så præsenteres en beskrivelsesboks for \textit{brugeren}, hvor fejlen beskrives med tekst. Derefter er rapporten klar, og \textit{brugeren} skal nu godkende rapporten, inden den bliver sendt til \textit{administratoren}.

Derudover er der en generel rapporteringsknap, der vedrører andre, generelle fejl på siden. Når denne knap benyttes, så dirigeres brugeren direkte hen til en beskrivelsesboks, hvor fejlen beskrives med tekst. Til slut skal rapporten godkendes af \textit{brugeren}, inden den bliver sendt til \textit{administratoren}. Det er altid muligt at annullere rapporteringen under alle tilstande i brugsmønstret.}
{Fejl, Opskrift}
{Opret fejlrapport}
{Brugmønsteret rapportering}
