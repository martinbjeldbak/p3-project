\brugtabel{Søgning}{soegning}
{En søgning igangsættes af \textit{brugeren}. \textit{Brugeren} bliver præsenteret for et tom søgefelt, der illustrerer, at man kan indtaste nogle søgekriterier i form af råvaretyper i det tomme felt. En søgning fuldføres, når man taster på knappen ``søg'', hvilket kun er muligt, når der mindst er én råvaretype som søgekriterie. Det er muligt at indtaste og fjerne så mange råvaretyper fra søgefeltet, som \textit{brugeren} ønsker. Se \figref{fig:bm-soegning}.

Når der er blevet foretaget en søgning, så bliver der vist et søgeresultat, der er en liste af opskrifter, der indeholder en eller flere af de indtastede råvaretyper. \textit{Brugeren} har nu endnu en gang mulighed for at fjerne eller indtaste nye råvaretyper fra denne resultatside. \textit{Brugeren} kan søge på de nye søgekriterier direkte fra denne side uden at genlæse visningen i sin internetbrowser.

Opskrifterne sorteres i første omgang efter, hvor godt deres ingredienser matcher de valgte råvaretyper. \textit{Brugeren} kan vælge to andre sorteringsmuligheder. Udover den primære sortering, så kan opskrifterne sorteres efter tilberedningstid og alfabetisk orden. Det er også muligt at begrænse søgeresultatet efter tilberedningstid. Her har \textit{brugeren} mulighed for at bestemme om, der skal bruges lidt eller meget tid på madlavningen. Når \textit{brugeren} har valgt, så bliver listen dynamisk opdateret.

Søgningen kan under alle tilstande afsluttes ved at lukke siden.}
{Opskrift, Råvaretype}
{Søg efter opskrifter, Skaler søgeresultat, Sorter søgeresultat, Begræns søgeresultat}
{Brugsmønster for søgning.}
