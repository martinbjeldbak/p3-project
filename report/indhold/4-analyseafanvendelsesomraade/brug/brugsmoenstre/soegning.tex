\brugtabel{Søgning}{soegning}
{En søgning igangsættes af \textit{brugeren}. Brugeren bliver præsenteret med en tom søgebar, der illustrerer, at man kan indtaste nogle søgekriterier i det tomme felt. En søgning fuldføres, når man taster på knappen ``søg''. Det er muligt at indtaste og fjerne så mange råvarer fra søgefeltet som \textit{brugeren} ønsker. Se \figref{fig:bm-soegning}.

Når der er blevet foretaget en søgning, så bliver der vist et søgeresultat, der er en liste af opskrifter, der indeholder en eller flere af de indtastede råvarer. Brugeren har nu endnu en gang mulighed for at fjerne eller indtaste nye råvarer fra denne resultatside. \textit{Brugeren} kan søge på de nye søgekriterier direkte fra denne side.

Opskrifterne sorteres i første omgang efter, hvor godt deres ingredienser matcher de valgte råvarer. \textit{Brugeren} kan vælge to andre sorteringsmuligheder. Udover den primære sortering, så kan opskrifterne sorteres efter tilberedningstid og alfabetisk orden. Det er også muligt at begrænse søgeresultatet efter tilberedningstid. Her har \textit{brugeren} mulighed for at bestemme om, der skal bruges lidt eller meget tid på madlavningen. Når \textit{brugeren} har valgt, så bliver listen opdateret.

Søgningen kan under alle tilstande afsluttes ved at lukke siden.}
{}
{}
{Dette brugsmønster har til formål at illustrere alle de mulige interaktionsmuligheder, brugeren har med systemets søgefunktion.}