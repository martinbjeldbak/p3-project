\brugtabel{Indlogning}{indlogning}
{Indlogning igangsættes af \textit{brugeren}. Der er to startmuligheder, og de ender begge i samme sluttilstand. Se \figref{fig:bm-indlogning}. Den første mulighed er, hvis \textit{brugeren} var logget ind fra en tidligere session, så hentes oplysningerne fra denne session automatisk og \textit{brugeren} vil blive logget ind. Den anden mulighed er, at \textit{brugeren} trykker på “Login / Opret”, som kan tilgås fra en vilkårlig \Foodl-underside. Systemet venter nu på \textit{brugerens} loginoplysninger. \textit{Brugeren} indtaster oplysningerne, og systemet påbegynder godkendelsesprocessen. Hvis oplysningerne bliver afvist, så skal \textit{brugeren} rapporterer systmet fejlen ved oplysninger og beder \textit{brugeren} genindtaste oplysnignerne. Hvis de bliver godkendt, så bliver \textit{brugeren} logget ind på siden. En anden mulighed er, hvis \textit{brugeren} ønsker at lave en ny konto i systemet. Dette sker på samme del af systemet, hvor \textit{brugeren} kan logge ind. \textit{Brugeren} skal nu indtaste e-mail og adgangskode i systemet. Hvis der opstår en fejl i oplysningerne, så skal \textit{brugeren} genindtaste oplysningerne. Når oplysningerne bliver godkendt, så bliver \textit{brugeren} logget ind med den e-mail og adgangskode, som er blevet indtastet. På hvilket som helst tidspunkt, har \textit{brugeren} mulighed for at annullere loginprocessen, hvorefter man naturligvis ikke bliver logget ind eller oprettet i systemet.

Hvis der er favoriseringer eller hvis indkøbslisten indeholder varer fra hvor \textit{brugeren} ikke var logget ind, så overføres disse til den nylig oprettet bruger til yderligere redigering og tilføjelse.

Hvis \textit{brugeren} er logget ind på en konto, så bliver \textit{brugerens} indkøbsliste og favoritter indlæst. Det muliggøre også fejlrapportering på de dele af systemet, der understøtter det. Disse kan tilgås fra en vilkårlig \Foodl-underside. \textit{Brugeren} kan nu oprette eller fjerne favoritter og tilgå og håndtere indkøbslisten.}
{Person}
{Registrer bruger, Log ind, Log ud, Ret brugerdata, Slet bruger}
{Brugsmønster for indlogning.}
