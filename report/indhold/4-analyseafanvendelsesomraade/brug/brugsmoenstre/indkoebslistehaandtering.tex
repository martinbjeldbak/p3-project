\brugtabel{Indkøbslistehåndtering}{indkoebslistehaandtering}
{Håndteringen af indkøbslisten følger materialemønsteret\cite[p.~128]{ooad}, hvor indkøbslisten er materialet, der arbejdes på. Håndteringen igangsættes af \textit{brugeren}. \textit{Brugeren} kan tilgå indkøbslisten fra en vilkårlig underside på \Foodl. Når \textit{brugeren} har tilgået indkøbslisten, så er redigeringen igangsat, og det er muligt at tilføje/fjerne varer fra indkøbslisten. Alle indgange i indkøbslisten er tekststrenge, hvilket gør det muligt for \textit{brugeren} at indtaste vilkårlige varer, ikke blot ingredienser, der er relateret til \fx en opskrift. \textit{Brugeren} har også mulighed for at tilføje/fjerne varer eller alle opskriftens ingredienser direkte fra søgeresultatet, der er en liste af opskrifter. Når \textit{brugeren} forlader søgeresultatet, så er indkøbslisten gemt, og den kan stadig tilgås fra en vilkårlig underside på \Foodl. 

Der kan altid kun være én indkøbsliste af gangen. Indkøbslisten kan fra samme side printes eller slettes. Håndteringen af indkøbslisten afsluttes når \textit{brugeren} lukker ned for redigering - altså forlader indkøbsliste-siden. Redigering startes igen, når man tilgår indkøbslisten, og alle de tilføjede var, der ikke er blevet slettet, vil stadig være tilgængelige.

Brugsmønsteret kan ses på diagramform i \figref{fig:bm-indkoebslistehaandtering}.}
{}
{Håndter indkøbsliste}
{Brugsmønster for håndteringen af indkøbslisten.}