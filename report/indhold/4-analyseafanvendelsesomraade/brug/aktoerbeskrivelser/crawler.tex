\aktortabel{Dataudtrækker}{ak-crawler}
{Dataudtrækkeren er et system, der tilføjer madopskrifter til \Foodl{} i form af data, repræsenteret på en måde, som \Foodl{} kan forstå. \Foodl{} forstår opskrifter som noget, der har et navn, billede, tilberedningstid, portionsstørrelse, en liste af råvarer og en fremgangsmåde. En Dataudtrækker kan \fx være et system, der besøger en hjemmeside med madopskrifter, udtrækker den nødvendige data omkring forskellige opskrifter, oversætter dataet og sender det til \Foodl{}. En opskrift kan bestå af ingredienser, som \Foodl{} ikke kender til. Derfor skal disse ingredienser også oversættes til en af de råvarer, som \Foodl{} kender.}
{Dataudtrækkeren er utrolig systematisk i måden, hvorpå den læser og oversætter opskrifter. Den ikke fleksibel dens arbejde, da den kun kan oversætte opskrifter med et layout, der er magen til det layout den har lært at læse. En dataudtrækker er heller ikke kritisk overfor det materiale den læser. Hvis en den pludselig støder på en opskrift, hvis billede er stødende, ville den ikke bemærke dette, og stadigvæk forsøger at indsende dette billede til \Foodl{}.}
{Dataudtrækker A fungerer på en hjemmeside med opskrifter. Den bevæger sig imellem alle de forskellige opskrifter på hjemmesiden og udtrækker informationen omkring opskrifterne fra sidernes kildekode. Opskrifternes billede er gemt i et, af \Foodl, ukendt format, hvorfor dataudtrækker A konverterer billedet. Tilberedningstiden er enten hurtig eller langsom, som også oversættes til henholdsvis 30 og 60 minutter for at det kan forstås af \Foodl{}}