\chapter{Problemstilling}
\label{chap:problemstilling}

%INTRODUKTION TIL HELE KAPITLET
Indledningsmæssigt beskrives og dokumenteres et overordnet og bredt problem. Læseren introduceres for forbrugerne, altså danskerne i de danske husstande, og de problemer, de muligvis har mht. madlavningen i husstanden. Vi arbejder sammen med to informanter (Merete og Keld), der begge bor i Aalborg omegn og sørger for madlavningen for deres familier. Informanterne står til rådighed, når vi har spørgsmål vedrørende \fx madlavnings- og madspildsområdet. Disse områder betegner vi som problemområdet. Derudover bruger vi deres hjælp til afprøvning af vores idéer. De hjælper vores udviklingsproces ved at komme med deres meninger om en given idé og forslag til forbedringer.

Herefter beskrives selve udviklingsopgaven, som vi skal arbejde med og udvikle en løsning til. Hertil bliver informanterne brugt i høj grad ved at give os respons på vores idéer. Vi tager deres idéer i betragtning, når vi skal definere og afgrænse problemet vha. en systemdefinition (beskrivelse af en IT-løsning, udtrykt i naturligt sprog).

\todo{Læseren skal introduceres for metodeafsnittet!}

%\section{Initierende problem}
\label{sec:initierendeproblem}

I Danmark findes der omkring 2,6 millioner husstande \cite{husstande}, der dagligt skal få madlavningen til at gå op i en højere enhed. Der er nemlig mange ting at tage højde for under madlavningen. Der skal tænkes på sundhed for den enkelte, i form af selve kosten, men også sundhed for alle på længere sigt, hvilket opnås ved, at vi i samlet flok skåner miljøet ved at anvende madrester.

En ensformig kost er langt fra lige så sund som en varieret kost. En årsag til ensformighed i madlavningen kan være en travl hverdag, hvor man knytter sig til faste vaner som \fx at lave den samme ret ofte, fordi man synes, den smager rigtig godt og samtidig er lynhurtig at lave. Kosten kan også blive ensformig, fordi man benytter resterne fra madlavningen i nøjagtig samme opskrift dagen efter.

Under madlavningen kan miljøet skånes ved at undgå at smide for meget mad væk. En parcelhusejer smider i gennemsnit 42 kilo spiseligt mad ud om året. \cite{madspildpol} Hvis man ikke vil benytte madresterne fra foregående aften i en lignende ret dagen efter, så kan det være svært at finde en ny ret, hvor man kan benytte de specifikke madrester. I kogebøger bliver det hurtigt uoverskueligt, at skulle søge igennem opskrifter for at finde en opskrift, hvor man benytter sig af de madrester, man har på det pågældende tidspunkt. 

Når man endelig står i supermarkedet, så sælges alt i kæmpe portioner. Hakket oksekød findes typisk i pakker med 500 gram som det mindste. Til en enkelt person er dette ofte for meget, og derved risikerer man enten at stå med 100-200 gram hakket oksekød til skraldespanden, eller et problem med hvor dette kød nu skal benyttes.

Tal fra Politiken viser, at en parcelhusejer i gennemsnit smider 42 kilo spiseligt mad ud om året. \cite{madspildpol}
Sådanne typer madspild koster desuden danske husholdninger 16 milliarder kroner om året, eller ca. 20 \% af madforbruget af en gennemsnitlig dansk børnefamilie. \cite{madspild16}

\section{Situation}
\label{sec:situation}

%initierende problem
I Danmark findes der omkring 2,6 millioner husstande \cite{husstande}, der dagligt skal få madlavningen til at gå op i en højere enhed. Der er nemlig mange ting at tage højde for under madlavningen. Der skal tænkes på sundhed for den enkelte, i form af selve kosten, men også sundhed for alle på længere sigt, hvilket opnås ved, at vi i samlet flok skåner miljøet ved at anvende madrester. Under madlavningen kan miljøet skånes ved at undgå at smide for meget mad væk. Hvis man ikke vil benytte madresterne fra foregående aften i en lignende ret dagen efter, så kan det være svært at finde en ny ret, hvor man kan benytte de specifikke madrester. I kogebøger bliver det hurtigt uoverskueligt, at skulle søge igennem opskrifter for at finde en opskrift, hvor man benytter sig af de madrester, man har på det pågældende tidspunkt.

En ensformig kost er langt fra lige så sund som en varieret kost. En årsag til ensformighed i madlavningen kan være en travl hverdag, hvor man knytter sig til faste vaner som \fx at lave den samme ret ofte, fordi man synes, den smager rigtig godt og samtidig er lynhurtig at lave. Kosten kan også blive ensformig, fordi man benytter resterne fra madlavningen i nøjagtig samme opskrift dagen efter.

Når man endelig står i supermarkedet, så sælges alt i kæmpe portioner. Hakket oksekød findes typisk i pakker med 500 gram som det mindste. Til en enkelt person er dette ofte for meget, og derved risikerer man enten at stå med 100-200 gram hakket oksekød til skraldespanden, eller et problem med hvor dette kød nu skal benyttes.

Tal fra Politiken viser, at en parcelhusejer i gennemsnit smider 42 kilo spiseligt mad ud om året. \cite{madspildpol}
Sådanne typer madspild koster desuden danske husholdninger 16 milliarder kroner om året, eller ca. 20 \% af madforbruget af en gennemsnitlig dansk børnefamilie. \cite{madspild16}

%indledning
To familier kan ikke repræsentere en befolknings madvaner eller madlavningspolitik. Vi er helt klare over, at vores to informanters madvaner og lignende ikke kan generaliseres til hele Danmark. Vi vurderer dog, at samtalerne med informanterne giver et godt billede af, hvordan stituationen, mht. madlavningen og madvaner, ser ud i danske husstande. Merete og Keld kommer fra to forskellige familier i Aalborg.

%situation
Situation er, at begge vore informanter står for madlavningen i deres husstande. De er begge opmærksomme på, at der er dele af aftensmaden, der bliver smidt ud. Denne madspild forekommer selvom, at de prøver at genbruge madresterne ved bl.a. at fryse madresterne ned eller genbruge dem den kommende dag i \fx biksemad, supper osv. Det forekommer ofte, at familierne får gårsdagens rester til dagens aftensmad.

%madvaner og madlavning
Det er vigtigt at være opmærksom på, at Merete har været vant til at lave aftensmad til to drenge og en pige ud over hende selv og hendes mand. Ægteparrets børn er flyttet hjemmefra, og de spiser derfor sjældent med hos forældrene. Det er svært at få mængden af aftensmad til at passe, så der ikke er nogen rester, når alle er mætte. Merete er også meget opmærksom på holdbarhedsdatoerne på de forskellige madvarer. Hvis den dato bliver overskredet, så bliver maden smidt ud med det samme.

Derudover forklarer Keld, at han med vilje laver ekstra store portioner til aftensmaden, så familien kan få resterne fra dagens aftensmad den næste dag. Denne strategi benytter Keld sig af, fordi han mener, at der ikke altid er meget tid tilovers til madlavningen. Ægteparret har to små børn, der skal passes og bruges tid på. Ud over at tage sig af børnene, så har de også hver deres arbejde, som skal ses til. Derfor er tid ikke noget, som ægteparret har meget af, og de bruger lignende tricks til at bruge mindre tid på madlavningen og mere tid på at være sammen. Det er helt tydeligt, at tiden er en vigtig faktor for Kelds familie, og det er netop derfor, at familien ofte får de samme retter til aftensmad.

%indkøbsliste
Når det kommer til indkøb af madvarer, så er det ikke altid, at der bliver brugt en indkøbsseddel til at planlægge indkøbbet. Den person, der har tid, handler ind. Merete og hendes mand kan bedst lide at gå på opdagelse i supermarkedet og se om de kan finde nogle gode tilbud, som de kan lave noget aftensmad ud af. Keld derimod står altid for indkøb, og han har ofte en plan i hovedet eller en liste i hånden over, hvad han skal have købt med hjem til aftensmaden. Han påpeger, at det ofte forekommer, at han får købt lidt andet godt (slik osv.) med hjem end der stod på indkøbssedlen.

%madplan
Hverken Merete eller Keld benytter sig af en madplan, når ugens aftensmad skal planlægges. De har ofte idéerne til aftensmaden i hovedet, og madlavningen er rutinepræget. Aftensmaden er meget ensformig, fordi fremgangsmåden er velkendt og derved nem og hurtig at lave. Derudover er det svært for familierne at planlægge tidspunktet for aftensmaden, fordi de alle har jobs, der skal ses til. Derfor ændrer deres planer sig pludseligt, og det vil være svært at styre en madplan, når arbejdstiderne kan variere.

%afslutning
På baggrund af samtalerne med informanterne, formulerer vi selve udviklingsopgaven, som vi skal arbejde videre med. Herunder skal vi udarbejde systemdefinitioner og udvælge en af disse, som vi skal basere vores videre arbejde på. Informanterne skal vurdere og give os respons på, hvad de mener om de definitioner, vi har udarbejdet. De skal kunne bruge systemet i deres travle hverdag. Formålet er at få valgt et system, der kan hjælpe familierne med at få genbrugt madrester på en fornuftig og intuitiv måde.
\section{Udviklingsopgaven}
\label{sec:udviklingsopaven}

%introduktion
Det primære formål med samtalerne med informanterne er at undersøge, hvordan en almindelig hverdag med madlavning foregår i danske husstande. Ud fra sådan en undersøgelse vil vi være i standt til at forstå, hvordan et system vil være i stand til at hjælpe familierne med madlavningen. Derudover er samtalerne blevet brugt til idégenerering undervejs, da informanterne har nogle gode løsningsidéer, som vi kan arbejde videre med i projektet. 

%opgaven
Vi har nu fået en indsigt i, hvordan de to familier håndterer og planlægger madlavningen i deres travle hverdage. Det kan meget hurtigt blive rutinepræget mad, der bliver serveret, fordi den madansvarlige har kendskab til retten og ved hvor lang tid, der skal bruges på den. Familierne prøver så vidt muligt at bruge så lidt tid på madlavningen som muligt, så de har mere tid til at hygge sig med familien. Det forekommer ofte, at der er madrester fra aftensmaden, og familierne gør deres bedste for at få spist eller genbrugt alle resterne. Dette lykkedes dog ikke hver aften, og derfor bliver der en gang i mellem smidt mad ud i affaldet. 

Når der er madrester fra gårsdagens aftensmad, så har den madansvarlige i familien sjældent tid til manuelt at kigge kogebøger eller hjemmesider med opskrifter igennem for at finde en ret, hvor madresterne kan genbruges. Dette medfører tit, at aftensmaden bliver ensformig og ofte er det blot den samme ret, der bliver genbrugt. På denne måde sparer familierne også tid på madlavningen.

Udviklingsopgaven lyder på, at vi skal udvikle et system, der hurtigt og nemt kan bruges til at planlægge madlavningen i danske husstande. 

%afslutning
Hvordan udføres denne opgave? Vi starter med at formulere nogle systemdefinitioner, der har til formål at løse problemerne mht. planlægningen og madlavningen. Disse systemdefinitioner bliver udarbejdet ud fra de samtaler, vi har haft med informanterne. Derefter præsenterer vi definitioner for informanter, der skal give os deres meninger og idéer vedrørende systemdefinitionerne.
\section{Systemvalg}
I det private køkken, sker det til tider, at der smides mad væk. En del af madspildet kan for eksempel ske efter endt madlavning, hvor man kan risikere at stå tilbage med nogle fødevarer, der helst skal bruges hurtigst muligt for ikke at blive for gamle. Man kunne for eksempel have lavet hele kyllingbryst med persille, og fået en masse af begge dele tilovers. Næste aften har man ikke lyst til at få samme ret igen, men maden kan ikke holde sig meget længere, så man fristes derfor til at smide maden væk. Dette problem forsøger vi at løse ved at udvikle et system. Systemet vælger vi at kalde Foodl.

\subsection{Møde 1 med informanter}
Formålet med møde 1 er at opnå en større indsigt i informantens mad- og indkøbsvaner og på den måde få en forståelse for hvilke problemer der er mulighed for at løse ved udviklingen af et system.

Det er på baggrund af møde 1 er det blevet gjort klart, at der blandt informanterne findes flere forskellige problemer vedrørende madlavning, hvoraf nogle er prioriterede højere end andre blandt informanterne. Ved at bruge BATOFF-modellen, er 2 systemdefinitioner blevet lavet. Systemdefinitionerne beskriver kort 2 forskellige systemer, der hver især løser nogle af de, af informanterne, rejste problemstillinger .

\section{Metodeovervejelser}
\label{sec:metodeovervejelser}

%INTRODUKTION
I det videre arbejde med projektet vil informanterne være en stor del af afprøvnings- og kvalitetssikringsdelene af projektet. De skal være med til at sikre, at vi udvikler et system, der stemmer overens med systemdefinitionen og dækker deres behov mht. madlavning i husstanden. Dette gør vi ved at udvikle forskellige prototyper, som vi præsenterer for informanterne. Disse prototyper repræsenterer de idéer, vi har til en given funktion eller systemsdel, som vi ønsker at få feedback på.

\subsection{Prototyper}
\label{subsec:prototyper}

Vi vil benytte os af papirsprototyper til de initierende afprøvninger. Denne form for prototype er ikke tidskrævende og er en billig form for præsentation. I og med at det er initierende afprøvninger, så mener vi, at det er vigtigt, at vi ikke bruger for mange kræfter på at udvikle en prototype. Der er en risiko for, at vi kan ende med at have brugt så meget tid på prototypen, at vi vil have svært ved at give slip på idéen, hvis informanterne ønsker noget helt andet end prototypen skal illustrere. I de senere afprøvninger kan vi bruge mere tid på prototyperne, da der på de senere tidspunkter er blevet fastlagt nogle stramme rammer, der har til formål at afgrænse systemets udviklingsproces. Dvs., at vi ikke burde ende med et system, der slet ikke har noget med de rammer at gøre. I og med at de rammer bliver fastlagt, så er risikoen for, at informanterne ønsker et system, der er helt anderledes end det, vi har udviklet, meget smal.

Vi kan i et lidt senere forløb i projektet præsentere informanterne for diasshow-prototyper, der har til formål at undersøge systemets funktioner. Med sådan en prototype bliver informanterne præsenteret for en prototype, der er dynamisk. Man kan klikke på knapper og navigere rundt i diasshowet. En diasshow-prototype giver os og informanterne mulighed for at lege lidt med vores idé for systemet. 

Systemets brugbarhed er en vigtig faktor for os. Vi ønsker at gøre det lettere for den madansvarlige i husstanden at bruge sine madrester i madlavningen. Derfor skal systemet være intuitivt og nemt at gå til. Med diasshow-prototyper får vi mulighed for at sikre os, at vores designidéer bliver forstået af informanterne. Hvis informanterne \fx har svært ved at finde nogle funktioner, så kan det være, at de skal gøres mere synlige. Lignende spørgsmål bliver besvaret relativt tidligt i systemets udviklingsfase, hvilket er en fornuftig ting. Det giver os mere tid til at rette fejl og komme på nye designidéer, hvis det bliver nødvendigt.

Som nogle af de sidste afprøvninger, vil vi præsentere et funktionsdygtigt system for informanterne. Dette system vil blive baseret på alle de foregående prototyper og designidéer. Ved at præsentere informanterne for det funktionsdygtige system, bliver vi i stand til at sikre os en vis kvalitet, inden vi afslutter projektet og produktudviklingen. Til de sidste afprøvninger er der mulighed for at opdage kritiske fejl, fordi vi lader en person, der ikke har været med til at udvikle systemet, lege med det. Her kan brugeren foretage sig nogle valg i systemet, som vi måske ikke har tænkt over. Disse uforudsete valg skal ikke være skyld i at systemet går ned, og denne sidste afprøvning er en god sikring for dette. 
\section{Evolutionær metode}
\label{sec:evolution}

Det er besværligt at planlægge en arbejdsproces for et virkelighedsnært problem. Det kræver meget forarbejde, og man bliver nødt til at fortolke og revidere problemstillingen mange gange, indtil man har tilstrækkelig nok viden og forståelse for problemet. Dette gøres på bedste vis ved at have tæt kommunikation og interaktion med brugerne af et system, der kan løse problemet. Den evolutionære arbejdsproces går ud på, at man skal eksperimentere med bl.a. prototyper til at skabe sig en forståelse for problemet. \cite{cic} Den evolutionære metode er en iterativ arbejdstilgang, hvor der er forskellige faser, hvor man reviderer og bearbejder analysen, designet af systemet, implementering og kvalitetssikring i hver fase.

Vi havde selvfølgelig nogle lineære tilgange i planlægningen, men ser vi på det store billede, så var det den iterative metode, der dominerede. Da projektet strakte sig over 4 måneder, valgte vi at dele projektforløbet op i 4 faser á 3 ugers varighed. Vi havde forskellige mål med hver fase, hvor vi definerede hovedhensigten og de undermål, der skulle løses i løbet af fasen. For at opnå de mål, ifølge den evolutionære arbejdsmetode, skulle vi arbejde på analysen, designet, implementeringen og kvalitetssikring af systemet i hver enkelt fase. Som skrevet, forsøgte vi at arbejde så evolutionært som muligt, dog lykkedes det ikke helt, da vi \fx i fase 1 ikke arbejdede på implementering eller kvalitetssikring af systemet.

%planlægning er usikkert
%trial and error
%problemet bliver fortolket og revideret (iterationer)
%tæt kommunikation og interaktion med brugere

\ourtable{iterationeroverblik}{3}{Denne tabel giver et hurtigt og kortfattet overblik over projektets arbejdsfaser. Her ses de forskellige faser af den iterative arbejdsmetode med tilhørende formål og de resultater, vi har fået ud af de forskellige faser. Desuden kan man se, hvordan informanterne er blevet inddraget i processen.}
             {Beskrivelser}
       {Fase}  {Formål                       & Resultat                           & Informanternes inddragelse}{
\ourrow{1   }{At få indblik i informanternes problemstillinger, mht. madlavning og anvendelse af deres madrester, og at modellere disse. At definere et system, og forstå hvilke funktioner informanterne har brug for. & \textit{Tilføjet:} \textbf{Problmeområdet.} Klasser. Hændelser. Hændelsestabel. Klassestruktur. Prototype med fokus på søgefunktionalitet.  & Møde med fokus på problemet. Møde med fokus på løsning. Prototype med fokus på søgefunktionalitet.}
\ourrow{2   }{At sikre os, at de funktioner vi har i systemet passer med informanternes behov og at dokumentere og modellere disse. & \textit{Tilføjet:} \textbf{Problemområdet. Anvendelsesområdet.} Brugsmønstre. Funktioner. Aktører. Kriterier. Forbilleder. \textit{Revideret:} \textbf{Problemområdet.} Prototype med fokus på funktionalitet.  & Prototype med fokus på funktionalitet.}
\ourrow{3   }{At modellere systemet, ved at implementere funktioner og sikre os, at de definerede kriterier bliver opfyldt.     & \textit{Tilføjet:} \textbf{Implementering.} Sammensætning af rapport. Komponenter. Udtrækning af data i opskrifter. \textit{Revideret:} \textbf{Problemområdet. Anvendelsesområdet. Design.}  &  }
\ourrow{4   }{  At implementere det designede system.                                    & \textit{Tilføjet:} \textbf{Implementering. Kvalitetssikring.} \textit{Revideret:} \textbf{Problemområdet. Anvendelsesområdet. Design.}                                &           %Bestemmelse af bedste metode til at mappe ingrediens $\rightarrow$ råvare. 
Usability-test. }
}

