\section{Udviklingsopgaven}
\label{sec:udviklingsopaven}

%introduktion
Det primære formål med samtalerne med informanterne er at undersøge, hvordan en almindelig hverdag med madlavning foregår i danske husstande. Ud fra sådan en undersøgelse vil vi være i stand til at forstå, hvordan et system vil være i stand til at hjælpe familierne med madlavningen. Derudover er samtalerne blevet brugt til idégenerering undervejs, da informanterne har nogle gode løsningsidéer, som vi kan arbejde videre med i projektet. 

%opgaven
Vi har nu fået en indsigt i, hvordan de to familier håndterer og planlægger madlavningen i deres travle hverdage. Det kan meget hurtigt blive rutinepræget mad, der bliver serveret, fordi den madansvarlige har kendskab til retten og ved hvor lang tid, der skal bruges på den. Familierne prøver så vidt muligt at bruge så lidt tid på madlavningen som muligt, så de har mere tid til at hygge sig med familien. Det forekommer ofte, at der er madrester fra aftensmaden, og familierne gør deres bedste for at få spist eller genbrugt alle resterne. Dette lykkedes dog ikke hver aften, og derfor bliver der en gang i mellem smidt mad ud i affaldet. 

Når der er madrester fra gårsdagens aftensmad, så har den madansvarlige i familien sjældent tid til manuelt at kigge kogebøger eller hjemmesider med opskrifter igennem for at finde en ret, hvor madresterne kan genbruges. Dette medfører tit, at aftensmaden bliver ensformig og ofte er det blot den samme ret, der bliver genbrugt. På denne måde sparer familierne også tid på madlavningen.

Udviklingsopgaven lyder på, at vi skal udvikle et system, der hurtigt og nemt kan bruges af den madansvarlige til at planlægge madlavningen i danske husstande, og derved mindske madspildet. 

%afslutning
Hvordan udføres denne opgave? Vi starter med at formulere systemdefinitioner, der har til formål at løse problemerne mht. planlægningen og madlavningen. En systemdefinition er en kortfattet og præcis beskrivelse af en IT-løsning, der er udtrykt i naturligt sprog. Disse systemdefinitioner bliver udarbejdet ud fra de samtaler, vi har haft med informanterne. Derefter præsenterer vi definitioner for informanter, der skal give os deres meninger og idéer om disse.