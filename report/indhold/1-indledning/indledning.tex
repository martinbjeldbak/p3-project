\chapter{Indledning}
\label{chap:indledning}

<<<<<<< HEAD
Mange mennesker vil på et tidspunkt opleve, at deres madlavning begynder at bære præg af ensformighed. Man har et par faste retter, som man er vant til at lave, og i en stresset hverdag er det fristende at holde sig til det, man kender, for at prioritere mere tid til andre ting end madlavning. Vil man dog prøve noget nyt, kan det tage lang tid at bladre en kogebog igennem, før man finder noget, man har lyst til at lave. Ville det ikke være meget lettere med en kogebog, der kun indeholdt de opskrifter, man rent faktisk kunne lave på baggrund af indholdet i ens køleskab. Et stort problem er, at størrelsen på de varer, man køber i supermarkederne, ofte er tilpasset familier på 2-4 personer. Tal fra Politiken viser, at en parcelhus-dansker i gennemsnit smider 42 kilo spiseligt mad ud om året \cite{madspildpol}. Eksempelvis er et halvt kilo oksekød eller en bøtte creme fraiche for meget til én person. Som dansker står man ofte med et køleskab fyldt med rester, som ikke fik plads i de foregående dages aftensmad. Sådanne typer madspild koster desuden danske husholdninger 16 milliarder kroner om året, eller ca. 20 \% af madforbruget af en gennemsnitlig dansk børnefamilie \cite{madspild16}. 
=======
Alle ved, at mad er livsnødvendigt. I Danmark findes der omkring 2,6 millioner husstande\cite{husstande}, der dagligt skal få madlavningen til at gå op i en højere enhed. Der er nemlig mange ting at tage højde for under madlavningen. Der skal tænkes på sundhed for den enkelte i form af kosten, men også sundhed for alle på længere sigt, hvilket opnås ved at vi i samlet flok skåner miljøet.
En ensformig mad er langt fra lige så sund som en varieret kost. En årsag til ensformighed i madlavningen kan være en travl hverdag, hvor man knytter sig til faste vaner som for eksempel at lave den samme ret ofte, fordi man synes, den smager rigtig godt og samtidig er lynhurtig at lave. Kosten kan også blive ensformig, fordi man benytter resterne fra madlavningen i nøjagtig samme opskrift dagen efter.
Under madlavningen kan miljøet skånes ved at undgå at smide for meget mad væk. En parcelhusejer smider i gennemsnit 42 kilo spiseligt mad ud om året \cite{madspildpol}. Hvis man ikke vil benytte resterne fra madlavningen i samme opskrift dagen efter, så kan det være svært at finde en ny opskrift at benytte resterne i. I kogebøger bliver det hurtigt uoverskueligt, gang på gang, at blive mødt af opskrifter, der alle kræver en tur i supermarkedet for at få fat på den selleri, man aldrig har. Når man endelig får bevæget sig ned i supermarkedet, så sælges alt i kæmpe portioner. Hakket oksekød findes typisk i pakker med 500 gram som det mindste. Til en enkelt person er dette ofte for meget, og derved risikerer man enten at stå med 100-200 gram hakket oksekød til skraldespanden, eller et problem med hvor dette kød nu skal benyttes.
>>>>>>> aa29c344fb34ac4e7d76dd0039cc7a823cc1c814

\section {Målgruppe}
Problemområdet for dette projekt er madspild, der opstår på grund af manglende kreativitet til hvordan rester kan anvendes. Der er ingen der er i tvivl om, at der findes madspild rigtig mange steder. I erhvervslivet findes madspild i form af slagteren, der ikke når at få solgt sit kød, Nettos mælk der er ved at overskride sidste salgsdato, og plejehjemmet, der fik lavet for meget suppe til de ældre. I det private køkken sker der også madspild. I den store familie, købes der stort ind, og ikke alle fødevarer når at blive anvendt, enten fordi man simpelthen har købt for stort ind, eller også fordi fødevarerne ikke har passet godt nok sammen. Man har måske haft lidt kylling liggende, og ikke syntes dette ville være passende i en lasagne. I en familie kan der være mange forskellige madønsker og kræsenheder. Oven i købet er der forskel på hvornår man bliver sulten, så samlet set er det ikke svært at forestille sig at lidt mad går til spilde fordi madlavningen ikke helt gik som tidligere planlagt, for eksempel under indkøbet.
Med alle disse steder man kan forestille sig at møde madspild, er det nødvendigt for os at tage en beslutning om hvilken gruppe vi vil fokusere på at mindske madspildet hos. En for stor målgruppe kan risikere at blive så uhomogen, at et system til målgruppen i sidste ende ikke vil kunne udvikles, fordi kravene til systemet vil være alt for forskellige.

For at indsnævre målgruppen vælger vi at fokusere på madspild i det private køkken. Det gør vi fordi de valgte informanter arbejder ikke med mad inden for erhvervslivet og kan dermed ikke kommentere på madspildet i det område.

I det private køkken, sker det til tider, at der smides mad væk. Årligt har man beregnet at hver dansker smider ca. 63 kg fødevarer væk. \url{http://politiken.dk/mad/madnyt/ECE527771/hver-dansker-smider-63-kg-mad-ud/} 
En del af madspildet kan for eksempel ske efter endt madlavning, hvor man kan risikere at stå tilbage med nogle fødevarer, der helst skal bruges hurtigst muligt for ikke at blive for gamle. Man kunne for eksempel have lavet hele kyllingbryst med persille, og fået en masse af begge dele tilovers. Næste aften har man ikke lyst til at få samme ret, men maden kan ikke holde sig meget længere, så man fristes derfor til at smide maden væk. Maden kan dog godt anvendes på en anden måde end sidst. Man kunne for eksempel skære kyllingen i tern i en sammenkogt ret og persillen kunne komme i en salat i stedet for ud over kyllingen som i sidst. Denne viden er det ikke alle, der bærer rundt på, og hvis man med en kreativ tilgang blot prøver at blande nogle forskellige ingredienser, kan man ende med et måltid, der smager knap så godt som man forestillede sig.

