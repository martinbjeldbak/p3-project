\chapter{Indledning}
\label{chap:indledning}

\section{Initierende problem}
Mange mennesker vil på et tidspunkt opleve, at deres madlavning begynder at bære præg af ensformighed. Man har et par faste retter, som man er vant til at lave, og i en stresset hverdag er det fristende at holde sig til det, man kender, for at prioritere mere tid til andre ting end madlavning. Vil man dog prøve noget nyt, kan det tage lang tid at bladre en kogebog igennem, før man finder noget, man har lyst til at lave. Ville det ikke være meget lettere med en kogebog, der kun indeholdt de opskrifter, man rent faktisk kunne lave på baggrund af indholdet i ens køleskab. Et stort problem er, at størrelsen på de varer, man køber i supermarkederne, ofte er tilpasset familier på 2-4 personer. Tal fra Politiken viser, at en parcelhus-dansker i gennemsnit smider 42 kilo spiseligt mad ud om året \cite{madspildpol}. Eksempelvis er et halvt kilo oksekød eller en bøtte creme fraiche for meget til én person. Som dansker står man ofte med et køleskab fyldt med rester, som ikke fik plads i de foregående dages aftensmad. Sådanne typer madspild koster desuden danske husholdninger 16 milliarder kroner om året, eller ca. 20 \% af madforbruget af en gennemsnitlig dansk børnefamilie \cite{madspild16}. 

Kunne man ikke lave en webapplikation, der fungerede som en almindelig kogebog, men som blot viser de opskrifter, man allerede har nogle af ingredienserne til? 

Hvis brugeren har en halv pakke hakket oksekød og en håndfuld tomater, der bliver for gammel i morgen, så ville det være rart at kunne få vist alle de opskrifter, hvor hakket oksekød og tomater indgår. På den måde kunne man undgå madspild og samtidig lave varieret mad på en spændende og kreativ måde.
