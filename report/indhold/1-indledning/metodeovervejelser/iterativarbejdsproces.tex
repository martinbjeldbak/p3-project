\subsection{Den iterative arbejdsproces}
\label{subsec:iterativarbejdsproces}

Det er meget svært at planlægge en arbejdsproces for et virkelighedsnært problem. Det kræver meget forarbejde og man bliver nødt til at fortolke og revidere problemstillingen mange gange, indtil man har tilstrækkelig nok viden og forståelse for problemet. Dette gøres på bedste vis ved at have tæt kommunikation og interaktion med brugerne af et system, der kan løse problemet. Den evolutionære arbejdsproces går ud på, at man skal eksperimentere med bl.a. prototyper til at skabe sig en forståelse for problemet. Den evolutionære metode er en iterativ proces, hvor man har forskellige faser, hvor man reviderer og bearbejder analysen, designet af systemet, programmering, aftestning og afprøvning.

Vi har selvfølgelig nogle lineære tilgange i planlægningen, men ser vi på det store billede, så er det den iterative metode, der dominerer.

\todo{smid lige en kilde ind.}

%planlægning er usikkert
%trial and error
%problemet bliver fortolket og revideret (iterationer)
%tæt kommunikation og interaktion med brugere

%Der findes i forvejen systemer, der har til formål at løse netop lignende problemer, som det problem, vi er kommet frem til. Men hvordan fungerer disse systemer? Hvad er deres stærke sider? Har de nogle negative sider? Dette og lignende problemstillinger vil vi undersøge i de kommende afsnit. Vi håber på at kunne lære noget af lignende systemer. På denne måde kan vi bruge de stærke sider og evt. implementere lignende egenskaber i vores eget system og lære af deres negative sider.

I de følgende afsnit, analyserer vi problem- og anvendelsesområdet. Disse analyser skal bruges til at skabe en bedre forståelse for selve problemet og anvendelsen af en mulig løsning.