\subsection{Prototyper}
\label{subsec:prototyper}

Vi vil benytte os af papirsprototyper til de initierende afprøvninger. Denne form for prototype er ikke tidskrævende og er en billig form for præsentation. I og med at det er initierende afprøvninger, så mener vi, at det er vigtigt, at vi ikke bruger for mange kræfter på at udvikle en prototype. Der er en risiko for, at vi kan ende med at have brugt så meget tid på prototypen, at vi vil have svært ved at give slip på idéen, hvis informanterne ønsker noget helt andet end prototypen skal illustrere. I de senere afprøvninger kan vi bruge mere tid på prototyperne, da der på de senere tidspunkter er blevet fastlagt nogle stramme rammer, der har til formål at afgrænse systemets udviklingsproces. Dvs., at vi ikke burde ende med et system, der slet ikke har noget med de rammer at gøre. I og med at de rammer bliver fastlagt, så er risikoen for, at informanterne ønsker et system, der er helt anderledes end det, vi har udviklet, meget smal.

Vi kan i et lidt senere forløb i projektet præsentere informanterne for diasshow-prototyper, der har til formål at undersøge systemets funktioner. Med sådan en prototype bliver informanterne præsenteret for en prototype, der er dynamisk. Man kan klikke på knapper og navigere rundt i diasshowet. En diasshow-prototype giver os og informanterne mulighed for at lege lidt med vores idé for systemet. 

Systemets brugbarhed er en vigtig faktor for os. Vi ønsker at gøre det lettere for den madansvarlige i husstanden at bruge sine madrester i madlavningen. Derfor skal systemet være intuitivt og nemt at gå til. Med diasshow-prototyper får vi mulighed for at sikre os, at vores designidéer bliver forstået af informanterne. Hvis informanterne \fx har svært ved at finde nogle funktioner, så kan det være, at de skal gøres mere synlige. Lignende spørgsmål bliver besvaret relativt tidligt i systemets udviklingsfase, hvilket er en fornuftig ting. Det giver os mere tid til at rette fejl og komme på nye designidéer, hvis det bliver nødvendigt.

Som nogle af de sidste afprøvninger, vil vi præsentere et funktionsdygtigt system for informanterne. Dette system vil blive baseret på alle de foregående prototyper og designidéer. Ved at præsentere informanterne for det funktionsdygtige system, bliver vi i stand til at sikre os en vis kvalitet, inden vi afslutter projektet og produktudviklingen. Til de sidste afprøvninger er der mulighed for at opdage kritiske fejl, fordi vi lader en person, der ikke har været med til at udvikle systemet, lege med det. Her kan brugeren foretage sig nogle valg i systemet, som vi måske ikke har tænkt over. Disse uforudsete valg skal ikke være skyld i at systemet går ned, og denne sidste afprøvning er en god sikring for dette. 