\section {Målgruppe}
Problemområdet for dette projekt er madspild, der opstår på grund af manglende kreativitet til hvordan rester kan anvendes. Der er ingen der er i tvivl om, at der findes madspild rigtig mange steder. I erhvervslivet findes madspild i form af slagteren, der ikke når at få solgt sit kød, Nettos mælk der er ved at overskride sidste salgsdato, og plejehjemmet, der fik lavet for meget suppe til de ældre. I det private køkken sker der også madspild. I den store familie, købes der stort ind, og ikke alle fødevarer når at blive anvendt, enten fordi man simpelthen har købt for stort ind, eller også fordi fødevarerne ikke har passet godt nok sammen. Man har måske haft lidt kylling liggende, og ikke syntes dette ville være passende i en lasagne. I en familie kan der være mange forskellige madønsker og kræsenheder. Oven i købet er der forskel på hvornår man bliver sulten, så samlet set er det ikke svært at forestille sig at lidt mad går til spilde fordi madlavningen ikke helt gik som tidligere planlagt, for eksempel under indkøbet.
Med alle disse steder man kan forestille sig at møde madspild, er det nødvendigt for os at tage en beslutning om hvilken gruppe vi vil fokusere på at mindske madspildet hos. En for stor målgruppe kan risikere at blive så uhomogen, at et system til målgruppen i sidste ende ikke vil kunne udvikles, fordi kravene til systemet vil være alt for forskellige.

For at indsnævre målgruppen vælger vi at fokusere på madspild i det private køkken. Det gør vi fordi de valgte informanter arbejder ikke med mad inden for erhvervslivet og kan dermed ikke kommentere på madspildet i det område.

I det private køkken, sker det til tider, at der smides mad væk. Årligt har man beregnet at hver dansker smider ca. 63 kg fødevarer væk. \url{http://politiken.dk/mad/madnyt/ECE527771/hver-dansker-smider-63-kg-mad-ud/} 
En del af madspildet kan for eksempel ske efter endt madlavning, hvor man kan risikere at stå tilbage med nogle fødevarer, der helst skal bruges hurtigst muligt for ikke at blive for gamle. Man kunne for eksempel have lavet hele kyllingbryst med persille, og fået en masse af begge dele tilovers. Næste aften har man ikke lyst til at få samme ret, men maden kan ikke holde sig meget længere, så man fristes derfor til at smide maden væk. Maden kan dog godt anvendes på en anden måde end sidst. Man kunne for eksempel skære kyllingen i tern i en sammenkogt ret og persillen kunne komme i en salat i stedet for ud over kyllingen som i sidst. Denne viden er det ikke alle, der bærer rundt på, og hvis man med en kreativ tilgang blot prøver at blande nogle forskellige ingredienser, kan man ende med et måltid, der smager knap så godt som man forestillede sig.
