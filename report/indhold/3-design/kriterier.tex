\section{Kriterier}
I dette afsnit vurderer vi hvor væsentlige forskellige kriterier er for det færdige system, idet vi har et begrænset antal ressourcer der skal håndteres og bruges på den mest fordelagtig måde. Vi opstiller prioriteter for hver af disse kriterier for at skabe overblik over hvor vi skal koncentrere arbejdskraften. Til at få overblikket benytter og prioriterer vi Vincent et al.'s 12 \emph{klassiske} kriterier, der lyder således: \emph{brugbart, sikkert, effektivt, korrekt, pålideligt, vedligeholdbart, testbart, fleksibelt, forståeligt, genbrugelidt, flytbart} og \emph{integrerbart}. Hver af disse kriterier består af 3-5 underkriterer, der enkeltvis vurderes med resultatet af en endelig prioritering af kriteriet. \Fx har kriteriet brugbart 3 underkriterier: \emph{træning, meddelsomhed} og \emph{funktionalitet}. Vi vælger disse kriterier fremfor de generelle tre: \emph{brugbart, fleksibelt} og \emph{forståeligt} \cite{crit}, da de 3 allerede indgår i Vincent et al.'s kriterier. Hvert kriterie består af underkriterier og har 5 forskellige ``vigtighedsgrader'': meget vigtig, vigtig, mindre vigtig, irrelevant og trivielt. Trivielt er anderledes på den måde, at et afkrydsning i det felt betyder, at kriteriet bliver opfyldt som følge af en af de andre kriterier. Nogle af kriterierne strider imod eller har større indflydelse på andre kriterier, \fx hænger \emph{genbrugbart} og \emph{effektivitet} ikke særlig godt sammen og kan have en stor forskel i vurderingen (der refereres til \cite[s.~18]{crit}, hvis mere information om ``trade-offs'' imellem kriterierne ønskes).

Vi benytter kriterierne til at forbedre fordelingen af de begrænsede ressourcer samt kvalitetssikre systemet og sikre os, at der ikke er nogle indlysende svagheder i designet. I forhold til ressourcer, skal dette projekt afleveres til et specifikt deadline, i modsætning til andre systemer med mere fleksible ressourcer og deadlines. Derfor lægger vi meget vægt i at vurderingen af kriterierne er korrekte samt alle gruppemedlemmerne er enige om prioriteterne. Det er bedst at specifikt kunne vurdere kriterier i detajler for at vide, hvilken vi skal lægge ekstra kraft bag. Kvalitetssikring af systemet foregår også selvsagt under vurdering af kriterierne, da vi naturligvis ønsker et produkt af høj kvalitet, der lever op til de opstillede krav som informanterne/kunderne kræver.

Kriterene er blevet vurderet ud fra systemdefinitionen, mængden af tid, mængden af arbejdkraft og tekniske begræsnsninger mm.\ og er subjektiv i forhold til vores system. Vurderingen af kriterierne ændres i løbet af den iterative arbejdsproces, vi benytter. På baggrund af det fastlægger vi, at der kan maksimum være 1-2 \emph{meget vigtige} kriterier i dette system.

\subsection{Begrundelser for hvert kriterie}
Som skrevet er kriterierne vurderet ud fra underkriterierne i Vincent et al.'s \emph{klassiske} kriterier \cite[s.~12]{crit}. Underkriterierne er nævnet i begrundelserne for hvert kriterie, det kræves dog ikke, at man har bekendskab til underkriterierne for at læse begrundelserne. Alle gruppemedlemmer har været med til at skabe beskrivelserne, så der er blevet dannet et fælles grundlag for implementeringen af systemet. Følgende er hvert kriterie og forklaringen for dets betydning i vores system:

\paragraph{Brugbart} Systemet skal være håndgribeligt for brugeren. Det skal være nemt at bruge og nemt at gå til uden nogen form for oplæring, fordi vi vurderer, at der ikke er nogen, der ønsker at bruge meget tid på at skulle sætte sig ind i et lille system. Dette medfører, at systemet skal være intuitivt, og vi mener, \tjek{[Følgende forklarer hvordan vi vil opfylde kravet, hvilket vi ikke gør for de andre kriterier. Flyt til et andet afsnit?]} at vi opnår dette ved at designe systemet på en lignende måde som andre velkendte systemer, såsom Googles søgefunktion og udseende. Hvorfor tages der udgangspunkt i præcis Google? Google er den mest brugte søgemaskine i verden, efterfulgt af Bing og Yahoo.\cite{googlesoeg}\cite{ebizmba} Derudover er Google også den meste besøgte hjemmeside på internettet, efterfulgt af Facebook og Youtube.\cite{alexadk} Vi ønsker at nå op på et højt abstraktionsniveau, ved at holde systemet simpelt og intuitivt. Det gør det nemmere for brugeren at forstå systemet. Brugeren opnår en høj forståelighed for systemets brug som følge af, at vi ønsker at gøre systemet så brugbart som muligt. Vi ønsker at brugeren skal være i stand til at kigge på systemet og genkende funktioner fra lignende systemer, såsom Google og Facebook. På den måde benytter vi os af mennesker evne til at genkende ting, og det gør det mere intuitivt, da de i forvejen kender funktionerne fra de velkendte systemer.

\paragraph{Sikkert} Systemet behandler ikke personfølsomme oplysninger. Derfor vurderes sikkerhedskriteriet som irrelevant for systemets design.

\paragraph{Effektivt} Det er vigtigt, at søgninger og navigation i systemet foregår hurtigt og responsivt. Alle ved, at det er utroligt træls, hvis man bruger et system, hvor søgefunktioner og navigation ikke reagerer i løbet af relativt kort tid, bliver man hurtigt træt af det pågældende system. Hvis der går for mange sekunder, før der sker noget på skærmen, så kan brugeren begynde at klikke på de samme funktioner flere gange, fordi man tror, at de måske ikke har klikket korrekt i første forsøg, og det er trættende i længden. 

Marissa Mayer, tidligere leder hos Google, nu præsident og direktør for Yahoo udtalte følgende:\cite{googlespeed}
\begin{quote}
``When the Google Maps home page was put on a diet, shrunk from 100K to about 70K to 80K, traffic was up 10 percent the first week and in the following three weeks, 25 percent more.''
\end{quote}  

Man kan i sidste ende vælge at forlade systemet, fordi det er langsomt og ineffektivt, og dette viser statistikken for Google Maps hjemmeside også, da hjemmesidens trafik steg med 25 \% på tre uger efter hjemmesiden blev komprimeret. Dette medvirkede også til, at hjemmesiden blev hurtigere til at reagere på brugerinteraktionen.

\paragraph{Korrekt} Korrektheden er kategoriseret som mindre vigtigt, fordi gruppens fokus ligger på brugbarhed og effektivitet. Èt forkert søgeresultatet, her og der, kan måske skræmme nogle brugere væk, men det er ikke livsnødvendigt, at dette punkt i systemet skal være 100 \% korrekt.

\paragraph{Pålideligt} Pålideligheden er kategoriseret mindre vigtig, fordi konsistens og nøjagtighed ikke er vigtige aspekter at tage hensyn til, da det ikke er vurderet som et stort problem, hvis systemet giver ét forkert søgeresultat. Dog, mener gruppen, at fejltolerance kan vurderes en smule højere end de andre underkriterier. Det er utroligt træls, når et system går ned, fordi brugeren har udført en handling, der ikke er taget højde for, hvilket medfører at systemet går ned. Systemet skal være i stand til at melde tilbage til brugeren, at der er sket en fejl uden, at systemet går ned. Derudover er det en god idé at have en beskrivende fejlbesked, når der opstår en fejl, så brugeren kan læse sig til, hvad der er gået galt.

\paragraph{Vedligeholdbart og fleksibelt} Det er vigtigt for implementeringen af systemet, at det er nemt at udbygge, videreudvikle og vedligeholde. I og med at gruppen benytter sig af en iterativ arbejdsproces, så er det vigtigt, at gruppen kan vende tilbage til koden efter en måned og, uden problemer eller forsinkelser, være i stand til at læse og modificere i systemets funktioner. Endnu en grund til, at vedligeholdelse og fleksibilitet er vigtige kriterier, er at muligheden for fremtidige udvidelse til f.eks. andre sprog skal være overskueligt.

\paragraph{Testbart} Testbarhed hænger meget sammen med kriteriet vedligeholdbarhed og fleksibilitet, og derfor skal underkriterierne vurderes på samme niveau. Det er relevant at holde implementeringen simpel og modulær samt selvbeskrivende, men da de ikke spiller en væsentlig rolle i systemet, \tjek{[Hvorfor er det relevant og hvorfor er det ikke væsentligt? uddyb]} vurderes de som mindre vigtige.

\paragraph{Forståeligt} Forståelighed er vigtig, fordi det skal være overkommeligt at forklare systemets design i denne udviklingsrapport så simpelt og forståeligt for læseren som muligt.
 
\paragraph{Genbrugbart} Gruppen ønsker ikke at tage højde for, at andre systemer skal være i stand til at bruge det, da det ikke er beskrevet i systemdefinitionen. Derfor er dette kriterie irrelevant for projektet.

\paragraph{Flytbart} Systemet skal kunne flyttes frit mellem forskellige tekniske platforme, såsom Windows og Unix hvis det bliver nødvendigt. For eksempel skal det være muligt at køre systemet på andre webhostingsservicer. Derfor skal gruppen tænke over hvilken eksterne systemer, der bliver benyttet i det færdige produkt. Nogle eksterne systemer har platformsafhængigheder, der begrænser valget af platform, som produktet kan køre på. Derfor skal gruppen være i stand til at benytte cross-platform biblioteker.

\paragraph{Integrerbart} Da modularitet og standardisering af datarepræsentationer er en væsentlig del i projektet og fleksibilitet samt vedligeholdbarhed er vurderet som vigtig, vil gruppen fortage nogle overvejelder i forhold til integrerbarhed. Det vurderes som mindre vigig fordi systemet ikke skal benyttes i andre systemer (se ``genbrugbart'').


\subsection{Kriterietabel}
Begrundelserne for hver kriterie giver anledning til at opstille en tabel, der gør det let at sammenligne vigtigheden af kriterierne. Det er nu muligt, at kunne tælle op hvor mange kriterier er i hver grad for at undgå for mange afkrydsninger i de venstre søjler, da vi er meget begrænset i visse omfang. I sådanne tilfælde må vi gå tilbage og justere på vigtigheden af kriterierne. Tabellen ses i \tableref{table:kriterietabel}:

%\begin{table}[H]
%  \centering
%    \begin{tabular}{ r | c  c  c  c  c }
%  \hline
%  & \multicolumn{5}{c}{\textbf{Vigtighed}} \\
%   \textbf{Kriterium}    & Meget vigtig &  Vigtig  & Mindre vigtig  & Irrelevant  & Trivielt \\ \hline
%         Brugbart        & \checkmark            &                    &                         &                      &                   \\ 
%         Sikkert         &                       &                    &                         &   \checkmark         &                   \\ 
%         Effektivt       &                       &    \checkmark      &                         &                      &                   \\ 
%         Korrekt         &                       &                    &    \checkmark           &                      &                   \\
%         Pålideligt      &                       &                    &   \checkmark            &                      &                   \\ 
%         Vedligeholdbart &                       &   \checkmark       &                         &                      &                   \\
%         Testbart        &                       &                    &  \checkmark             &                      &                   \\ 
%         Fleksibelt      &                       &  \checkmark        &                         &                      &                   \\ 
%         Forståeligt     &                       &   \checkmark       &                         &                      &                   \\ 
%         Genbrugbart     &                       &                    &                         &       \checkmark     &                   \\ 
%         Flytbart        &                       &                    &   \checkmark            &                      &                   \\ 
%         Integrerbart    &                       &                    &   \checkmark            &                      &                   \\
%    \hline
%    \end{tabular}
%    \capt{Oversigt over vigtigheden af designkriterierne for projektet.}
%    \label{table:kriterietabel}
%\end{table}

\ourtable{kriterietabel}{5}{Oversigt over vigtigheden af designkriterierne for projektet.}
                                                            {Vigtighed}
       {Kriterium      }{ Meget vigtig   & Vigtig         & Mindre vigtig  & Irrelevant     & Trivielt       }{
\ourrow{Brugbart       }{ \checkmark     &                &                &                &                }
\ourrow{Sikkert        }{                &                &                & \checkmark     &                }
\ourrow{Effektivt      }{                & \checkmark     &                &                &                }
\ourrow{Korrekt        }{                &                & \checkmark     &                &                }
\ourrow{Pålideligt     }{                &                & \checkmark     &                &                }
\ourrow{Vedligeholdbart}{                & \checkmark     &                &                &                }
\ourrow{Testbart       }{                &                & \checkmark     &                &                }
\ourrow{Fleksibelt     }{                & \checkmark     &                &                &                }
\ourrow{Forståeligt    }{                & \checkmark     &                &                &                }
\ourrow{Genbrugbart    }{                &                &                & \checkmark     &                }
\ourrow{Flytbart       }{                &                & \checkmark     &                &                }
\ourrow{Integrerbart   }{                &                & \checkmark     &                &                }
}


Denne aktivitet giver indblik i nogle væsentlige dele af systemdesignet. Der er blevet dannet et helhedsindtryk af hvor fokusområderne i implementationen af systemet lægger og hvor programmerings- og designmæssige beslutninger skal lægge. Vi er nu sikre på, at der ikke burde være nogle betydningsfulde svagheder i det færdige design, nu når de mest essentielle designkriterier er blevet vurderet, kommenteret og kategoriseret. Som det kan iagttages, er \emph{brugbart} det eneste kriterie, der er vurderet som meget vigtig, hvilket passer meget godt i forhold til det initierende påstand på 1-2 ``meget vigtige'' kriterier. Brugbarhed er vurderet så på grund af, at brugbarhed er kernen i vores system og kræver derfor ekstra opmærksomhed under udviklingen af produktet. I følge systemudviklingslærebogen \cite{ooad} opfylder vores design også de tre kriterier, der danner et godt design, nemlig: \emph{brugbart, fleksibelt} og \emph{forståeligt} da de er vurderet som vigtige eller derover.

Gruppen er blevet enige om hvad, rent designmæssigt, er væsentlig for projektets succes. Det står nu klart, hvad vi skal bruge vores kræfter (tid) på. Det giver anledning til videre arbejde på design og implementering af systemet.
