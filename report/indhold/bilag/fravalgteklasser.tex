\chapter{Fravalgte klasser og hændelser}
\label{ap:fravalgteklasseroghaendelser}

\section{Fravalgte klasser}
\label{ap:fravalgteklasser}
Herunder ses de fravalgte klasser, som gruppen ikke fandt relevante i forhold til problemområdet. De listes her, da der har været meget diskussion, om hvorvidt klasserne skulle med i systemet eller ej.

\begin{description}
\item[Husholdning] \hfill \\
En husholdning repræsenterer et hjem, som indeholder én til flere personer. Det er ikke en del af problemområdet at holde styr på eller at kommunikere med andre husstande.

\item[Køkken] \hfill \\
Køkken og husholdning dækker over samme del af problemområdet, og vi fjerner derfor også køkken, af samme grund som vi fjernede husholdning.

\item[Køleskab/skab/opbevaringsskab] \hfill \\
Om råvarene befinder sig i et køleskab eller i en skuffe er, for os, uinteressant, derfor er vi ikke interesseret i at modellere disse skabe som en klasse.

\item[Køkkenredskab/komfur] \hfill \\
Ligesom ved køleskab/anden opbevaring er vi ikke interesseret i at modellere hvilke redskaber husholdningen har adgang til.

\item[Service] \hfill \\ 
Vi har valgt at fjerne klassen service (bestik), da service er noget, der bliver brugt når man er i færd med at spise maden, og ikke under selve madlavningen.

\item[Butik] \hfill \\
Vi har valgt at fjerne klassen Butik, da vores system fokuserer på madspild, og ikke madindkøb. En modellering af butikker ville være relevant, hvis vores fokus lå på at begrænse udgifter på mad, men da dette ikke er situationen i problemområdet, bliver den udeladt.

\item[Typisk/atypisk ingrediens] \hfill \\
Det er ikke en del af problemområdet at folk ikke er klar over hvilke ingredienser der er normale/unormale at have.

\item[Enhed/Mængde] \hfill \\
Enhed og mængde er ikke klasser, men attributter til en ingrediens.

\item[Madplan] \hfill \\
Informanterne syntes ikke at en madplan var særlig nødvendig. Den indgik i systemdefinition S2, som informanterne fravalgte. Madplanen anses derfor som overflødig, hvorfor denne fjernes som klasse.
\end{description}

\section{Fravalgte hændelser}
De hændelser som vi har fravalgt ses herunder. De fravalgte hændelser har en kort forklaring, der beskriver hvorfor hændelsen er blevet fravalgt.

\begin{itemize} [noitemsep]
\item Køkkenredskab benyttet (systemet skal ikke holde styr på køkkenredskaber)
\item Råvare benyttet (systemet skal ikke holde styr på mængden af råvarer hos brugeren)
\item Mæthed opnået (fra fravalgte klasser: bruger, person)
\item Madrest opstået (systemet skal ikke behandle råvarer forskelligt om det er rester eller ej)
\item Bord opdækket (fra fravalgt klasse: husholdning)
\item Opvask taget (fra fravalgt klasse: køkken)
\item Service benyttet (fra fravalgt klasse: service)
\item Køleskab åbnet (fra fravalgt klasse: opbevaringsskab)
\item Køleskab lukket (fra fravalgt klasse: opbevaringsskab)
\item Opskrift vurderet (opskrift valgt indebærer, at man har vurderet opskriften)
\item Opskrift anmeldt (ikke en del af problemområdet at anmelde opskrifter)
\item Sult opstået (fra fravalgte klasser: bruger, person)
\item Madlavning afsluttet (fra fravalgte klasser: bruger, person)
\item Madlavning påbegyndt (fra fravalgte klasser: bruger, person)
\item Råvare identificeret (overvåges i form af “råvare købt”-hændelsen med samme resultat)
\item Ingrediens identificeret (indgår i hændelsen opskrift valgt)
\item Madplan lagt (fra fravalgt klasse: Madplan)
\item Madplan startet (fra fravalgt klasse: Madplan)
\item Madplan afsluttet (fra fravalgt klasse: Madplan)
\item Opskrift fravalgt (fra fravalgt klasse: Madplan)
\end{itemize}