\section{Møde 2}

\begin{description}

\item[Formål]
For at kunne begynde at modellere avendelsesområdet, har vi behov for mere viden om informanternes tanker omkring systemet. En sådan viden vil vi gerne opnå igennem dette møde. Informanterne præsenteres for systemdefinitionerne S1 og S2, hvorefter vi gerne vil høre hvilket system de ønsker at vi skal udvikle. Derefter har vi et par ret lukkede spørgsmål omkring måden de vil bruge systemet på og hvilke funktioner der er nødvendige for at opfylde deres behov.

\item[Huskeliste] Listen herunder er en huskeliste for gruppen, når vi skal tale med informanterne.

\begin{itemize}[noitemsep]
\item Præsenter systemdefinition S1 og S2 for informanten
\item Hvor ville du bruge et sådan system? Bærbar, IPhone, Ipad, stationær?
\item Skal programmet kunne huske ingredienser til næste gang, og hvilke ingredienser? Vil du fjerne de ingredienser du bruger under madlavningen?
\item Hvilke opskrifter skal der findes? Forretter, hovedretter, desserter, m.m?
\item Hvordan skal opskrifter sorteres?
\item Retter med eller uden billeder?
\item Andre forslag til programmet?
\end{itemize}

\item[Noter fra møde med Merete Munthe, 13. september, Gistrup]
Merete mener hun helt sikkert ville bruge et program i stil med det nævnt i vores systemdefinition 1. Systemdefinition 2 siger hende ikke rigtig noget. Hun er ret sikker på hun ikke vil bruge en madplan. Hun kan godt lide at have frihed til at lave det hun lige har lyst til. Systemdefinition 1 mener hun vil være nyttigt til at få brugt alle resterne i køleskabet på en smart måde, så de ikke skal smides ud. Hun vil primært bruge programmet på sin bærbar.
Hun ville også bruge programmet selv om hun ingen rester havde, for at få gode idéer til retter hun kan lave.
Køleskabet skal ikke holde styr på ens varelager, hun vil kun bruge programmet inden madlavningen, og ikke efter. Det vil blive for bøvlet hvis man hver gang efter madlavning skal huske at fjerne ingredienser fra programmet. Man har nok at gøre med at rydde op efter maden.
Første gang man bruger programmet skal man kunne indtaste alt hvad man har, også krydderier og mælk. Med en "husk mig" knap, kan man gemme de ingredienser, man ikke vil skrive på hver gang.
Programmet skal kun finde almindelige hverdagsretter, ikke desserter, morgenmad og så videre.
Man skal kunne vælge tilberedningstid. Hvis hun har travlt vil hun ikke have foreslået retter der tager halvanden time at lave. Merete foreslår 3 knapper: 0 - 30 min, 30 - 60 min, > 60 min
En knap at sætte flueben i "Vis mig kun retter uden kød", ville være rar.
Programmet skal sortere opskrifter efter "dem man kan lave", så "dem man mangler 1 ting til", “2 ting til”, osv.
Inden for hver af de netop nævnte lister ville det være rart at kunne sortere efter kalorier, popularitet og tilberedningstid.
Vis kun retter med billeder.

\item[Noter fra møde med Keld Kjær, 18. september, Aalborg]
Jeg forklarede hvordan systemet vil virke efter systemdefinition S1 og S2 og stillede følgende spørgsmål:

\begin{itemize}[noitemsep]
\item Hvad systemdefinition kan du bedst lide? Systemdfinition S1.
\item Hvor ville du bruge sådan et system? Bærbar, mobil, tablet, stationær
\item Skal systemet kunne huske ingredienser til næste gang? Ja
\item Hvilke opskrifter skal der findes? forretter, hovedretter, dessert? Jeg laver normalt ikke forretter og dessert, når vi bare er derhjemme - det er kun, når vi får besøgende
\item Hvordan skal opskrifter sorteres? Årstiden, mængde af passende ingredienser
\item Andre forslag til programmer?
\begin{itemize}[noitemsep]
\item Kommentarer på opskrifter
\item Ingrediensmængdeberegning i forhold til antal personer
\item Printfunktion, så man kan få det på papir
\end{itemize}
\item Retter med eller uden billeder? med billeder!
\end{itemize}

Han lavede tit ensformig mad, når han har tid, kan noget nyt godt laves vha. inspiration fra kogebøger samt internetsider. Info om energi/kulhydrater/vitaminer + mineraler betyder ikke så meget for ham, så længe billedet og opskriften ser lækker og inspirerende ud. Krydderier er ikke vigtige, når man søger på opskrifter.
\end{description}

\subsection{Sammendrag}

\textbf{Enighed blandt begge informanter:}
\begin{itemize}[noitemsep]
\item Systemet bruges på en bærbar
\item Programmet skal fokusere på hovedretter
\item Der skal vises billeder af opskrifterne
\item Opskrifterne skal sorteres efter mængde af passende ingredienser
\item Opskrifter skal kunne skaleres i forhold til personer
\item Systemet skal kunne huske ingredienser til næste søgning
\end{itemize} 

\textbf{Foreslået af enkelt informant:}
\begin{itemize}[noitemsep]
\item Kommentarer på opskrifter
\item Udprinte opskrifter
\item Sorter opskrifter efter årstid, kalorier, tilberedningstid
\item Krydderier er ikke vigtige når man søger på opskrifter
\end{itemize}
