\section{Møde 1}
\begin{description}
\item[Formål] Igennem et semistruktureret interview med vores to informanter, ønsker vi opnå viden omkring, hvilke problemer informanterne har, i forbindelse med madlavning i det private. På baggrund af denne viden vil vi lave en eller flere systemdefinitioner, som beskriver et eller flere systemer, som vi forventer vil kunne løse disse problemer.

\item[Spørgmål] Informanterne blev stillet følgende spørgsmål.

\begin{itemize}[noitemsep]
\item Hvad gør du for at undgå madspild?
\item Hvordan planlægger du dine indkøb? Hvordan foregår de?
\item Hvordan finder du ud af, hvad du skal spise til aftensmad?
\item Gør du noget for at spise varieret? (Hvordan/hvorfor ikke?)
\item Beskriv hvordan I i jeres husstand håndterer madlavningen?
\begin{itemize}[noitemsep]
\item Hvem laver mad?
\item Hvem bestemmer hvad I skal have?
\item Hvem køber ind?
\item Hvor tit handler I ind?
\end{itemize}
\item Føler du, at du smider meget mad ud?
\item Laver du en madplan? (hvordan/hvorfor ikke?)
\begin{itemize}[noitemsep]
\item Hvad skal der til for, at du vil anvende en madplan?
\end{itemize}
\end{itemize}

\item[Møde med informant Merete Munthe, Gistrup]
Merete bor med sin mand og det er hende, der bestemmer, hvad for noget mad de skal have til aften. Det er altid hende der laver den, men nogle gange kommer manden dog hjem med takeaway.

Merete føler at hun smider meget mad ud. Maden bliver smidt ud når den bliver for gammel, da hun er meget opmærksom på holdbarhedsdatoer. En anden grund til madspild, er at hun før i tiden, har været vant til at lave mad til en hel familie, dengang hendes to sønner og datter boede hjemme, men nu er der kun hende og hendes mand tilbage i husstanden, hvilket har været svært at vænne sig til. Derfor bliver der lavet for store portioner. Hvis mad bliver tilovers, bruges det ofte i en sammenkogt ret næste dag (eksempelvis supper, biksemad osv.).

Merete bruger ikke madplaner af flere grunde. Med en madplan føler de ikke de får særlig meget mad for pengene, da der ikke er nogle madplaner, som tager højde for gode tilbud. Samtidig har Merete og hendes mand et job, som gør at deres planer ofte ændrer sig, og når de er på farten, er det let selv at lave mad, eller at tage højde for en madplan. Hvis Merete skulle bruge en madplan skulle den være baseret på hvad hun har i køleskabet, i en kombination med supermarkedernes gode tilbud. ``Mad leveret til døren''-tilbud fungerer ikke for hende, da hende og manden som før nævnt ofte er på farten, og deres planer ofte ændrer sig impulsivt. Både Merete og hendes mand kan stå for at handle ind. De planlægger sjældent indkøbet, men kan bedre lide at gå på opdagelse efter gode tilbud i forretningen. Merete gør ingenting for at spise varieret, da hun synes det tager for lang tid at tage højde for at få kosten til at blive varieret.

\item[Møde med informant Keld Kjær, Aalborg]

Keld bor med sin kone og to døtre ved Østre Anlæg i Aalborg. Han bestemmer, hvad familien skal have at spise hver aften, og han køber ind og laver al maden til familien. Når der bliver lavet mad, så bliver portionen som regel lavet så står, at der er nok til to dage. Familien ønsker nemlig ikke at lave mad hver aften, da der ikke er meget tid i hverdagen. De får oftest spist hele portionen i løbet af to dage, men hvis der bliver noget ekstra tilovers, fryser de den ned, så der bliver smidt så lidt mad ud som muligt. Keld planlægger oftest aftensmad to til tre dage i fremtiden. Når der skal handles ind, så bruges der en indkøbsliste. Dvs., at de har en liste klar, når der skal handles ind. Der kommer oftest også andre sager med i indkøbskurven, fordi de er nemme ofre for impulskøb. Keld handler ca. tre til fire gange om ugen.

Der bliver sjældent spist varieret mad. Det er oftest de samme ``almindelige'' og hurtige danske retter, fordi de har erfaring med disse og de mener, at det er nemt at lave dem. Dvs., at tid er en vigtig faktor, når det kommer til familiens aftensmad. Dog kan det hænde, at der eksperimenteres med nye retter, men dette sker kun i weekenden, når der er lidt ekstra tid.

Madplan er ikke noget, som de bruger, fordi Keld normalvis har en plan i hovedet, som han går efter. Resten af familien har ikke den store indflydelse på madlavningen. Han kommenterer dog, hvis han skulle bruge en madplan, så skulle denne have opskrifter, der ikke tog meget tid, og hvor ingredienserne var håndgribelige. Med dette menes der, at ingredienserne ikke skulle være alt for forskellige, da man pga. af dette ville have sværere ved at mindske sine madrester fra aftensmaden før. Derudover skulle der være nogle billeder til hver opskrift, så man kunne få et hurtigt indblik i hvordan maden skal se ud, og på den måde kan man se, om en opskrift er noget for en. 

\end{description}

\subsection{Sammendrag}
Vi har nu hørt om 2 informanters erfaringer inden for privat madlavning. De 2 informanter har det til fælles, at de begge oplever madspild og ikke benytter en madplan. Netop fordi ingen af informanterne benytter en madplan, kan det være at det blot er en sådan der skal til for at mindske deres madspild. Det kan også være at ingen af dem benytter en madplan fordi de simpelthen ikke kan overtales til dette. For nærmere at undersøge hvordan vi skal løse problemet med madspild, konstruerer vi 2 forskellige systemdefinitioner. Begge systemer forsøger at mindske madspild. Systemdefinition S1 benytter en løsning der ikke involverer en madplan, mens systemdifinition S2 netop benytter en madplan.