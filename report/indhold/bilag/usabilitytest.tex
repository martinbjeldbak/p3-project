\chapter{Usabilitytest}
\label{ap:usabilitytest}

Usabilitytest med Merete er blevet optaget\cite{usabilitymerete}.

Usabilitytest med Keld er blevet optaget\cite{usabilitykeld}.

I forbindelse med en usabilitytest foretaget på de 2 informenter, benyttede vi følgende case:

\section{Case}
\begin{enumerate}
\item Gå på Foodl.dk
\item Find en opskrift som indeholder ``gulerødder'' og ``piskefløde''.
\item Gå ind på opskriften: ``Lakridsfisk på fennikelbund''
\item Gå tilbage til siden du kom fra, hvor du ser søgeresultatet
\item Favoriser opskriften ``Lakridsfisk på fennikelbund''
\item Skalér opskriften, så ingredienserne passer til 8 personer
\item Fra opskriften ``Lakridsfisk på fennikelbund'', tilføj lakridspulver til indkøbslisten
\item Tilføj alle ingredienser fra ``Lakridsfisk på fennikelbund'' til indkøbslisten
\item Gå ind på din indkøbsliste
\item Lakridspulver er der 2 gange - fjern den første
\item Du mangler også noget mælk - tilføj ``mælk'' til indkøblisten
\item Print indkøbslisten ud
\item Indkøbslisten er nu udskrevet. Slet alle varer på din indkøbsliste
\item Gå ud på forsiden
\item Opret en bruger på Foodl med brugernavn: test@test.dk, kodeord: ``test123''.
\item Skift kodeordet til at være “1234567”.
\item Lav en ny søgning på ``Baconskiver'' og ``Hakket oksekød''.
\item Sortér opskrifter efter alfabetisk orden.
\item Gør sådan at du kun ser de opskrifter, der er hurtigst at lave.
\item Send en mail til Foodl-udviklerne
\item Find ud af hvor mange, der har været med til at udvikle Foodl.
\item Du er nu færdig på foodl.dk - log ud
\item Du har ikke været på Foodl i lang tid. Prøv nu at finde den opskrift du favoriserede.
\item Du kan ikke længere lide opskriften, fjern den fra dine favoritter.
\end{enumerate}
