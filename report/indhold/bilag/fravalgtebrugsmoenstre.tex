\chapter{Fravalgte brugsmønstre}
\label{ap:fravalgtebrugsmoenstre}
\begin{description}
\item[Skalering] \hfill \\
Skalering af ingredienser i opskrifter i forhold til antallet af personer.
Fjernet, da skalering indgår i søgningen, netop fordi der ikke er nogen grund til at skalere opskriften i hver visning og kan dermed indgå som en slags “filter” i søgningen.

\item[Råvarehåndtering] \hfill \\
Holde styr på hvilke råvarer man har i sit køleskab.
Fjernet, da råvarehåndtering indgår i søgningen.

\item[Overvågning] \hfill \\
Vi mener ikke at vi har behov for at overvåge noget og har derfor fjernet dette brugsmønster.

\item[Begrænsning] \hfill \\
At tilføje en begrænsning, så man kun får vist opskrifter uden gluten, uden svinekød, m.m.
Fjernet, da dette hører under søgning.

\item[Sortere] \hfill \\
At sortere opskrifter efter anmeldelse (antal stjerner ), navn eller lignende.
Fjernet, da dette hører under søgning.

\item[Madplanlægning] \hfill \\
Fjernet, da klassen madplan ikke er en del af problemområdet.

\item[Synkronisering] \hfill \\
I stedet for et loginsystem, kunne vi benytte cookies og tilbyde muligheden for at synkronisere flere enheder, så de er tilknyttet hinandne. Således at en ændring på indkøbslisten på én enhed betyder at samtlige enheder kan se ændringen. På baggrund af møde 4 med informanterne er det gjort klart, at vi skal benytte et login-system i stedet, og fjerner derfor dette brugsmoenster.
\end{description}