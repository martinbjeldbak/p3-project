\section{Evolutionær metode}
\label{sec:evolution}

Det er besværligt at planlægge en arbejdsproces for et virkelighedsnært problem. Det kræver meget forarbejde, og man bliver nødt til at fortolke og revidere problemstillingen mange gange, indtil man har tilstrækkelig nok viden og forståelse for problemet. Dette gøres på bedste vis ved at have tæt kommunikation og interaktion med brugerne af et system, der kan løse problemet. Den evolutionære arbejdsproces går ud på, at man skal eksperimentere med bl.a. prototyper til at skabe sig en forståelse for problemet. \cite{cic} Den evolutionære metode er en iterativ arbejdstilgang, hvor der er forskellige faser, hvor man reviderer og bearbejder analysen, designet af systemet, implementering og kvalitetssikring i hver fase.

Vi havde selvfølgelig nogle lineære tilgange i planlægningen, men ser vi på det store billede, så var det den iterative metode, der dominerede. Da projektet strakte sig over 4 måneder, valgte vi at dele projektforløbet op i 4 faser á 3 ugers varighed. Vi havde forskellige mål med hver fase, hvor vi definerede hovedhensigten og de undermål, der skulle løses i løbet af fasen. For at opnå de mål, ifølge den evolutionære arbejdsmetode, skulle vi arbejde på analysen, designet, implementeringen og kvalitetssikring af systemet i hver enkelt fase. Som skrevet, forsøgte vi at arbejde så evolutionært som muligt, dog lykkedes det ikke helt, da vi \fx i fase 1 ikke arbejdede på implementering eller kvalitetssikring af systemet.

%planlægning er usikkert
%trial and error
%problemet bliver fortolket og revideret (iterationer)
%tæt kommunikation og interaktion med brugere

\ourtable{iterationeroverblik}{3}{Denne tabel giver et hurtigt og kortfattet overblik over projektets arbejdsfaser. Her ses de forskellige faser af den iterative arbejdsmetode med tilhørende formål og de resultater, vi har fået ud af de forskellige faser. Desuden kan man se, hvordan informanterne er blevet inddraget i processen.}
             {Beskrivelser}
       {Fase}  {Formål                       & Resultat                           & Informanternes inddragelse}{
\ourrow{1   }{At få indblik i informanternes problemstillinger, mht. madlavning og anvendelse af deres madrester, og at modellere disse. At definere et system, og forstå hvilke funktioner informanterne har brug for. & \textit{Tilføjet:} \textbf{Problmeområdet.} Klasser. Hændelser. Hændelsestabel. Klassestruktur. Prototype med fokus på søgefunktionalitet.  & Møde med fokus på problemet. Møde med fokus på løsning. Prototype med fokus på søgefunktionalitet.}
\ourrow{2   }{At sikre os, at de funktioner vi har i systemet passer med informanternes behov og at dokumentere og modellere disse. & \textit{Tilføjet:} \textbf{Problemområdet. Anvendelsesområdet.} Brugsmønstre. Funktioner. Aktører. Kriterier. Forbilleder. \textit{Revideret:} \textbf{Problemområdet.} Prototype med fokus på funktionalitet.  & Prototype med fokus på funktionalitet.}
\ourrow{3   }{At modellere systemet, ved at implementere funktioner og sikre os, at de definerede kriterier bliver opfyldt.     & \textit{Tilføjet:} \textbf{Implementering.} Sammensætning af rapport. Komponenter. Udtrækning af data i opskrifter. \textit{Revideret:} \textbf{Problemområdet. Anvendelsesområdet. Design.}  &  }
\ourrow{4   }{  At implementere det designede system.                                    & \textit{Tilføjet:} \textbf{Implementering. Kvalitetssikring.} \textit{Revideret:} \textbf{Problemområdet. Anvendelsesområdet. Design.}                                &           Bestemmelse af råvaretype fra ingrediens. 
Usability-test. }
}

