\section{Evolutionær metode}
\label{sec:evolution}
\tjek{Revideret af Elias d. 9.12.12}

Det er vanskeligt at planlægge et helt projekt for et stort problem. Det kræver meget forarbejde, og man bliver nødt til at fortolke og revidere problemstillingen mange gange, indtil man har tilstrækkelig viden og forståelse for problemet. Dette gøres på bedste vis ved at have tæt kommunikation og interaktion med brugerne af et system, der kan løse problemet. Vi har udviklet et system til brugerne, og det er dem, der skal bruge systemet. Derfor er det vigtigt, at de kan finde ud af at bruge systemet, og på samme tid udvikler vi en bedre forståelse for problemet igennem interaktionen med brugerne. 

Denne arbejdsproces er iterativ og igennem denne iteration sker der en udvikling i forståelsen. Dette kaldes en evolution, hvilket denne arbejdsmetode også hedder. Den evolutionære arbejdsmetode bygger på eksperimentering med bl.a. prototyper til at skabe sig en forståelse for problemet. Arbejdsmetoden består af forskellige faser, hvor man reviderer og bearbejder analysen, designet af systemet, implementering og kvalitetssikringen i hver fase. \cite{cic} Når vi påstår, at vi har benyttet en iterativ arbejdsproces, så betyder det ikke, at der ikke har været linæere tilgange i planlægningen, men ser man på det store billede, så var det den iterative tilgang, der dominerede.

Projektet strakte sig over fire måneder, og vi vurderede, at fire faser af tre ugers varighed var en fornuftig opdeling af tiden, og at det gav os tilstrækkelig tid til de iterative faser. Disse fire faser kan ses i \tableref{table:iterationeroverblik}. Hver fase havde et formål, og det vi fik ud af den enkelte fase kan ses i \tableref{table:iterationeroverblik} under ``Resultat''. Herunder står, hvilke dele af rapporten, der er blevet tilføjet eller revideret. I den første fase er der ikke noget at revidere, da vi ikke havde udført noget arbejde endnu. I tabellen under ``Informaternes inddragelse'' beskriver vi, hvordan vi har interageret med brugere af systemet.

Informanterne, der bliver beskrevet i \secref{sec:samarbejde}, hjalp os med at forstå både problemet og en eventuel løsning på problemet. Vi holdte et møde for at forstå problemet, og det møde ledte os til en eventuel løsning på problemet. Vi præsenterede informanterne for prototyper, der havde til formål at visualisere, hvilke tanker vi havde gjort os, og vi ønskede at undersøge, hvordan de interagerede med disse prototyper. Vi havde to prototyper, hvis fokus lå på søgefunktionaliteten i systemet og en prototype, hvis fokus lå på andre funktioner, systemet kunne have. Disse prototyper er beskrevet yderligere i \secref{sec:samarbejde}. Tabel \ref{table:iterationeroverblik} har til formål at give læseren et hurtigt overblik over, hvad der er foregået i de forskellige faser.


\ourtable{iterationeroverblik}{3}{Denne tabel giver et hurtigt og kortfattet overblik over projektets arbejdsfaser. Her ses de forskellige faser af den iterative arbejdsmetode med tilhørende formål og de resultater, vi har fået ud af de forskellige faser. Desuden kan man se, hvordan informanterne er blevet inddraget i processen.}
             {Beskrivelser}
       {Fase}  {Formål                       & Resultat                           & Informanternes inddragelse}{
\ourrow{1   }{At få indblik i informanternes problemstillinger, mht. madlavning og anvendelse af deres madrester, og at modellere disse. At definere et system, og forstå hvilke funktioner informanterne har brug for. & \textit{Tilføjet:} \textbf{Problmeområdet.} Klasser. Hændelser. Hændelsestabel. Klassestruktur. Prototype med fokus på søgefunktionalitet.  & Møde med fokus på problemet. Møde med fokus på løsning. Prototype med fokus på søgefunktionalitet.}
\ourrow{2   }{At sikre os, at de funktioner vi har i systemet passer med informanternes behov og at dokumentere og modellere disse. & \textit{Tilføjet:} \textbf{Problemområdet. Anvendelsesområdet.} Brugsmønstre. Funktioner. Aktører. Kriterier. Forbilleder. \textit{Revideret:} \textbf{Problemområdet.} Prototype med fokus på funktionalitet.  & Prototype med fokus på funktionalitet.}
\ourrow{3   }{At modellere systemet, ved at implementere funktioner og sikre os, at de definerede kriterier bliver opfyldt.     & \textit{Tilføjet:} \textbf{Implementering.} Sammensætning af rapport. Komponenter. Udtrækning af data i opskrifter. \textit{Revideret:} \textbf{Problemområdet. Anvendelsesområdet. Design.}  &  }
\ourrow{4   }{  At implementere det designede system.                                    & \textit{Tilføjet:} \textbf{Implementering. Kvalitetssikring.} \textit{Revideret:} \textbf{Problemområdet. Anvendelsesområdet. Design.}                                &           Bestemmelse af råvaretype fra ingrediens. 
Usability-test. }
}

