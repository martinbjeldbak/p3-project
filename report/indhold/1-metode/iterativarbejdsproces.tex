\section{Evolutionær metode}
\label{sec:evolution}

Det er meget svært at planlægge en arbejdsproces for et virkelighedsnært problem. Det kræver meget forarbejde og man bliver nødt til at fortolke og revidere problemstillingen mange gange, indtil man har tilstrækkelig nok viden og forståelse for problemet. Dette gøres på bedste vis ved at have tæt kommunikation og interaktion med brugerne af et system, der kan løse problemet. Den evolutionære arbejdsproces går ud på, at man skal eksperimentere med bl.a. prototyper til at skabe sig en forståelse for problemet. Den evolutionære metode er en iterativ proces, hvor man har forskellige faser, hvor man reviderer og bearbejder analysen, designet af systemet, programmering, aftestning og afprøvning.

Vi har selvfølgelig nogle lineære tilgange i planlægningen, men ser vi på det store billede, så er det den iterative metode, der dominerer.

% indhold i hver iteration

\todo{smid lige en kilde ind.}

%planlægning er usikkert
%trial and error
%problemet bliver fortolket og revideret (iterationer)
%tæt kommunikation og interaktion med brugere

%Der findes i forvejen systemer, der har til formål at løse netop lignende problemer, som det problem, vi er kommet frem til. Men hvordan fungerer disse systemer? Hvad er deres stærke sider? Har de nogle negative sider? Dette og lignende problemstillinger vil vi undersøge i de kommende afsnit. Vi håber på at kunne lære noget af lignende systemer. På denne måde kan vi bruge de stærke sider og evt. implementere lignende egenskaber i vores eget system og lære af deres negative sider.

\ourtable{iterationeroverblik}{3}{Denne tabel giver et hurtigt og kortfattet overblik over projektets arbejdsfaser. Her ses de forskellige faser af den iterative arbejdsmetode med tilhørende formål og de resultater, vi har fået ud af de forskellige faser. Desuden kan man se, hvordan informanterne er blevet inddraget i processen.}
             {Beskrivelser}
       {Fase}  {Formål                       & Resultat                           & Informanternes inddragelse}{
\ourrow{1   }{At få indblik i informanternes problemstillinger, mht. madlavning og anvendelse af deres madrester, og at modellere disse. At definere et system, og forstå hvilke funktioner informanterne har brug for. & \textit{Tilføjet:} \textbf{Problmeområdet.} Klasser. Hændelser. Hændelsestabel. Klassestruktur. Prototype med fokus på søgefunktionalitet.  & Møde med fokus på problemet. Møde med fokus på løsning. Prototype med fokus på søgefunktionalitet.}
\ourrow{2   }{At sikre os, at de funktioner vi har i systemet passer med informanternes behov og at dokumentere og modellere disse. & \textit{Tilføjet:} \textbf{Problemområdet. Anvendelsesområdet.} Brugsmønstre. Funktioner. Aktører. Kriterier. Forbilleder. \textit{Revideret:} \textbf{Problemområdet.} Prototype med fokus på funktionalitet.  & Prototype med fokus på funktionalitet.}
\ourrow{3   }{At modellere systemet, ved at implementere funktioner og sikre os, at de definerede kriterier bliver opfyldt.     & \textit{Tilføjet:} \textbf{Implementering.} Sammensætning af rapport. Komponenter. Udtrækning af data i opskrifter. \textit{Revideret:} \textbf{Problemområdet. Anvendelsesområdet. Design.}  &  }
\ourrow{4   }{  At implementere det designede system.                                    & \textit{Tilføjet:} \textbf{Implementering. Kvalitetssikring.} \textit{Revideret:} \textbf{Problemområdet. Anvendelsesområdet. Design.}                                &           Bestemmelse af råvaretype fra ingrediens. 
Usability-test. }
}

