\section{Evolutionær metode}
\label{sec:evolution}
\tjek{Revideret af Elias d. 9.12.12}

Det er vanskeligt at planlægge et helt projekt for et stort problem. Det kræver meget forarbejde, og man bliver nødt til at fortolke og revidere problemstillingen mange gange, indtil man har tilstrækkelig viden og forståelse for problemet. Dette gøres på bedste vis ved at have tæt kommunikation og interaktion med potentielle brugere af det fremtidige system. Vi har udviklet et system til brugerne, og det er dem, der skal bruge systemet. Derfor er det vigtigt, at de kan finde ud af at bruge systemet, og på samme tid udvikler vi en bedre forståelse for problemet igennem interaktionen med brugerne. 

Den evolutionære metode, bygger på en iterativ arbejdsproces, hvor der gennem hver iteration sker en udvikling i forståelsen af et problem. Metoden bygger på eksperimentering med bl.a. prototyper til at skabe sig en forståelse for problemet. Arbejdsmetodens iterationer, kaldes også for faser. I hver fase reviderer og bearbejder man analysen, designet af systemet, implementering og kvalitetssikringen. \cite{cic} Når vi siger, at vi har benyttet en iterativ arbejdsproces, så betyder det ikke, at der ikke også har været linæere tilgange i planlægningen. Men det betyder, at hvis man ser på den samlede udviklingsproces, så er det den iterative tilgang, der dominerer.

Projektet strakte sig over fire måneder, og vi vurderede, at fire faser af tre ugers varighed var en fornuftig opdeling af tiden, og at det gav os tilstrækkelig tid til hver fase. Disse fire faser kan ses i \tableref{table:iterationeroverblik}. Hver fase havde et formål, og et resultat. Tabellen har til formål at give læseren et hurtigt overblik over, hvad der er foregået i de forskellige faser. I tabellen står der eksempel hvilke dele af rapporten, der er blevet tilføjet eller revideret i den enkelte fase. I den første fase er der ikke noget at revidere, da vi ikke havde udført noget arbejde endnu. Derudover beskriver tabellen, hvordan vi har interageret med informanterne.

Informanterne, der bliver beskrevet i \secref{sec:samarbejde}, hjalp os bl.a. med at forstå problemet og hvordan en eventuel løsning på problemet skulle udformes. Vi holdt et initierende møde med dem, for at få en bred forståelse af problemet. Vi præsenterede informanterne for prototyper, der havde til formål at visualisere, hvilke tanker vi havde gjort os i forhold til en eventuel løsning, og vi ønskede at undersøge, hvordan informanterne interagerede med disse. Vi havde to prototyper, hvis fokus lå på søgefunktionaliteten i systemet og en prototype, hvis fokus lå på andre funktioner, systemet kunne have. Prototyperne er beskrevet yderligere i \secref{sec:samarbejde}. 


\ourtable{iterationeroverblik}{3}{Denne tabel giver et hurtigt og kortfattet overblik over projektets arbejdsfaser. Her ses de forskellige faser af den iterative arbejdsmetode med tilhørende formål og de resultater, vi har fået ud af de forskellige faser. Desuden kan man se, hvordan informanterne er blevet inddraget i processen.}
             {Beskrivelser}
       {Fase}  {Formål                       & Resultat                           & Informanternes inddragelse}{
\ourrow{1   }{At få indblik i informanternes problemstillinger, mht. madlavning og anvendelse af deres madrester, og at modellere disse. At definere et system, og forstå hvilke funktioner informanterne har brug for. & \textit{Tilføjet:} \textbf{Problmeområdet.} Klasser. Hændelser. Hændelsestabel. Klassestruktur. Prototype med fokus på søgefunktionalitet.  & Møde med fokus på problemet. Møde med fokus på løsning. Prototype med fokus på søgefunktionalitet.}
\ourrow{2   }{At sikre os, at de funktioner vi har i systemet passer med informanternes behov og at dokumentere og modellere disse. & \textit{Tilføjet:} \textbf{Problemområdet. Anvendelsesområdet.} Brugsmønstre. Funktioner. Aktører. Kriterier. Forbilleder. \textit{Revideret:} \textbf{Problemområdet.} Prototype med fokus på funktionalitet.  & Prototype med fokus på funktionalitet.}
\ourrow{3   }{At modellere systemet, ved at implementere funktioner og sikre os, at de definerede kriterier bliver opfyldt.     & \textit{Tilføjet:} \textbf{Implementering.} Sammensætning af rapport. Komponenter. Udtrækning af data i opskrifter. \textit{Revideret:} \textbf{Problemområdet. Anvendelsesområdet. Design.}  &  }
\ourrow{4   }{  At implementere det designede system.                                    & \textit{Tilføjet:} \textbf{Implementering. Kvalitetssikring.} \textit{Revideret:} \textbf{Problemområdet. Anvendelsesområdet. Design.}                                &           Bestemmelse af råvaretype fra ingrediens. 
Usability-test. }
}

