\section{Initierende problem}
\label{sec:initierendeproblem}

I Danmark findes der omkring 2,6 millioner husstande \cite{husstande}, der dagligt skal få madlavningen til at gå op i en højere enhed. Der er nemlig mange ting at tage højde for under madlavningen. Der skal tænkes på sundhed for den enkelte, i form af selve kosten, men også sundhed for alle på længere sigt, hvilket opnås ved, at vi i samlet flok skåner miljøet ved at anvende madrester.

En ensformig kost er langt fra lige så sund som en varieret kost. En årsag til ensformighed i madlavningen kan være en travl hverdag, hvor man knytter sig til faste vaner som \fx at lave den samme ret ofte, fordi man synes, den smager rigtig godt og samtidig er lynhurtig at lave. Kosten kan også blive ensformig, fordi man benytter resterne fra madlavningen i nøjagtig samme opskrift dagen efter.

Under madlavningen kan miljøet skånes ved at undgå at smide for meget mad væk. En parcelhusejer smider i gennemsnit 42 kilo spiseligt mad ud om året. \cite{madspildpol} Hvis man ikke vil benytte madresterne fra foregående aften i en lignende ret dagen efter, så kan det være svært at finde en ny ret, hvor man kan benytte de specifikke madrester. I kogebøger bliver det hurtigt uoverskueligt, at skulle søge igennem opskrifter for at finde en opskrift, hvor man benytter sig af de madrester, man har på det pågældende tidspunkt. 

Når man endelig står i supermarkedet, så sælges alt i kæmpe portioner. Hakket oksekød findes typisk i pakker med 500 gram som det mindste. Til en enkelt person er dette ofte for meget, og derved risikerer man enten at stå med 100-200 gram hakket oksekød til skraldespanden, eller et problem med hvor dette kød nu skal benyttes.

Tal fra Politiken viser, at en parcelhusejer i gennemsnit smider 42 kilo spiseligt mad ud om året. \cite{madspildpol}
Sådanne typer madspild koster desuden danske husholdninger 16 milliarder kroner om året, eller ca. 20 \% af madforbruget af en gennemsnitlig dansk børnefamilie. \cite{madspild16}
