\chapter{Metode}
\label{chap:metode}
\tjek{Revideret af Elias d. 9.12.12}

Gennem hele projektforløbet har der været brug for planlægning og strukturering af arbejdsopgaver for at få projektet til at gå op i en højere enhed. Dette kapitel beskriver, hvilke arbejdsmetoder vi har brugt, og hvordan vi har samarbejdet med eventuelle brugere af det system, der er produktet af dette projektforløb.

Ydermere har vi brugt udvalgte metoder og aktiviteter fra bogen, Objektorienteret Analyse \& Design (OOAD) \cite{ooad}, hvis formål er at fastlægge systemspecifikationer. Det betyder, at OOAD-bogen er blevet brugt til at forstå et system, dets kontekst og de betingelser, der forudsætter dets implementation. Bogens metoder har gjort det muligt at udvikle et system med mindre uvished om, hvorvidt systemet løser et givent problem, end hvis metoderne og aktiviteterne ikke var blevet brugt.
\tjek{Giver dette mening?}

\section{Evolutionær metode}
\label{sec:evolution}

Det er besværligt at planlægge en arbejdsproces for et virkelighedsnært problem. Det kræver meget forarbejde, og man bliver nødt til at fortolke og revidere problemstillingen mange gange, indtil man har tilstrækkelig nok viden og forståelse for problemet. Dette gøres på bedste vis ved at have tæt kommunikation og interaktion med brugerne af et system, der kan løse problemet. Den evolutionære arbejdsproces går ud på, at man skal eksperimentere med bl.a. prototyper til at skabe sig en forståelse for problemet. \cite{cic} Den evolutionære metode er en iterativ arbejdstilgang, hvor der er forskellige faser, hvor man reviderer og bearbejder analysen, designet af systemet, implementering og kvalitetssikring i hver fase.

Vi havde selvfølgelig nogle lineære tilgange i planlægningen, men ser vi på det store billede, så var det den iterative metode, der dominerede. Da projektet strakte sig over 4 måneder, valgte vi at dele projektforløbet op i 4 faser á 3 ugers varighed. Vi havde forskellige mål med hver fase, hvor vi definerede hovedhensigten og de undermål, der skulle løses i løbet af fasen. For at opnå de mål, ifølge den evolutionære arbejdsmetode, skulle vi arbejde på analysen, designet, implementeringen og kvalitetssikring af systemet i hver enkelt fase. Som skrevet, forsøgte vi at arbejde så evolutionært som muligt, dog lykkedes det ikke helt, da vi \fx i fase 1 ikke arbejdede på implementering eller kvalitetssikring af systemet.

%planlægning er usikkert
%trial and error
%problemet bliver fortolket og revideret (iterationer)
%tæt kommunikation og interaktion med brugere

\ourtable{iterationeroverblik}{3}{Denne tabel giver et hurtigt og kortfattet overblik over projektets arbejdsfaser. Her ses de forskellige faser af den iterative arbejdsmetode med tilhørende formål og de resultater, vi har fået ud af de forskellige faser. Desuden kan man se, hvordan informanterne er blevet inddraget i processen.}
             {Beskrivelser}
       {Fase}  {Formål                       & Resultat                           & Informanternes inddragelse}{
\ourrow{1   }{At få indblik i informanternes problemstillinger, mht. madlavning og anvendelse af deres madrester, og at modellere disse. At definere et system, og forstå hvilke funktioner informanterne har brug for. & \textit{Tilføjet:} \textbf{Problmeområdet.} Klasser. Hændelser. Hændelsestabel. Klassestruktur. Prototype med fokus på søgefunktionalitet.  & Møde med fokus på problemet. Møde med fokus på løsning. Prototype med fokus på søgefunktionalitet.}
\ourrow{2   }{At sikre os, at de funktioner vi har i systemet passer med informanternes behov og at dokumentere og modellere disse. & \textit{Tilføjet:} \textbf{Problemområdet. Anvendelsesområdet.} Brugsmønstre. Funktioner. Aktører. Kriterier. Forbilleder. \textit{Revideret:} \textbf{Problemområdet.} Prototype med fokus på funktionalitet.  & Prototype med fokus på funktionalitet.}
\ourrow{3   }{At modellere systemet, ved at implementere funktioner og sikre os, at de definerede kriterier bliver opfyldt.     & \textit{Tilføjet:} \textbf{Implementering.} Sammensætning af rapport. Komponenter. Udtrækning af data i opskrifter. \textit{Revideret:} \textbf{Problemområdet. Anvendelsesområdet. Design.}  &  }
\ourrow{4   }{  At implementere det designede system.                                    & \textit{Tilføjet:} \textbf{Implementering. Kvalitetssikring.} \textit{Revideret:} \textbf{Problemområdet. Anvendelsesområdet. Design.}                                &           %Bestemmelse af bedste metode til at mappe ingrediens $\rightarrow$ råvare. 
Usability-test. }
}



% hvem er de
% hvad laver de
% hvad laver vi med dem

% hvad har de været ind over?
% inddrage, deltagelse

\section{Samarbejde med brugere}
\label{sec:samarbejde}

Vi mener, at systemet skal være rettet mod brugerenes behov og ønsker. Derfor valgte vi to informanter, som hører til to relative forskellige forbrugsområder, hvilket gav et større indblik i andre forbrugere, som vi ikke kan nå. Vi har konsulteret vores 2 informanter med hensyn til inspiration og opklaring af gruppediskussioner. Hvis vi har været i tvivl om eksempelvis en klasse eller struktur skulle indgå i systemet, tog vi kontakt til vores informanter for at høre deres meninger. Informanterne repræsenterer nemlig de brugere, som vi ønsker vores systemet skal appellere til, og derfor er det i sidste ende deres mening der tæller.

Informanter vil også være en stor del af afprøvnings- og kvalitetssikringsdelene af projektet. De skal være med til at sikre, at vi udvikler et system, der stemmer overens med systemdefinitionen og dækker deres behov med hensyn til madlavning i husstanden. Dette gør vi blandt andet ved at lave prototyper af systemet, og præsenterer og teste disse på informanterne. Prototyperne har indeholdt de givne funktioner eller designs, som vi har ønsket at få feedback på.

\subsection{Prototyper}
\label{subsec:prototyper}

Vi vil benytte os af papirsprototyper til de initierende afprøvninger. Denne form for prototype er ikke tidskrævende og er en billig form for præsentation. I og med at det er initierende afprøvninger, så mener vi, at det er vigtigt, at vi ikke bruger for mange kræfter på at udvikle en prototype. Der er en risiko for, at vi kan ende med at have brugt så meget tid på prototypen, at vi vil have svært ved at give slip på idéen, hvis informanterne ønsker noget helt andet end prototypen skal illustrere. I de senere afprøvninger kan vi bruge mere tid på prototyperne, da der på de senere tidspunkter er blevet fastlagt nogle stramme rammer, der har til formål at afgrænse systemets udviklingsproces. Dvs., at vi ikke burde ende med et system, der slet ikke har noget med de rammer at gøre. I og med at de rammer bliver fastlagt, så er risikoen for, at informanterne ønsker et system, der er helt anderledes end det, vi har udviklet, meget smal.

Vi kan i et lidt senere forløb i projektet præsentere informanterne for diasshow-prototyper, der har til formål at undersøge systemets funktioner. Med sådan en prototype bliver informanterne præsenteret for en prototype, der er dynamisk. Man kan klikke på knapper og navigere rundt i diasshowet. En diasshow-prototype giver os og informanterne mulighed for at lege lidt med vores idé for systemet. 

Systemets brugbarhed er en vigtig faktor for os. Vi ønsker at gøre det lettere for den madansvarlige i husstanden at bruge sine madrester i madlavningen. Derfor skal systemet være intuitivt og nemt at gå til. Med diasshow-prototyper får vi mulighed for at sikre os, at vores designidéer bliver forstået af informanterne. Hvis informanterne \fx har svært ved at finde nogle funktioner, så kan det være, at de skal gøres mere synlige. Lignende spørgsmål bliver besvaret relativt tidligt i systemets udviklingsfase, hvilket er en fornuftig ting. Det giver os mere tid til at rette fejl og komme på nye designidéer, hvis det bliver nødvendigt.

Som nogle af de sidste afprøvninger, vil vi præsentere et funktionsdygtigt system for informanterne. Dette system vil blive baseret på alle de foregående prototyper og designidéer. Ved at præsentere informanterne for det funktionsdygtige system, bliver vi i stand til at sikre os en vis kvalitet, inden vi afslutter projektet og produktudviklingen. Til de sidste afprøvninger er der mulighed for at opdage kritiske fejl, fordi vi lader en person, der ikke har været med til at udvikle systemet, lege med det. Her kan brugeren foretage sig nogle valg i systemet, som vi måske ikke har tænkt over. Disse uforudsete valg skal ikke være skyld i at systemet går ned, og denne sidste afprøvning er en god sikring for dette. 
