\ourtable{iterationeroverblik}{3}{Denne tabel giver et hurtigt og kortfattet overblik over projektets arbejdsprocesser. Her ses de forskellige faser af arbejdsprocessen med tilhørende formål og de resultater, vi har fået ud af de forskellige faser. Desuden kan man se, hvordan informanterne er blevet inddraget i processen.}
                                                     {Beskrivelser}
       {Fase}{Formål                                & Resultat                                            & Informanternes inddragelse}{
\ourrow{1   }{At få indblik i informanternes problemstillinger, mht. madlavning og anvendelse af deres madrester, og at modellere disse. At definere et system, og forstå hvilke funktioner informanterne har brug for. & \textit{Tilføjet:} \textbf{PO.} Klasser. Hændelser. Hændelsestabel. Klassestruktur. Prototype 1.  & Møde 1. Møde 2. Prototype 1.           }
\ourrow{2   }{At sikre os, at de funktioner vi har i systemet passer med informanternes behov og at dokumentere og modellere disse. & \textit{Tilføjet:} \textbf{PO. AO.} Brugsmønstre. Funktioner. Aktører. Kriterier. Forbilleder. \textit{Revideret:} \textbf{PO.} Prototype 2.  & Prototype 2.              }
\ourrow{3   }{At modellere systemet, ved at implementere funktioner og sikre os, at de definerede kriterier bliver opfyldt.     & \textit{Tilføjet:} \textbf{Implementering.} Sammensætning af rapport. Komponenter. Parsing af opskrifter. \textit{Revideret:} \textbf{PO. AO. Design.}      &                           }
\ourrow{4   }{                                      & \textit{Tilføjet:} \textit{Revideret:}                                 &                           }
}
