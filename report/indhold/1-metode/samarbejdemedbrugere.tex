% hvem er de
% hvad laver de
% hvad laver vi med dem

% hvad har de været ind over?
% inddrage, deltagelse

\section{Samarbejde med brugere}
\label{sec:samarbejde}

Vi mener, at systemet skal være rettet mod brugerenes behov og ønsker. Derfor valgte vi to informanter, som hører til to relative forskellige forbrugsområder, hvilket gav et større indblik i andre forbrugere, som vi ikke kan nå. Vi har konsulteret vores 2 informanter med hensyn til inspiration og opklaring af gruppediskussioner. Hvis vi har været i tvivl om eksempelvis en klasse eller struktur skulle indgå i systemet, tog vi kontakt til vores informanter for at høre deres meninger. Informanterne repræsenterer nemlig de brugere, som vi ønsker vores systemet skal appellere til, og derfor er det i sidste ende deres mening der tæller.

Informanter vil også være en stor del af afprøvnings- og kvalitetssikringsdelene af projektet. De skal være med til at sikre, at vi udvikler et system, der stemmer overens med systemdefinitionen og dækker deres behov med hensyn til madlavning i husstanden. Dette gør vi blandt andet ved at lave prototyper af systemet, og præsenterer og teste disse på informanterne. Prototyperne har indeholdt de givne funktioner eller designs, som vi har ønsket at få feedback på.