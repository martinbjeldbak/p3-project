\section{Samarbejde med brugere}
\label{sec:samarbejde}

Gennem projektet har vi arbejdet sammen med to informanter, som vi vurderede kunne være potentielle brugere af et system, der kan løse problemet med madspild i danske husstande. Dette afsnit har til formål at præsentere disse to informanter for læseren og beskrive, hvordan vi har interageret med dem igennem projektforløbet. De to informanter vil blive refereret til via navnene Merete og Keld igennem hele rapporten.

\begin{description}
\item[Merete] er en familiemor, der bor med sin mand. De har fire børn, som alle er flyttet hjemmefra. Merete står for madlavningen i husstanden. Både Merete og hendes mand har et arbejde, der skal passes, derfor føler de, at det ofte er svært at planlægge madlavningen i forvejen.

\item[Keld] er en familiefar, der bor med sin kone og deres to små døtre. Keld står for både madlavningen og indkøb af varer til familien. Familien føler ikke, at de har meget fritid i hverdagen, derfor vælger Keld at lave store portioner aftensmad, så han ikke behøver at lave mad hver aften. Både Keld og hans kone har et arbejde, der skal passes, og når de kommer hjem fra arbejde, så ønsker de at bruge så meget tid som muligt på at være sammen med deres døtre.
\end{description}

Begge informanter var med til at gøre det muligt for os at fortolke og forstå problemstillingen set fra deres perspektiv. De var en stor del af afprøvnings- og kvalitetssikringen af systemet. Denne samarbejdsproces er kort illustreret i \tableref{table:iterationeroverblik}, hvor man kan se, hvorledes informanterne er blevet inddraget i arbejdet i de forskellige faser.

%møder
Indledningsvis mødtes vi med informanterne hver for sig, og diskuterede problemet omkring madspild, og hvordan de oplevede madlavningen i deres respektive husstande. Efter vi havde diskuteret og fortolket problemet, var vi nu i stand til at formulere to systemdefinitioner, beskrevet i \secref{sec:systemdefinition}, som vi kunne præsentere for informanterne. Der var nu tale om møder, hvor fokus lå på, hvordan vi kunne løse dette problem, vi havde diskuteret i de foregående møder. 

%prototyper
Efter disse initierende møder, havde vi brug for at afprøve vores idéer og tanker vedrørende løsningsmetoder. Dette gjorde vi ved at udarbejde prototyper af de idéer, vi havde fået som resultat af diskussionerne med informanterne samt arbejdet med problemet. Tidligt i forløbet brugt vi papirsprototyper, som ikke krævede meget tid eller mange ressourcer at fremstille. Dette er en prototype, der er lavet af papir, som styres af et gruppemedlem, mens informanten lader som om, at der bliver trykket på en knap. Alt efter, hvad informanten gør, ændrer prototypen sig i forhold til denne interaktion.

Lidt længere inde i projektforløbet benyttede vi diasshow-prototyper, der havde til formål at undersøge systemets funktionalitet. Med sådan en prototype blev informanterne præsenteret for en prototype, der var dynamisk. Man kunne klikke på knapper og navigere rundt i diasshowet. Denne interaktion foregik på en computer. Begge prototyper er forklaret yderligere i \chapref{chap:designafbrugergraenseflade}, hvor systemets brugergrænseflade bliver præsenteret. Systemets design er blevet til som følge af de afprøvninger, vi havde foretaget med prototyper.

Afslutningsvis præsenterede vi det funktionsdygtige system for informanterne. Systemet var baseret på alle de foregående prototyper og designidéer, som var blevet præsenteret for og diskuteret med informanterne. Ved at præsentere informanterne for det funktionsdygtige system, blev vi i stand til at sikre os en vis kvalitet, inden vi afsluttede projektet og produktudviklingen. Til de sidste afprøvninger var der mulighed for at opdage kritiske fejl, i og med vi lod personer, der ikke havde været med til at udvikle systemet, bruge og teste det. Her kan brugeren foretage sig nogle valg i systemet, som vi måske ikke havde overvejet. Disse uforudsete valg skulle ikke være skyld i, at systemet blev ubrugelig eller gik ned, og denne sidste afprøvning var en god sikring mod dette. Disse afsluttende test bliver præsenteret i \secref{sec:usability}.

%GAMLE AFSNIT

%Vi mener, at systemet skal være rettet mod brugerenes behov og ønsker. Derfor valgte vi to informanter, som hører til to relative forskellige forbrugsområder, hvilket gav et større indblik i andre forbrugere, som vi ikke kan nå. Vi har konsulteret vores 2 informanter med hensyn til inspiration og opklaring af gruppediskussioner. Hvis vi har været i tvivl om eksempelvis en klasse eller struktur skulle indgå i systemet, tog vi kontakt til vores informanter for at høre deres meninger. Informanterne repræsenterer nemlig de brugere, som vi ønsker vores systemet skal appellere til, og derfor er det i sidste ende deres mening der tæller.

%Informanter vil også være en stor del af afprøvnings- og kvalitetssikringsdelene af projektet. De skal være med til at sikre, at vi udvikler et system, der stemmer overens med systemdefinitionen og dækker deres behov med hensyn til madlavning i husstanden. Dette gør vi blandt andet ved at lave prototyper af systemet, og præsenterer og teste disse på informanterne. Prototyperne har indeholdt de givne funktioner eller designs, som vi har ønsket at få feedback på.

% hvem er de
% hvad laver de
% hvad laver vi med dem

% hvad har de været ind over?
% inddrage, deltagelse
% Personas?
% ingen navne. anonyme informanter.

%PROTOTYPER (SKAL MULIGVIS RYKKES NED I AKADEMISK RAPPORT)

%Vi vil benytte os af papirsprototyper til de initierende afprøvninger. Denne form for prototype er ikke tidskrævende og er en billig form for præsentation. I og med at det er initierende afprøvninger, så mener vi ikke at det er en god ide, at bruge for mange kræfter på at udvikle prototypen. Der er nemlig en risiko for, at vi ender med at have brugt så meget tid på prototypen, at vi vil have svært ved at give slip på idéen, hvis informanterne ønsker noget helt andet end prototypen illustrerer. I de senere afprøvninger, kan vi bruge mere tid på prototyperne, da der her er blevet fastlagt nogle rammer, og vi er blevet mere bevidste om informanternes behov. Med denne viden, vil vi kunne afgrænse systemets udviklingsproces, og vi vil undgå at ende med et system, som ikke ligger indenfor rammerne. Dermed er risikoen, for at det system vi ender op med, er helt anderlede i forhold til informanterne ønsker, meget lille.

%Vi kan i et lidt senere forløb i projektet præsentere informanterne for diasshow-prototyper, der har til formål at undersøge systemets funktioner. Med sådan en prototype bliver informanterne præsenteret for en prototype, der er dynamisk. Man kan klikke på knapper og navigere rundt i diasshowet. En diasshow-prototype giver os og informanterne mulighed for at afteste vores idéer for systemet. 

%Systemets brugbarhed er en vigtig faktor for os. Vi ønsker at gøre det lettere for den madansvarlige i husstanden at bruge sine madrester i madlavningen. Derfor skal systemet være intuitivt og nemt at gå til. Med diasshow-prototyper får vi mulighed for at sikre os, at vores designidéer bliver forstået af informanterne. Hvis informanterne \fx har svært ved at finde nogle funktioner, så kan det være, at de skal gøres mere synlige. Lignende spørgsmål bliver besvaret relativt tidligt i systemets udviklingsfase, hvilket er en fornuftig ting. Det giver os mere tid til at rette fejl og komme på nye designidéer, hvis det bliver nødvendigt.

%Som nogle af de sidste afprøvninger, vil vi præsentere et funktionsdygtigt system for informanterne. Dette system vil blive baseret på alle de foregående prototyper og designidéer. Ved at præsentere informanterne for det funktionsdygtige system, bliver vi i stand til at sikre os en vis kvalitet, inden vi afslutter projektet og produktudviklingen. Til de sidste afprøvninger er der mulighed for at opdage kritiske fejl, fordi vi lader en person, der ikke har været med til at udvikle systemet, bruge og afteste det. Her kan brugeren foretage sig nogle valg i systemet, som vi måske ikke har tænkt over. Disse uforudsete valg skal ikke være skyld i at systemet går ned, og denne sidste afprøvning er en god sikring for dette. 
