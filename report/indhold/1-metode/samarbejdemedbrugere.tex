% hvem er de
% hvad laver de
% hvad laver vi med dem

% hvad har de været ind over?
% inddrage, deltagelse

\section{Samarbejde med brugere}
\label{sec:samarbejde}

Vi mener, at systemet skal være rettet til brugerenes behøv og ønsker. Derfor valgte vi to informanter, som hver hører til to relative forskellige forbrugsområder hvilket gav et større indblik til andre forbrugere, som vi ikke kan nå. Vi har konsulteret vores 2 informanter med hensyn til inspiration og opklaring af gruppediskussioner. Hvis vi har været i tvivl om hverken en klasse eller struktur skulle indgå i systemet, tog vi kontakt til vores informanter for at høre deres meninger.

Informanter vil også være en stor del af afprøvnings- og kvalitetssikringsdelene af projektet. De skal være med til at sikre, at vi udvikler et system, der stemmer overens med systemdefinitionen og dækker deres behov med hensyn til madlavning i husstanden. Dette gr vived at udvikle forskellige prototyper, som vi præsenterer for informanterne til møderne . Diss prototyper repræsenterer de id\'{e}er, i har til en given funktion eller systemsdel, som vi ønsker at få feedback på.
